% GenCom.Tex in Tetexmf
% General Commands that can be used in any TeX document
% Revised: 08/20/2013
   \newcommand{\Ref}[1]{(\ref{#1})}
		 \newcommand{\I}[2]{#1_{1}\cdots #1_{#2}}
			\newcommand{\J}[2]{#1_{1},\ldots ,#1_{#2}}
			\newcommand{\K}[2]{#1_{1} < \cdots < #1_{#2}}
			\newcommand{\Q}{ \hfill \rule{.5em}{1ex}}
			\newcommand{\sgn}{\mbox{sgn}}
			\newcommand{\Div}{\mbox{div}}
			\newcommand{\Grad}{\mbox{grad}}
			\newcommand{\grad}{\mbox{grad}}
			\newcommand{\T}[5]{\frac{\partial #1^{#3_{1}}}{\partial #2^{#4_{1}}} \cdots \frac{\partial #1^{#3_{#5}}}{\partial #2^{#4_{#5}}}}
			\newcommand{\D}{\,d}
			\newcommand{\E}{\noindent {\bf Example}: \hspace{.4em}}
			\newcommand{\Es}{\noindent {\bf Examples}: \hspace{.4em}}
			\newcommand{\Exer}{\noindent {\bf Exercise}: \hspace{.4em}}
			\newcommand{\Exers}{\noindent {\bf Exercises}: \hspace{.4em}}
			\newcommand{\R}{\noindent {\bf Remark}: \hspace{.4em}}
			\newcommand{\Rs}{\noindent {\bf Remarks}: \hspace{.4em}}
			\newcommand{\h}{\hspace{1.3em}}
			\newcommand{\V}{\vspace{1ex}}
			\newcommand{\VV}{\vspace{2ex}}
			\newcommand{\VA}{\vspace{.5ex}}
			\newcommand{\VHalf}{\vspace{.5ex}}
   \newcommand{\StartSkip}[1]{}
			\newcommand{\Prf}{\noindent {\bf Proof}: \hspace{.4em}}
			\newcommand{\Proof}[1]{\noindent {\bf Proof}{\ #1} \hspace{.4em}}
			\newcommand{\LU}[3]{{{#1}_#2}^{#3}}
			\newcommand{\UL}[3]{{{#1}^#2}_{#3}}
			
						% Commands for Index Forming
\newcommand{\IndA}[1]{\index{#1}} % Single Index for #1; no original #1
\newcommand{\IndAA}[1]{#1\index{#1}} % Single Index for #1; type #1

\newcommand{\IndB}[2]{\index{#1!#2}} % Index #2 under #1; no original #2
\newcommand{\IndBB}[2]{#2\index{#1!#2}} % Index #2 under #1; type #2

\newcommand{\IndC}[3]{\index{#1!#2!#3}} % Index #3 under subindexed #2; no #3
\newcommand{\IndCC}[3]{#3\index{#1!#2!#3}} % Index #3 under subindexed #2; type #3

\newcommand{\IndD}[2]{\index{#1|see{#2}}} % To find #1, see #2. Does not type #1
\newcommand{\IndDD}[2]{#1\index{#1|see{#2}}} % To find #1, see #2. Does type #1

\newcommand{\IndBD}[2]{\IndB{#1}{#2}\IndD{#2}{#1}} % Combination of IndB and IndD on same topics: indexes #2 under #1, to find #2 see #1. Does not type #2 
\newcommand{\IndBBD}[2]{#2\IndB{#1}{#2}\IndD{#2}{#1}} % Same as \IndBD, but also types #2 in text


   \newcommand{\C}[2]{\left(\begin{array}{c} {#1} \\ {#2} \end{array} \right)}

   \newcommand{\RR}{\rm I\kern-.25em R} % Blackboard Bold R
   \newcommand{\CC}{\rm I\kern-.50em C} % Blackboard Bold C
   \newcommand{\HH}{\rm I\kern-.25em H} % Blackboard Bold H
   \newcommand{\QQ}{\rm I\kern-.50em Q} % Blackboard Bold Q
   \newcommand{\NN}{\rm I\kern-.25em N} % Blackboard Bold N
   \newcommand{\ZZ}{\it I\kern-.70em Z} % Blackboard Bold Z
   
   %% Old version included `\mbox{}"; e.g.,
   %%  \newcommand{\ZZ}{\mbox{\rm I\kern-.25em Z}}

\newcommand{\Subsetneq}{
\raisebox{.4ex}{\mbox{$\subset$}}_{{\mbox{{\kern-.7em $\neq$}}}}}

\newcommand{\Supsetneq}{
\raisebox{.4ex}{\mbox{$\supset$}}_{{\mbox{{\kern-.7em $\neq$}}}}}


   \newcommand{\subsetne}{\mbox{\raisebox{-1.10ex}{$\stackrel{\textstyle{\subset}}{\scriptscriptstyle{\neq}}$
}}} % Subset of, but not equal to

   \newcommand{\supsetne}{\mbox{\raisebox{-1.10ex}{$\stackrel{\textstyle{\supset}}{\scriptscriptstyle{\neq}}$}}} % Superset of, but not equal to

   \newcommand{\Vect}[1]{\mbox{$\stackrel{\textstyle{\rightarrow}}{#1}$}} % Same as \vec, but with longer arrow
   % NOTE: \Vec seems to conflict with amsart document class

   \newcommand{\spo}{{\sf so}} % In place of Gothic (Lie Algebra)

\newcommand{\Bfm}[1]{\mbox{\boldmath ${#1}$}} %Bold Math Symbols

\newcommand{\Arcsin}{\mbox{Arcsin}} % Principle branch of arcsine: values in [-pi/2,pi/2]

\newcommand{\Arctan}{\mbox{Arctan}} % Principle branch of arctan: values in (-pi/2, pi/2)

\newcommand{\Sin}{Sin} % Restriction of standard sine function to [-pi/2,pi/2]

\newcommand{\Tan} % Restriction of standard tangent function to (-pi/2,pi/2)

