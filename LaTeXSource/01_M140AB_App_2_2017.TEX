% M140AB_A.TeX  Notes for `Single-Variable Analysis': Appendix B

%
% Revised: 01/23/2013 
%



                        \chapter{Further Results in Set Theory}
                        \label{SectAAdd2}

\VV

        \underline{Quotes for Appendix~B}\IndB{chapter quotes}{for Appendix~B (Further Results in Set Theory)}:

\V

\begin{quotation}
{\footnotesize

        (1) Fake Quote

}%EndFootNoteSize
\end{quotation}


        In this appendix we consider some standard results in Set Theory with which every mathematician should eventually become familiar;
    however, this material is not required for reading the preceding chapters.

\VV

                 \subsection{\small{\bf Theorem (DeMorgan's Laws for Sets)}}
            %\label{ThmA10.30}
            \index{DeMorgan's laws for sets}

        Let ${\cal A}$ be a nonempty family of sets, and let $X$ be a set; we do {\em not} assume that $X$ is a member of the family ${\cal A}$.
    Let ${\cal B}$ be the family of all sets $Y$ which can be expressed in the form $Y \,=\, X{\setminus}A$ for at least one set $A$ in the family ${\cal A}$. Then

\V

        (i) $X{\setminus}\left({\bigcup}\, {\cal A}\right) \,=\, {\bigcap}\, {\cal B}$.
    \h
        (ii) $X{\setminus}\left({\bigcap}\, {\cal A}\right) \,=\, {\bigcup}\, {\cal B}$.

\noindent Using the alternate notation introduced in Part~(d) of Definition~\Ref{DefA10.15} above,
    these set identities can be written, without needing to introduce the family~${\cal B}$, as follows:
        \begin{displaymath}
        (i)\,\, X{\setminus}\left({\bigcup}_{A{\in}{\cal A}}\, A\right) \,=\, {\bigcap}_{A{\in}{\cal A}} (X{\setminus}A) \h
        (ii)\,\, X{\setminus}\left({\bigcap}_{A{\in}{\cal A}}\, A\right) \,=\, {\bigcup}_{A{\in}{\cal A}} (X{\setminus}A).
        \end{displaymath}
\V
\V

        \underline{Proof of (i)} (The proof of (ii) is similar, and is left as an exercise): Suppose that $c$ is an element of the set $X{\setminus}\left({\bigcup}\, {\cal A}\right)$.
	Then $c$ is in $X$ but $c$ is not in ${\bigcup}\,{\cal A}$.
    To say that $c$ is not in ${\bigcup}\,{\cal A}$ means that for each $A$ in the family ${\cal A}$, $c$ is not in $A$.
	Thus, for each such $A$ one has $c$ in $X$ but $c$ not in $A$;
    hence one has $c{\in}(X{\setminus}A)$.
    Thus, $c$ is in each of the sets of the family ${\cal B}$, hence $c$ is in ${\bigcap}\,{\cal B}$.
	It follows that
        \begin{displaymath}
        X{\setminus}\left({\bigcup}\, {\cal A}\right) \,{\subseteq}\, {\bigcap}\,{\cal B} \h ({\ast})
        \end{displaymath}

		Likewise, suppose that $p$ is an element of ${\bigcap}\,{\cal B}$.
    Then $p$ is in each of the sets of the family ${\cal B}$;
    that is, $p$ is in every set of the form $X{\setminus}A$ with $A$ in the family ${\cal A}$.
    That is, for each such $A$ the point $p$ is in $X$ but not in $A$.
    Since $p$ fails to be in any of these sets $A$, it follows that $p$ is also not in ${\bigcup}\,{\cal A}$;
    and since $p$ is in $X$, it then follows that $p$ is in $X{\setminus}\left({\bigcup}\,{\cal A}\right)$.
    That is,
        \begin{displaymath}
        {\bigcap}\,{\cal B} \,{\subseteq}\, X{\setminus}\left({\bigcup}\, {\cal A}\right) \h ({\ast}{\ast})
        \end{displaymath}


        The set relations $({\ast})$ and $({\ast}{\ast})$ then imply, by Theorem~\Ref{ThmA10.10},
    that $ X{\setminus}\left({\bigcup}\, {\cal A}\right) \,=\, {\bigcap}\,{\cal B}$, as claimed.
    

\V

        {\bf Remark} The preceding theorem is sometimes expressed informally as follows:

        \h (i)\, The complement of a union is the intersection of the complements;

       \h (ii) The complement of an intersection is the union of the complements.


\VV

\V

        \subsection{\small{\bf Remark} Russell's Paradox}
        %\label{RemrkA10.35}
        \IndBD{set theory}{Russell's Paradox}

        The theory of sets is a very deep subject, and is still an area of active research.
    Fortunately, the use of sets in {\TheseNotes} does not require a deep knowledge of this theory.
    Nevertheless, it is appropriate to mention one topic from the wider theory of sets.

        The basic issue arises directly from the definition of a `set' as being simultaneously a collection of objects and an object in its own right.
    As has already been seen, this allows the construction of sets whose individual elements are themselves sets.

\V

        {\bf Example}: Let us say that a set $Y$ has Property~R provided that the set $Y$,
    thought of as an object in its own right, is {\em not} an element of the set~$Y$, thought of as a collection of objects.
    (Note that all the sets we have considered so far do have Property~R.)
    Now define the set $X$ be the family of all sets $Y$ such that $Y$ has Property~R.

        \underline{Question} Does the set $X$ have Property~R?

        \underline{Analysis} Suppose that the set $X$ {\em does} have Property~R.
    Then, by the very definition of the set $X$ just given, the elements of $X$ include {\em every} set having Property~R and thus include $X$ itself.
    That is, $X$, viewed as an object, is an element of $X$, viewed as a set. However, by definition of Property~R,
    this last fact means that $X$ does not have Property~R, which contradicts the supposition that $X$ does have Property~R.

        Now suppose that $X$ does {\em not} have Property~R. Then, by definition of `Property~R',
    the set $X$, thought of as an object, {\em is} an element of the set $X$.
    But by the very definition of $X$ given above, each element of the set $X$ -- including now $X$ itself --
    is a set which has Property~R, contradicting the supposition that $X$ does not have Property~R.

    In summary: assuming that $X$ {\em does} have Property~R implies that $X$ does {\em not} have Property~R;
    while assuming $X$ does {\em not} have Property~R implies that $X$ {\em does} have Property~R.
        This mysterious state of affairs is called {\bf Russell's Paradox},
    in honor of the British philosopher Bertrand Russell, who discovered it in~$1901$.
    It is well beyond the scope of {\TheseNotes} to consider a proper analysis of such set-theoretic paradoxes;
    for that, the reader should pick up a book on the logical foundations of mathematics.
    Instead, we follow the common approach in Analysis and assume that the sets we deal with are all subsets of some fixed,
    but unspecified, `universal set' which is big enough for our purposes but `small enough' to avoid such paradoxes.
    In particular, this approach allows us to use only sets $X$ which satisfy the relation $X \not \in X$.

\VV


        (4) Suppose that $X$ is a nonempty set and let $k$ be a natural number. If $X_{1}$, $X_{2}$,\,{\ldots}\,$X_{k}$ are nonempty subsets of $X$,
    then the set of all $k$-tuples $g:{\NN}_{k} \,{\rightarrow}\, X$ such that $g(j){\in}X_{j}$ for each $j \,=\, 1,2,\,{\ldots}\,k$ is denoted $X_{1}{\times}X_{2}{\times}\,{\ldots}\,{\times}X_{k}$; it is called the {\bf Cartesian product} of these sets.
    This is sometimes written more briefly by expressions such as ${\prod}_{j \,=\, 1}^{n} X_{j}$, where, as usual, the symbol $\prod$ stands for `product'.

        More generally, suppose that ${\varphi}:Z \,{\rightarrow}\, {\cal P}(X){\setminus}\{{\emptyset}\}$ is a function defined on a nonempty set $Z$, 
    with values being nonempty subsets of $X$.
    For each $i$ in $Z$ let $X_{i} \,=\, {\varphi}(i)$.
    Then the corresponding {\bf Cartesian product of the indexed family $\Bfm{X}_{i}$}, denoted ${\prod}_{i{\in}Z} X_{i}$,
    consists of the functions $g:Z \,{\rightarrow}\, X$ such that $g(i){\in}X_{i}$ for each $i$ in $Z$.
    The set $X_{i}$ is called the {\bf $\Bfm{i}$-th factor} in this product.

        It is convenient to extend these ideas to cases in which one of the `factors' is the empty set by declaring that such a product to also be the empty set.

        \underline{Remark} For small values of $k$ it is customary to use older language.
    For instance, an ordered $2$-tuple is often called an {\bf ordered pair}. Of course this conflicts with the primitive (i.e., Kuratowski) concept of `ordered pair.
    The way to get around this ambiguity is to think of the definition of $2$-tuple (as a function defined on the set $\{1,2\}$) as the `new, improved' type of ordered pair, while the Kuratowski definition describes the 'primitive ordered pair'.
    From this point on in {\TheseNotes} the phrase `ordered pair' means this new, improved version unless stated otherwise.

\V
\V

        Associated with a set $X$ is a second set whose elements are precisely the subsets of $X$,
    each such subset being thought of as a single object in its own right.

\V

        \subsection{\small{\bf Definition}}
        %\label{DefA10.40}

        Let $X$ be a set. The {\bf power set of $\Bfm{X}$}\IndBD{set theory}{power set} is the collection whose elements are precisely the subsets of $X$.
    That is, to say that an object $C$ is an element of the power set of $X$ means that $C$ is a subset of $X$.
    Some authors denote the power set of $X$ by the symbol ${\cal P}(X)$; some denote it by the symbol $2^{X}$ (whence the name `{\em power} set').
    We shall use the ${\cal P}(X)$ notation in {\TheseNotes}.
    (See Example~(5) below for the origin of the `exponential' notation $2^{X}$ mentioned here.)

\V

        \subsection{\small{\bf Examples}}
        %\label{ExampA10.50}

\hspace*{\parindent}
        (1) Suppose that $X \,=\, \{c\}$ for some object $c$, so that $X$ is a singleton  set.
    It is obvious that ${\cal P}(X)$ has exactly two (different) subsets, namely $X$ itself and the empty set ${\emptyset}$.
    (You should try to provide a correct proof before going on. You might also wish to read the \Note\, `on the power set of a singleton set'.)

\V

        (2) A similar analysis, which is left as an exercise, shows that if $X \,=\, \{c,d\}$ where 
$c \,{\neq}\, d$, (i.e., if $X$ is a doubleton set),
    then ${\cal P}(X) \,=\, \{{\emptyset}, \{c\}, \{d\}, X\}$.
	In particular, if $X$ has exactly two elements, then ${\cal P}(X)$ has exactly four elements.

\V

        (3) The empty set ${\emptyset}$ has a subset, namely itself; see Part~(b) of Theorem~\Ref{ThmA10.25}.
    It is also clear that ${\emptyset}$ has no other subsets, and thus that ${\cal P}({\emptyset}) \,=\, \{{\emptyset}\}$.
    In particular, the power set of the empty set is \underline{not} the empty set, since ${\cal P}({\emptyset})$ does have an element, namely the set ${\emptyset}$, thought of as an object in its own right.

\V

        (4) It is possible that $X$ and ${\cal P}(X)$ can have one or more elements in common.
    For instance, let $c$ be an object, and let $X \,=\, \{c,\{c\}\}$.
    Thus, $X$ is a set with two distinct elements, namely the object $c$ and the singleton set $\{c\}$, thought of as an object in its own right.
    It follows from the results of Example~(2) above that the power set ${\cal P}(X)$ consists of the four elements ${\emptyset}$, $\{c\}$, $\{\{c\}\}$ and $X$.
    Notice that the second entry in this list is the set $\{c\}$, which is also an element of $X$.

        A simpler example of this type is $X \,=\, \{{\emptyset}\}$;
    that is, $X$ is as in Example~(1) above, with $c$ equal to the empty set ${\emptyset}$, thought of as an object in its own right.
    Then the result of Example~(1) takes the form ${\cal P}(X) \,=\, \{{\emptyset},X\}$.
    In particular, the object ${\emptyset}$ is both an element of $X$ and an element of ${\cal P}(X)$.

\V

        (5) We shall see later that if $X$ is a set with exactly $n$ members, where $n$ is some natural number,
    then ${\cal P}(X)$ has exactly~$2^{n}$ members; and of course if $X$ has no members (i.e., if $X \,=\, {\emptyset}$),
    then ${\cal P}(X)$ has~$1 \,=\, 2^{0}$ members. The exponential notation $2^{X}$ grew out of these facts.

\V

%%% 
\begin{quotation}
{\footnotesize \underline{\Note} (on the power set of a singleton set)\IndB{\notes}{on the power set of a singleton set}: 
    In Example (1) above it is asserted that the fact that $X \,=\, \{c\}$ has exactly two subsets, namely, $X$ itself and ${\emptyset}$.
    Whenever you see such an assertion of `obviousness' in a math context, beware:
    the author may be trying to slide something by you. And even if you agree that the stated result is `obvious',
    you should always be prepared to give a {\em proper} proof; that is, a proof which arises from definitions and previously-accepted, results using correct logic.
    For example, the following is a `fake proof':

\VA

        \h `The sets $X$ and ${\emptyset}$ are certainly subsets of $X$, and I can't think of any others.'

\VA

\noindent Here is a proper proof:

        First note that, by Part~(b)(i) of Theorem~\Ref{ThmA10.25}, the sets ${\emptyset}$ and $X$ are, indeed, subsets of $X$;
    in particular, by the definition of `power set' and `subset', $\{{\emptyset}, X\} \,{\subseteq}\, {\cal P}(X)$.
    (This is essentially how the `fake proof' above starts.)

        To see that there are no other subsets of $X$, i.e., no other elements of ${\cal P}(X)$, let $Y$ be any nonempty subset of $X$.
    Since, by hypothesis, $Y \,\,{\neq}\,\, {\emptyset}$, it follows,
    by the definition of `empty set', that there must be at least one element in the set $Y$.
    Let $d$ be any element of $Y$. Since, by hypothesis, $Y \,{\subseteq}\, X$, it follows (from the definition of `subset') that $d$ is an element of $X$.
    However, by the Fundamental Principle of Set Theory, combined with the hypothesis that $X \,=\, \{c\}$,
    so that $c$ is the {\em only} element of $X$, it follows that $d \,=\, c$.
    That is, every element of $Y$ equals $c$, so $Y$ and $X$ have exactly the same elements,
    namely the single element $c$. Thus, by the Fundamental Principle of Set Theory, one has $Y \,=\, X$.
    That is, if $Y \,{\subseteq}\, X$ and $Y \,\,{\neq}\,\, {\emptyset}$, then $Y \,=\, X$.
    It follows that ${\cal P}(X) \,{\subseteq}\, \{{\emptyset},X\}$. However, it was already noted above that ${\cal P}(X) \,{\supseteq}\, \{{\emptyset},X\}$.

    
    In the latter case, $Y$ must have at least one element (by definition of `not the empty set'),
    and any such element must be an element of $X$ (by definition of $Y$ being a {\em subset} of $X$).
    If $d$ is any element of $Y$, then, as just shown, $d$ must be an element of $X$ and thus, by the hypothesis that $X \,=\, \{c\}$, one must have $d \,=\, c$.
}%EndFootnotesize
\end{quotation} 
%##


\V
\V


        \underline{Preliminary Comment} One knows that if $X$ is a finite set with exactly $n$ elements,
    then the corresponding power set ${\cal P}(X)$ has exactly $2^{n}$ elements. In particular, the power set ${\cal P}(X)$ has more elements than the original set $X$;
    moreover, the larger $n$ is, the greater the difference in the sizes of the sets.
    This observation may make the next result seem almost `obvious'.

\V

            \subsection{\small{\bf Theorem} (Cantor's Power-Set Theorem)} \IndBD{set theory}{Cantor's Power-Set Theorem}
            %\label{ThmA20.90}

        Let $X$ be a set, and let $Y \,=\, {\cal P}(X)$ be the corresponding power set of $X$.

    Then $X$ does {\em not} have the same cardinality as $Y$; that is, there does not exist a bijection of $X$ onto ${\cal P}(X)$.
    Indeed, even more can be said; namely, if $X \,\,{\neq}\,\, {\emptyset}$ and $F:X \,{\rightarrow}\, {\cal P}(X)$ is any function with domain $X$ and values in ${\cal P}(X)$,
    then $F$ does {\em not} map $X$ onto ${\cal P}(X)$; that is, $F$ is {\em not} a surjection of $X$ onto~${\cal P}(X)$.
    In particular, anyone who claims to have constructed an example of a bijection of $X$ onto ${\cal P}(X)$ is wrong,
    since there does not exist a surjection, much less a bijection.

\V


        \underline{Proof}: \underline{Case 1} Suppose that $X$ is the empty set.
    Recall that ${\cal P}({\emptyset}) \,=\, \{{\emptyset}\}$, so ${\cal P}({\emptyset})$ is a nonempty set.
    Since, as has already been noted, the empty set does not have the same cardinality as a nonempty set,
    it follows that the claimed result is true when $X$ is empty. (More intuitively: The empty set has no elements, while ${\cal P}({\emptyset})$ has one element.)

\V

        \underline{Case 2} Suppose that $X$ is nonempty, and suppose that $F:X \,{\rightarrow}\, {\cal P}(X)$ is a function with domain $X$ and values in ${\cal P}(X)$.
    Let $S$ be the set of all $x$ in $X$ such that $x$ is {\em not} an element of the set $F(x)$.

        \h \underline{Claim} The set $S$ is not in the image of the function~$F$.

        \h \underline{Proof of Claim} Suppose, to the contrary, that $S$ can be expressed as $F(x_{0})$ for some $x_{0}$ in $X$.
    If $x_{0}$ is in $S$, then (by definition of $S$) $x_{0}$ is {\em not} an element of  $F(x_{0})$; that is, since $S$ is supposed to equal $F(x_{0})$, one has $x_{0}$ {\em not} in $S$.
    Similarly, if $x_{0}$ is {\em not} in $S$, then (again by the definition of $S$) $x_{0}$ must be an element of $F(x_{0})$; that is, $x_{0}$ is an element of $S$.
    Thus, assuming that there exists $x_{0}$ in $X$ such that $S \,=\, F(x_{0})$ leads to the existsence of an element, namely this $x_{0}$,
    which is simultaneously an element of $S$ and {\em not} an element of~$S$. No such $x_{0}$ can exist, hence $S$ is not in the image of $F$.


\V
\V

            \subsection{\small{\bf Corollary}}
            %\label{RemrkA20.92}

        There exist sets which are uncountable.

\V

        {\bf Proof}\, 

\V

        Let $Y \,=\, {\cal P}({\NN})$, the power set of ${\NN}$.
    It is clear that $Y$ is not finite since it has the infinite subset
    $\{\{n\}: n{\in}{\NN}\}$ whose elements are the singleton sets of natural numbers.
    In addition, by Cantor's Power-Set Theorem above, $Y$ does not have the same cardinality as ${\NN}$;
    that is, $Y$ is not countably infinite. It follows that the set $Y$ is not countable.

\V
\V

        The uncountable set ${\cal P}({\NN})$ is difficult to visualize. Fortunately, it is possible to interpret it geometrically as a subset of the real number line.
    The basic idea is to use the representation of subsets of ${\NN}$, in terms of sequences of $0$s and~$1$s, discussed in Example~\Ref{ExampA40.30}~(6).
    For convenience we break up the discussion into several parts.

\V



                        \section{`Ordered Pairs' in Terms of Sets (Kuratowski)}
                        %\label{SectA12A}

\V


    In the current section we illustrate the `reduction to set theory' process by expressing the important concept of `an ordered pair of objects' purely in terms of sets.

        Most students first encounter the idea of an `ordered pair of objects' in analytical geometry:
    one characterizes the location of a point (i.e., a `dot') in the $xy$-plane in terms of a pair of numbers,
    the `Cartesian coordinates' of the point (relative to a choice of origin and coordinate axes).
    The order in which these numbers are written makes a difference; for instance,
    the pair $(2,-3)$ corresponds to a geometric point in the fourth quadrant of the $xy$-plane,
    while the pair $(-3,2)$ corresponds to a geometric point in the second quadrant.

        More generally, if $x$ and $y$ are any objects, one seeks a purely set-theoretic way to encode the main information contained in intuitive concept of `the ordered pair $(x,y)$':

\VA

        \h (i) The concept should be expressed purely in terms of sets formed from the objects $x$ and~$y$.

        \h (ii) If $x \,\,{\neq}\,\, y$, then the definition should allow one to distinguish which of these objects is to be the `first' and which is to be the `second'.

\VA

\noindent Notice that the `obvious' solution, namely to define the desired ordered pair as the set $\{x,y\}$, does not satisfy Condition~(ii);
    indeed, one has $\{x,y\} \,=\, \{y,x\}$, so this doubleton set does not single out one of the objects $x$ or $y$ as somehow being `preferred'.

        Around 1914 the American mathematician Norbert Wiener developed a suitable set-theoretic treatment of `ordered pairs';
    this was simplified by the Polish mathematician Kasimir Kuratowski in~1921. We follow Kuratowski's approach here.

\V
\V

        \subsection{\small{{\bf Definition} (Kuratowski Ordered Pairs})}
        \label{DefA12.10}\IndB{ordered pairs}{Kuratowski definition}

        Let $x$ and $y$ be objects. Then the {\bf Kuratowski ordered pair whose first entry is $\Bfm{x}$ and whose second entry is $\Bfm{y}$} is the set $\{\{x\},\{x,y\}\}$.
    If $x \,=\, y$ one says that the ordered pair is of {\bf Type~I}\IndC{ordered pairs}{Kuratowski definition}{Type I}, while if $x \,\,{\neq}\,\, y$ then it is of {\bf Type~II}\IndC{ordered pairs}{Kuratowski definition}{Type II}.

\V

        \underline{Remarks}

\V

        (1) For convenience one normally denotes the ordered pair whose first entry is $x$ and whose second entry is $y$ by the standard symbol $(x,y)$,
    where the first entry of the ordered pair appears on the left in this notation and the second on the right;
    in the spoken form, one then refers to `the ordered pair $x$\,~$y$', with the first entry of the pair spoken first.
    Note that neither of these conventions distinguishes `first entry' in a purely set-theoretic way:
    the phrase `on the left' introduces a visual aspect, while the phrase `spoken first' introduces a temporal aspect.

\V

        (2) In the Kuratowski formulation, the ordered pair with first entry $x$ and second entry $y$ is either a singleton set, if $x \,=\, y$, or a doubleton set, if $x \,\,{\neq}\,\, y$.

\V

        (3) As was indicated above, other set-theoretic definitions of `ordered pairs' have been proposed.
    These are not simply rewordings of the Kuratowski definition which describe the same objects.
    For instance, in the Kuratowski formulation the ordered pair whose first entry is the number $x$
    and whose second entry is the number~$y$ is the set $\{\{x\}, \{x,y\}\}$, thought of as a single object.
    In the earlier formulation of Norbert Wiener, however, the same ordered pair is defined to be the set $\{\{\{x\},{\emptyset}\}, \{\{y\}\}\}$.
    In particular, the Wiener version of the ordered pair $(x,y)$ is a set which includes the object $\{\{y\}\}$ as an element;
    in contrast, the Kuratowski version is a set which does {\em not} have $\{\{y\}\}$ as an element.
    It follows from the Fundamental Principle of Set Theory that these objects (i.e., sets) cannot be equal,
    so the definitions themselves, while superficially similar, are fundamentally different.
    More precisely, one cannot take, say, the Kuratowski definition of the ordered pair $(x,y)$
    and then prove that $(x,y)$ is also given by the Wiener definition.
    (Compare this situation with the discussion above on defining even and odd integers.)
    Instead, the purpose of such constructions in modern mathematics is to provide a precise construction, purely in terms of sets,
    of concrete objects whose set-theoretic properties correspond precisely to the properties our understanding of the given intuitive concept.

\V
\V

%%% 
\begin{quotation}
{\footnotesize \underline{\Note} (on Kuratowski ordered pairs)\IndB{\notes}{on Kuratowski ordered pairs}:
    The symbols used to formulate the preceding definition, when combined with the standard $(x,y)$ notation,
    may make it appear that almost nothing has really happened. However, this perception changes when one tries to use this definition `with one's eyes closed'.
    Indeed, the true test of whether the Kuratowski definition `works', i.e., whether it does accurately encode the intuitive idea of `ordered pair'
    in a purely set-theoretic way, is this: given any object $Z$, one ought to be able to tell,
    using set-theoretic ideas alone, whether $Z$ is an ordered pair in the sense of Kuratowski;
    if it is, one ought to be able to determine the original objects from which the ordered pair $Z$ is formed,
    and also to determine which of these objects is to be the `first'.
    This determination should involve only set-theoretic ideas, and not involve spatial notions such as `the object on the left',
    or temporal notions such as `the first object mentioned'. Here is how the Kuratowski formulation carries this out:

       \h Step 1: Given an object $Z$, determine whether it is a set.
    If it is not a set, then it cannot be an ordered pair, so stop. Otherwise:

       \h Step 2: If $Z$ is a set, determine whether it is either a singleton or a doubleton set.
    If $Z$ is neither a singleton set nor a doubleton set, then (Remark~2 above) it cannot be an ordered pair, so stop. Otherwise:

       \h Step 3:

            \h \h (a) If $Z$ is a singleton, determine whether its unique element is itself a singleton set.
    If it is not, then $Z$ cannot be an ordered pair, so stop. However, if the unique element of $Z$ {\em is} a singleton set $A$,
    let $x$ be the unique element of $A$. Then this analysis shows that $Z$ is the ordered pair $\{\{x\}\} \,=\, \{\{x\},\{x\}\} \,=\, \{\{x\},\{x,x\}\}$ 
    whose first entry is $x$ and whose second entry is also $x$. Success!

        \h \h (b) If $Z$ is a doubleton set, determine whether one of its two elements is a singleton set and the other is a doubleton set.
    If this is not the case, then $Z$ cannot be an ordered pair, so stop. Otherwise:

        Step 4: For convenience, call the element of $Z$ that is a singleton set $B$, and call the element of $Z$ that is a doubleton set $C$.
    If $B$ is not a subset of $C$, then $Z$ cannot be an ordered pair, so stop. Otherwise:

        Step 5: Let $x$ be the unique element of $B$, and let $D \,=\, C\,{\setminus}\,B$.
    Since $B$ is a subset of $C$, it is clear that $D$ is obtained by removing $x$ from $C$.
    It follows that $D$ is also a singleton set whose unique element does {\em not} equal $x$.
    Now let $y$ be the unique element of $D$. Then it is clear that $Z \,=\, \{\{x\},\{x,y\}\}$. Success!

\V

       \underline{Note} The preceding analysis may seem overly fussy, but it is forced on us because the object $Z$ may be quite complicated.

        \underline{Example}: Consider the following object:         \begin{displaymath}
Z \,=\, 
                \{
            \{
        \{
           \{
            \{
              \{a\},\{a,b\}
            \}
           \}
             ,
        \{
           \{\{a\},\{a,b\}\} , c
            \}
        \}
            \}
                         ,
            \{
        \{
           \{
            \{
              \{a\},\{a,b\}
            \}
           \}
             ,
        \{
           \{\{a\},\{a,b\}\} , c
            \}
        \},
          \{\{a\},\{a,\{\{b\},\{b,c\}\}\}\}
            \}
                \}
        \end{displaymath}
%% MAKE SURE THIS OBJECT ACTUALLY WORKS SHOULD BE (x,y), WHERE
%%  x=((a,b),c), y=(a,(b,c))
    where $a$, $b$ and $c$ are distinct objects.
    The reader is encouraged to determine whether $Z$ is an ordered pair in the Kuratowski sense.
}%EndFootNoteSize
\end{quotation} 
%##

\V
\V


        It is fairly straight forward to generalize the Kuratowski construction to define `ordered triples', `ordered quadruples', and so on.
    However, the results become increasingly complicated. Instead we handle such extensions in an alternate way in Section~~\Ref{SectA30}.
    At that time it also will be convenient to provide a `new, improved' formulation for ordered pairs.
    The formulation there will ultimately be based on the Kuratowski construction given above,
    but from then on the Kuratowski ordered pair will be treated as the `primitive' version.
    This process of evolving from a `primitive definition' of ordered pair to an `improved definition' should sound familiar:
    a similar evolution process was described in `Warning~(b)' above for `Definitions'.

\V
\V

        One of the advantages of formulating concepts in terms of sets is that one has a precise way of determining when two sets are equal to each other; namely, when they have exactly the same elements (Axiom of Extension).
    In particular, the Kuratowski definition gives actual content to the following statement:


        \subsection{\small{\bf Theorem}}
            \label{ThmA12.20}

        A necessary and sufficient condition for ordered pairs $Z$ and $W$ to be equal is that the first entry of $Z$ equal the first entry of $W$ and the second entry of $Z$ equal the second entry of $W$.

\V

        \underline{Proof}:

        Let $a$ be the first entry of the ordered pair $Z$ and let $b$ be the second entry of $Z$.
    Likewise, let $c$ and $d$ be the first and second entries, respectively, of $W$.
    Then, in accordance with Kuratowski's definition, the statement to be proved is this:
        \begin{displaymath}
        \mbox{A necessary and sufficient condition for } \{\{a\},\{a,b\}\} \,=\, \{\{c\},\{c,d\}\} \mbox{ is that }
    a \,=\, c \mbox{ and } b \,=\, d.
        \end{displaymath}

        \underline{`Sufficient' Half} Suppose that $a \,=\, c$ and $b \,=\, d$.
    Then $\{a\} \,=\, \{c\}$ and $\{a,b\} \,=\, \{c,d\}$, so $\{\{a\}, \{a,b\}\} \,=\, \{\{c\},\{c,d\}\}$.
    (We have simply replaced `equals by equals'.) That is, $Z \,=\, W$, as claimed.

\V

        \underline{`Necessary' Half}: Suppose that $Z \,=\, W$.
    Then, by the Kuratowski definition of `ordered pairs', this can be written
        \begin{displaymath}
        \{\{a\},\{a,b\}\} \,=\, \{\{c\},\{c,d\}\} \h ({\ast})
        \end{displaymath}

        \underline{Case 1}: Suppose that $a \,=\, b$, so that $(a,b)$ is an ordered pair of Type~I (see Definition~\Ref{DefA12.10} above).
    Then $\{\{a\},\{a,b\}\} \,=\, \{\{a\}\}$; in particular, the left side of Equation~$({\ast})$ is a singleton set.
    Thus, from the `Axiom of Extension', it follows that the right side of Equation~$({\ast})$ must also be a singleton.
    That is, $(c,d)$ is also an ordered pair of Type~I, so $c \,=\, d$.
    Thus,
        \begin{displaymath}
        \{\{a\}\} \,=\, \{\{c\}\}.
        \end{displaymath}
    Apply the Axiom of Extension to the singleton sets $\{\{a\}\}$ and $\{\{c\}\}$ to get $\{a\} \,=\, \{c\}$;
    Apply the axiom again to the singleton sets $\{a\}$ and $\{c\}$ to get $a \,=\, c$.
    Since in this case one also has $a \,=\, b$ and $c \,=\, d$, it follows that $a \,=\, c$ and $b \,=\, d$, as claimed.

\V

        \underline{Case 2}: Suppose $a \,\,{\neq}\,\, b$. Then $Z \,=\, (a,b)$ is a Type~II ordered pair,
    hence $W \,=\, (c,d)$ is also of Type~II (since, by hypothesis, $Z \,=\, W$), and thus $c \,\,{\neq}\,\, d$.
    The equality $Z \,=\, W$ also implies that the element of $Z$ which is a singleton set must equal the element of $W$ which is a singleton set. That is, $\{a\} \,=\, \{c\}$, so by the Axiom of Extension again one has $a \,=\, c$.
    Likewise, the element of $Z$ which is a doubleton set must equal the element of $W$ which is a doubleton set,
    so $\{a,b\} \,=\, \{c,d\}$. Since, as was just shown, one has $a \,=\, c$, one can then write $\{a,b\} \,=\, \{a,d\}$;
    just replace $c$ by $a$ in the set $\{c,d\}$. The fact that $\{a,b\} \,=\, \{a,d\}$ means that $b$ must be one of the elements of the doubleton set $\{a,d\}$. Since we are assuming $a \,\,{\neq}\,\, b$, it follows that $b \,=\, d$, as required, and the desired result follows.

\VV

        In analysis, as elsewhere in mathematics, one encounters sequences which are constructed in a step-by-step manner,
    using the early entries to determine the later ones. For example, in the definition of the quantity $k!$,
    where $k$ is a natural number, one first sets $1! \,=\, 1$, and then $(k+1)! \,=\, k!(k+1)$.
    Such definitions are said to be {\bf recursive}.
    Likewise, the construction of the canonical bijection ${\Psi}_{C}$ in Definition~\Ref{DefA30.70} is given recursively.
    The main theoretical fact about recursive definitions is the following result.

\V

        \subsection{\small{{\bf Theorem}} (Dedekind's Theorem on Recursive Definitions)}\IndBD{recursive definitions}{Dedekind's recursive-definition theorem}
        \label{ThmA40.100}

        Let $Y$ be a nonempty set, let $y_{0}$ be a point of $Y$, and let $G:Y \,{\rightarrow}\, Y$ be a function with domain $Y$ and values in $Y$.
    Then the exists a unique function $f:{\NN} \,{\rightarrow}\, Y$ such that $f(1) \,=\, y_{0}$ and $f(k+1) \,=\, G(f(k))$ for all $k$ in ${\NN}$.

\V

        {\bf Proof}\,  The `uniqueness' portion is a simple consequence of the Principle of Mathematical Induction, and its proof is left as an exercise.
        To show that such $f$ exists, we use Theorem~\Ref{ThmA30.27} to reduce the problem to the existence of analogous functions on the subsets ${\NN}_{k}$ of ${\NN}$.

\V

        \underline{Claim} For each $k$ in ${\NN}$ there is a function $f_{k}:{\NN}_{k} \,{\rightarrow}\, Y$
    such that $f(1) \,=\, y_{0}$ and if $j{\in}{\NN}_{k}$ and $j\,<\,k$ then $f_{k}(j+1) \,=\, G(f_{k}(j))$.

        \underline{Proof of Claim} Let $A$ be the set of $k$ in ${\NN}$ such that such $f_{k}$ exists.
    Clearly $1{\in}A$: just define $f_{1}:{\NN}_{1} \,{\rightarrow}\, Y$ by the rule $f_{1}(1) \,=\, y_{0}$.

        Next, suppose that $k{\in}A$. Define $f_{k+1}:{\NN}_{k+1} \,{\rightarrow}\, Y$ by the rule
        \begin{displaymath}
        f_{k+1}(j) \,=\, f_{k}(j) \mbox{ if $1\,\,{\leq}\,\,j\,\,{\leq}\,\,k$};
        f_{k+1}(k+1) \,=\, G(f_{k}(k)).
        \end{displaymath}
    It is clear that $f_{k+1}$ has the desired properties, so that $k+1$ is also in $A$.
    Now the Principle of Mathematical Induction implies that $A \,=\, {\NN}$, so the Claim follows.

    It is also clear from the construction above that if $i$ and $j$ are elements of ${\NN}$ with $i\,<\,j$, then $f_{i}$ is the restriction to ${\NN}_{i}$ of $f_{j}$.
    Hence it follows from Theorem~\Ref{ThmA30.27}, the `Union-of-Functions Theorem',
    that the union of the functions $f_{k}$ for $k$ in ${\NN}$ is a function $f:{\NN} \,{\rightarrow}\, Y$;
    it is also clear that this function has the desired properties.

\V
\V

        \subsection{\small{{\bf Example}}}
        \label{ExampA40.110}

\V
\V

        The statement of Dedekind's Theorem above is phrased along the lines of the original Principle of Mathematical Induction.
    There is an alternate version which follows the phrasing of the Strong Principle of Mathematical Induction~(\Ref{PrinA20.04A}).

\V

        \subsection{\small{{\bf Theorem}} (Strong Form of Dedekind's Theorem on Recursive Definitions)}\IndC{recursive definitions}{Dedekind's recursive-definition theorem}{strong form}
        \label{ThmA40.120}


        Let $Y$ be a nonempty set, let $y_{0}$ be a point of $Y$, and let ${\cal B}$ be the set of all tuples in $Y$.
    Let $g:{\cal B} \,{\rightarrow}\, Y$ be a function with domain ${\cal B}$ and values in $Y$,
    and let $G:{\cal B} \,{\rightarrow}\, {\cal B}$ be the function defined by the following rule:

        If ${\sigma} \,=\, (y_{1},y_{2},\,{\ldots}\,y_{k})$ is an element of ${\cal B}$, then $G({\sigma})$ is the tuple $({\sigma},g({\sigma}))$;
    more precisely,
        \begin{displaymath}
        G((y_{1},y_{2},\,{\ldots}\,y_{k})) \,=\, (y_{1},x_{2},\,{\ldots}\,y_{k}, g((y_{1},y_{2},\,{\ldots}\,y_{k}))).
        \end{displaymath}
    Then the exists a unique function $f:{\NN} \,{\rightarrow}\, Y$ such that $f(1) \,=\, y_{0}$ and $f(k+1) \,=\, g((f(1),f(2),\,{\ldots}\,f(k))$ for all $k$ in ${\NN}$.

\V

        The simple proof is left as an exercise. (Hint: Apply the original form of the Dedekind Theorem to the map $G:{\cal B} \,{\rightarrow}\, {\cal B}$.)

\V
\V



        In {\TheseNotes} we have many occasions to construct sequences of objects.
    Often a proper treatment of these constructions should involve one or the other of the preceding Dedekind Theorems.
    In most of these situations, however, we leave the details of such a proper treatment to the reader,
    and follow a more informal approach in order to simplify the discussion.




%%% 
\begin{quotation}
{\footnotesize \underline{\Note} (on the Schr\"{o}der-Bernstein Theorem)\IndB{\notes}{on the Schr\"{o}der-Bernstein Theorem}: 
    The next result is not needed for the rest of {\TheseNotes} -- which is why it appears in a \Note -- but it is a powerful tool in set theory.

\V

        {\bf The Schr\"{o}der-Bernstein Theorem}\IndD{Schr\"{o}der-Bernstein Theorem}{\notes}

\V

        Suppose that $A$ and $B$ are nonempty sets such that $A$ has a subset $X$ with the same cardinality as $B$,
    and $B$ has a subset $Y$ with the same cardinality as $A$.
    Then $A$ has the same cardinality as $B$.

\V

        {\bf Proof}\,  Let $f:A \,{\rightarrow}\, Y$ and $g:B \,{\rightarrow}\, X$ and be bijections;
    such bijections exist because of the `equal cardinality' hypotheses.
    The key idea is given in the following analysis:

\V

       \underline{The Basic Schr\"{o}der-Bernstein Construction}

        Let $A' \,=\, g[Y]$. Clearly $A'$ is a subset of $X$, and $g$ maps $Y$ one-to-one onto $A'$;
    that is, the restriction of $g$ to $Y$ is a bijection of $Y$ onto $A'$. It follows that the composition $g{\circ}f$ maps $A$ bijectively onto $A'$.
    Likewise, let $B' \,=\, f[X]$. Then by a similar argument one sees that the composition $f{\circ}g$ maps $B$ bijectively onto $B'$

        Now let $C \,=\, A{\setminus}X$ and $D \,=\, X{\setminus}A'$; note that $C$ and $D$ are mutually disjoint.
    Since $A' \,{\subseteq}\, X \,{\subseteq}\, A$, it is also clear that $A \,=\, A'\,{\cup}\,C\,{\cup}\,D$, a disjoint union.
    Likewise, let $E \,=\, B{\setminus}Y$ and $F \,=\, Y{\setminus}B'$. Then it is clear that $B$ is the disjoint union $B \,=\, B'\,{\cup}\,E\,{\cup}\,F$ of the subsets $B'$, $E$ and $F$.

        It is also easy to see that the sets $C \,=\, A{\setminus}X$ and $F \,=\, Y{\setminus}B'$ have the same cardinality.
    More precisely, the function $f$ maps $A$ one-to-one onto $Y$ and $f$ maps $X$ one-to-one onto $B'$,
    so $f$ maps $A{\setminus}X$ one-to-one onto $Y{\setminus}B'$.
    Similarly, the function $g$ maps $E \,=\, B{\setminus}Y$ bijectively onto $D \,=\, X{\setminus}A'$.

        Finally, notice that $f$ maps $A'$ into $B' \,=\, f[X]$ since $A' \,{\subseteq}\, X$; likewise, $g$ maps $B'$ into $A'$.
    In addition, the restrictions of $f$ to $A'$ and $g$ to $B'$ are one-to-one functions, since the original $f$ and $g$ are one-to-one.
    In particular, $A'$ has a subset $X'$ with the same cardinality as $B'$, namely $X' \,=\, f[A']$;
    likewise, $B'$ has a subset $Y'$ with the same cardinality as $A'$, namely $Y' \,=\, g[B']$.
    Thus, the `Basic Construction' can be repeated on the sets $A'$, $B'$, $X'$ and $Y'$ to express $A'$ and $B'$ as disjoint unions of the form $A' \,=\, A''\,{\cup}\,C'\,{\cup}\,D'$ and $B' \,=\, B''\,{\cup}\,E'\,{\cup}\,F'$.


\V

        The rest of the proof consists primarily of carrying out the `Basic Construction' above infinitely many times.
    To do this, however, it is convenient to introduce numerical subscripts instead of `primes'.
    Thus, let us write $A_{0}$, $X_{0}$, and $A_{1}$ instead of $A$, $X$, $A'$; and write $C_{0} \,=\, A_{0}{\setminus}X_{0}$ and $D_{0} \,=\, X_{0}{\setminus}A_{1}$ 
    instead of $C \,=\, A{\setminus}X$ and $D \,=\, X{\setminus}A'$.
    Likewise, write $B_{0}$, $Y_{0}$ and $B_{1}$ instead of $B$, $Y$ and $B'$; and write $E_{0}$ and $F_{0}$ instead of $E$ and $F$.
    And when one repeats the `Basic Construction', simply increase the indices by~$1$ instead of slapping on another `prime'.

        By doing this, for each $k \,=\, 0,1,2,\,{\ldots}\,$ one obtains sets $A_{k}$, $X_{k}$, $C_{k}$, $D_{k}$, and $B_{k}$, $Y_{k}$, $E_{k}$ and $F_{k}$, which satisfy the following relations:

\V

        \h (a) $A_{0}$, $B_{0}$, $X_{0}$ and $Y_{0}$ equal the original sets $A$, $B$, $X$ and $Y$.

        \h (b) $Y_{k} \,=\, f[A_{k}]$ and $X_{k} \,=\, g[B_{k}]$.

        \h (c) $C_{k} \,=\, A_{k}{\setminus}X_{k}$ and $D_{k} \,=\, X_{k}{\setminus}A_{k+1}$;
    likewise, $E_{k} \,=\, B_{k}{\setminus}Y_{k}$ and $F_{k} \,=\, Y_{k}{\setminus}B_{k+1}$.

        \h (d) $A_{k}$ is the disjoint union of $A_{k+1}$, $C_{k}$ and $D_{k}$;
    likewise, $B_{k}$ is the disjoint union of $B_{k+1}$, $E_{k}$ and $F_{k}$.

        \h (e) $C_{k}$ and $F_{k}$ have the same cardinality; likewise, $D_{k}$ and $E_{k}$ have the same cardinality.

\V

\noindent From the preceding one then can write
        \begin{displaymath}
        A \,=\, A_{0} \,=\, A_{1}\,{\cup}\,C_{0}\,{\cup}\,D_{0} \,=\, A_{2}\,{\cup}\,C_{1}\,{\cup}\,D_{1}\,{\cup}\,C_{0}\,{\cup}\,D_{0} \,=\, 
A_{3}\,{\cup}\,C_{2}\,{\cup}\,D_{2}\,{\cup}\,C_{1}\,{\cup}\,D_{1}\,{\cup}\,C_{0}\,{\cup}\,D_{0} \,=\, \,{\ldots}\,.
        \end{displaymath}
    Likewise,
        \begin{displaymath}
        B \,=\, B_{0} \,=\, B_{1}\,{\cup}\,E_{0}\,{\cup}\,F_{0} \,=\, B_{2}\,{\cup}\,E_{1}\,{\cup}\,F_{1}\,{\cup}\,E_{0}\,{\cup}\,F_{0} \,=\,
B_{3}\,{\cup}\,E_{2}\,{\cup}\,F_{2}\,{\cup}\,E_{1}\,{\cup}\,F_{1}\,{\cup}\,E_{0}\,{\cup}\,F_{0} \,=\, \,{\ldots}\,.
        \end{displaymath}
    All the unions shown here are {\em disjoint} unions.

        Since $A_{k+1} \,{\subseteq}\, A_{k}$ for each $k$, one can conclude that
        \begin{displaymath}
        A \,=\, P\,{\cup}\, C_{0}\,{\cup}\,D_{0}\,{\cup}\,C_{1}\,{\cup}\,D_{1}\,{\cup}\,\,{\ldots}\,\,{\cup}\,C_{k}\,{\cup}\,D_{k}\,{\cup}\,\,{\ldots}\, \h ({\ast})
        \end{displaymath}
    where $P \,=\, A_{0}\,{\cap}\,A_{1}\,{\cap}\,\, {\ldots}\,\,{\cap}\,A_{k}\,{\cap}\,\,{\ldots}\,$.
    Likewise,
        \begin{displaymath}
        B \,=\, Q\,{\cup}\, E_{0}\,{\cup}\,F_{0}\,{\cup}\,E_{1}\,{\cup}\,F_{1}\,{\cup}\,\,{\ldots}\,\,{\cup}\,E_{k}\,{\cup}\,F_{k}\,{\cup}\,\,{\ldots}\, \h ({\ast}{\ast})
        \end{displaymath}
    where $Q \,=\, B_{0}\,{\cap}\,B_{1}\,{\cap}\,\, {\ldots}\,\,{\cap}\,B_{k}\,{\cap}\,\,{\ldots}\,$.

        \underline{Claim} The sets $P$ and $Q$ have the same cardinality.

        \underline{Proof of Claim} First note that $A_{k+1} \,{\subseteq}\, X_{k} \,{\subseteq}\, A_{k}$, so $P \,=\,X_{1}\,{\cap}\,X_{2}\,{\cap}\,\,{\ldots}\,\,{\cap}\,X_{k}\,{\cap}\,\,{\ldots}\,$.
    Likewise, $B_{k+1} \,{\subseteq}\, Y_{k} \,{\subseteq}\, B_{k}$, so $Q \,=\, Y_{1}\,{\cap}\,Y_{2}\,{\cap}\,\,{\ldots}\,\,{\cap}\,Y_{k}\,{\cap}\,\,{\ldots}\,$.

        Suppose first that $P$ is nonempty, and let $x$ be an element of $P$. Then for each $k$, $x{\in}A_{k}$, hence $f(x){\in}Y_{k}$ (since $f$ maps $A_{k}$ to $Y_{k}$).
    Thus, $f(x)$ is in the intersection of the sets $Y_{1}$, $Y_{2}$,\,{\ldots}\,, hence $f(x){\in}Q$. That is, $f$ maps $P$ {\em into} $Q$.
    To see that $f$ maps $P$ {\em onto} $Q$, let $y$ be any element of $Q$. Then for each $k$, one has $y{\in}Y_{k}$; hence, since $Y_{k} \,=\, f[A_{k}]$,
    one has $f(x_{k})$ for some element $x_{k}$ in $A_{k}$. And since $f$ is a bijection on $X$, one has $x_{1} \,=\, x_{2} \,=\, \,{\ldots}\,$.
    Call this common value $x$, so $x{\in}A_{k}$ for each $k$ and thus $x{\in}P$.
    That is $f$ maps $P$ {\em onto} $Q$.  It follows that $f$ maps $P$ bijectively onto $Q$, hence $P$ and $Q$ have the same cardinality, as claimed.

        It is easy to verify by a similar argument that if $P \,=\, {\emptyset}$ then $Q \,=\, {\emptyset}$;
    thus $P$ and $Q$ have the same cardinality in this case as well.

        Now consider the right sides of Equations~$({\ast})$ and~$({\ast}{\ast})$ above,
    which express $A \,=\, A_{0}$ and $B \,=\, B_{0}$ as disjoint unions of sets.
    We have already proved that for each $k$ the sets $C_{k}$ and $F_{k}$ have the same cardinality,  as do the sets $D_{k}$ and $E_{k}$.
    And since $P$ and $Q$ also have the same cardinality, Theorem~\Ref{ThmA50.88A} now implies that $A$ and $B$ have the same cardinality, as claimed.

\V

The reader is encouraged to review the examples considered in Sections~\Ref{SectA15}, \Ref{SectA20} and~\Ref{SectA25} and see whether some of them could have been obtained more easily using the Schr\"{o}der-Bernstein theorem.

%\StartSkip{

\V

        {\bf Remark} The reader is encouraged to review the examples considered in Sections~\Ref{SectA15}, \Ref{SectA20} and~\Ref{SectA25} and see how the results obtained there could have been proved using the Schr\"{o}der-Bernstein theorem.
%}%\EndSkip
}%EndFootnotesize
\end{quotation} 
%##



        {\bf Problem} Precisely define the concept of `relation'.%\\\

\V

        As a starting point, consider the definition of `relation' given in one dictionary:

        \h `A relation is a logical or natural association between two or more things.'
    (Of course in the special case of a {\em binary} relation, the phrase `or more' would be omitted.)


\noindent That is, a relation is defined to be a type of association. However, the same dictionary goes on to define `association' as follows:


        \h `An association is a mental connection or relation between thoughts, feelings, ideas, or sensations.'

\noindent One can combine these to say, in effect, that

        \h `A relations is an association, and an association is a relation'.


\noindent This is a `circular definition'. It appears that the concept of `relation' must fall under Justice Stewart's dictum:

        \h `I can't define it, but I know it when I see it'.

\V

        Nevertheless, mathematicians have formulated a precise definition, based purely on set-theoretic concepts; this definition is given below.
    In principle it could be given here, with no further introduction, since all the words which appear in it already make sense.
    However,as was observed above in the discussion of the `Definition-Theorem-Proof' style of mathematical exposition,
    in reality formal definitions arise only {\em after} one understands what the relevant features of the situation being described.
    One normally gets that understanding through a supply of examples which indicate what features any reasonable formal definition ought to have.
    Here is a short list of such examples of relations, both from within mathematics and from ordinary life:

\V

       \h (1) `John is a son of Jane.' (Relation: `is a son of')

\VA

       \h (2) `Jane is the mother of Jill.' (Relation: is the mother of')

\VA

       \h (3) `John is a son of Jane and George.' (Relation: is a son of')

\VA

       \h (4) `The number $x$ is greater than the number $y$. (Relation: `is greater than')

\VA

       \h (5) `The number $y$ is greater than the number $x$'. (Relation: `is greater than')

\VA

       \h (6) `The number $x$ is the square of the number $y$.' (Relation: `is the square of')



\VA

        These examples indicate some features that any reasonable definition of the concept `relation' ought to have,
    where by `reasonable' is meant a definition which corresponds well to one's intuitive notion of the concept.

        \h (a) A `relation' should relate two or more objects; for instance, Example~(3) relates three objects (John, Jane and George),
    while the other examples relate pairs of objects. In {\TheseNotes} the focus is on relations of the latter type, called {\em binary} relations.
    (This is partially because such relations are the most common in mathematics,
    and partially because more general relations can often be reduced to the binary case.)

        \h (b) The order in which the objects are listed may make a difference. For instance, in Example~(2)
    the statements `Jane is the mother of Jill' and `Jill is the mother of Jane' both make grammatical sense, but they mean very different things.
    Likewise, the only difference between Statements~(4) and~(5) is the order in which the numbers $x$ and~$y$ are listed.
    If  $x \,\,{\neq}\,\, y$ then the order makes a difference in the truth of the stated relationship.

        \h (c) The sets from which the objects being related are drawn must be specified, and these sets need not be the same.
    For instance, in Example~(1) the context suggests that, in the relation `$x$ is a son of $y$',
    the object $x$ (`John' here) should be chosen from, say, the set of all human males;
    in contrast, it is not so clear whether the $y$ should be drawn fron the set of, say, all humans, or all human females, or all human parents.
    Likewise, in Example~(6) it makes a difference whether the numbers $x$ and $y$ are to be chosen from the natural numbers or from the real numbers.

        \h (d) Of course the real question for any such binary relation is this: which pairs are in the given relationship?
    (Once that is answered, the issue of which pairs are {\em not} in the relationship is automatically settled.)

        \h (e) The definition must allow for the possibility that there are no pairs in the given relationship.
    For instance, the relation described by the inequality $x^{2}\,<\,-y^{2}$' is satisfied by no pair of real numbers.

        Having said all that, here is the standard definition of `binary relation' in mathematics:

\V

        \subsection{\small{{\bf Definition}}}
        \label{DefA12.60}\IndC{relations}{binary}{from a given set to a given set}\IndD{binary relations}{relations}

\V

        (1) Let $X$ and $Y$ be a pair of nonempty sets.
    A {\bf binary relation from $\Bfm{X}$ to $\Bfm{Y}$} is a (possibly empty) subset of the Cartesian product $X{\times}Y$.
    If $R$ is such a relation, i.e., if $R \,{\subseteq}\, X{\times}Y$, and if $(x,y)$ is an element of $R$, one says that {\bf \Bfm{x} and \Bfm{y} are in the  relation~$R$}.
    One frequently indicates this state of affairs by writing $x\,R\,y$ instead of the more usual $(x,y){\in}R$.

        If $Y \,=\, X$, then one normally speaks of a {\bf relation \underline{on} $\Bfm{X}$}\IndC{relations}{binary}{on a given set} instead of a relation \underline{from} $X$ to~$X$.

        (2) If $R$ is a binary relation from $X$ to $Y$, as above, then the corresponding
    {\bf negated binary relation}\IndC{relations}{binary}{negated relation} from $X$ to~$Y$, often denoted ${\not}R$,
    is the complement $X{\times}Y\,{\setminus}\,R$; that is, it is the set of all ordered pairs $(x,y)$, with $x$ in $X$ and $y$ in $Y$, such that $(x,y)$ is {\em not} in the set~$R$.

        (3) A {\bf formal binary relation}\IndC{relations}{binary}{formal} is an ordered pair $(U,R)$,
    where $U$ is an ordered pair $(X,Y)$ of nonempty sets and $R$ is a subset of $X{\times}Y$.

\VV

        \subsection{\small{{\bf Some Nonmathematical Examples of Binary Relations}}}
        \label{ExampA12.65A}

\V


        (1) Let $W$ be the set consisting of the following women; the numbers following their names are their ages as of January~1, 2000:

        \begin{center}
        Abigail (85), Betty (65), Carrie (22), Debbie (45), Edie (34) and Francine (42)
        \end{center}

        Here are the familial relations between these women:


        Betty is the daughter of Abigail; Carrie is the daughter of Francine; Debbie is the daughter of Betty.

        \underline{Problem} Let $X \,=\, Y \,=\, W$, and consider the binary relation on $W$, `is the daughter of'.
    Describe this relation in terms of the preceding definition; that is, as a set of ordered pairs.

        \underline{Solution} To save a little writing, abbreviate the names of the six women to their first initials, so that $W \,=\, \{A,B,C,D,E,F\}$.
    Then the relation is the following set $R$ of ordered pairs:
        \begin{displaymath}
        R \,=\, \mbox{\{(B,A), (C,F), (D,B)\}.}
        \end{displaymath}

\V

        (2) Let $W$ be the same set as in the preceding example.

        \underline{Problem} As before, let $X \,=\, Y \,=\, W$, but now consider the binary relation,
    again on $W$, described by `is exactly twenty years younger than' (on January~1, 2000, of course).
    Describe this relation in terms of a set of ordered pairs.

        \underline{Solution} It is easy to see that the answer is exactly the same set $R$ as in the preceding example.

\V

        (3) Let $\hat{W} \,=\, W\,{\setminus}\,\{E\} \,=\, \{A, B, C, D, F\}$, and once again consider the binary relation described by `is the daughter of', but now on the set~$\hat{W}$.
    It is clear that the corresponding set of ordered pairs is the same as in the preceding examples.
    Nevertheless, in light of Part~(3) of Definition~\Ref{DefA12.60} they are considered to be different.
    Indeed, as `formal' binary relations the first is the ordered pair $((W,W),R)$ while the second is $(\hat{W},\hat{W},R)$.
    Since $W \,\,{\neq}\,\, \hat{W}$, these two ordered pairs are physically different objects.

\V

        \underline{Remarks} (a) The relations discussed in Examples~(1) amd~(2) above are, intuitively speaking, quite dissimilar:
    one concerns the specific relation of daughters to mothers, while the other concerns age differences.
    Nevertheless, from the viewpoint of Definition~\Ref{DefA12.60} they are literally the same object.

        In contrast, the binary relation described in Example~(3) appears to be the same as that in Example~(1), even when speaking intuitively.
    Nevertheless, in terms of Definition~\Ref{DefA12.60} they must be considered to be different, because the underlying sets $W$ and $\hat{W}$ are not equal.
    If this seems strange, consider the alternative: if these relations were literally the same, then the corresponding negated relations would be equal.
    However, the set of ordered pairs associated with the {\em negated} relation for Example~(1) includes,
    among several others, the pair $(E,A)$: Edie is not the daughter of Abigail.
    In contrast, the pair $(E,A)$ cannot be part of the negated relation for Example~(3), since $E$ is not in the set~$W'$.


\VV

        \subsection{\small{{\bf Some Mathematical Examples of Binary Relations}}}
        \label{ExampA12.65B}

\V

        (1) Some of the standard well-known examples of mathematical binary relations have been alluded to before:
    `greater than' `less than', `equal to' and so on. To formulate these precisely in the terms of Definition~\Ref{DefA12.60},
    one needs to specify the sets which pay the role of $X$ and $Y$ in that definition.
    For instance, the relation `is greater than', denoted `$\,>\,$', certainly applies when $X \,=\, Y \,=\, {\RR}$, the set of all real numbers.
    In this case, the corresponding set of ordered pairs is $R \,=\, \{(x,y): x{\in}{\RR}, y{\in}{\RR}, x\,>\,y\}$.

        The corresponding negated relation, denoted $\not>$ and pronounced `not greater than',
    consists of those elements $(x,y)$ in ${\RR}{\times}{\RR}$ for which $x\not>y$.
    This is equivalent to the relation `less than, or equal to', denoted 	`$\,\,{\leq}\,\,$' on the same set.

\V

        (2) Let $X \,=\, \{1,2,3,4,5\}$ and let $R$ be the binary relation on $X$ decribed as the set of all ordered pairs $(x,y)$ in $X{\times}X$
    such that $x \,\,{\neq}\,\, y$ and $x-y$ is a whole-number multiple of~$3$. One easily verifies that $R \,=\, \{(4,1), (5,2)\}$.

\V

        (3) Let $X$ be as in the preceding example but now let $R'$ be the binary relation on $X$
    decribed as the set of all ordered pairs $(x,y)$ in $X{\times}X$ such that $x-y$ is a whole-number multiple of~$6$;
    that is, $x-y \,=\, 6\,k$ for some $k$ in ${\NN}$. Clearly, $R' \,=\, {\emptyset}$.

\V

        (4) Let $\hat{X} \,=\, \{4, 5, 6, 7, 8\}$ and let $R''$ be described as the set of all ordered pairs $(x,y)$ in $\hat{X}{\times}\hat{X}$ such that $x-y$ is a whole-number multiple of~$10$
    Clearly $R'' \,=\,{\emptyset}$, so $R''$ equals the corresponding set $R'$ from the preceding example.
    Nevertheless, the binary relations $R'$ and $R''$ are treated as being different:
    the sets $X$ and $\hat{X}$ on which these relations are defined are not the same.



        \underline{Remarks} (1) The phrasing `relation {\em from} $X$ {\em to} $Y$' emphasizes the requirement, already mentioned above,
    that the definition should allow the possibility that the order ($X$ is first, $Y$ is second) in a relation might make a difference.
    Indeed, the phrases `from $X$ to $Y$' and `from $Y$ to $X$' certainly do not signify the same ideas.
    However, some authors use instead the phrasing `relation of $X$ {\em with} $Y$'.
    Linguistically speaking, this means the same as `relation of $Y$ with $X$', which hides the order dependence.
    Such authors rely on the convention that by writing $X{\times}Y$, as opposed to $Y{\times}X$,
    one tacitly treats $X$ as the first set, $Y$ as the second. (See the {\Note} below on the `left-to-right bias' in mathematics.)
    This usage is mildly sloppy, but seems to cause no confusion.

        (2) Some authors abbreviate the content of Definition~\Ref{DefA12.60} to the following:

\VA

        \h `A binary relation is a set $R$ of ordered pairs of objects'

\VA

\noindent In this formulation there is no reference to any given sets from which the entries of the ordered pairs in $R$ are to be drawn.
    Stated this way, the binary relation {\em is} the set $R$ itself. With that viewpoint, it would follow that if $R \,{\subseteq}\, X{\times}Y$,
    and $X'$ and $Y'$ are proper supersets of $X$ and $Y$, respectively, then the same object, $R$,
    would be a binary relation from $X$ to $Y$ and, simultaneously, a binary relation from $X'$ to $Y'$.
    This is {\em not} the intent of Definition~\Ref{DefA12.60}, as is made clear in the examples.
    

        (3) The restriction that $X$ and $Y$ be nonempty is to avoid situations which are of no interest whatsoever.
    In contrast, the definition {\em does} allow the possibility that the subset $R$ might be empty, since that fact might reflect something of interest;
    namely, that no pairs $(x,y)$ in $X{\times}Y$ happen to be in the given relation.

        (4) The preceding definition uses only subsets of the Cartesian product $X{\times}Y$, and the latter concept is defined purely in set-theoretic terms.
    Thus, Definition~\Ref{DefA12.60} does formulate the concept of `relation' purely in terms of set theory.
    However, by identifying a relation from $X$ to $Y$ with a subset of $X{\times}Y$,
    the definition does allow the possibility of two relations, which intuitively to be seem very different, being treated as the same; see below.


\V
\V

        One of the most important types of binary relations on a set is a so-called `equivalence relation'.

\V


        \subsection{\small{{\bf Definition}}}
        \label{DefA50.90}\IndC{relations}{binary}{equivalence relations}\IndD{equivalence relations}{relations}

        Let $X$ be a nonempty set.
    An {\bf equivalence relation on $X$} is a subset $W$ of $X{\times}X$ with the following properties:

        \h (i) (Reflexivity) If $x{\in}X$ then $(x,x){\in}W$.

\V

        \h (ii) (Symmetry) If $(x,y){\in}W$ then $(y,x){\in}W$.

\V

        \h (iii)(Transitivity) If $(x,y){\in}W$ and $(y,z){\in}W$ then $(x,z){\in}W$.

\V

        One often indicates that $(x,y){\in}W$ by the notation $x \stackrel{W}{\,\sim\,}y$;
    the expression  $x\stackrel{W}{\,\sim\,y}$ is pronounced `$x$ is equivalent to $y$ with respect to~$W$'.
    If the choice of equivalence relation $W$ is understood from the context, then this notation may be simplified to $x\,{\sim}\,y$,
    which is pronounced `$x$ is equivalent to $y$'; in this situation we may also refer to `the equivalence relation ${\sim}$'.

        If $x$ is any element of $X$ then $[x]_{W}$ denotes the set of all $y$ in $X$ such that $x \stackrel{W}{\,\sim\,}y$.
    The set $[x]_{W}$ is called the {\bf equivalence class of $\Bfm{x}$ relative to~$W$};
    as usual, the explicit reference to the equivalence relation $W$ may be dropped if it is clear from the context.


        \subsection{\small{{\bf Examples}}}
        \label{ExampA50.100}

\hspace*{\parindent}
        (1) The relation of `equality' is an equivalence relation on any nonempty set.
    For this equivalence relation, the equivalence class $[x]$ of $x$ is simply the singleton set~$\{x\}$.

\V

        (2) Suppose that ${\cal F}$ is a partition of a nonempty set $X$.
    Then ${\cal F}$ determines an equivalence relation $W_{{\cal F}}$ on $X$ by the rule that $x \stackrel{W_{{\cal F}}}{\,{\sim}\,} y$ if, and only if,
    $x$ and $y$ lie in the same element of the family ${\cal F}$. Indeed, suppose that $x{\in}X$.
    Then the fact that ${\bigcup} {\cal F} \,=\, X$ implies that there is an element $S$ in the family ${\cal F}$ such that $x{\in}S$.
    Clearly $x$ and $x$ are in $S$. Likewise, if $x$ and $y$ are both in the same element $S$ of ${\cal F}$ then $y$ and $x$ are both in $S$.
    Finally, suppose that $x$ and $y$ are both in the same element $S$ of ${\cal F}$ and that $y$ and $z$ are both in the same element $T$ of ${\cal F}$.
    Then $S$ and $T$ have an element in common, namely $y$, so $S\,{\cap}\,T \,\,{\neq}\,\, {\emptyset}$.
    By the fact that ${\cal F}$ is a partition it follows that $S \,=\, T$, and thus $x$ and $z$ are both in $T$.

        It is clear that if $x{\in}X$ then the equivalence class $[x]$ is precisely the unique set $A$ in the partition ${\cal F}$ such that $x{\in}A$.

\V

        (3) Suppose that $f:X \,{\rightarrow}\, Y$ is a surjection of a set $X$ onto a set~$Y$.
    Then $f$ determines an equivalence relation $W$ on $X$ by the rule $x \stackrel{W}{\,\sim\,}y$ if, and only if, $f(x) \,=\, f(y)$.
    We refer to this as the {\bf equivalence relation on $\Bfm{X}$ determined by~$\Bfm{f}$}, and we denote it by $W_{f}$.
    We also sometimes write $x\,{\sim}_{f}\,y$ instead of the more proper $x \stackrel{W_{f}}{\,\sim\,}y$.

\VV

        The next result summarizes the relations between these concepts.

        \subsection{\small{{\bf Theorem}}}
        \label{ThmA50.110}

        Let $X$ be a nonempty set.

\V

        (1) A subset $W$ of $X{\times}X$ is an equivalence relation on $X$ if, and only if,
    there exists a partition ${\cal F}$ on $X$ such that $W \,=\, W_{{\cal F}}$, where $W_{{\cal F}}$ is described above.
    The partition ${\cal F}$ is unique; that is, if ${\cal F}$ and ${\cal G}$ are partitions of $X$ such that $W_{{\cal F}} \,=\, W_{{\cal G}}$, then ${\cal F} \,=\, {\cal G}$.

\V

        (2) A subset $W$ of $X{\times}X$ is an equivalence relation on $X$ if, and only if,
    there exists a surjection $f:X \,{\rightarrow}\, Y$ of $X$ onto some set $Y$ such that such that $W \,=\, W_{f}$.
    (The set $Y$ is {\em not} at all unique, and therefore neither is the map~$f$.)

\V

        \underline{Proof}: (1) The `if' portion is essentially the content of Example~(2) above.
    The rest of the proof is left as a straight-forward exercise.

\V

        (2) This follows from Part~(1) together with the results of Theorem~\Ref{ThmA50.80}.

 
%\StartSkip{
      The next result says, in effect, that an equivalence relation on a set $X$ determines corresponding equivalence relations on the nonempty subsets of $X$.
    Likewise, a partition on $X$ determines corresponding partitions on the nonempty subsets of $X$.

\V

        \subsection{\small{{\bf Theorem}}}
                \label{ThmA50.115}

        Let $X$ be a nonempty set, and let $Y$ be a nonempty subset of $X$.

\V

        (a) Suppose that $W$ is an equivalence on $X$, and let $W|_{Y}$ denote the set $W\,{\cap}\,(Y{\times}Y)$.
    Then $W|_{Y}$ is an equivalence relation on $Y$.

\V

        (b) Let ${\cal F}$ be a partition of $X$, and let ${\cal F}_{Y}$ denote the family of all nonempty sets of the form $S\,{\cap}\,Y$, where $S$ is in the family ${\cal F}$.
    Then ${\cal F}_{Y}$ is a partition of $Y$.

\V

        (c) Suppose that $W$ is an equivalence relation on $X$, and let ${\cal F}$ be the partition of $X$ such that $W \,=\, W_{{\cal F}}$
    (see Part~(a) of Theorem~\Ref{ThmA50.110}).
    Then $W|_{Y} \,=\, W_{{\cal F}_{Y}}$.

\V

        {\bf Proof}\, 

\V

        (a) Suppose that $y{\in}Y$.
    Then $y{\in}X$ (since $Y$ is a subset of $X$), hence $(y,y){\in}W$ (since $W$ satisfies the Reflexivity condition).
    But $(y,y){\in}Y{\times}Y$ (by definition of Cartesian Product), so $(y,y){\in}W\,{\cap}\,Y{\times}Y$ (by definition of intersection).
    In particular, $W_{Y}$ satisfies the Reflexivity condition on $Y$.
    Similar arguments can be used to show that $W_{Y}$ also satisfies the Symmetry and Transitivity conditions on $Y$.

\V

        (b) First note that the family ${\cal F}_{Y}$ is nonempty.
    Indeed, let $y$ be an element of $Y$; such $y$ exists because $Y$ is assumed to be nonempty.
    Then (by definition of `partition') there exists a unique set $S$ in the family ${\cal F}$ such that $y{\in}S$.
    Thus $y$ is in both $S$ and $Y$, so $y{\in}(S\,{\cap}\,Y)$, hence the intersection $S\,{\cap}\,Y$ is nonempty.
    Thus $S\,{\cap}\,Y$ is an element of ${\cal F}_{Y}$, which implies that the family ${\cal F}_{Y}$ is nonempty.

        Next, consider a pair of sets $U$ and $V$ in the family ${\cal F}_{Y}$, with $U \,\,{\neq}\,\, V$.
    Then, by definition of this family, there exists elements $S$ and $T$ of ${\cal F}$ such that $U \,=\, S\,{\cap}\,Y$ and $V \,=\, T\,{\cap}\,Y$.
    Since $U\,\,{\neq}\,\, V$, it follows that $S \,\,{\neq}\,\, T$.
    Thus $S\,{\cap}\,T \,=\, {\emptyset}$, by the `Disjointness Property' of partitions.
    It follows that
        \begin{displaymath}
        U\,{\cap}\,V \,=\, (S\,{\cap}\,Y)\,{\cap}\,(T\,{\cap}\,Y) \,=\, (S\,{\cap}\,T)\,{\cap}\,(Y\,{\cap}\,Y) \,=\, {\emptyset}\,{\cap}\,Y \,=\, {\emptyset},
        \end{displaymath}
    by basic properties of `intersection'.
    That is, the family ${\cal F}_{Y}$ satisfies the `Disjointness Property' as well.

        Finally, it follows from the fact that $X \,=\, {\bigcup}_{S{\in}{\cal F}} S$, the definition of ${\cal F}_{Y}$,
    and basic properties of `union' and `intersection', that
        \begin{displaymath}
        Y \,=\, X\,{\cap}\,Y \,=\, \left({\bigcup}_{S{\in}{\cal F}} S\right)\,{\cap}\,Y \,=\, {\bigcup}_{S{\in}{\cal F}} (S\,{\cap}\,Y) \,=\, 
    {\bigcup}_{U{\in}{\cal F}_{Y}} U.
        \end{displaymath}
    (In the final equation any sets of the form $S\,{\cap}\,Y$ for which this intersection is empty can be ignored, since they do not contribute to the union.)
    Thus, ${\cal F}_{Y}$ satisfies the `Union Property' for partitions.

\V

        (c) The simple proof is left to the reader as an exercise.


\V
\V

        \subsection{\small{{\bf Definition}}}
                \label{DefA50.117}

        Let $X$ be a nonempty set and let $Y$ be a nonempty subset of $X$.

\V

        (a) If $W$ is an equivalence relation on $X$, then the {\bf equivalence relation on $Y$ induced from $W$} is the equivalence relation $W|_{Y} \,=\, Y\,{\cap}\,(Y{\times}Y)$ discussed in the preceding theorem.

\V

        (b)  If ${\cal F}$ is a partition of $X$, then the {\bf partition of $Y$ induced from ${\cal F}$} is the partition ${\cal F}_{Y}$ described in the preceding theorem.
    
%}%\EndSkip

 
        \subsection{\small{{\bf Definition}}}
                \label{DefA50.120}

        Let $W$ be an equivalence relation on a nonempty set $X$.

\V

        (1) For each $x$ in $X$ the set $W[x] \,=\, \{y{\in}X: (x,y){\in}W\})$ is called the {\bf equivalence class of $x$ with respect to $W$}.
    (If the equivalence relation $W$ under consideration is clear from the context,  we may abbreviate the notation $W[x]$ to $[x]$.)
    If $S$ is an equivalence class for the equivalence relation $W$ then each element of $S$ is called a {\bf representative of $S$} (relative to the given equivalence relation).
    (The element $x$ `represents' the equivalence class $S$ in the sense that one can write $S \,=\, [x]$;
    but of course the choice of $x$ used to represent $S$ this way is not unique unless $S$ happens to be a singleton set.)

\V

        (2) The set whose elements are the equivalence sets of $W$ is called the {\bf quotient set of $X$ with respect to $W$}; this set is denoted $X/W$.

\V

        Note: It is clear that the quotient set $X/W$ is the same as the partition ${\cal F}$ described in Part~(a) of Theorem~\Ref{ThmA50.110}.
    It is also clear that $X/W \,=\, X/(f)$, where $f$ is the map described in Part~(b) of the same theorem.


\V
\V

        The next result ties together much of what we have just seen.


\V
\V

        \subsection{\small{{\bf Theorem}}}
                \label{ThmA50.125}

        Suppose that $f:X \,{\rightarrow}\, Y$ is a surjection of a nonempty set $X$ onto a set $Y$,
    and let $Z \,=\, X/(f)$ be the corresponding quotient set.
    Define a function $\hat{f}:Z \,{\rightarrow}\, Y$ as follows: if $S$ is an element of $Z$, so $S \,=\, f^{-1}[y]$ for some element $y$ in $Y$, set $\hat{f}(S) \,=\, y$.
    Then the function $\hat{f}$ is a bijection from $Z$ onto $Y$.

\V

        The simple proof is left as an exercise.

\VV

        \subsection{\small{{\bf Definition}}}
                \label{DefA50.127}

        The bijection $\hat{f}$ described in the preceding theorem is called the {\bf canonical bijection from $W$ onto $Y$ determined by the bijection $f:X \,{\rightarrow}\, Y$}.

\VV

                        \section{Construction of ${\QQ}$ from ${\NN}$}
                        \label{SectAAdd3}%\\\
\VV

        {\bf Introduction} In Chapter~\Ref{ChaptA} we accepted as `primitive truths' basic properties of the natural numbers and the rational numbers;
    but there was a promise to delve more deeply into the structure of such systems.
    Indeed, in Appendix~A the internal structure of the natural numbers is reduced to a consideration of the Dedekind-Peano axioms.
    Of course, the actual nature of these numbers, that is, the answer to the question
    `Exactly what is a natural number in some philosophical sense', is left open:
    We continue to treat these numbers as `primitive objects' which we do not define, but `we know them when we see them'. In particular,
    we do not prove that there exists a system of objects which satifies the Dedekind-Peano axioms; we accept our intuitive notions of ${\NN}$ as a given.

        Having done this, however, it is now easy to derive from this primitive notion of `Natural Number',
    in a rigorous manner which uses only these accepted facts about ${\NN}$ and accepted set-theoretic concepts,
    the rigorous construction of systems which correspond in every important way to our intuitive ideas of `integer' and `rational number'.
    The process lets the intuitive notion lead lead us, in a fairly natural way, to a rigorous construction of~${\QQ}$.
    The goal of the present section is such a contruction.

\V

    \underline{Note} The analogous questions for the real number system are postponed to Chapter~\Ref{ChaptB}.

\VV

%-----------------


        \subsection{\small{{\bf Construction 1: The Positive Rationals from the Natural Numbers}}}
                \label{ConstA50.130A}

 \V

        Let $X$ be the set ${\NN}{\times}{\NN}$ of all ordered pairs of natural numbers.
    The set $X$ is `rigorously defined', in the sense that we are taking both the set ${\NN}$ and the concept of `ordered pairs' as primitive concepts..
    Let $Y$ be the set ${\QQ}^{+}$ of all positive rational numbers', viewed as an intuitively understood system which eventually must be defined rigorously. 
    In terms of this intuition, there is a well-known surjective map ${\rho}:X \,{\rightarrow}\, {\QQ}^{+}$ defined by
    ${\rho}(j,k) \,=\, j/k$ for each ordered pair $(j,k)$ in~$X$; the Greek lettre `${\rho}$' stands for `ratio'.
    It is easy to see that the equivalence relation determined on $X$ by the surjection ${\rho}$ is this:
        \begin{displaymath}
        (j_{1},k_{1}) \,{\sim}\, (j_{2}, k_{2}) \mbox{ if, and only if, }
        j_{1}{\cdot}k_{2} \,=\, j_{2}{\cdot}k_{1} \h ({\ast})
        \end{displaymath}
    This equivalence relation, even though it arises in this discussion from an intitive understanding of `rational number', which includes the map~${\rho}$,
    is formulated in~$({\ast})$ purely in terms of the primitive notions of ordered pairs and the algebraic system ${\NN}$ as described by the Dedekind-Peano axioms.
    In particular, there is no mention in Condition~$({\ast})$ of either the `intuitive' concept of the set ${\QQ}^{+}$ or the `intuitive' map~${\rho}$.
    For that reason we use the symbol ${\sim}$ instead of the more proper ${\sim}_{{\rho}}$ in Condition~$({\ast})$.

        Return now to the intuitive discussion of~${\QQ}^{+}$. Let $A \,=\, X/({\rho})$ be the set of equivalence classes associated with the equivalence relation described above,
    and let $\hat{{\rho}}:A \,{\rightarrow}\, {\QQ}^{+}$ be the corresponding canonical bijection; see Definition~\Ref{DefA50.127} above.
   Note also that the inverse map $\hat{{\rho}}^{-1}:{\QQ}^{+} \,{\rightarrow}\, A$ is easy to compute:
    if $r{\in}{\QQ}^{+}$, express $r$ in the form $j/k$ as above; then $\hat{{\rho}}^{-1}(r)$ is the equivalence class with representative~$(j,k)$.
    Note that there is no need at this stage to verify that if $r$ is expressed as $j'/k'$ then $(j',k'){\sim}(j,k)$,
    since that is built into the definition of the equivalence relation determined by the surjection~${\rho}$.

        The bijection $\hat{{\rho}}$, together with the binary operation of `addition' already (intuitively) defined on the set~${\QQ}^{+}$,
    determines a corresponding operation of `addition' on $A$ for which the map ${\varphi}$ is an isomorphism; see Example~\Ref{ExampA60.80}~(4).
    More precisely, suppose that $z_{1}$ and $z_{2}$ are elements of~$A$,
    so that $z_{1} \,=\, \hat{{\rho}}^{-1}[\{r_{1}\}]$ and $z_{2} \,=\, \hat{{\rho}}^{-1}[\{r_{2}\}]$ for positive rationals $r_{1}$ and~$r_{2}$. Then define the sum $z_{1} + z_{2}$ by the rule
        \begin{displaymath}
        z_{1} + z_{2} \,=\, \hat{{\rho}}^{-1}[\{r_{1}+r_{2}\}],
        \end{displaymath}
    where the addition on the right is the (intuitive) addition of positive rational numbers.
    It is clear from the description of $\hat{{\rho}}^{-1}$ just given that if $r_{1} \,=\, j_{1}/k_{1}$ and $r_{2} \,=\, j_{2}/k_{2}$,
    then 
        \begin{displaymath}
        \hat{{\rho}}^{-1}(r_{1} + r_{2})
    \,=\, 
        \hat{{\rho}}^{-1}\left(\frac{j_{1}{\cdot}k_{2} + j_{2}{\cdot}k_{1}}{k_{1}{\cdot}k_{2}}\right),
        \end{displaymath}
    which is the equivalence class with representative $(j_{1}{\cdot}k_{2} + j_{2}{\cdot}k_{1}, k_{1}{\cdot}k_{2})$.
    Likewise, one sees that the bijection ${\rho}$, together with the (intuitive) concept of `multiplication' on ${\QQ}^{+}$,
    determines a `multiplication' on $A$, given by
        \begin{displaymath}
        z_{1}{\cdot}z_{2} \,=\, \hat{{\rho}}^{-1}[\{r_{1}{\cdot}r_{2}\}] \mbox{ for all $z_{1}$ and $z_{2}$ in $A$}.
        \end{displaymath}
    Using the same notation as above, one sees that $z_{1}{\cdot}z_{2}$ is the equivalence class
    with the representative $(j_{1}{\cdot}j_{2}, k_{1}{\cdot}k_{2})$.
    Furthermore, the `order' relation on ${\QQ}^{+}$ likewise determines a corresponding order relation on~$A$:
        \begin{displaymath}
        z_{1}\,<\,z_{2} \mbox{ if, and only if, } {\rho}(z_{1})\,<\,{\rho}(z_{2}).
        \end{displaymath}
    In terms of fractions used above, one has $z_{1}\,<\,z_{2}$ if, and only if, $j_{1}{\cdot}k_{2}\,<\,j_{2}{\cdot}k_{1}$.
    It is clear that the bijection $\hat{{\rho}}$ is an isomorphism between the operations of addition and multiplication,
    just defined on the set $A$, and the corresponding `intuitive' operations on~${\QQ}^{+}$;
    likewise, the bijection~$\hat{{\rho}}$ is an order-preserving map from $A$ onto~${\QQ}^{+}$.
    It follows that the algebraic properties of $A$ are the same as those on ${\QQ}^{+}$.
    For example, the `intuitive' understanding of ${\QQ}^{+}$ implies that there is a unique element in ${\QQ}^{+}$, the number~$1$, such that $r{\cdot}1\,=\, r$
    for all $r$ in~${\QQ}^{+}$. Likewise, for each $r$ in ${\QQ}^{+}$ there is a unique $s$ in ${\QQ}^{+}$ such that $r{\cdot}s \,=\, 1$.
    The existence of the isomorphism $\hat{{\rho}}$ then implies that the analogous properties hold for~$A$.

        As the formulas given above for $+$, ${\cdot}$ and~$\,<\,$ on $A$ indicate,
    the definitions of these concepts can also be obtained directly using only the equivalence relation~$\sim$, given by~$({\ast})$, and properties of~${\NN}$.
    That is, by using only the {\em results} of the intuitive discussion as a new starting point, and eliminating the intuitive discussion itself entirely,
    one can give a rigorous development of ${\QQ}^{+}$ directly from the properties of ${\NN}$ (and elementary set theory),
    as if one had never heard of ${\QQ}^{+}$ before. Here is how such a rigorous construction would normally be carried out:

        Given the set $X \,=\, {\NN}{\times}{\NN}$ and the equivalence relation~$(\sim)$ described above in Condition~$({\ast})$,
    let $A$ being the corresponding quotient set $X/\sim$. Define the binary operations of $+$ and~${\cdot}$ on $A$, and the order $\,<\,$, as follows:
    Let $z_{1}$ and $z_{2}$ be elements of $A$, i.e., equivalence classes associated with~${\sim}$,
    and let $(j_{1},k_{1})$ and $(j_{2},k_{2})$ in $X$ be representatives of $z_{1}$ and~$z_{2}$, respectively.
    Define $z_{1} + z_{2}$ to be the equivalence class which is represented by the ordered pair $(j_{1}{\cdot}k_{2} + j_{2}{\cdot}k_{1}, k_{1}{\cdot}k_{2})$.
    Likewise, define $z_{1}{\cdot}z_{2}$ to be the equivalence class represented by the ordered pair $(j_{1}{\cdot}j_{2}, k_{1}{\cdot}k_{2})$.
    Finally, define the relation $z_{1}\,<\,z_{2}$ to mean $j_{1}{\cdot}k_{2}\,<\,j_{2}{\cdot}k_{1}$.
    Of course these definitions might seem somewhat arbitrary without the earlier intuitive discussion using the surjection~${\rho}$;
    but technically speaking they do not refer to that discussion, and thus are ultimately based purely on the rigorously defined ${\NN}{\times}{\NN}$.
    Using the standard notation $[(j,k)]$ for the equivalence class with repressentative $(j,k)$,
    one can write 
        \begin{displaymath}
        [(j_{1},k_{1})] + [(j_{2},k_{2})] \,=\, [(j_{1}{\cdot}k_{2} + j_{2}{\cdot}k_{1}, k_{1}{\cdot}k_{2})], \h
        [(j_{1},k_{1})]{\cdot}[(j_{2},k_{2})] \,=\, [(j_{1}{\cdot}j_{2},{\cdot} k_{1}{\cdot}k_{2})], \h
        \end{displaymath}
        \begin{displaymath}
        [(j_{1},k_{1})] \,<\, {(j_{2},k_{2})} \mbox{ if, and only if, }
        j_{1}{\cdot}k_{2}\,<\,j_{2}{\cdot}k_{1}.
        \end{displaymath}
    The first need is to check that these constructions are {\bf well defined}\IndA{well defined constructions},
    in the sense that they does not depend on the choice of representatives of the classes $z_{1}$ and~$z_{2}$ used.
    Consider in more detail, for example, the definition of `multiplication' just given.
    Suppose that $(m_{1}, n_{1})$ and $(m_{2}, n_{2})$ are also representatives of $z_{1}$ and~$z_{2}$, respectively,
    so that by~$(\sim)$ one has $j_{1}{\cdot}n_{1} \,=\, k_{1}{\cdot}m_{1}$ and $j_{2}{\cdot}n_{2} \,=\, k_{2}{\cdot}m_{2}$.
    Then $(m_{1}{\cdot}m_{2},n_{1}{\cdot}n_{2})$ is equivalent to $((j_{1}{\cdot}j_{2}, k_{1}{\cdot}k_{2}))$,
    since $(m_{1}{\cdot}m_{2}){\cdot}(k_{1}{\cdot}k_{2}) \,=\, (n_{1}{\cdot}n_{2}{\cdot}(k_{1}{\cdot}k_{2}))$.
    That is, using different representatives for the equivalence classes $z_{1}$ and $z_{2}$ produce the same equivalence class $z_{1}{\cdot}z_{2}$, as required.
    The proofs that the operation of addition and the order relation are also well-defined are left as exercises.

%% EXERCISES +, \,<\, well-defined.

        The set $A$ of equivalence classes just constructed from ${\NN}$, together with the operations $+$ and ${\cdot}$, and the order relation $\,<\,$,
    can now  be considered as our our `official' definition of the system ${\QQ}^{+}$ of positive rational numbers, at least for the time being.
    (Later on we'll `improve' the construction.)

        Notice that system just constructed has a subsystem which behaves just like our original primitive system ${\NN}$ of natural numbers,
    namely the subset ${\NN}'$ of ${\QQ}^{+}$ consisting of all equivalence classes $z$ in $A$ of the form $[(j,1)]$ with $j$ in ${\NN}$.
    It is easy to see that $[(j_{1},1)] + [(j_{2},1)] \,=\, [((j_{1}+j_{2}),1)$ and $[(j_{1},1)]{\cdot}[(j_{2},1)] \,=\, [(j_{1}{\cdot}j_{2},1)]$.
    Likewise, $[(j_{1},1)]\,<\,[(j_{2},1)]$ if, and only if, $j_{1}\,<\,j_{2}$ in the original set~${\NN}$.
    We think of the set ${\NN}'$ so obtained as the `new and improved natural numbers':
    `new' because obviously the set ${\NN}'$ is different from the original set ${\NN}$;
    `improved', because these new numbers extend the original algebra of ${\NN}$ to the more inclusive arithmetic of~${\QQ}^{+}$.
    It is easy to see that in this arithmetic one has $[(1,1)]{\cdot}[(j,k)] \,=\, [(j,k)]$
    and $[(k,j)]{\cdot}[(j,k)] \,=\, [(k{\cdot}j,k{\cdot}j)] \,=\, [(1,1)]$. In particular,
    $[(1,1)]$ is the multiplicative unit, while $[(k,j)]$ is the multipicative inverse of $[(j,k)]$.
    By introducing the obvious `division' operation by $[(j_{1},k_{1})/(j_{2},k_{2})] \,=\, [(j_{1},k_{1})]{\cdot}[(k_{2},j_{2})]$,
    one gets the formula $[(j,k)] \,=\, [(j,1)]/[(k,1)]$. This is the `new and improved'
    version of the formula ${\rho}(j,k) \,=\, j/k$ used in the intuitive discussion above.

        Note that, by the preceding constructions, a positive rational number is an equivalence class of ordered pairs of the primitive natural numbers.
    In particular, the `new and improved' natural numbers are special types of such classes.



        \subsection{\small{{\bf Construction 2: The Rationals from the Positive Rationals}}}
                \label{ConstA50.130B}

 \V

        The next step is to rigorously extend the system ${\QQ}^{+}$, just obtained,
    to the system consisting of {\em all} the rational numbers, not just the positives.
    The procedure is, in spirit, similar to -- but, in the details, rather different from -- that used in Construction~(1), so we can be somewhat briefer.
    Indeed, if we let ${\QQ}$ denote for now the rational numbers of our intuition,
    there is a natural surjection ${\delta}:{\QQ}^{+}{\times}\,{\QQ}^{+}  \,{\rightarrow}\,  {\QQ}$ given by
        \begin{displaymath}
        {\delta}(z_{1},z_{2}) \,=\, z_{1} - z_{2}.
        \end{displaymath}
    (The symbol ${\delta}$ here stands for `difference'.) The equivalence relation on ${\QQ}^{+}{\times}\,{\QQ}^{+}$ determined by this surjection is
        \begin{displaymath}
        (z_{1},z_{2}){\sim}(z_{1}',z_{2}') \mbox{ if, and only if, }
        z_{1} + z_{2}' \,=\, z_{1}' + z_{2}
        \end{displaymath}
    Let ${\QQ}$ denote the set of equivalence classes arising from this equivalence relation on ${\QQ}^{+}{\times}\,{\QQ}^{+}$.
    It is easy to define the concepts of addition, multiplication and order on the set ${\QQ}$,
    using only the structures on~${\QQ}^{+}$, but not the intuitive map ${\Delta}$.
    For example, if $s \,=\, [(z_{1},z_{2}))]$ and $s' \,=\, [(z_{1}',z_{2}')]$ are elements of ${\QQ}$,
    then $s+s' \,=\, [(z_{1} + z_{1}', z_{2} + z_{2}')]$. Likewise, $s{\cdot}s' \,=\, [(z_{1}{\cdot}z_{1}' + z_{2}{\cdot}z_{2}', z_{1}{\cdot}z_{2}' + z_{2}{\cdot}z_{1}')]$.
    Likewise, $s\,<\,s'$ if, and only if, $z_{1} + z_{2}'\,<\,z_{1}' + z_{2}$. These constructions use only the rigorously defined structure of ${\QQ}^{+}$.

        There is a zero element in ${\QQ}$, namely the equivalence class with representative $(1,1)$,
    where $1$ now denotes the multiplicative identity in~${\QQ}^{+}$; any representative of the form $(c,c)$ with $c$ in ${\QQ}^{+}$ works too.
    One then defined the {\em positive rationals} to be those elements $[(z_{1},z_{2})]$ of ${\QQ}$ such that $[(z_{1},z_{2})]\,>\,[(1,1)]$.
    It is easy to check that this is equivalent to $z_{1}\,>\,z_{2}$ in ${\QQ}^{+}$,
    and that this set of `positive rationals' is isomorphic to~${\QQ}^{+}$ in the usual sense.
    Likewise, there is a subset of the set ${\QQ}$ just constructed which corresponds to the set ${\ZZ}$ of integers.

    Of course one needs to verify that these concepts do not depend on choices of representatives of equivalence classes in ${\QQ}^{+}{\times}\,{\QQ}^{+}$,
    and one needs to show that the copy of ${\QQ}^{+}$ in ${\QQ}$ just described is equivalent to the ${\QQ}^{+}$ already described.
    The details are tedious, so the usual custom, gladly followed here, is to leave the details to the reader.

\VV

 
%-----------------
\StartSkip{

        \subsection{\small{{\bf Examples}}
                \label{ExampA50.130}

 \V

\hspace*{\parindent} (1) Let $X$ be the set ${\NN}{\times}{\NN}$ of all ordered pairs of natural numbers, treated as a `rigorous' primitive concept.
    Let $Y$ be the set ${\QQ}^{+}$ of all positve rational numbers', viewed as an intuitively understood system which eventually must be defined rigorously. 
    In terms of this intuition, there is a well-known surjective map ${\rho}:X \,{\rightarrow}\, {\QQ}^{+}$ defined by
    ${\rho}(j,k) \,=\, j/k$ for each ordered pair $(j,k)$ in~$X$; `${\rho}$' stands for `ratio'.
    It is esay to see that the equivalence relation determined on $X$ by the surjection ${\rho}$ is this:
        \begin{displaymath}
        (j_{1},k_{1}) \,{\sim}\, (j_{2}, k_{2}) \mbox{ if, and only if, }
        j_{1}{\cdot}k_{2} \,=\, j_{2}{\cdot}k_{1} \h ({\ast})
        \end{displaymath}
    This equivalence relation, even though it arises in this discussion from an intitive understanding of `rational number', which includes the map~${\rho}$,
    is formulated here purely in terms of the primitive notions of ordered pairs and the algebraic system ${\NN}$ as described by the Dedekind-Peano axioms.
    In particular, there is no mention in Condition~$({\ast})$ of either the `intuitive' concept of the set ${\QQ}^{+}$ or the `intuitive' map~${\rho}$.
    For that reason we use the symbol ${\sim}$ instead of the more proper ${\sim}_{{\rho}}$ in Condition~$({\ast})$.

        Return now to the intuitive discussion of~${\QQ}$. Let $Z \,=\, X/({\rho})$ be the set of equivalence classes associated with the equivalence relation described above,
    and let $\hat{{\rho}}:Z \,{\rightarrow}\, {\QQ}^{+}$ be the corresponding canonical bijection; see Definition~\Ref{DefA50.127} above.
   Note also that the inverse map $\hat{{\rho}}^{-1}:{\QQ}^{+} \,{\rightarrow}\, Z$ is easy to compute:
    if $r{\in}{\QQ}^{+}$, express $r$ in the form $j/k$ as above; then $\hat{{\rho}}^{-1}(r)$ is the equivalence class with representative~$(j,k)$.
    Note that there is no need at this stage to verify that if $r$ is expressed as $j'/k'$ then $(j',k'){\sim}(j,k)$,
    since that is built into the definition of the equivalence relation determined by the surjection~${\rho}$.

        The bijection $\hat{{\rho}}$, together with the binary operation of `addition' already (intuitively) defined on the set~${\QQ}^{+}$,
    determines a corresponding operation of `addition on $Z$ for which the map ${\varphi}$ is an isomorphism; see Example~\Ref{ExampA60.80}~(4).
    More precisely, suppose that $z_{1}$ and $z_{2}$ are elements of~$Z$,
    so that $z_{1} \,=\, \hat{{\rho}}^{-1}[\{r_{1}\}]$ and $z_{2} \,=\, \hat{{\rho}}^{-1}[\{r_{2}\}]$ for positive rationals $r_{1}$ and~$r_{2}$. Then
        \begin{displaymath}
        z_{1} + z_{2} \,=\, \hat{{\rho}}^{-1}[\{r_{1}+r_{2}\}],
        \end{displaymath}
    where the addition on the right is the (intuitive) addition of positive rational numbers,
    while that on the left is the addition of elements of $Z$ being defined.
    It is clear from the description of $\hat{{\rho}}^{-1}$ just given that if $r_{1} \,=\, j_{1}/k_{1}$ and $r_{_{2} \,=\, j_{2}/k_{2}}$,
    then 
        \begin{displaymath}
        \hat{{\rho}}^{-1}(r_{1} + r_{2})
    \,=\, 
        \hat{{\rho}}^{-1}\left(\frac{j_{1}{\cdot}k_{2} + j_{2}{\cdot}k_{1}}{k_{1}{\cdot}k_{2}}\right),
        \end{displaymath}
    which is the equivalence class with representative $(j_{1}{\cdot}k_{2} + j_{2}{\cdot}k_{1}, k_{1}{\cdot}k_{2})$.
    Likewise, one sees that the bijection ${\rho}$, together with the (intuitive) concept of `multiplication' on ${\QQ}^{+}$,
    determines a `multiplication' on $Z$, given by
        \begin{displaymath}
        z_{1}{\cdot}z_{2} \,=\, \hat{{\rho}}^{-1}[\{r_{1}{\cdot}r_{2}\}] \mbox{ for all $z_{1}$ and $z_{2}$ in $Z$}.
        \end{displaymath}
    Using the same notation as above, one sees that $z_{1}{\cdot}z_{2}$ is the equivalence class
    with the representative $(j_{1}{\cdot}j_{2}, k_{1}{\cdot}k_{2})$.
    Furthermore, the `order' relation on ${\QQ}^{+}$ likewise determines a corresponding order relation on~$Z$:
        \begin{displaymath}
        z_{1}\,<\,z_{2} \mbox{ if, and only if, } {\rho}(z_{1})\,<\,{\rho}(z_{2}).
        \end{displaymath}
    In terms of fractions used above, one has $z_{1}\,<\,z_{2}$ if, and only if, $j_{1}{\cdot}k_{2}\,<\,j_{2}{\cdot}k_{1}$.
    It is clear that the bijection $\hat{{\rho}}$ is an isomorphism between the operations of addition and multiplication,
    just defined on the set $Z$, and the corresponding `intuitive' operations on~${\QQ}^{+}$;
    likewise, the bijection~$\hat{{\rho}}$ is an order-preserving map from $Z$ onto~${\QQ}^{+}$.
    It follows that the algebraic properties of $Z$ are the same as those on ${\QQ}^{+}$.
    For example, the `intuitive' understanding of ${\QQ}^{+}$ implies that there is a unique element in ${\QQ}^{+}$, the number~$1$, such that $r{\cdot}1\,=\, r$
    for all $r$ in~${\QQ}^{+}$. Likewise, for each $r$ in ${\QQ}^{+}$ there is a unique $s$ in ${\QQ}^{+}$ such that $r{\cdot}s \,=\, 1$.
    The existence of the isomorphism $\hat{{\rho}}$ then implies that the analogous properties hold for~$Z$.

        As the formulas given above for $+$, ${\cdot}$ and~$\,<\,$ on $Z$ indicate,
    the definitions of these concepts can also be obtained directly using only the equivalence relation~$\sim$, given by~$({\ast})$, and properties of~${\NN}$.
    That is, by using only the {\em results} of the intuitive discussion as a new starting point, and eliminating the intuitive discussion itself entirely,
    one can give a `rigorous' development of ${\QQ}^{+}$ directly from the properties of ${\NN}$ (and elementary set theory),
    as if one had never heard of ${\QQ}^{+}$ before. Here is how such a `rigorous' construction would normally be carried out:

        Given the set $X \,=\, {\NN}{\times}{\NN}$ and the equivalence relation~$(\sim)$ described above in Condition~$({\ast})$,
    let $Z$ being the corresponding quotient set $X/\sim$. Define the binary operations of $+$ and~${\cdot}$ on $Z$, and the order $\,<\,$, as follows:
    Let $z_{1}$ and $z_{2}$ be elements of $Z$, i.e., equivalence classes associated with~${\sim}$,
    and let $(j_{1},k_{1})$ and $(j_{2},k_{2})$ in $X$ be representatives of $z_{1}$ and~$z_{2}$, respectively.
    Define $z_{1} + z_{2}$ to be the equivalence class which is represented by the ordered pair $(j_{1}{\cdot}k_{2} + j_{2}{\cdot}k_{1}, k_{1}{\cdot}k_{2})$.
    Likewise, define $z_{1}{\cdot}z_{2}$ to be the equivalence class represented by the ordered pair $(j_{1}{\cdot}j_{2}, k_{1}{\cdot}k_{2})$.
    Finally, define the relation $z_{1}\,<\,z_{2}$ to mean $j_{1}{\cdot}k_{2}\,<\,j_{2}{\cdot}k_{1}$.
    Of course these definitions might seem somewhat arbitrary without the earlier intuitive discussion using the surjection~${\rho}$;
    but technically speaking they do not refer to that discussion, and thus are ultimately based purely on the rigorously defined ${\NN}{\times}{\NN}$.
    Using the standard notation $[(j,k])$ for the equivalence class with repressentative $(j,k)$,
    one can write 
        \begin{displaymath}
        [(j_{1},k_{1})] + [(j_{2},k_{2})] \,=\, [(j_{1}{\cdot}k_{2} + j_{2}{\cdot}k_{1}, k_{1}{\cdot}k_{2})], \h
        [(j_{1},k_{1})]{\cdot}[(j_{2},k_{2})] \,=\, [(j_{1}{\cdot}j_{2},{\cdot} k_{1}{\cdot}k_{2})], \h
        [(j_{1},k_{1})] \,<\, {(j_{2},k_{2})} \mbox{ if, and only if, }
        j_{1}{\cdot}k_{2}\,<\,j_{2}{\cdot}k_{1}.
        \end{displaymath}
    The first need is to check that these constructions are {\bf well defined}IndA{well defined constructions},
    in the sense that they does not depend on the choice of representatives of the classes $z_{1}$ and~$z_{2}$ used.
    Consider in more detail, for example, the definition of `multiplication' just given.
    Suppose that $(m_{1}, n_{1})$ and $(m_{2}, n_{2})$ are also representatives of $z_{1}$ and~$z_{2}$, respectively,
    so that by~$(\sim)$ one has $j_{1}{\cdot}n_{1} \,=\, k_{1}{\cdot}m_{1}$ and $j_{2}{\cdot}n_{2} \,=\, k_{2}{\cdot}m_{2}$.
    Then $(m_{1}{\cdot}m_{2},n_{1}{\cdot}n_{2})$ is equivalent to $((j_{1}{\cdot}j_{2}, k_{1}{\cdot}k_{2}))$,
    since $(m_{1}{\cdot}m_{2}){\cdot}(k_{1}{\cdot}k_{2}) \,=\, (n_{1}{\cdot}n_{2}{\cdot}(k_{1}{\cdot}k_{2}))$.
    That is, using different representatives for the equivalence classes $z_{1}$ and $z_{2}$ produce the same equivalence class $z_{1}{\cdot}z_{2}$, as required.
    The proofs that the operation of addition and the order relation are also well-defined are left as exercises.

%% EXERCISES +, \,<\, well-defined.

        The set $Z$ of equivalence classes just constructed from ${\NN}$, together with the operations $+$ and ${\cdot}$, and the order relation $\,<\,$,
    can now  be considered as our our `official' definition of the system ${\QQ}^{+}$ of positive rational numbers, at least for the time being.
    Notice that this system has a subsystem which behaves just like our original primitive system ${\NN}$ of natural numbers,
    namely the subset ${\NN}'$ of ${\QQ}^{+}$ consisting of all equivalence classes $z$ in $Z$ of the form $[(j,1)]$ with $j$ in ${\NN}$.
    It is easy to see that $[(j_{1},1)] + [(j_{2},1)] \,=\, [((j_{1}+j_{2}),1)$ and $[(j_{1},1)]{\cdot}[(j_{2},1)] \,=\, [(j_{1}{\cdot}j_{2},1)]$.
    Likewise, $[(j_{1},1)]\,<\,[(j_{2},1)]$ if,and only if, $j_{1}\,<\,j_{2}$ in the original set~${\NN}$.
    We think of the set ${\NN}'$ so obtained as the `new and improved natural numbers':
    `new' because obviously the set ${\NN}'$ is different from the original set ${\NN}$;
    `improved', because these new numbers extend the original algebra of ${\NN}$ to the more inclusive arithmetic of~${\QQ}^{+}$.
    It is easy to see that in this arithmetic one has $[(1,1)]{\cdot}[(j,k)] \,=\, [(j,k)]$
    and $[(k,j)]{\cdot}[(j,k)] \,=\, [(k{\cdot}j,k{\cdot}j)] \,=\, [(1,1)]$. In particular,
    $[(1,1)]$ is the multiplicative unit, while $[(k,j)]$ is the multipicative inverse of $[(j,k)]$.
    By introducing the obvious `division' operation by $[(j_{1},k_{1})/(j_{2},k_{2})] \,=\, [(j_{1},k_{1})]{\cdot}[(k_{2},j_{2})]$,
    one gets the formula $[(j,k)] \,=\, [(j,1)]/[(k,1)]$. This is the `new and improved'
    version of the formula ${\rho}(j,k) \,=\, j/k$ used in the intuitive discussion above.

        Note that, by the preceding constructions, a positive rational number is an equivalence class of ordered pairs of the primitive natural numbers.
    In particular, the `new and improved' natural numbers are special types of such classes.


\VV

        (2) The next step is to extend the system ${\QQ}^{+}$ just obtained to the system consisting of {\em all} the rational numbers, not just the positives.
    The procedure is, in spirit, similar to -- but, in the details, rather different from -- that used in Example~(1), so we can be somewhat briefer.
    Indeed, if we let ${\QQ}$ denote for now the rational numbers of our intuition,
    there is a natural surjection ${\delta}:{\QQ}^{+}{\times}{\QQ}^{+} \,{\supseteq}\, {\QQ}$ given by
        \begin{displaymath}
        {\delta}(z_{1},z_{2}) \,=\, z_{1} - z_{2}.
        \end{displaymath}
    (The symbol ${\delta}$ here stands for `difference'.) The equivalence relation on ${\QQ}^{+}{\times}{\QQ}^{+}$ determined by this surjection is
        \begin{displaymath}
        (z_{1},z_{2}){\sim}(z_{1}',z_{2}') \mbox{ if, and only if, }
        z_{1} + z_{2}' \,=\, z_{1}' + z_{2}
        \end{displaymath}
    Let ${\QQ}$ denote the set of equivalence classes arising from this equivalence relation on ${\QQ}^{+}{\times}{\QQ}^{+}$.
    It is easy to define the concepts of addition, multiplication and order on the set ${\QQ}$ using only the structures on~${\QQ}^{+}$, and not the intuitive map ${\delta}$.
    For example, if $s \,=\, [(z_{1},z_{2}))]$ and $s' \,=\, [(z_{1}',z_{2}')]$ are elements of ${\QQ}$,
    then $s+s' \,=\, [(z_{1} + z_{1}', z_{2} + z_{2}')]$. Likewise, $s{\cdot}s' \,=\, [(z_{1}{\cdot}z_{1}' + z_{2}{\cdot}z_{2}', z_{1}{\cdot}z_{2}' + z_{2}{\cdot}z_{1}')]$.
    Likewise, $s\,<\,s'$ if, and only if, $z_{1} + z_{2}'\,<\,z_{1}' + z_{2}$. These constructions use only the rigorously defined structure of ${\QQ}^{+}$.

        There is a zero element in ${\QQ}$, namely the equivalence class with representative $(1,1)$,
    where $1$ now denotes the multiplicative identity in~${\QQ}^{+}$; any representative of the form $(c,c)$ with $c$ in ${\QQ}^{+}$ works too.
    One then defined the {\em positive rationals} to be those elements $[(z_{1},z_{2})]$ of ${\QQ}$ such that $[(z_{1},z_{2})]\,>\,[(1,1)]$.
    It is easy to check that this is equivalent to $z_{1}\,>\,z_{2}$ in ${\QQ}^{+}$,
    and that this set of `positive rationals' is isomorphic to~${\QQ}^{+}$ in the usual sense.
    Likewise, there is a subset of the set ${\QQ}$ just constructed which corresponds to the set ${\ZZ}$ of integers.

    Of course one needs to verify that these concepts do not depend on choices of representatives of equivalence classes in ${\QQ}^{+}{\times}{\QQ}^{+}$,
    and one needs to show that the copy of ${\QQ}^{+}$ in ${\QQ}$ just described is equivalent to the ${\QQ}^{+}$ already described.
    The details are tedious, so the custom is to leave them to the reader.

\VV

}%\EndSkip
%------------------------

%--------------------------
\StartSkip{

\hspace*{\parindent} (3) The preceding examples first extended the positive integers to the positive rationals,
    then extended further to obtain all the rationals, including negative numbers.
    Historically speaking, this makes sense: the positive rationals appeared long before negative numbers.
    In modern mathematics, however, the custom is to start with ${\NN}$, then extend to ${\ZZ}$, the set of all integers,
    and finally extend from ${\ZZ}$ to~${\QQ}$, the set of all rationals. Here is the first stage of that process.

\V

        As before, start with with one's intuitive understanding of the set ${\ZZ}$ of all integers.
    There is a corresponding (intuitive) map ${\Delta}:{\NN}{\times}{\NN} \,{\rightarrow}\, {\ZZ}$ of ${\NN}{\times}{\NN}$ onto ${\ZZ}$ given by the rule
        \begin{displaymath}
        {\Delta}(j,k) \,=\, j-k \mbox{ for all $(j,k)$ in ${\NN}{\times}{\NN}$}
        \end{displaymath}
    (${\Delta}$ stands for `difference'.) It is easy to see that ${\Delta}$ is a surjection of (rigotously defined) set ${\NN}{\times}{\NN}$ onto the (intuitive) set ${\ZZ}$.
    Note also that the operation of addition of integers corresponds to a simple operation on pairs of natural numbers.
    Indeed, suppose that $m_{1} \,=\, {\Delta}(j_{1},k_{1}) \,=\, j_{1}-k_{1}$ and $m_{2} \,=\, {\Delta}(j_{2},k_{2}) \,=\, j_{2}-k_{2}$, with $(j_{1},k_{1})$, and $(j_{2},k_{2})$ being elements of ${\NN}{\times}{\NN}$.
    Then
        \begin{displaymath}
        m_{1} + m_{2} \,=\, (j_{1} - k_{1}) + (j_{2} - k_{2}) \,=\,(j_{1} + j_{2}) - (k_{1} + k_{2})
    \,=\,
        {\Delta}(j_{1} + j_{2}, k_{1} + k_{2})
        \end{displaymath}
    Similarly, the operation of multiplication of integers corresponds to a (somewhat less simple) operation on elements of ${\NN}{\times}{\NN}$,
    Indeed, if $m_{1} \,=\, j_{1}-k_{1}$ and $m_{2} \,=\, j_{2}-k_{2}$, then
        \begin{displaymath}
        m_{1}{\cdot}m_{2} \,=\, (j_{1}-k_{1}) {\cdot} (j_{2}-k_{2}) \,=\, (j_{1}{\cdot}j_{2} + k_{1}{\cdot}k_{2}) - (j_{1}{\cdot}k_{2} + k_{1}{\cdot}j_{2})
        \end{displaymath}
    That is,
        \begin{displaymath}
        m_{1}{\cdot}m_{2} \,=\, {\Delta}(j_{1}{\cdot}j_{2} + k_{1}{\cdot}k_{2},j_{1}{\cdot}k_{2} + k_{1}{\cdot}j_{2}) \h ({\ast}{\ast})
        \end{displaymath}

        The surjection ${\Delta}$ determines an equivalence relation $W_{{\Delta}}$ on ${\NN}{\times}{\NN}$ by the rule
        \begin{displaymath}
        (j,k) {\sim}_{{\Delta}} (j',k') \mbox{ if, and only if, } j-k \,=\, j'-k';
        \end{displaymath}
 see Example~\Ref{ExampA50.100}~(2).
    Note that this condition can be reformulated purely in terms of natural numbers as
        \begin{displaymath}
        (j,k) {\sim}_{{\Delta}} (j',k') \mbox{ if, and only if, } j + k' \,=\, j' + k \h ({\ast}{\ast}{\ast})
        \end{displaymath}
    Let $W$ be the quotient set corresponding to this equivalence relation on ${\NN}{\times}{\NN}$.
    Then Theorem~\Ref{ThmA50.125} produces a natural bijection $\hat{{\Delta}}:W \,{\rightarrow}\, {\ZZ}$.
    As in the preceding examples, this intuitive treatment can be reformulated purely in terms of the structure on ${\NN}$,
    with no reference to the map ${\Delta}$, to give a rigorous construction of ${\ZZ}$
    the tedious details are omitted here.

        Now that a rigorous construction of the set ${\ZZ}$ of all integers has been outlined,
    it is easy to use our experience from Example~(1) above to go directly from this set ${\ZZ}$ to a rigorous definition of the set ${\QQ}$ of all rationals.
    Indeed, let $X'$ be the set of all ordered pairs $(m,n)$ with $m$ and $n$ in ${\ZZ}$ and $n\,>\,0$.
    (Note that here $0$ means the integer, i.e., equivalence class of ordered pairs $(j,k)$ of natural numbers, for which $j \,=\, k$.
    Likewise, $m$ and $n$ are themselves equivalence classes of ordered pairs of natural numbers.)
    Define an equivalence relation ${\sim}'$ on $X'$ by the rule
        \begin{displaymath}
        (m_{1}, n_{1}) \,{\sim}'\, (m_{2},n_{2}) \mbox{ if, and only if, }
        m_{1}{\cdot}n_{2} \,=\, m_{2}{\cdot}n_{1};
        \end{displaymath}
     the multiplication indicated here is, of course, that defined rigotously on the set ${\ZZ}$ above.
    The addition, multiplication and order are then defined much as in Example~(1).
    As usual, the details that all of this works is left to the reader as an exercise.

\VV

        {\bf Remarks} (1) In both Examples (2) and (3) one rigorously constructs, from the same `primitive' system~${\NN}$,
    new systems which correspond to one's intuitive concept of `rational number'.
    However the {\em natures} of the individual objects in these systems are quite different. For instance,
    in Example~(3) the rational number~$0$ is an equivalence class consisting of ordered pairs of integers, which in turn are sets of ordered pairs of natural numbers.
    In contrast, the corresponding element~$0$ in Example~(2) is also an equivalence class,
    but one whose elements are ordered pairs of positive rationals, which in turn are also sets of ordered pairs of natural numbers, but different sets.
    In set-theoretic terms, these objects are unequal because of the Axiom of Extension.
    However, it is straight-forward, but extrewmely boring, to check that the two systems have the same algebraic properties as each other.
    Indeed, the main purpose of such constructions is to rigorously establish the existence of systems which correspond to our intuitive concept of~${\QQ}$.
    Of course both of these constructions are based on the set ${\NN}$ of natural numbers, which we have taken as a primitive concept:
    we do not here provide a rigorous construction of ${\NN}$ from objects which are even more primitive.

\V

        (2) It is clear that the original natural numbers are certainly not elements of the set ${\QQ}$,
    whether one uses the first construction or the second. Nevertheless, the set ${\QQ}$ does have a natural subset which correponds to the natural numbers.
    We follow the usual custom and treat these as `new, improved' natural numbers and thus treat ${\NN}$ as this subset of~${\QQ}$.
    The sam holds for the set ${\ZZ}$ of integers: we treat it as a subset of~${{\QQ}}$,
    Later on we `improve' these numbers again and replace them with a subset of the real numbers.

}%\EndSkip
%---------------------

%\\\\\
\StartSkip{
        In the preceding construction we assume that both ${\NN}$ and ${\ZZ}$ are already known;
    the bijection $\hat{{\Delta}}$ then tells us that the (known) set ${\ZZ}$ has a lot of similarities with the quotient set $W$.
    However, because of Condition~$({\ast}{\ast}{\ast})$, it is possible to define the given equivalence relation on ${\NN}{\times}{\NN}$
    purely in terms of the set ${\NN}$ alone, without explicitly mentioning the set ${\ZZ}$ or the function ${\Delta}$.
    Likewise, the arguments of ${\Delta}$ on the right sides of Equations~$({\ast})$ and~$({\ast}{\ast})$ describe operations on elements of ${\NN}{\times}{\NN}$ which make sense without mentioning either ${\ZZ}$ or ${\Delta}$.
    Many authors exploit these facts to {\em define} the set ${\ZZ}$, together with its operations of addition and multiplication,
    purely in terms of the more primitive set ${\NN}$ and its operations. Here is how it works:

        \h (i)\,\, Inspired by Condition~$({\ast}{\ast}{\ast})$ above, let ${\sim}$ be the relation on ${\NN}{\times}{\NN}$ given by the condition
        \begin{displaymath}
        (j,k) {\sim} (j',k') \mbox{ if,and only if, } j+k' \,=\, j'+k.
        \end{displaymath}
    It is a simple exercise to show, using only the properties of ${\NN}$ (and without mentioning `zero' or `negative numbers') that ${\sim}$ is an equivalence relation on ${\NN}{\times}{\NN}$.
    Now one simply {\em defines} ${\ZZ}$ to be the quotient set of this equivalence relation.

        \h (ii)\, Inspired by Condition~$({\ast})$ above, define an operation of `addition' on the set ${\ZZ}$ just obtained as follows:
    Suppose that $m_{1}$ and $m_{2}$ are elements of ${\ZZ}$; that is, $m_{1}$ and $m_{2}$ are equivalence classes of the equivalence relation $\sim$ on ${\NN}{\times}{\NN}$.
    Let $(j_{1},k_{1})$ and $(j_{2},k_{2})$ be elements of the equivalence classes $m_{1}$ and $m_{2}$, respectively;
    using the notation of Definition~\Ref{DefA50.120}, one then has $m_{1} \,=\, [(j_{1},k_{1})]$ and $m_{2} \,=\, [(j_{2},k_{2})]$.
    Now {\em define} $m_{1}+m_{2}$ to be the equivalence class containing the ordered pair $(j_{1}+j_{2},k_{1}+k_{2})$.
    That is,
        \begin{displaymath}
        [(j_{1},k_{1})] + [(j_{2},k_{2})] \,=\, [(j_{1}+j_{2},k_{1}+k_{2})].
        \end{displaymath}
        This `definition' of $m_{1}+m_{2}$ has an obvious flaw: it uses a choice of representatives from $m_{1}$ and $m_{2}$,
    so it appears that the value of $m_{1}+m_{2}$ depends not only on the sets $m_{1}$ and $m_{2}$, but may depend on how one chooses these representatives.
    Fortunately, this apparent dependence is an illusion.
    Indeed, it is easy to show directly from the definition of the equivalence relation ${\sim}$ that if we replace $(j_{1},k_{1})$ by an equivalent element $(j_{1}',k_{1}')$,
    and replace $(j_{2},k_{2})$ by an equivalent $(j_{2}',k_{2}')$, then the resulting pair $(j'_{1}+j'_{2},k'_{1}+k'_{2})$ is equivalent to the original $(j_{1}+j_{2},k_{1}+k_{2})$;
    in other words, $[(j'_{1}+j'_{2},k'_{1}+k'_{2})] \,=\, [(j_{1}+j_{2},k_{1}+k_{2})]$.


        \h (iii) Inspired by Condition~$({\ast}{\ast})$ above, define an operation of `multipication' on the set ${\ZZ}$ as follows:
    Suppose that $m_{1}$ and $m_{2}$ are elements of ${\ZZ}$; that is, $m_{1}$ and $m_{2}$ are equivalence classes of the equivalence relation $\sim$ on ${\NN}{\times}{\NN}$.
    Let $(j_{1},k_{1})$ and $(j_{2},k_{2})$ be elements of the equivalence classes $m_{1}$ and $m_{2}$, respectively.
    Now {\em define} $m_{1}{\cdot}m_{2}$ to be the equivalence class containing the ordered pair $(j_{1}{\cdot}j_{1} + k_{1}{\cdot}k_{2}, j_{1}{\cdot}k_{2}+k_{1}{\cdot}j_{2}])$.
    That is,
        \begin{displaymath}
        [(j_{1},k_{1})] {\cdot} [(j_{2},k_{2})] \,=\, [(j_{1}{\cdot}j_{2} + k_{1}{\cdot}k_{2}, j_{1}{\cdot}k_{2} + k_{1}{\cdot}j_{2})].
        \end{displaymath}
    As with the definition of `addition' on ${\ZZ}$, one must verify that this construction of $m_{1}{\cdot}m_{2}$ does not depend on the choice of representatives of these sets.
    The analysis is slightly tedious, but not difficult; it is left as an exercise.

        \underline{Note} In the preceding construction, an `integer' turns out to be set of ordered pairs of natural numbers.
    Of course each such `ordered pair' $(j,k)$ is a set of sets of natural numbers, $\{\{j\},\{j,k\}\}$, if one uses the Kuratowski definition of `ordered pair';
    or, if one uses the `new and improved' definition of `ordered pair' given in Definition~\Ref{DefA30.30},
    $(j,k)$ is a set of sets of sets of natural numbers.
    In any event, an integer is a  type of object much more complicated than our original notion of `natural number'.
    In particular, the set ${\ZZ}$ just constructed does {\em not} have the set ${\NN}$ as a subset, since (as was just noted) none of the elements of ${\ZZ}$ is actually a natural number!
    This  should  seem  disconcerting, but the solution is simple:  the set ${\ZZ}$ does have a subset that behaves just like ${\NN}$;
    namely, the set of all equivalence classes of the form $[(k+1,1)]$ with $k$ in ${\NN}$.
    If we were going to stop with the set of integers, it would make sense to choose this subset as our preferred `new, improved' set of natural numbers.
    In fact, however, that honor will be saved for a set to be described in the next chapter.

\V

        (2) Let ${\QQ}$ denote the set of all rational numbers, and let ${\ZZ}^{*}$ denote the set of all nonzero integers.
    There is a natural function ${\rho}:{\ZZ}{\times}{\ZZ}^{*} \,{\rightarrow}\, {\QQ}$,
    given by the rule ${\rho}(j,k) \,=\, j/k$ for all $(j,k)$ in ${\ZZ}{\times}{\ZZ}^{*}$.
    It is easy to see that this function maps ${\ZZ}{\times}{\ZZ}^{*}$ surjectively onto ${\QQ}$,
    and thus ${\rho}$ determines an equivalence relation on ${\ZZ}{\times}{\ZZ}^{*}$.
    The function ${\rho}$ is certainly not an injection, however, since if a rational number $r$ can be expressed as $r \,=\, j/k$ with $j$ and $k$ in ${\ZZ}$ and $k \,\,{\neq}\,\, 0$,
    then it can also be expressed as $r \,=\, j'/k'$ with $j' \,=\, mj$ and $k' \,=\, mk$ for any $m$ in ${\ZZ}^{*}$.

       The surjection ${\rho}$ determines an equivalence relation on ${\ZZ}{\times}{\ZZ}^{*}$:
    $(j,k) {\sim} (j',k')$ provided ${\rho}(j,k) \,=\, {\rho}(j',k')$, i.e., $j/k \,=\, j'/k'$.
    As in  Example~(1) above, however, this condition can be formulated without explicitly mentioning either ${\rho}$ or the rational numbers:
        \begin{displaymath}
        (j,k) {\sim} (j',k') \mbox{ if, and only if, } j{\cdot}k' \,=\, j'{\cdot}k.
        \end{displaymath}
    And, as in Example~(1), one can define appropriate operations of `addition' and `multiplication' on the quotient set $\left({\ZZ}{\times}{\ZZ}^{*}\right)/{\sim}$ to make this set into a model of the rational numbers.
    This is the approach followed by many authors in their construction of ${\QQ}$.
    We shall not carry it  any further here, however, since we'll deal with ${\QQ}$ in a different manner in the next chapter.
}%\EndSkip %% NOT CLEAR WHY THIS EXTRA ENDSKIP IS NEEDED
}%\Endskip

