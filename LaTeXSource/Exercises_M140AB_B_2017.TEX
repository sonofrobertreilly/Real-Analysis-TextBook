% Exercises_M140AB_B.TeX   Exercises for Chapter B

%
% Revised: 06/28/2016  Encoding: Western ASCII
%

%% NOTE: Copy the 52 lines below to each chapter, and change the Chapter letters.

%\thispagestyle{myheadings}


%\markboth{Exercises for Chapter~\ref{ChaptB} -}{Exercises for Chapter~\ref{ChaptB} -}

\newcommand{\ExBa}{{\bf \ref{ChaptB} - \,1} }
\newcommand{\ExBb}{{\bf \ref{ChaptB} - \,2} }
\newcommand{\ExBc}{{\bf \ref{ChaptB} - \,3} }
\newcommand{\ExBd}{{\bf \ref{ChaptB} - \,4} }
\newcommand{\ExBe}{{\bf \ref{ChaptB} - \,5} }
\newcommand{\ExBf}{{\bf \ref{ChaptB} - \,6} }
\newcommand{\ExBg}{{\bf \ref{ChaptB} - \,7} }
\newcommand{\ExBh}{{\bf \ref{ChaptB} - \,8} }
\newcommand{\ExBi}{{\bf \ref{ChaptB} - \,9} }
\newcommand{\ExBj}{{\bf \ref{ChaptB} -  10} }
\newcommand{\ExBk}{{\bf \ref{ChaptB} -  11} }
\newcommand{\ExBl}{{\bf \ref{ChaptB} -  12} }
\newcommand{\ExBm}{{\bf \ref{ChaptB} -  13} }
\newcommand{\ExBn}{{\bf \ref{ChaptB} -  14} }
\newcommand{\ExBo}{{\bf \ref{ChaptB} -  15} }
\newcommand{\ExBp}{{\bf \ref{ChaptB} -  16} }
\newcommand{\ExBq}{{\bf \ref{ChaptB} -  17} }
\newcommand{\ExBr}{{\bf \ref{ChaptB} -  18} }
\newcommand{\ExBs}{{\bf \ref{ChaptB} -  19} }
\newcommand{\ExBt}{{\bf \ref{ChaptB} -  20} }
\newcommand{\ExBu}{{\bf \ref{ChaptB} -  21} }
\newcommand{\ExBv}{{\bf \ref{ChaptB} -  22} }
\newcommand{\ExBw}{{\bf \ref{ChaptB} -  23} }
\newcommand{\ExBx}{{\bf \ref{ChaptB} -  24} }
\newcommand{\ExBy}{{\bf \ref{ChaptB} -  25} }
\newcommand{\ExBz}{{\bf \ref{ChaptB} -  26} }


\newcommand{\ExBaa}{{\bf \ref{ChaptB} - 27} }
\newcommand{\ExBab}{{\bf \ref{ChaptB} - 28} }
\newcommand{\ExBac}{{\bf \ref{ChaptB} - 29} }
\newcommand{\ExBad}{{\bf \ref{ChaptB} - 30} }
\newcommand{\ExBae}{{\bf \ref{ChaptB} - 31} }
\newcommand{\ExBaf}{{\bf \ref{ChaptB} - 32} }
\newcommand{\ExBag}{{\bf \ref{ChaptB} - 33} }
\newcommand{\ExBah}{{\bf \ref{ChaptB} - 34} }
\newcommand{\ExBai}{{\bf \ref{ChaptB} - 35} }
\newcommand{\ExBaj}{{\bf \ref{ChaptB} - 36} }
\newcommand{\ExBak}{{\bf \ref{ChaptB} - 37} }
\newcommand{\ExBal}{{\bf \ref{ChaptB} - 38} }
\newcommand{\ExBam}{{\bf \ref{ChaptB} - 39} }
\newcommand{\ExBan}{{\bf \ref{ChaptB} - 40} }
\newcommand{\ExBao}{{\bf \ref{ChaptB} - 41} }
\newcommand{\ExBap}{{\bf \ref{ChaptB} - 42} }
\newcommand{\ExBaq}{{\bf \ref{ChaptB} - 43} }
\newcommand{\ExBar}{{\bf \ref{ChaptB} - 44} }
\newcommand{\ExBas}{{\bf \ref{ChaptB} - 45} }
\newcommand{\ExBat}{{\bf \ref{ChaptB} - 46} }
\newcommand{\ExBau}{{\bf \ref{ChaptB} - 47} }
\newcommand{\ExBav}{{\bf \ref{ChaptB} - 48} }
\newcommand{\ExBaw}{{\bf \ref{ChaptB} - 49} }
\newcommand{\ExBax}{{\bf \ref{ChaptB} - 50} }
\newcommand{\ExBay}{{\bf \ref{ChaptB} - 51} }
\newcommand{\ExBaz}{{\bf \ref{ChaptB} - 52} }



                       \section{EXERCISES FOR CHAPTER~\ref{ChaptB}}
                        \label{SectBEX}

\V
\V
\V
\V

\begin{center}
PART I -- THESE EXERCISES REQUIRE ONLY THE FIELD AND/OR ORDER AXIOMS FOR ${\RR}$; DO \underline{NOT} USE ANY RESULTS BASED ON `COMPLETENESS' IN THE SOLUTIONS.
\end{center}

\V
\V
\V
\V

 \noindent \ExBa \label{ExBa} (a) Prove Parts~(d) and (e) of Theorem~\Ref{ThmB10.25}.

\V

        (b) Prove Parts~(f) and (g) of Theorem~\Ref{ThmB10.25} directly from the field axioms, Axioms A0--A6. %Theorem B.1.3.

\V
\V

\noindent \ExBb \label{ExBb} Prove the {\bf Cancellation Law}: Let $a$, $b$ and $c$ be real numbers such that $a \,\,{\neq}\,\, 0$.
    Then there is exactly one real number $x$ such that $a\,x+b \,=\, c$.

\V
\V

\noindent \ExBc NEED NEW EXERCISE %Prove Theorem~\Ref{ThmB10.27} %Theorem B.1.4 on Page 93

\V
\V

\noindent \ExBd Let $f:{\RR}^{2} \,{\rightarrow}\, {\RR}$ be given by the formula $f(x,y) \,=\, |x-y|$ for all $(x,y)$ in ${\RR}^{2}$.

\V

        (a) Prove that the binary operation $f$ commutative but not associative.

\V

        (b) Determine whether the operation $f$ satisfies the `Extended Commutative Law'; see Theorem~\Ref{ThmB10.32}.

\V
\V

\noindent \ExBe Let $D':({\RR}{\setminus}\{0\}){\times}({\RR}{\setminus}\{0\}) \,{\rightarrow}\, {\RR}$ be the restriction to ${\RR}{\setminus}\{0\}$ of the `division' function $D$ described in Definition~\Ref{DefB10.22}. % Definition B.1.2

\V

        (a) Explain why $D'$ is a binary operation but $D$ is not.

\V

        (b) Determine whether the binary operation $D'$ satisfies the Associative Law.

\V
\V

\noindent \ExBf NEED NEW EXERCISE %Prove Equation~\Ref{EqnB.20B} % Equation B.4 on Page 100

\V
\V

\noindent \ExBg NEED NEW EXERCISE %Prove Theorem~\Ref{ThmB10.90} %Thm B.1.15 Page 103

\V
\V

\noindent \ExBh Use the Extended Commutative and Associative Laws for Addition (ThemB10.32) to prove
        \begin{displaymath}
        x_{1} + ((x_{2}+x_{3}) + x_{4}) + x_{5} + ((x_{6} + (x_{7}+x_{8})) + x_{9}) + x_{10} \,=\, 
    x_{10} + x_{7} + x_{9} + x_{6} + x_{1} + x_{5} + x_{4} + x_{3} + x_{2} + x_{8}
        \end{displaymath}
    for all real numbers $x_{j}$, $j \,=\, 1,2,\,{\ldots}\,10$.


\V
\V

\noindent \ExBi NEED NEW EXERCISE

\StartSkip{The following fact is sometimes referred to as `the generalized associative law for addition':

        \h If $x_{1}, x_{2},\,{\ldots}\,x_{m}$ are real numbers and if $k$ is a natural number such that $1\,<\,k\,<\,m$, then
        \begin{displaymath}
        (x_{1}+x_{2}+\,{\ldots}\,+x_{k}) + (x_{k+1}+\,{\ldots}\,+x_{m}) \,=\, x_{1} + x_{2} +\,{\ldots}\,+x_{m}.
        \end{displaymath}

        \underline{Problem} Show that this version of the `generalized associative law' follows from the version stated in Theorem~\Ref{ThmB10.60}.
}%\EndSkip

\V
\V


\noindent \ExBj NEED NEW EXERCISE

\StartSkip{(a) Definition~\Ref{DefB10.30} defines the repeated sum $x_{1}+x_{2}+x_{3}+\,{\ldots}\,+x_{k}$ using the `left-to-right' bias (see Remark~\Ref{RemrkB10.31}~(1)). %% Def B.1.5 Page 93 Rem B.1.6 Page 94
    Give the definition of $x_{1}+x_{2}+x_{3}+\,{\ldots}\,+x_{k}$ that would have resulted had we followed instead the corresponding `right-to-left' bias.

\V

        (b) Compute the repeated sum $1+3-4+5-2-6+9$ two ways:

        \h (i)\, directly from Definition~\Ref{DefB10.30};
        \h (ii) using instead the definitiion you obtained in Part~(a).
}%\EndSkip

\V
\V

\noindent \ExBk NEED NEW EXERCISE

\StartSkip{Let $X \,=\, {\NN}_{10}$, and let $f:X \,{\rightarrow}\, {\RR}$ be given by the rule $f(k) \,=\, k^{3} - 8k^{2} + 12k$ for each $k$ in~$X$.

\V

        (a) Determine $X_{f;-}$, $X_{f;0}$ and $X_{f;+}$.

\V

        (b) By direct calculation (i.e., without using Theorem~\Ref{ThmB20.25A}), %% Theorem B.2.4 on Page 109
    calculate the numbers ${\Sigma}_{X} f$, ${\Sigma}_{X_{f;+}}$ and ${\Sigma}_{X_{f;-}}$.

\V

        (c) Use the results of Part~(b) to verify for this $X$ and $f$ that Equation~\Ref{EqnB.25} %% Eqn B.8 on Page 109
    is correct.
}%\EndSkip

\V
\V

% Redo so ThmB20.45 becomes an exercise
\noindent \ExBl (a) Prove Theorem~\Ref{ThmB20.45} % Thm B.2.10 Page 113

\V

        (b) Find a number $B\,>\,0$ such that if $M\,>\,B$ then ${\displaystyle \frac{1}{M}\,<\,\frac{1}{2}\,<\,M}$.

\V
\V

\noindent \ExBm Prove Corollary~\Ref{CorB20.61} % Cor B.2.13 Page 115

\V
\V

\noindent \ExBn Prove that if $X$ is a finite nonempty set of real numbers then the set $X$ has both a maximum element and a minimum element.
    That is, show that there are real numbers $m$ and $M$ such that

        \h (i)\, $m$ and $M$ are elements of $X$.

        \h (ii) If $x$ is any number in $X$ such that $x \,\,{\neq}\,\, m$ and $x \,\,{\neq}\,\, M$, then  $m\,<\,x\,<\,M$.

\noindent Also, show that the numbers $m$ and $M$ with these properties are unique.

\V
\V

\noindent \ExBo
        (a) Prove Parts (e) and (f) of Theorem~\Ref{ThmB20.150}. 

\V

        (b) Prove Part (g) of Theorem~\Ref{ThmB20.150}.

\V

        (c) Prove Part (h) of Theorem~\Ref{ThmB20.150}.
        
        \V
        \V

\noindent \ExBp Suppose that $f$, $g$ and $h$ are real-valued functions defined on a nonempty set $X$.
    Assume that there are positive numbers $A$, $B$ and $C$ such that $|f(x)|\,\,{\leq}\,\,A$, $|g(x)|\,\,{\leq}\,\,B$ and $|h(x)|\,\,{\geq}\,\,C$ for all $x$ in $X$.
    Show that $ \left|\frac{f(x) - 3g(x)}{h(x)}\right|\,\,{\leq}\,\,\frac{A+3B}{C}$ for all $x$ in $X$.

\V
\V

\noindent  \ExBq
        (a) Prove Parts~(a) and (b) of Theorem~\Ref{ThmB25.50}.

\V

        (b) Prove Part~(c) of Theorem~\Ref{ThmB25.50}.

\V

        (c) Prove Part~(d) of Theorem~\Ref{ThmB25.50}.

\V
\V

\noindent \ExBr NEED NEW EXERCISE

\StartSkip{In Definition~\Ref{DefB10.50} the expression $c^{k}$ is defined for every real number $c$ and every natural number~$k$.
    The next exercise is devoted to extending the notion of `exponent' from natural numbers to all integers.
   To avoid the possibility of division by~$0$, throughout this Exercise the number $c$ is assumed to be nonzero.
    Furthermore, it is understood that the sets ${\NN}$ and ${\bf Z}$  are assumed to be the versions of these sets that are subsets of ${\RR}$,
    as in Section~\Ref{SectB25}.

\V

        {\bf Definition} Let $c$ be a nonzero real number. If $m$ is a {\em nonpositive} integer then
        \begin{displaymath}
        c^{m} \,=\, 
        \left \{
        \begin{array}{cl}
        1 & \mbox{if $m \,=\, 0$} \\
        1/c^{-m} & \mbox{if $m\,<\,0$}
        \end{array}
                        \right.
        \end{displaymath}

\V

        (a) Show that $c^{-m} \,=\, 1/c^{m}$ for all $m$ in ${\bf Z}$.

\V

        (b) Show that $c^{m+n} \,=\, c^{m}{\cdot}c^{n}$ for all $m$ and $n$ in ${\bf Z}$.

\V

        (c) Show that $\left(c^{m}\right)^{n} \,=\, c^{(m{\cdot}n)}$ for all $m$ and $n$ in ${\bf Z}$.
}%\EndSkip

\V
\V



\noindent \ExBs Prove Parts (a), (b), (c) and (d) of Theorem~\Ref{ThmB20.60}  %% Theorem~B.2.12 on Page 115

\V
\V

\noindent \ExBt Prove Parts (e), (f), (g) and (h) of Theorem~\Ref{ThmB20.60}  %% Theorem~B.2.12 on Page 115

\V
\V

\noindent \ExBu Prove Corollary~\Ref{CorB20.61}  %% Corollary~B.2.13 on Page 115

\V
\V


\noindent \ExBv Prove {\bf Bernoulli's Inequality}: If $c$ is a real number such that $c\,>\,-1$, then $(1+c)^{k}\,\,{\geq}\,\,1+kc$ for each natural number~$k$.

\V
\V

\noindent \ExBw Let $p:{\RR} \,{\rightarrow}\, {\RR}$ be a quadratic polynomial function;
    that is, there are numbers $a$, $b$ and $c$ in ${\RR}$ such that $p(t) \,=\, at^{2} + bt + c$ for all $t$ in ${\RR}$.
    To simplify the discussion, assume that $a\,>\,0$.

\V

        (a) Prove that that the function $p$ assumes a minimum value on ${\RR}$.
    More precisely, prove that there exists a unique number $t_{0}$ such that $p(t_{0})\,\,{\leq}\,\,p(t)$ for all $t$ in ${\RR}$.
    (Hint: Does the phrase `Complete the Square' ring a bell?)

\V

        (b) Give simple criteria, in terms of the coefficients $a$, $b$ and $c$ of the polynomial $p$,
    for the minimum value of the polynomial $p$ to be positive, negative or zero.

\V
\V

\noindent \ExBx This exercise has as its goal to obtain an important criterion for a pair of vectors $\Vect{x} \,=\, (x_{1},\,{\ldots}\,x_{n})$ and $\Vect{y} \,=\, (y_{1},\,{\ldots}\,y_{n})$ in ${\RR}^{n}$ to be linearly dependent.
    Since this property is trivially true if either of the vectors is the zero vector,
    throughout most of this exercise we assume that $\Vect{x}$ and $\Vect{y}$ are nonzero vectors.

        Recall from elementary linear algebra that a necessary and sufficient condition for a pair of nonzero vectors in ${\RR}^{n}$ to be linearly dependent is that each of them is a multiple of the other.
    Using the notation established above, this means that the vectors $\Vect{x}$ and $\Vect{y}$ are linearly dependent if, and only if, there is a number $t$ such that $x_{i} \,=\, ty_{i}$ for each $i \,=\, 1,2,\,{\ldots}\,n$.

\V

        (a) Let $\Vect{x}$ and $\Vect{y}$ be as above, and let $p:{\RR} \,{\rightarrow}\, {\RR}$ be the function given by the rule
        \begin{displaymath}
        p(t) \,=\, (x_{1}-ty_{1})^{2} + (x_{2}-ty_{2})^{2} + \,{\ldots}\, + (x_{n}-ty_{n})^{2} \mbox{ for all $t$ in ${\RR}$},
        \end{displaymath}
     Show that the vectors $\Vect{x}$ and $\Vect{y}$ are linearly dependent if, and only if, the equation $p(t) \,=\, 0$ has a real solution.

\V

        (b) Apply the results of Exercise~\ExBt above to prove that for all vectors $\Vect{x}$ and $\Vect{y}$ in ${\RR}^{n}$ as above one has
        \begin{displaymath}
        (x_{1}y_{1} + x_{2}y_{2} + \,{\ldots}\, + x_{n}y_{n})^{2}\,\,{\leq}\,\,\left(x_{1}^{2} + x_{2}^{2} + \,{\ldots}\,+x_{n}^{2}\right){\cdot}\left(y_{1}^{2} + y_{2}^{2} + \,{\ldots}\,+ y_{n}^{2}\right) \h ({\ast})
        \end{displaymath}
    Moreover, one gets equality in~$({\ast})$ if, and only if, the vectors $\Vect{x}$ and $\Vect{y}$ are linearly dependent.

\V

        (c) In the preceding it was always assumed that neither $\Vect{x}$ nor $\Vect{y}$ equals the zero vector.
    Determine the status of the inequality~$({\ast})$, and its relation to linear independence, when one of the vectors equals $\Vect{0}$.

\V

        \underline{Remark} Inequality~$({\ast})$ above is usually called the Cauchy-Schwarz Inequality;
    or, more precisely, the Cauchy-Schwarz Inequality for $n$-tuples. It is due to Cauchy.
    There is also an analogous result, but for definite integrals, which is due to Schwarz.
    It is also known as the Cauchy-Schwarz Inequality, but for integrals. Cauchy was,
    one presumes, unaware of this integral form, since he was already dead when it was published.
    It seems likely that Schwarz's name is added to the original Cauchy result simply to distinguish it from the myriad of other results in mathematics also called `Cauchy's Theorem'.
    Finally, it turns out that Bunyakowski proved the same result as Schwarz, so some authors add his name to the list.
    The order in which the names are listed in texts and research papers seems dependent on the nationality of the writer.

\V
\V

\noindent \ExBy Prove Parts (e), (f), (g) and (h) of Theorem~\Ref{ThmB20.150}  %% Theorem~B.2.28 on Page 125


\V
\V

\noindent \ExBz \label{ExBz} Suppose that $x_{1}$, $x_{2}$,\,{\ldots}\,$x_{k}$ are real numbers which satisfy a string of inequalities of the following form:
        \begin{displaymath}
        x_{1} \,L_{1}\,x_{2} \,L_{2}\,x_{3} \,L_{3}\,\,{\ldots}\,x_{k-1} \,L_{k-1}\,x_{k},
        \end{displaymath}
    where each of the symbols $L_{1}$, $L_{2}$,\,{\ldots}\,$L_{k-1}$ stands for one of the relations `$\,<\,$' or `$\,\,{\leq}\,\,$'.
    Let $i$ and $j$ be indices such that $1\,\,{\leq}\,\,i\,<\,j\,\,{\leq}\,\,k$.

\V

        (a) Show that if at least one of the expressions $L_{i}$, $L_{i+1}$,\,{\ldots}\,$L_{j-1}$ equals `$\,<\,$', then $x_{i}\,<\,x_{j}$.

\V

        (b) Show that a necessary and sufficient condition for the equation $x_{i} \,=\, x_{j}$ to hold is that $x_{i} \,=\, x_{i+1} \,=\, \,{\ldots}\, \,=\, x_{j-1} \,=\, x_{j}$.
    Make it clear which part of your solution corresponds to the `necessary' portion of the statement, and which corresponds to the `sufficient' portion.

\V
\V

\noindent \ExBaa Let $X$ be the set of all real numbers $x$ such that $x^{2}\,\,{\leq}\,\,2$.
    Show that $X$ is a convex set.

\V
\V

\noindent \ExBab Let $X$ be a nonempty set of real numbers, and let ${\cal F}_{X}$ be a nonempty family of convex subsets $Y$ of ${\RR}$ such that $X \,{\subseteq}\, Y$.
    Then the intersection of the family ${\cal F}_{X}$ is a convex subset of ${\RR}$ which has $X$ as a subset.

\V
\V

\noindent \ExBac NOTE: The purpose of this exercise is to show that there are several equivalent ways to formulate a suitable notion of `completeness' for ${\RR}$.
    The choice of which statement to take as the `official' Completeness Axiom is largely a matter of taste and pedagogy.

\V

        Consider the following statements:

\V

        (1) If $[a_{1},b_{1}]$, $[a_{2},b_{2}]$,\,{\ldots}\, is a bisection sequence in ${\RR}$,
    then the intersection ${\bigcap}_{k=1}^{{\infty}} [a_{k},b_{k}]$ is a singleton set.

\V

        (2) If $X$ is a nonempty subset of ${\RR}$ which is bounded above, and if $Y$ is the set of all upper bounds of $X$, then $Y$ has a minimum element.

\V

        (3) If $X$ is a nonempty subset of ${\RR}$ which is bounded below, and if $Y$ is the set of all lower bounds of $X$, then $Y$ has a maximum element.

\V

        (4) Every convex subset of ${\RR}$ with at least two elements is an interval.

\V

        (5) For every closed interval $[a,b]$ in ${\RR}$ the following is true: 
    if ${\cal F}$ is a family of open intervals in ${\RR}$ whose union contains $[a,b]$ as a subset,
    then there is a finite subfamily ${\cal F}'$ of ${\cal F}$ whose union also contains $[a,b]$ as a subset.

\V

        The simplest way to label the various parts of this exercise is with {\em pairs} of integers $i$ and $j$, with $1\,\,{\leq}\,\,i,j\,\,{\leq}\,\,5$ and $i \,\,{\neq}\,\, j$:

        \underline{Problem $(i,j)$} Show that Statement $i$ implies Statement~$j$.


\V
\V

\noindent \ExBad Let $(x_{1},x_{2},\,{\ldots}\,x_{k})$ be a $k$-tuple of real numbers in an interval $[a,b]$ in ${\RR}$, with $k\,\,{\geq}\,\,2$.
    Prove that there exist indices $i$ and $j$, with $1\,\,{\leq}\,\,i,j\,\,{\leq}\,\,k$ and $i \,\,{\neq}\,\, j$,
    such that ${\displaystyle |x_{i}-x_{j}|\,\,{\leq}\,\,\frac{b-a}{k-1}}$.

\V
\V
\V

\newpage

\begin{center}
PART II -- EXERCISES WHICH MAY INVOLVE ALL THE AXIOMS, INCLUDING `COMPLETENESS'
\end{center}

\V
\V
\V
\V


\noindent \ExBae \underline{Prove or Disprove} A necessary and sufficient condition for two real numbers $A$ and $B$ to be equal is that $|A-B|\,<\,1/k$ for all $k$ in ${\NN}$.

\V
\V

\noindent \ExBaf Prove Theorem~\Ref{ThmB30.150}. %% Thm B.4.21 p.147

\V
\V

\noindent \ExBag Prove Parts (a), (b) and (c) of Theorem~\Ref{ThmB30.200}. %% Thm B.4.27

\V
\V

\noindent \ExBah Prove Parts (d), (e) and (f) of Theorem~\Ref{ThmB30.200}. %% Thm B.4.27

\V
\V


\noindent \ExBai Let $A$ be a nonempty set of real numbers.

\V


        (a) If $c$ is a real number such that ${\inf}\,A\,<\,c$, then there exists a real number $x$ in $A$ such that ${\inf}\,A\,\,{\leq}\,\,x\,<\,c$;
    equality is possible if, and only if, $A$ has a minimum element; namely, $x \,=\, {\inf}\,A$.

\V

        (b) If $d$ is a real number such that $d\,<\,{\sup}\,A$, then there exists a real number $y$ in $A$ such that $d\,<\,y\,\,{\leq}\,\,{\sup}\,A$;
    equality is possible if, and only if, $A$ has a maximum element; namely $y \,=\, {\sup}\,A$.

\V
\V

\noindent \ExBaj Suppose that $A$ and $B$ are nonempty subsets of ${\RR}$, and let $C \,{\subseteq}\, {\RR}$ be defined by the rule
        \begin{displaymath}
        C \,=\, \{c \mbox{ in } {\RR}: c \,=\, a+b \mbox{ for some choice of $a$ in $A$ and $b$ in $B$}\}.
        \end{displaymath}

        \underline{Problem} (a) Show that if either of the sets $A$ or $B$ is unbounded above, then so is $C$, so ${\sup}\,C \,=\, +{\infty}$.
    Likewise, if either of the sets $A$ or $B$ is unbounded below, then so is $C$, and ${\inf}\,C \,=\, -{\infty}$.

\V

        (b) Show that if $A$ and $B$ are both bounded above, then so is $C$, and ${\sup}\,C \,=\, {\sup}\,A \,+\, {\sup}\,B$.
    Likewise, if both $A$ and $B$ are bounded below, then so is $C$, and ${\inf}\,C \,=\, {\inf}\,A \,+\, {\inf}\,B$.

\V
\V

\noindent \ExBak For each of the sets given below, determine the supremum and infimum of the set.
    Also, determine whether the set has a maximum or minimum element.

        \underline{Note} Resist any temptation to use `limits'; they are not introduced until Chapter~\Ref{ChaptC}.

\V

        (a) $X \,=\, \{x: x \,=\, 1-(-1)^{k}\frac{1}{k} \mbox{ for all $k$ in ${\NN}$}\}$

\V

        (b) $Y \,=\, \{x: x \,=\, (-k)^{k} \mbox{ for $k$ in {\bf Z}}\}$

\V

        (c) $Z$ is the set of all the digits which appear in the decimal expansion of the number $1/7$.
    (You may use the usual rules for decimals here, even though they won't be proved until later.)

\V
\V

\noindent \ExBal {\bf Definition} Let $A$ be a nonempty subset of ${\RR}$. Define subsets ${\cal L}_{A}$, ${\cal R}_{A}$ and ${\cal I}_{A}$ of ${\RR}$ by the rules
        \begin{displaymath}
        {\cal L}_{A} \,=\, \{x: x\,\,{\leq}\,\,a \mbox{ for some element $a$ of $A$}\}; \h
        {\cal R}_{A} \,=\, \{x: x\,\,{\geq}\,\,a \mbox{ for some element $a$ of $A$}\}; \h
        {\cal I}_{A} \,=\, {\cal L}_{A}\,{\cap}\,{\cal R}_{A}
        \end{displaymath}

\V

        (a) Show that if $A$ is a nonempty subset of ${\RR}$, then ${\cal I}_{A}$ is the intersection of the family of all convex subsets of ${\RR}$ which contain $A$ as a subset.

\V

        (b) \underline{Prove or Disprove} If $X$ and $Y$ are nonempty subsets of ${\RR}$, then ${\cal L}_{X\,{\cup}\,Y} \,=\, {\cal L}_{X}\,{\cup}\,{\cal L}_{Y}$.

\V

        (c) \underline{Prove or Disprove} If $X$ and $Y$ are nonempty subsets of ${\RR}$, then ${\cal R}_{X\,{\cup}\,Y} \,=\, {\cal R}_{X}\,{\cup}\,{\cal R}_{Y}$.


\V
\V

\StartSkip{
\noindent \ExBam In high-school math courses one defines the {\bf greatest integer function} $G:{\RR} \,{\rightarrow}\, {\bf Z}$ by the rule
        \begin{displaymath}
        G(x) \mbox{ is the largest integer which is not larger than $x$}.
        \end{displaymath}

        \underline{Problem} Show that this function is actually defined on all of ${\RR}$, and that it satisfies the conditions
        \begin{displaymath}
        (i)\, G(x) \,=\, x \mbox{ if $x{\in}{\bf Z}$}; \h
        (ii) x-G(x) \,=\, (0,1) \mbox{ if $x\not {\in} {\bf Z}$}.
        \end{displaymath}
        NOTE: The usual notation for the quantity $G(x)$ described here is $[x]$.
} %\EndSkip
