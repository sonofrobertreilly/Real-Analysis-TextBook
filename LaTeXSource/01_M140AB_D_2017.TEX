% M140AB_D.TeX  Notes for `Single-Variable Analysis': 

%
% Revised: 01/20/2017 Encoding: Western Ascii
%

                  \chapter{Continuity -- The Basic Theory}
                  \label{ChaptD}

\VV


        \underline{Quotes for Chapter~\Ref{ChaptD}}\IndB{chapter quotes}{for Chapter~\Ref{ChaptD} (Continuity -- The Basic Theory}:

\V

\begin{quotation}
{\footnotesize
        (1) `How fleeting are all human passions compared to the massive continuity of ducks.'

        (Lord Peter Wimsey to Miss Harriet Vane, in the novel {\em Gaudy Nights}, by Dorothy L. Sayers)

\V

}%EndFootNoteSize
\end{quotation}


            \small{\bf Introduction}

\V

        In the standard textbooks of elementary calculus one learns about the concepts of `limits' and `continuity'
    in the context of a real-valued function defined on a fairly simple set in ${\RR}$ -- usually an interval, or at worst the union of disjoint intervals.
    Generally speaking, the treatment of this concept in such texts is neither rigorous nor carried out in depth. For example, such texts do not normally consider
    `complicated' subsets of~${\RR}$, or carefully develop the structure of the real number system, or discuss in depth the convergence of real sequences.
    That is, calculus texts typically omit most of the material found in Chapters~\Ref{ChaptA}, \Ref{ChaptB} and~\Ref{ChaptC} of {\ThisText}.

        In texts on real analysis, such as {\ThisText}, one major goal is to revisit elementary calculus, but in a much more rigorous manner.
    Another major goal, however, is to prepare the reader for even more advanced modern analysis which involves `spaces', for example, `metric spaces' or `topological spaces',
    that are much more general than ${\RR}$ or even~${\RR}^{n}$. The structure of the proofs given here is often the same as in the more general contexts,
    so studying the theory for real-valued functions defined on subsets of~${\RR}$ is good preparation, in a familiar context, for the more advanced theories.

        In {\ThisText} the basic theory of `limits' and `continuity' that is presented in Chapters~\Ref{ChaptC} above and the present chapter provides ample background
    to cover the rigorous treatment of  elementary calculus which is carried out in Chapters~\Ref{ChaptE} and~\Ref{ChaptH} below.

        In addition to the basic theory on limits and continuity referred to above, Chapter~\Ref{ChaptF} expands on that theory.
        Any reader who prefers to finish the full theory of limits and continuity before starting the study of calculus can do so:
    simply procede directly from the present chapter to Chapter~\Ref{ChaptF}, and return to the calculus chapters later.
    The references to calculus in Chapter~\Ref{ChaptF} usually involve facts which are familiar from elementary calculus and can be temporarily accepted `on faith'.

\VV

                \section{{\bf Continuity for Real-Valued Functions Defined in ${\RR}$}}
                \label{SectD20}\IndB{ZZ Sections}{\Ref{SectD20} Continuity for Real-Valued Functions Defined in ${\RR}$}


\VV

        Throughout this chapter all the functions under consideration are real-valued
with domains being nonempty subsets of~${\RR}$, unless stated explicitly otherwise.


        There are two main approaches to the concept of `continuity' that are commonly used for the theory of
    real-valued functions defined on a subset of~${\RR}$. The next result says that it doesn't matter which approach one adopts as the `official' definition.

\V

            \subsection{\small{\bf Theorem}}
            \label{ThmD20.20}

        Let $f:X \,{\rightarrow}\, {\RR}$ be a real-valued function defined on a (nonempty) subset $X$ of ${\RR}$.
    Let $c$ be a point of the set $X$. Then the following statements are equivalent:

\V

        \h (i) For every sequence ${\xi} \,=\, (x_{1},x_{2},\,{\ldots}\,)$ of numbers in $X$ such that $\lim_{k \,{\rightarrow}\, {\infty}} x_{k} \,=\, c$,
    one has $\lim_{k \,{\rightarrow}\, {\infty}} f(x_{k}) \,=\, f(c)$.

\V

        \h (ii)\, For every number ${\varepsilon}\,>\,0$ there exists a number ${\delta}\,>\,0$ such that if $d$ in $X$ satisfies $|d-c|\,<\,{\delta}$, then $|f(d)-f(c)|\,<\,{\varepsilon}$.

\V

        \underline{Proof} Suppose that Statement~(ii) holds. Let ${\xi} \,=\, (x_{1},x_{2},\,{\ldots}\,)$
    be a sequence of numbers in $X$ such that $\lim_{k \,{\rightarrow}\, {\infty}} x_{k} \,=\, c$, and let ${\varepsilon}\,>\,0$ be given.
    Statement~(i) guarantees that there exists ${\delta}\,>\,0$ so that if $d{\in}X$ and $|d-c|\,<\,{\delta}$, then $|f(d)-f(c)|\,<\,{\varepsilon}$.
    For that ${\delta}$ the `convergence' hypothesis for the sequence ${\xi}$ impies that there exists a number $N$ in ${\NN}$ such that if $k\,\,{\geq}\,\,N$, then $|x_{k}-c|\,<\,{\delta}$.
    Combining these facts, one sees that for each ${\varepsilon}\,>\,0$ there exists a number $N$ in ${\NN}$ such that if $k\,\,{\geq}\,\,N$,
    then $|f(x_{k})-f(c)|\,<\,{\varepsilon}$. That is, $\lim_{k \,{\rightarrow}\, {\infty}} f(x_{k}) \,=\, f(c)$. It follows that Statement~(i) holds.

\V

        To show that Statement~(i) implies Statement~(ii), it suffices to show the truth of the contrapositive assertion;
    that is, to prove that Statement~(ii) being false implies that Statement~(i) is also false.

        Thus, suppose that Statement~(ii) does not hold for the given $f$ and $c$.
    This implies that there exists some ${\varepsilon}_{0}\,>\,0$ so that for every ${\delta}\,>\,0$
    there exists $d$ in $X$ such that $|d-c|\,<\,{\delta}$ but $|f(d)-f(c)|\,\,{\geq}\,\,{\varepsilon}_{0}$.
    Apply this successively for ${\delta}$ of the form $1/k$ with $k$ in~${\NN}$ to see that for each $k$ in ${\NN}$ there exists $x_{k}$ in $X$
    such that $|x_{k}-c|\,<\,1/k$ but $|f(x_{k})-f(c)|\,\,{\geq}\,\,{\varepsilon}_{0}$.
    The first of these properties implies that $c \,=\, \lim_{k \,{\rightarrow}\, {\infty}} x_{k}$, while the second implies that the sequence
    $(f(x_{1}), f(x_{2}),\,{\ldots}\,f(x_{k}),\,{\ldots}\,)$ does {\em not} converge to~$f(c)$. In particular, Statement~(i) fails to hold.

\VV

            \subsection{\small{\bf Definition}}
            \label{DefD20.30}%\IndA{continuity}

\V

        Let $f:X \,{\rightarrow}\, {\RR}$ be a real-valued function whose domain is a (nonempty) subset $X$ of ${\RR}$. Let $c$ be a point of $X$.

\V

        (1) One says that {\bf $f$ is continuous at $\Bfm{c}$}\IndB{continuity}{continuity at a point} if either Statement~(i) or Statement~(ii) of the preceding theorem holds;
    of course it then follows that {\em both} statements hold.
    One then refers to Statement~(i) as the {\bf sequential characterization of continuity}\IndBD{continuity}{sequential characterization of continuity},
    and Statement~(ii) as the {\bf $\Bfm{{\varepsilon}\,{\delta}}$ characterization of continuity}\IndBD{continuity}{${\varepsilon}\,{\delta}$ characterization of continuity}.

        If $f:X \,{\rightarrow}\, {\RR}$ is {\em not} continuous at $c$, then one says that {\bf $\Bfm{f}$ is discontinuous at $\Bfm{c}$},
    and that {\bf the point $\Bfm{c}$ is a discontinuity of $f$}\IndB{continuity}{discontinuity at a point}.

\V

        (2) Let $S$ be a nonempty subset of the set $X$.
    One says that {\bf $\Bfm{f}$ is continuous on the subset $\Bfm{S}$}\IndB{continuity}{continuity on a subset}
    provided $f:X \,{\rightarrow}\, {\RR}$ is continuous at $c$ (in the sense of Part~(1) of this definition) for each number $c$ in $S$.

\V

        (3) The function $f:X \,{\rightarrow}\, {\RR}$ is said to be a {\bf continuous function}\IndB{continuity}{continuous function}
    if it is continuous on $X$ (in the sense of Part~(b)); that is, if $f$ is continuous at each point of its domain~$X$.

\VV

            \subsection{\small{\bf Remarks}}
            \label{RemrkD20.50}

\hspace*{\parindent}(1) Many texts in real analysis use the `${\varepsilon}\,{\delta}$' characterization of continuity as their `official' definition.
    For such texts the `sequential' characterization becomes a theorem to be proved.
    Likewise, other such texts prefer to use the `sequential' characterization as their `official' definition of continuity.
    In those texts the `${\varepsilon}\,{\delta}$' characterization then becomes a theorem. In any event, both types of texts must prove Theorem~\Ref{ThmD20.20}. 
    The approach taken in {\ThisText} avoids the problem of choosing which of these characterizations to take as the `official' definition.

\V

        (2) Note also that to say that `$f:X \,{\rightarrow}\, {\RR}$ has a discontinuity at a number $c$' requires that $c$ be in~$X$.
    This usage, which appears to be the norm in real analysis texts such as {\ThisText}, is {\em not} what is taught in elementary calculus.
    Indeed, a typical problem in a freshman calculus course might be to `find the discontinuites of the function $f(x) \,=\, 1/\sqrt{x^{2}-1}$',
    and the `correct' answer would be `$x \,=\, -1$ and $x \,=\, 1$'; correct, that is, in the context of elementary calculus.
    In real analysis, in contrast, the domain of this function consists of all $x$ in ${\RR}$ such that $x\,<\,-1$ or $x\,>\,1$,
    so it is continuous at each point of its domain, and thus, in accordance with Definition~\Ref{DefD20.30}, has no discontinuities.

\V

        (3) Part~(2) of the preceding definition appears to be innocuous, but it can lead to confusion.
    The problem is that the phrase `$f$ is continuous on $S$' is sometimes interpreted as meaning (using the `sequential characterization'):

\VA

    \h `{\em For every point $c$ in $S$ and for every sequence ${\xi} \,=\, (x_{1},x_{2},\,{\ldots}\,)$ of points {\em in $S$} converging to $c$,
    the corresponding sequence of values $(f(x_{1}),f(x_{2}),\,{\ldots}\,)$ converges to $f(c)$.}'

\VA

\hspace*{\parindent}This is equivalent to saying that the restriction $g \,=\, f|_{S}$ of $f$ to $S$, whose domain is the set $S$, is continuous at each point of~$S$.
    This is {\em not} the intended meaning of Part~(2) of the definition. For example, suppose that $f:X \,{\rightarrow}\, {\RR}$
    is a function with domain $X$ such that $f$ is not contiuous at a particular point~$c$ of~$X$. Let $S$ be the singleton set $\{c\}$.
    Then the function $f$ is {\em not} continuous on the subset~$S$, in the sense of Part~(2) of the definition,
    since it fails to be continuous at a point of~$S$, namely~$c$. However, it is easy to see that the restriction $f|_{S}$
    {\em is} continuous on~$S$, since constant functions are continuous; see Example~\Ref{ExampD20.53}~(1) below.

\V

        (4) It is useful, for future reference, to reformulate the condition for $f:X \,{\rightarrow}\,{\RR}$ to be continuous on~$X$,
    stated in Part~(2) of the preceding definition with $S \,=\, X$, the full domain of~$f$, in the following alternate form:

\VA

        \h {\em `For every $c$ in $X$ and for every ${\varepsilon}\,>\,0$ there exists ${\delta}\,>\,0$
    such that for every $d$ in $X$ such that $|d-c|\,<\,{\delta}$, one has $|f(d)-f(c)|\,<\,{\varepsilon}$'}.

\VA

        This alternate formulation does not refer to a previously defined concept of continuity at a single point of~$X$.
    In that sense, this formulation is closer to the original definitions of continuity given by Bolzano and Cauchy,
    which focused on the continuity of a function on a full interval; see the following \Note.

\VV

\begin{quotation}
{\footnotesize \underline{\Note}\IndB{\notes}{on the approaches to continuity of Bolzano and Cauchy} (on the approaches to continuity of Bolzano and Cauchy)

\V

        The modern concept of `continuity' of a function $y \,=\, f(x)$, namely that small changes in the input $x$ of the function cause small changes in the corresponding output~$y$,
    took many years to evolve. Historians of mathematics tell us that the earliest treatments of continuity
    that agree in essence with the modern definitions were carried out by Bolzano and Cauchy in publications of $1817$ and~$1821$, respectively.
    Both definitions focus on real-valued functions with domain some intervals in~${\RR}$, instead of more general sets of numbers.

        Bolzano's definition can be translated as follows:

\VA

       \h `{\em According to a correct definition, the expression that a function {\em f\,x}
    varies according to the law of continuity for all values of $x$ inside or outside certain limits means just that:
    if $x$ is some such value, the difference $f(x+{\omega})-f\,x$ can be made smaller than any given quantity provided
    ${\omega}$ can be taken as small as we please.}' (Translation given in `{\em Bolzano's Philosophy and the Emergence of Modern Mathematics'} by P.~Rusnock.)

\VA

\noindent {\bf Remark} Bolzano's definition is essentaily the same as the ${\varepsilon}\,{\delta}$ characterization described in Remark~\Ref{RemrkD20.50}~(4) above;
    in particular,  the `given quantity' corresponds to our~${\varepsilon}$, and the quantity ${\omega}$ is our ${\delta}$; `as small as we please' should then be interpreted to mean `sufficiently small'. Of course when Bolzano says
    `smaller than any given quantity' and `as small as we please', he is referring to the absolute value of the quantities involved.
    The `absolute value notation' $|\,|$ now taught in arithmetic did not become common in mathematics until after~$1840$.


\VA

        The Cauchy definition expresses the same idea:

\VA

       \h `{\em Let $f(x)$ be a function of the variable $x$, and suppose that for each value of $x$ between two given limits,
    the function always takes a unique finite value. If, beginning with a value of $x$ contained between these limits,
    we add to the variable $x$ an infinitely small increment $a$, the function itself is incremented by the difference $f(x+a)- f(x)$,
    which depends both on the new variable $a$ and on the value of $x$. Given this,
    the function $f(x)$ is a continuous function of $x$ between the assigned limits if,
    for each value of $x$ between these limits, the numerical value of the difference $f(x+a) - f(x)$
    decreases indefinitely with the numerical value of $a$. In other words, the function $f (x)$ is continuous with respect to $x$ between the given limits if, 
    between these limits, an infinitely small increment in the variable always produces an infinitely small increment in the function itself.}'
    (Translation given in {\em Cauchy's Cours d'analyse: An Annotated Translation} by Bradley and Sandifer.)

\VA

\noindent \underline{Remark} By `infinitesimal quantity' Cauchy means `a quantity which has $0$ as its limit'.
    This evokes a sense of a quantity `approaching' a value and thus a sense of `time' and  `motion',
    physical concepts which lie outside the purely analytical treatment of analysis which Cauchy claimed to present.
    Indeed, Cauchy even uses the phrase `successive values' of such a quantity in this context, which again suggests the idea of `time'.
    However, Cauchy goes on to provide an explicit example of what he means by an `infinitesimal quantity':
    $a \,=\, {\displaystyle \frac{1}{4}, \frac{1}{3}, \frac{1}{5}, \frac{1}{6},\,{\ldots}\,}$.
    This example shows that, whatever Cauchy meant by `infinitesimal quantity', he included all sequences of real numbers which converge to~$0$.
    The Cauchy approach to `continuity' thus corresponds essentially to the standard `sequence characterization' of continuity given above.
    Indeed, the sequence $(x_{1}-c, x_{2}-c,\,{\ldots}\,x_{k}-c, \,{\ldots}\,)$ corresponds to Cauchy's infinitesimal~$a$: note that $c + (x_{k}-c) \,=\, x_{k}$.

        Cauchy's `numerical value' is the same as the modern `absolute value'.

\V

        Note that both Cauchy and Bolzano use the word `limit' in the sense of a quantity which bounds the variable~$x$,
    not as a limit as in `limit of a sequence'. For example, Bolzano allows his $x$ to vary `inside or outside certain limits'.
    This means restrictions such as $a\,<\,x\,<\,b$ or $-{\infty}\,<\,x\,<\,b$
 and so on. Similarly, Cauchy `limits' his variable $x$ to lie in some sort of interval.

        In addition, each author's definition could be translated in more modern language as
    `$f$ is continuous inside an interval provided it is continuous at each $x$ in the interval'.
    However, at this time neither author assigns a name to what we now refer to as `continuity at a single point~$c$';
    perhaps both would have found it silly to even consider a function if it were continuous at, say, only a single point.
    Indeed, later on Cauchy does define `continuity at a single point $x$', but for him this means `continuity on a full interval containing~$x$'.
    In particular, at this early stage both Bolzano and Cauchy would describe `continuity of $f:X \,{\rightarrow}\, {\RR}$ on a subset $S$ of~$X$'
    along the lines of Remark~\Ref{RemrkD20.50}~(4) above. Of course neither author would have considered functions defined on arbitrary nonempty subsets of~${\RR}$,
    as is done in {\ThisText}; that is a much more recent viewpoint.
}%EndFootNoteSize
\end{quotation}

\VV

            \subsection{\small{\bf Examples}}
            \label{ExampD20.53}

\V

\hspace*{\parindent}(1) Let $X$ be a nonempty set of real numbers.

\VA

        \h (i)\, If $f:X \,{\rightarrow}\,{\RR}$ is a constant function with domain~$X$, then $f$ is continuous on~$X$.

        \h (ii) If $g:X \,{\rightarrow}\, X$ is the identity function with domain~$X$, then $g$ is continuous on~$X$.

\VA

\noindent Indeed, suppose ${\xi} \,=\, (x_{1}, x_{2},\,{\ldots}\,x_{k},\,{\ldots}\,)$
    is a sequence of points in~$X$ which converges to a point~$c$ in~$X$. Let $b$ be the constant value of the function $f$,
    so that $f(x_{k}) \,=\, b \,=\, f(c)$ for each index~$k$. Then clearly $\lim_{k \,{\rightarrow}\, {\infty}} f(x_{k}) \,=\, b \,=\, f(c)$.
    Likewise, since $g(x) \,=\, x$ for each $x$ in~$X$, it follows that $g(x_{k}) \,=\, x_{k}$,
    and thus $\lim_{k \,{\rightarrow}\, {\infty}} g(x_{k}) \,=\, \lim_{k \,{\rightarrow}\, {\infty}} x_{k} \,=\, c \,=\, g(c)$.
    The claimed continuity follows.

\V

        (2) Let $f:{\RR} \,{\rightarrow}\, {\RR}$ be the Dirichlet function which was discussed in Example~\Ref{ExampA30.29}.
    Thus, if $x{\in}{\RR}$ then $f(x) \,=\, 0$ if $x$ is irrational, while $f(x) \,=\, 1$ if $x$ is rational.

        It is easy to see that $f$ is discontinuous at every point of ${\RR}$.
    Indeed, suppose that $c$ is an irrational number, and let ${\xi} \,=\, (x_{1},x_{2},\,{\ldots}\,)$ be a sequence of {\em rational} numbers such that $c \,=\, \lim_{k \,{\rightarrow}\, {\infty}} x_{k}$.
    Then $f(x_{k}) \,=\, 1$ for all $k$ in ${\NN}$, hence $\lim_{k \,{\rightarrow}\, {\infty}} f(x_{k}) \,=\, 1$.
    However, $f(c) \,=\, 0$, so it is {\em not} the case that $\lim_{k \,{\rightarrow}\, {\infty}} f(x_{k}) \,=\, f(c)$;
    thus, $f$ is discontinuous at $c$.

        Similarly, if $c$ is a rational number, then let ${\zeta} \,=\, (z_{1},z_{2},\,{\ldots}\,)$
    be a sequence of {\em irrational} numbers which converges to $c$.
    One then notes that $f(z_{k}) \,=\, 0$ for all $k$, hence $\lim_{k \,{\rightarrow}\, {\infty}} f(z_{k}) \,=\, 0 \,\,{\neq}\,\, f(c)$
     since $f(c) \,=\, 1$ when $c$ is rational.

\V

        (3) Let $g:{\RR} \,{\rightarrow}\, {\RR}$ be given by the rule $g(x) \,=\, x\,f(x)$,
    where $f$ is the Dirichlet function studied in the preceding example. It is a simple exercise to show that if $c \,=\, 0$ then $g$ is continuous at~$c$,
    but if $c \,\,{\neq}\,\, 0$ then $g$ is discontinuous at~$c$. That is, with the modern definition of `continuity'
    it is possible for a function defined on an interval to be continuous at just a single point of its domain. %%EXERCISE

\V

        (4) Let $f:[0,+{\infty}) \,{\rightarrow}\, {\RR}$ be given by the formula $f(x) \,=\, \sqrt{x}$ for all $x$ in the domain $[0,+{\infty})$ of~$f$.

        \underline{Claim 1}\,If $c\,>\,0$, then $f$ is continuous at~$c$.

        \underline{Proof of Claim 1}\,Note that, by the usual algebraic trick, one has
        \begin{displaymath}
        |f(d) - f(c)|
    \,=\,
        |\sqrt{d} - \sqrt{c}|
    \,=\,
        \left|(\sqrt{d} - \sqrt{c})\,
        \left(\frac{\sqrt{d} + \sqrt{c}}{\sqrt{d} + \sqrt{c}}\right)\right|
     \,=\, 
        \frac{|d-c|}{\sqrt{d} + \sqrt{c}}\,\,{\leq}\,\,\frac{|d-c|}{\sqrt{c}} \mbox{ for all $d\,\,{\geq}\,\,0$}
        \end{displaymath}
    Now let ${\varepsilon}\,>\,0$ be given, and let ${\delta} \,=\, {\varepsilon}\,\sqrt{c}$.
    Then it is clear that if $d{\in}X$ and $|d-c|\,<\,{\delta}$, then $|f(d)-f(c)|\,<\,{\varepsilon}$, as required.

        \underline{Claim 2}\,The function $f$ is also continuous at $c \,=\, 0$.

        \underline{Proof of Claim 2}\,Let ${\xi} \,=\, (x_{1}, x_{2},\,{\ldots}\,x_{k},\,{\ldots}\,)$ be any sequence in $X$ converging to~$0$,
    and consider the corresponding sequence of values, namely ${\rho} \,=\, (r_{1}, r_{2},\,{\ldots}\,r_{k},\,{\ldots}\,)$,
    where $r_{k} \,=\,f(x_{k}) \,=\, \sqrt{x_{k}}$ for each index~$k$. Since the (convergent) sequence ${\xi}$ is bounded, so is the sequence ${\rho}$.
    Let ${\sigma} \,=\, (s_{1}, s_{2},\,{\ldots}\,s_{m},\,{\ldots}\,)$ be any convergent subsequence of ${\rho}$, and let its limit be~$L$.
    Then the sequence $(s_{1}^{2}, s_{2}^{2},\,{\ldots}\,s_{m}^{2},\,{\ldots}\,)$ is a subsequence of the original sequence ${\xi}$,
    hence by the Product Rule for Limits of Sequences, together with the fact that ${\xi}$ converges to~$0$,
    one has $L^{2} \,=\, 0^{2} \,=\, 0$, so $L \,=\, 0$. Now Part~(b) of Theorem~\Ref{ThmC30.20} applies to show that ${\rho}$ converges to~$0$;
    that is, $\lim_{k \,{\rightarrow}\, {\infty}} f(x_{k}) \,=\, 0 \,=\, f(0)$. Thus $f$ is also continuous at~$0$, as claimed.

\V
        (5) Let $P_{0} \,=\, (x_{0}, y_{0})$, $P_{1} \,=\, (x_{1},y_{1})$,\,{\ldots}\,$P_{n} \,=\, (x_{n},y_{n})$
    be $n+1$ points in ${\RR}^{2}$ such that $x_{0}\,<\,x_{1}\,<\,\,{\ldots}\,\,<\,x_{n-1}\,<\,x_{n}$. For convenience set $a \,=\, x_{0}$ and $b \,=\, x_{n}$, and let $I$ denote the closed interval $[a,b]$.
    Now let $h:I \,{\rightarrow}\, {\RR}$ be the piecewise-linear interpolation through these $n+1$ points; see Example~\Ref{ExampA30.29}. Then $h$ is continuous at every point of~$I$.
    More precisely, for each ${\varepsilon}\,>\,0$ there exists ${\delta}\,>\,0$ such that if $c,d{\in}I$ are such that $|d-c|\,<\,{\delta}$, then $|h(d)-h(c)|\,<\,{\varepsilon}$.

        More precisely, for each $j \,=\, 1,2,\,{\ldots}\,n$ let $s_{j} \,=\, (y_{j}-y_{j-1})/(x_{j}-x_{j-1})$;
    in geometric terms, $s_{j}$ is the slope of the line segment in the plane ${\RR}^{2}$ joining the points $P_{j-1}$ and~$P_{j}$.
    Let $M$ be any positive number such that $M\,\,{\geq}\,\,\max\,\{|s_{1}|, |s_{2}|,\,{\ldots}\,|s_{n}|\}$.
    Then for each pair of numbers $c$ and $d$ in $I$ one has $|h(d)-h(c)|\,\,{\leq}\,\,M\,|d-c|$.

        To see this, note first that if $x$ is a number such that $x_{j-1}\,\,{\leq}\,\,x\,\,{\leq}\,\,x_{j}$ for some index~$j$,
    then, by the formula for `linear interpolation', one has
        \begin{displaymath}
        h(x) \,=\, y_{j-1} + \left(\frac{y_{j}-y_{j-1}}{x_{j}-x_{j-1}}\right)\,(x-x_{j-1})
     \,=\, 
        y_{j-1} + s_{j}\,(x - x_{j-1}).
        \end{displaymath}
    Without loss of generality, assume that $c\,<\,d$. Then there exist indices $i$ and $k$, with $i\,\,{\leq}\,\,k$,
    such that $x_{i-1}\,\,{\leq}\,\,c\,\,{\leq}\,\,x_{i}$ and $x_{k-1}\,\,{\leq}\,\,d\,\,{\leq}\,\,x_{k}$.
    Assume that $i\,<\,k$; the case $i \,=\, k$ is even easier, and is left as an exercise. Using the formula for $h$ obtained above, together with the old `Add-and-Subtract' trick, one gets
        \begin{displaymath}
        h(d)-h(c)
         \,=\, 
        (h(d) - h(x_{k-1})) + (h(x_{k-1})-h(x_{k-2})) + \,{\ldots}\, + (h(x_{i}) - h(c))
     \,=\, 
        \end{displaymath}
        \begin{displaymath}
        s_{k}\,(d-x_{k-1}) + s_{k-1}\,(x_{k-1} - x_{k-2}) + \,{\ldots}\, + s_{i}\,(x_{i}-c).
        \end{displaymath}
    Now use the Triangle Inequality, together with the fact that the coefficients $d-x_{k-1}$, $x_{k-1}-x_{k-2}$,\,{\ldots}\,$x_{i}-c$ of $s_{k}$, $s_{k-1}$,\,{\ldots}\,$s_{i}$ are nonnegative, to get
        \begin{displaymath}
        |h(d)-h(c)|\,\,{\leq}\,\,|s_{k}|\,(d-x_{k-1}) + |s_{k-1}|\,(x_{k-1}-x_{k-2}) + \,{\ldots}\,+ |s_{i}|\,(x_{i}-c)
    \,\,{\leq}\,\,
        \end{displaymath}
        \begin{displaymath}
        M\,((d-x_{k-1}) + (x_{k-1}-x_{k-2}) + \,{\ldots}\, + (x_{i}-c))
     \,=\, 
        M\,|d-c|\,<\,M\,\left(\frac{{\varepsilon}}{M}\right) \,=\, {\varepsilon}.
        \end{displaymath}

        Finally, suppose that $c$ is any point of the interval~$I$. Let ${\varepsilon}\,>\,0$ be given, and let ${\delta} \,=\, {\varepsilon}/M$.
    Then, by the preceding result, for each $d$ in $I$ such that $|d-c|\,<\,{\delta}$ one has
        \begin{displaymath}
        |g(d)-g(c)|\,\,{\leq}\,\,M\,|d-c|\,<\,M\,\left(\frac{{\varepsilon}}{M}\right) \,=\, {\varepsilon},
        \end{displaymath}
    as required.



%% EXERCISE Give a physical interpretation of `claim' in terms of speeds

\V

        (6) Let $h$, $I \,=\, [a,b]$ and $M$ be as in the preceding example. Let $h:{\RR} \,{\rightarrow}\, {\RR}$ be the extension of $g$ to ${\RR}$ (see Definition~\Ref{A30.127}) defined by the rule
        \begin{displaymath}
        g(x) \,=\, 
        \begin{array}{cl}
        h(x) & \mbox{if $a\,\,{\leq}\,\,x\,\,{\leq}\,\,b$} \\
        h(a) & \mbox{if $x\,<\,a$} \\
        h(b) & \mbox{if x\,>\,b}
        \end{array}
        \end{displaymath}
    It is easy to see, from the preceding exercise, that $g$ satisfies the analogous inequality
        \begin{displaymath}
        |g(d) - g(c)|\,\,{\leq}\,\,M\,|d-c| \mbox{ for every $c$}
        \end{displaymath}

% EXERCISE -- prove it

\V

        \underline{Note} The preceding examples illustrate the fact that sometimes it is easier to attack a `continuity' problem
    by using the sequential characterization, while sometimes the ${\varepsilon}{\delta}$ characterization works more easily.

\VV

        The conditions enjoyed by the piecewise-linear functions $h$ in Example~(5) and $g$ in Example~(6) above appear frequently in analysis and is given a name.

\V


            \subsection{\small{\bf Definition}}
            \label{DefD20.54}

\V

        Let $f:X \,{\rightarrow}\, {\RR}$ be a real-valued function defined on a nonempty subset $X$ of~${\RR}$.
    One says that {\bf $\Bfm{f}$ satisfies a Lipschitz condition on $\Bfm{X}$}, or that {\bf $\Bfm{f}$ is a Lipschitz function on~$\Bfm{X}$},
    \IndBD{functions}{Lipschitz condition, Lipschitz function}provided there exists a real number $M\,>\,0$
    such that $|f(d)-f(c)|\,\,{\leq}\,\,M\,|d-c|$ for all $c$ and $d$ in the set~$X$.

\V

        {\bf Remarks} (1) The condition in question is named after the nineteenth-century German mathematician Rudolf Lipschitz,
    who recognized its importance in the theory of differential equations.

\V

        (2) The result in Example (5) above can be phrased to say that a continuous piecewise-linear function on an interval $[a,b]$ satisfies a Lipschitz condidtion on the set~$[a,b]$.
    A similar remark applies to Example~(6).

\V

        (3) It is obvious that if a function $f:X \,{\rightarrow}\, {\RR}$ satisfies a Lipschitz condition on a set~$X$, then $f$ is continuous on~$X$.
    However, the converse is not true. For example, it is an instructive exercise to show that
    the square-root function is {\em not} Lipschitz on, say, the closed interval $[0,1]$. %% EXERCISE

\VV

%-------------
\StartSkip{
            \subsection{\small{\bf Theorem} (The Concatenation Theorem for Continuous Functions)}\IndBD{concatenation}{theorem for continuous functions}
            \label{ThmD30.05}\IndA{concatenation}

        In this theorem, $I$ is an interval in ${\RR}$ of any type: open, closed, bounded, unbounded  etc.

\V

               (a) Let $f:I \,{\rightarrow}\, {\RR}$ and $g:I \,{\rightarrow}\, {\RR}$ be continuous functions on the interval~$I$.
    Suppose that there is a point $c$ in the interior of $I$ and a number $C$ such that $f(c) \,=\, g(c) \,=\, C$.
    Define a new function $h:I \,{\rightarrow}\, {\RR}$ by the following rule: when $x{\in}I$ then
        \begin{equation}
        \label{EqnD.20}
        h(x) \,=\, 
                \left\{
        \begin{array}{ll}
      f(x) & \mbox{if $x\,\,<\,c$} \\
       C   & \mbox{if $x \,=\, c$} \\
      g(x) & \mbox{if $x\,>\,c$}
        \end{array}
                 \right.
        \end{equation}
    Then $h$ is also continuous on $I$.

\V

        (b) More generally, suppose that $F$ and $G$ are real-valued functions, with $F$ defined and continuous at all $x$ in $I$ such that $x\,<\,c$,
    and $G$ defined and continuous at all $x$ in $I$ such that $x\,>\,c$. Suppose, in addition, that there is a number $C$ such that
    $\lim_{x{\nearrow}c} F(x) \,=\, C$ and $\lim_{x{\searrow}c} G(x) \,=\, C$.
    Define a new function $H:I \,{\rightarrow}\, {\RR}$ by the following rule: when $x{\in}I$, then 
        \begin{equation}
        \label{EqnD.20A}
        h(x) \,=\, 
                \left\{
        \begin{array}{ll}
      F(x) & \mbox{if $x\,\,<\,c$} \\
       C   & \mbox{if $x \,=\, c$} \\
      G(x) & \mbox{if $x\,>\,c$}
        \end{array}
                 \right.
        \end{equation}
    Then $H$ is continuous on~$I$. (Note that we allow the possibity that $F$ and $G$ are defined at other values of~$x$; in particular, at $x \,=\, c$.)

\V

        {\bf Proof} (a)\,It is clear that if $\overline{x}{\in}I$ but $\overline{x} \,\,{\neq}\,\, c$ then $h$ is continuous at $\overline{x}$.
    Indeed, suppose that $\overline{x}\,<\,c$, and let ${\varepsilon}\,>\,0$ be given.
    Since $\overline{x}\,<\,c$, it is possible to choose ${\delta}_{1}\,>\,0$ small enough that if $x{\in}I$ and $|x-\overline{x}|\,<\,{\delta}_{1}$ then $x\,<\,c$;
    for instance, let ${\delta}_{1} \,=\, (c-\overline{x})/2$.
    Next, use the assumed continuity of $f$ to choose ${\delta}_{2}\,>\,0$ so that if $|\overline{x}-x|\,<\,{\delta}_{2}$ then $x{\in}I$ and $|f(x)-f(\overline{x})|\,<\,{\varepsilon}$.
    Now set ${\delta} \,=\, {\min}\,\{{\delta}_{1},{\delta}_{2}\}$.
    One sees that ${\delta}\,>\,0$, and if $|x-\overline{x}|\,<\,{\delta}$ then $|f(x)-f(\overline{x})|\,<\,{\varepsilon}$.
    However, $h(x) \,=\, f(x)$ for $x$ in $I$ such that $x\,<\,c$, so the preceding result can be stated:
    if $x{\in}I$ and $|x-\overline{x}|\,<\,{\delta}$  then $|h(x)-h(\overline{x})|\,<\,{\varepsilon}$.
    That is, $h$ is continuoous at such $\overline{x}\,<\,c$.
    A similar argument shows that if $\overline{x}{\in}I$ and $\overline{x}\,>\,c$ then $h$ is continuous at $\overline{x}$.

        Finally, consider the continuity situation at $c$.
    By Theorem~\Ref{ThmD20.55}, it suffices to show that if ${\xi} \,=\, (x_{1},x_{2},\,{\ldots}\,)$ is any {\em monotonic} sequence in $I$ converging to $c$,
    then $\lim_{k \,{\rightarrow}\, {\infty}}  h(x_{k}) \,=\, h(c)$. Note that if the exists a value of the index $k$ such that $x_{k} \,=\, c$,
    then (by the monotonic convergence to~$c$) one has $x_{m} \,=\, c$ for all $m\,\,{\geq}\,\,k$,
    so that $\lim_{k \,{\rightarrow}\, {\infty}} f(x_{k}) \,=\, f(c)$, as required.
    Thus, there are two remaining cases to consider:

        \h \underline{Case (i)}\, Suppose that ${\xi}$ is monotonic {\em up}, but never equal to~$c$.
    Then clearly $x_{k}\,<\,c$ for all $k$, hence $h(x_{k}) \,=\, f(x_{k})$, by the definition of $h$.
    Then the desired equation
        \begin{displaymath}
        \lim_{k \,{\rightarrow}\, {\infty}} h(x_{k}) \,=\, h(c)
        \end{displaymath}
    is simply a rewriting of the equation
        \begin{displaymath}
        \lim_{k \,{\rightarrow}\, {\infty}} f(x_{k}) \,=\, f(c) \,=\, C.
        \end{displaymath}
    But this latter equation is true, by the continuity hypothesis on $f$;
    thus the corresponding equation on $h$ also holds, as desired.

        \h \underline{Case (ii)} Suppose instead that ${\xi}$ is monotonic {\em down}. The argument is similar to Case~(i) and is left to the reader.

         In summary: the function $h$ is continuous at each point of $I$, as claimed.

\V

        (b)\,This follows easily from Part~(a) by defining $f,g:I \,{\rightarrow}\, {\RR}$ by the following rule: when $x{\in}I$ then
        \begin{equation}
        f(x) \,=\, 
                \left\{
        \begin{array}{ll}
      F(x) & \mbox{if $x\,\,<\,c$} \\
       C   & \mbox{if $x\,\,{\geq}\,\,c$}
        \end{array}
                 \right.
        \mbox{ and }
        g(x) \,=\, 
                \left\{
        \begin{array}{ll}
       C   & \mbox{if $x\,\,{\leq}\,\,c$} \\
      G(x) & \mbox{if $x\,>\,c$}
        \end{array}
                 \right.
        \end{equation}
    It is clear that the functions $f$ an $g$ then satisfy the hypotheses of Part~(a),
    and that the conclusion of Part~(a) when applied to this $f$ and $g$ implies the conclusion of Part~(b).

\V
            \subsection{\small{\bf Remarks}}
            \label{RemrkD30.05A}


 \hspace*{\parindent}(1) The reason for the two-fold formulation of the preceding theorem is that sometimes one formulation is more convenient to use than the other.

\V

        (2) The title `Concatenation' of this theorem comes from the Latin word for `link' or `chain'.
    Indeed, the process described in this theorem consists of `linking together' functions.

\VV

        It is clear that the `concatenation' process described above can be extended
    to form a continuous function by linking together more than two continuous functions.
    For simplicity we formulate this extension only for the situation given in Part~(a) of the preceding theorem;
    it is easy to modify the extension to include the situation in Part~(b).

\V

            \subsection{\small{\bf Corollary}}
            \label{CorD30.05B}

    Let $I$ be an interval in ${\RR}$, and let $c_{1}$, \,{\ldots}\,$c_{n}$ be interior points of $I$
    such that $c_{1}\,<\,c_{2}\,<\,\,{\ldots}\,\,<\,c_{n}$.
    Suppose that for each $j \,=\, 0,1,\,{\ldots}\,n$ one is given a continuous function
    $f_{j}:I \,{\rightarrow}\, {\RR}$ such that
        \begin{displaymath}
        f_{0}(c_{1}) \,=\, f_{1}(c_{1}), \,f_{1}(c_{2}) \,=\, f_{2}(c_{2}),\,{\ldots}\,
        f_{n-1}(c_{n}) \,=\, f_{n}(c_{n})
        \end{displaymath}
    For each $j \,=\, 1,\,{\ldots}\,n$ let $C_{j} \,=\, f_{j-1}(c_{j})$.
Define a function $h:I \,{\rightarrow}\, {\RR}$ as follows: when $x{\in}I$, then
        \begin{displaymath}
    h(x) \,=\,
        \left\{
        \begin{array}{ll}
        f_{0}(x) & \mbox{if $x\,<\,c_{1}$}   \\
        C_{1}    & \mbox{if $x \,=\, c_{1}$} \\
        f_{1}(x) & \mbox{if $c_{1}\,<\,x\,<\,c_{2}$}   \\
        C_{2}    & \mbox{if $x \,=\, c_{2}$} \\
    \,{\vdots}\, &    \,{\vdots}\,           \\
        f_{n-1}(x) & \mbox{if $c_{n-1}\,<\,x\,<\,c_{n}$}   \\
        C_{n}    & \mbox{if $x \,=\, c_{n}$} \\
        f_{n}(x) & \mbox{if $x\,>\,c_{n}$}
        \end{array}
        \right.
        \end{displaymath}

}%\EndSkip
%-----------------

%------------------
\StartSkip{
        The word `concatenation' in the name of the Theorem~\Ref{ThmD20.56} derives from the Latin verb `to link together' or `to `chain together'.
    It reflects the process by which the function $h$ in the previous theorem is constructed; namely,
    by `linking together' the functions $f$ and $g$ at the point $c$ as described in Equation~\Ref{EqnD.20}.
    The word `concatenation' and the symbol~$\&$ are used in 
    The next definition makes this more precise.

\V

            \subsection{\small{\bf Definition}}
            \label{DefD20.58A}

\hspace*{\parindent}
        (1) Let $I$ be an interval in ${\RR}$ and let $c$ be an interior point of $I$.
    Suppose that $f$ and $g$ are real-valued functions such that $f(x)$ is defined for all $x$ in $I$ such that $x\,<\,c$, and $g(x)$ is defined for all $x$ in $I$ such that $x\,>\,c$.
    (The functions $f$ and $g$ are allowed to be defined at other points -- even at $c$.)
    Then {\bf a concatenation\IndA{concatenation} in $I$ of $f$ with $g$ at $c$} is a function $h:I \,{\rightarrow}\, {\RR}$ such that if $x{\in}I$ then
        \begin{displaymath}
        \mbox{$h(x) \,=\, f(x)$ for $x\,<\,c$ and $h(x) \,=\, g(x)$ for $x\,>\,c$};
        \end{displaymath}
    the value $C \,=\, h(c)$ of $h$ at $c$ can be any real number.
    The symbol for the specific concatenation formed this way from $f$ and $g$ and the data $(c,C)$ is $f\mbox{\&}_{(c,C)}g$.

    The rule defining the concatenation $f\stackrel{(c,C)}{\&}g$ is often written as follows:
    if $x{\in}I$, then
        \begin{displaymath}
        \left(f \stackrel{(c,C)}{\&}g\right) \,=\, 
        \left\{
        \begin{array}{cl}
        f(x) & \mbox{if $x\,<\,c$}   \\
          C  & \mbox{if $x \,=\, c$} \\
        g(x) & \mbox{if $x\,>\,c$}.
        \end{array}
                            \right.
        \end{displaymath}

\V

        (2) More generally, suppose that $c_{1}$, $c_{2}$, \,{\ldots}\,$c_{k}$ are interior points of $I$ such that
        \begin{displaymath}
        c_{1}\,<\,c_{2}\,<\,\,{\ldots}\,\,<\,c_{k},
        \end{displaymath}
	and let $C_{1}$, $C_{2}$,\,{\ldots}\,$C_{k}$ be any real numbers.
    Suppose that $f_{1}$, $f_{2}$,\,{\ldots}\,$f_{k+1}$ are real-valued functions such that $f_{1}$ is defined at all $x$ in $I$ such that $x\,<\,c_{1}$,
    $f_{k+1}$ is defined at all $x$ in $I$ such that $x\,>\,c_{k}$, and if $2\,\,{\leq}\,\,j\,\,{\leq}\,\,k$ then $f_{j}(x)$ is defined for all $x$ such that $c_{j-1}\,<\,x\,<\,c_{j}$.
    (Once again, these functions may  be defined at other points as well.)
    Then the symbol
        \begin{displaymath}
f_{1}\stackrel{(c_{1},C_{1})}{\&}f_{2}\stackrel{(c_{2},C_{2})}{\&}\,{\ldots}\,\stackrel{(c_{k},C_{k})}{\&}f_{k+1}:I \,{\rightarrow}\, {\RR}
        \end{displaymath}
    denotes the real-valued function $h:I \,{\rightarrow}\, {\RR}$ such that if $x{\in}I$ then $h(x)$ is given by the following (complicated!) rule:
        \begin{displaymath}
        h(x) \,=\, \left\{
        \begin{array}{cl}
        f_{1}(x) & \mbox{if $x\,<\,c_{1}$}             \\
        C_{1}    & \mbox{if $x \,=\, c_{1}$}           \\
        f_{2}(x) & \mbox{if $c_{1}\,<\,x\,<\,c_{2}$}   \\
        C_{2}    & \mbox{if $x \,=\, c_{2}$}           \\
        {\vdots} &        {\vdots}                     \\
        f_{k}(x) & \mbox{if $c_{k-1}\,<\,x\,<\,c_{k}$} \\
        C_{k}    & \mbox{if $x \,=\, c_{k}$}           \\
        f_{k+1}(x) & \mbox{if $x\,>\,c_{k}$}        
        \end{array}
                                \right.
        \end{displaymath}
    The subintervals of $I$ of the form $\{x{\in}I:x\,<\,c_{1}\}$, $\{x{\in}I:c_{1}\,<\,x\,<\,c_{2}\}$,\,{\ldots}\,$\{x{\in}I:c_{k-1}\,<\,x\,<\,c_{k}\}$, $\{x{\in}I: x_{k}\,>\,c_{k}\}$ are called the {\bf subintervals of the concatenation}.
    (Note that if the interval $I$ is not an open interval, then a concatenation subinterval which contains an endpoint of $I$ will also not be an open interval.)

    The numbers $c_{1}$, $c_{2}$, \,{\ldots}\,$c_{k}$ are called the {\bf nodes of the concatenation},
    and the numbers $C_{1}$, $C_{2}$,\,{\ldots}\,$C_{k}$ are the corresponding {\bf nodal values of $h$}.

\V

        (3) A function $h:I \,{\rightarrow}\, {\RR}$ is said to be {\bf piecewise continuous on $I$} provided $h$ can be expressed as a concatenation
    $h \,=\, f_{1}\stackrel{(c_{1},C_{2})}{\&}f_{2}\stackrel{(c_{2},C_{2})}{\&}\,{\ldots}\,\stackrel{(c_{k},C_{k})}{\&}f_{k+1}$
    on $I$ such that each function $f_{j}$ is continuous on the corresponding concatenation subinterval of $I$.

        If, in addition, each of the functions $f_{1}$, $f_{2}$,\,{\ldots}\,$f_{k+1}$ is linear on the corresponding subinterval, then one says that $h$ is {\bf continuous piecewise linear}.

\V

        (4) A function $h:I \,{\rightarrow}\, {\RR}$ is said to be {\bf piecewise linear on $I$} provided $h$ can be expressed as a concatenation
    $h \,=\, f_{1}\mbox{\&}_{(c_{1},C_{1})}f_{2}\mbox{\&}_{(c_{2},C_{2})}\,{\ldots}\,\mbox{\&}_{(c_{k},C_{k})}f_{k+1}$
    on $I$ such that each function $f_{j}$ is linear on the corresponding concatenation subinterval of $I$.

\V

        (5) More generally, let ${\cal A}$ be an adjective for which it makes sense to say that a function `is ${\cal A}$' on a given interval';
    in~(3) ${\cal A}$ is `continuous', while in~(4) ${\cal A}$ is `linear'.
    A function $h:I \,{\rightarrow}\, {\RR}$ is said to be {\bf piecewise ${\cal A}$ on $I$} provided $h$ can be expressed as a concatenation
    $h \,=\, f_{1}\mbox{\&}_{(c_{1},C_{1})}f_{2}\mbox{\&}_{(c_{2},C_{2})}\,{\ldots}\,\mbox{\&}_{(c_{k},C_{k})}f_{k+1}$
    on $I$ such that each function $f_{j}$ is ${\cal A}$ on the corresponding concatenation subinterval of $I$.

        Some common choices for ${\cal A}$:

        \h `monotonic'; `differentiable'; `$C^{k}$'; `smooth''; the last three terms are defined later, in Chapter~\Ref{ChaptE}.


\V
\V

            \subsection{\small{\bf Examples}}
            \label{ExampD20.58B}

\V

\hspace*{\parindent}(1) Let $f$, $g$ and $h$ be the functions which appear in Theorem~\Ref{ThmD20.56}.
    Then one can write
        \begin{displaymath}
        h \,=\, f\mbox{\&}_{(c,f(c))}g \,=\, f\mbox{\&}_{(c,g(c))}g.
        \end{displaymath}

\V

        (2) Let $h:{\RR} \,{\rightarrow}\, {\RR}$ be the absolute value function, given by the rule $h(x) \,=\, |x|$ for all $x$ in ${\RR}$.
    This function can be written in the form $f\mbox{\&}_{(0,0)}g$, where $g$ is the identity function and $f \,=\, -g$.
    This function is continuous on ${\RR}$, piecewise linear and piecewise monotonic on ${\RR}$.
    The open intervals associated with this concatenation are $(-{\infty},0)$ and $(0,+{\infty})$, and the node is~$0$.

\V

        (3) Let ${\cal D} \,=\, \{(x_{0},y_{0}),  (x_{2},y_{2}),\,{\ldots}\, (x_{k},y_{k})\}$ be a set of $(k+1)$ points of ${\RR}^{2}$ such that $x_{0}\,<\,x_{1}\,<\,\,{\ldots}\,\,<\,x_{k-1}\,<\,x_{k}$;
    think of ${\cal D}$ as a set of `data points'.
    Then the {\bf piecewise-linear interpolating function for the data set ${\cal D}$} is the concatenation
        \begin{displaymath}
        F_{{\cal D}} \,=\, L_{1}\mbox{\&}_{(x_{1},y_{1})}L_{2}\mbox{\&}_{(x_{2},y_{2})}\,{\ldots}\,L_{k-1}\mbox{\&}_{(x_{k-1},y_{k-1})}L_{k}:[x_{0},x_{k}] \,{\rightarrow}\, {\RR},
        \end{displaymath}
    where for each $j \,=\, 1,2,\,{\ldots}\,k$ the function $L_{j}:{\RR} \,{\rightarrow}\, {\RR}$ is the linear function whose graph passes through the data points $(x_{j-1},y_{j-1})$ and $(x_{j},y_{j})$.
    That is,
        \begin{displaymath}
        L_{j}(x) \,=\, \left(\frac{y_{j}-y_{j-1}}{x_{j}-x_{j-1}}\right)(x-x_{j-1}) + y_{j-1} \mbox{ for all $x$}.
        \end{displaymath}
    It is useful to extend the definition of $F_{{\cal D}}(x)$ to make sense for all $x$ in ${\RR}$ by setting $F_{{\cal D}}(x) \,=\, y_{0}$ if $x\,<\,x_{0}$ and $F_{{\cal D}}(x) \,=\, y_{k}$ if $x\,>\,x_{k}$.
    This extension can also be expressed as a concatenation, namely
        \begin{displaymath}
        L_{0}\mbox{\&}_{(x_{0},y_{0})}\,{\ldots}\,\mbox{\&}_{(x_{k-1},y_{k-1})}L_{k}\mbox{\&}_{(x_{k},y_{k})}L_{k+1},
        \end{displaymath}
    where $L_{0}(x) \,=\, x_{0}$ and $L_{k+1}(x) \,=\, x_{k}$ for all $x$ in ${\RR}$.

        \underline{Note} The most common use of such interpolations occurs when the data points $(x_{j},y_{j})$ form a table of values for a difficult-to-compute function $f$;
    for example, the values $y_{j} \,=\, f(x_{j})$ may have been determined through physical experiments, and the actual `formula' for $f$ is unknown.
    In this case one can use the values $F_{{\cal D}}(x)$ of the interpolation function $F_{{\cal D}}$
    to approximate the (unknown) values $f(x)$ when $x$ is not equal to a node $x_{j}$.
    Theorem~\Ref{ThmD20.56A} shows that if $f$ is continuous then the linear interpolation process can produce accurate results.
}%\EndSkip
%-------------------  

\VV

                \section{{\bf Standard Continuity Laws in ${\RR}$}}
                \label{SectD25}\IndB{ZZ Sections}{\Ref{SectD25} Standard Continuity Laws in ${\RR}$}


\VV


        The next several results allow one to construct functions which are
    continuous at a point,
    from other such functions, using the ordinary processes of algebra. If one uses the `sequential characterization' of continuity,
    then proving these standard continuity laws becomes a trivial exercise in applying the standard limit laws
    for convergent sequences of real numbers from the preceding chapter; most such proofs are left to the reader.
    Some of these results are proved here using the `${\varepsilon}\,{\delta}$' characterization,
    mainly to illustrate how such `${\varepsilon}\,{\delta}$' proofs work.

\V

            \subsection{\small{\bf Theorem}}
            \label{ThmD20.60A}


\hspace*{\parindent} Let $f:X \,{\rightarrow}\, {\RR}$ be a real-valued function defineed on a (nonempty) subset $X$ of ${\RR}$.

\V

        (a) If $f$ is continuous at a point $c$ of~$X$, then the function $|f|$ is also continuous at $c$.

\V

        (b) If $f$ is continuous at a point~$c$ of $X$ and $A$ is any real number, then $A{\cdot}f$ is continuous at~$c$.


\V

        (c) Suppose that $S$ is a subset of $X$ and that $c$ is a point of~$S$. Let $g \,=\, f|_{S}:S \,{\rightarrow}\, {\RR}$.
    If $f$ is continuous at $c$, then $g$ is also continuous at~$c$. However, the converse need not be true.
    That is, it can happen that $g$ is continuous at $c$ but $f$ is not. (Compare with Remark~\Ref{RemrkD20.50}.)

\V

        (d) Suppose that $X$, $S$ and $c$ are as in Part~(c), and suppose further that there is an open interval $I \,=\, (a,b)$ such that
    $c{\in}I$ and $I \,{\subseteq}\, S$. With this extra hypothesis, the converse of the conclusion in Part~(c) {\em is} true.
    That is, if $g \,=\, f|_{S}:S \,{\rightarrow}\, {\RR}$ is continuous at~$c$, then $f:X \,{\rightarrow}\, {\RR}$ is also continuous at~$c$.

        Similarly, suppose that $X$ has a least element $a$, and suppose that there is an interval $I \,=\, [a,b)$ with $c{\in}I$ and $I \,{\subseteq}\, X$.
    Then the converse of Part~(c) is true. Likewise, if $X$ has a maximum element $b$ and there is an interval $I \,=\, (a,b]$ such that $c{\in}I$ and $I \,{\subseteq}\, X$, then the converse of Part~(c) is true.

\V

        \underline{Partial Proof}

\V

        (a) By the definition of the function $|f|$, together with the Reverse Triangle Inequality, one has
        \begin{displaymath}
        |\,|f|(d) - |f|(c)\,| \,=\, |\,|f(d)|-|f(c)|\,|\,\,{\leq}\,\,|f(d)-f(c)|,
        \end{displaymath}
    The rest of the proof of the desired result is now left as an easy exercise.

\V

        (b) The case in which $A \,=\, 0$ is trivial, and is left as an exercise.

        Thus, assume that $A \,\,{\neq}\,\, 0$, and let ${\varepsilon}\,>\,0$ be given.
    The hypothesis that $f$ is continuous at~$c$ then implies that there is ${\delta}\,>\,0$ such that if $d{\in}X$ and $|d-c|\,<\,{\delta}$,
    then $|f(d)-f(c)|\,<\,{\varepsilon}/|A|$. Multiply both sides of this inequality by $|A|$ and do the obvious simplification to get
        \begin{displaymath}
        |(A{\cdot}f)(d) - (A{\cdot}f)(c)| \,=\, |A|{\cdot}|f(d) - f(c)|\,<\,{\varepsilon} \mbox{ if $d{\in}X$ and $|d-c|\,<\,{\delta}$}
        \end{displaymath}
    The desired result follows.

\V

        (c) The proof that $g$ is continuous at $c$ is trivial, and is left as an exercise.

        One simple example of the `converse' not holding is this: Suppose that $X \,=\, {\RR}$, $S \,=\, {\QQ}$ and $c \,=\, 0$.
    Let $f:{\RR} \,{\rightarrow}\, {\RR}$ be the Dirichlet function, which is known to be discontinuous at every point; see Example~\Ref{ExampD20.53}~(2).
    But by the definition of the Dirichlet function, $g \,=\, f|_{{\QQ}}:{\QQ} \,{\rightarrow}\, {\RR}$
    is a constant function, hence continuous at each point of its domain, including at~$0$.

\V

        (d) Write the open interval $I$ in the form $I \,=\, (a,b)$, so that $a\,<\,c\,<\,b$.
    Let ${\delta}_{1}\,>\,0$ be small enough that $a\,<\,c-{\delta}_{1}\,<\,c\,<\,c+{\delta}_{1}\,<\,b$.
    Now let ${\varepsilon}\,>\,0$ be given, and let ${\delta}_{2}\,>\,0$ be small enough that if $d{\in}S$ and $|d-c|\,<\,{\delta}_{2}$,
    then $|g(d)-g(c)|\,<\,{\varepsilon}$. Let ${\delta} \,=\, \min\,\{{\delta}_{1}, {\delta}_{2}\}$.
    If $d$ in $X$ satisfies $|d-c|\,<\,{\delta}$, then $|d-c|\,<\,{\delta}_{1}$ and thus $d{\in}I \,{\subseteq}\, S$; in particular, $d{\in}S$.
    It follows that $|f(d)-f(c)| \,=\, |g(d)-g(c)|$, by the definition of `restriction'.
    Furthermore, one also has $|d-c|\,<\,{\delta}_{2}$, so $|g(d)-g(c)|\,<\,{\varepsilon}$.
    Combine these facts to conclude that if $d{\in}X$ and $|d-c|\,<\,{\delta}$, then $|f(d)-f(c)|\,<\,{\varepsilon}$.
    It follows that $f:X \,{\rightarrow}\, {\RR}$ is continuous at~$c$, as required.

        The other statements in this part of the theorem are proved in a similar manner.


\VV

            \subsection{\small{\bf Theorem}}
            \label{ThmD20.60B}

\V

        Let $f:X \,{\rightarrow}\, {\RR}$ and $g:Y \,{\rightarrow}\, {\RR}$ be real-valued functions defined on $X$ and $Y$, respectively,
    where $X$ and $Y$ are nonempty subsets of ${\RR}$. Suppose that $Z \,=\, X\,{\cap}\,Y \,\,{\neq}\,\, {\emptyset}$,
    and that $c$ is a point of~$Z$. Assume that $f:X \,{\rightarrow}\, {\RR}$ and $g:Y \,{\rightarrow}\, {\RR}$ are both continuous at~$c$,
    when viewed as function diagrams $f:X \,{\rightarrow}\, {\RR}$ and $g:Y \,{\rightarrow}\, {\RR}$, resepctively. Then:

\V

        (a) Then $(f+g):Z \,{\rightarrow}\, {\RR}$ and $(f-g):Z \,{\rightarrow}\, {\RR}$ are both continuous at $c$.

\V

        (b) Likewise, the function $(f{\cdot}g):Z \,{\rightarrow}\, {\RR}$ is continuous at~$c$.

\V

        (c) Suppose, in addition, that $g(c) \,\,{\neq}\,\, 0$, and let $W \,=\, \{x{\in}Z: g(x) \,\,{\neq}\,\, 0\}$.
    Then $(f/g):W \,{\rightarrow}\, {\RR}$ is continuous at $c$.

\V

        \underline{Partial Proof} (a) and (b): The reader is encouraged to prove these parts using the ${\varepsilon}\,{\delta}$ characterization of contiinuity.

\V

        (c) For simplicity, set $h \,=\, f{\cdot}g:Z \,{\rightarrow}\, {\RR}$.

        \underline{Special Case} Suppose that $f(c) \,=\, 0$ and $g(c) \,=\, 0$. Then one has $h(c) \,=\, f(c)\,g(c) \,=\, 0$,
    hence $|h(d) - h(c)| \,=\, |f(d)\,g(d)|$. Let ${\varepsilon}\,>\,0$ be given; without loss of generality one may assume that ${\varepsilon}\,<\,1$.
    By hypothesis, $f:X \,{\rightarrow}\, {\RR}$ and $g:Y \,{\rightarrow}\, {\RR}$ are continuous at $c$.
    It then follows from Part~(c) of the preceding theorem that their restrictions to $Z$ are continuous at $c$.
    Now let ${\varepsilon}\,>\,0$; without loss of generality, assume that $0\,<\,{\varepsilon}\,<\,1$.
    It follows from this continuity that there exists ${\delta}\,>\,0$ such that if $d{\in}Z$ and $|d-c|\,<\,{\delta}$,
    then $|f(d)|\,<\,{\varepsilon}$ and $|g(d)|\,<\,{\varepsilon}$. Since, by hypothesis, $0\,<\,{\varepsilon}\,<\,1$,
    it then follows that for such $d$ one has $|h(d)| \,=\, |f(d){\cdot}g(d)|\,<\,{\varepsilon}^{2}\,<\,{\varepsilon}$, as required.

        \underline{General Case} Define $F:Z \,{\rightarrow}\, {\RR}$ and $G:Z \,{\rightarrow}\, {\RR}$
    given by the rules $F(x) \,=\, f(x)-f(c)$ and $G(x) \,=\, g(x)-g(c)$.
Consider  $H:Z \,{\rightarrow}\, {\RR}$
    given by $H(x) \,=\, F(x){\cdot}G(x)$, so that $F(c) \,=\, G(c) \,=\, 0$.
    It follows from Part~(b), together with the fact that constant functions are continuous,
    that $F$ and $G$ satisfy the hypothesis of the special case just considered, and thus that $H$ is continuous at $c$.
    However, one computes that for all $x$ in~$Z$ one has
        \begin{displaymath}
        H(x) \,=\,
        f(x){\cdot}g(x) - f(c){\cdot}g(x) - f(x){\cdot}g(c) + f(c){\cdot}g(c);
        \end{displaymath}
    that is,
        \begin{displaymath}
        f(x)\,{\cdot}\,g(x) \,=\,
    H(x) + f(c){\cdot}g(x) + f(x){\cdot}g(c) - f(c){\cdot}g(c).
        \end{displaymath}
    It follows easily by repeated use of Parts~(a) and~(b) of the present theorem, together with Part~(b) of the preceding theorem,
    that $f{\cdot}g:Z \,{\rightarrow}\, {\RR}$ is continuous at~$c$, as claimed.
\VV


            \subsection{\small{\bf Examples}}
            \label{ExampD20.65}

\V

\hspace*{\parindent}(1) Suppose that $f:{\RR} \,{\rightarrow}\, {\RR}$ is a polynomial function on ${\RR}$;
    that is, there exist finitely many real numbers $c_{0}$, $c_{1}$,\,{\ldots}\,$c_{m}$ such that
        \begin{displaymath}
        f(x) \,=\, c_{0} + c_{1}\,x + c_{2}\,x^{2} + \,{\ldots}\,+c_{m}\,x^{m} \mbox{ for all $x$ in ${\RR}$}.
        \end{displaymath}
     Then $f$ is continuous on ${\RR}$.

\V

        (2) Suppose that $h$ is a rational function; that is, there exist polynomial functions $f$ and $g$ on ${\RR}$, with $g$ not equal to the zero polynomial, such that the domain $D$ of $h$ equals the set of all $x$ such that $g(x) \,\,{\neq}\,\, 0$, and $h(x) \,=\, f(x)/g(x)$ for all $x$ in $D$.
    Then $h$ is continuous on $D$.

\V

        Both of these statements follow directly from the previous theorems, combined with Mathematical Induction. The details are left as an exercise.
%% EXERCISE

\VV

        The next result says, in effect, that `continuity' behaves nicely under composition.

\V

            \subsection{\small{\bf Theorem}}
            \label{ThmD20.70}

        Let $f:X \,{\rightarrow}\, {\RR}$ be a real-valued function defined on a nonempty subset $X$ of ${\RR}$.
    Likewise, let $g:Y \,{\rightarrow}\, {\RR}$ be such a function defined on a nonempty subset $Y$ of ${\RR}$. Assume that $f[X] \,{\subseteq}\, Y$.
    Let $c$ and $b$ be points of $X$ and $Y$, respectively, such that $b \,=\, f(c)$. Suppose that $f$ is continuous at $c$ and that $g$ is continuous at $b$.
    Then the composition $h \,=\, g{\circ}f$ is continuous at $c$.

\V

        \underline{Proof} For sake of variety let us prove this result using the `sequences characterization' of continuity.

\V

    Let ${\xi} \,=\, (x_{1},x_{2},\,{\ldots}\,)$ be a sequence in $X$ such that $\lim_{k \,{\rightarrow}\, {\infty}} x_{k} \,=\, c$.
    It follows from the continuity hypothesis on $f$ that $\lim_{k \,{\rightarrow}\, {\infty}} f(x_{k}) \,=\, f(c)$.
    Let $y_{k} \,=\, f(x_{k})$ for each $k$ in ${\NN}$, so that the previous equation can be written $\lim_{k \,{\rightarrow}\, {\infty}} y_{k} \,=\, b$.
    (Recall that, by hypothesis, $f(c) \,=\, b$.)
    Then, since $f[X] \,{\subseteq}\, Y$, it also follows that $y_{k}{\in}Y$ for each $k$.
    Now the continuity hypothesis on $g$ implies that $\lim_{k \,{\rightarrow}\, {\infty}} g(y_{k}) \,=\, g(b)$.
    Combine this with the fact that $h(x_{k}) \,=\, g(f(x_{k})) \,=\, g(y_{k}) \mbox{ for each $k$ in ${\NN}$, and } h(c) \,=\, g(f(c)) \,=\, g(b)$
    to get $\lim_{k \,{\rightarrow}\, {\infty}} h(x_{k})  \,=\, \lim_{k \,{\rightarrow}\, {\infty}} g(y_{k}) \,=\, g(b) \,=\, h(c)$, as required.

\VV


            \subsection{\small{\bf Example}}
            \label{ExampD20.80}
\V

        Let $f:[0,+{\infty}) \,{\rightarrow}\, {\RR}$ be the standard Square-Root Function, given by $f(x) \,=\, \sqrt{x}$ for each $x\,\,{\geq}\,\,0$.
    In Example~\Ref{ExampD20.53}~(4) it is proved that $f$ is continuous at each point of its domain~$[0,+{\infty})$.
    Since $f$ maps $[0,+{\infty})$ into itself, it follows that the composition $f{\circ}f:[0,+{\infty}) \,{\rightarrow}\, {\RR}$ is defined,
    and by the preceding theorem this composition is continuous. This process can be repeated as often as one wishes.

        Thus, for each $n$ in ${\NN}$ let $f_{n}:[0,+{\infty}) \,{\rightarrow}\, {\RR}$ be the function obtained by composing
    $f$ with itself $(n-1)$~times; thus, $f_{1} \,=\, f$, $f_{2} \,=\, f{\circ}f$, $f_{2} \,=\, f{\circ}f{\circ}f$, and so on.
    Clearly each of these functions is also continuous on~$[0,+{\infty})$. It is easy to see that for each $x\,\,{\geq}\,\,0$,
    $f_{n}(x)$ is the $2^{n}$-th root of~$x$; that is, $\left(f_{n}(x)\right)^{2^{n}} \,=\, x$.
    Using the `exponent' notation from high-school algebra, this can be written $f_{n}(x) \,=\, x^{1/2^{m}}$.
    Thus the function $g(x) \,=\, x^{1/2^{m}}$ is continuous on~$[0.+{\infty})$.

        Now let $m$ be any natural number, and $p_{m}:{\RR} \,{\rightarrow}\, {\RR}$ denote the `$m$-th power function' on ${\RR}$.
    This is a polynomial function, hence continuous on~${\RR}$, so the composition
    $h(x) \,=\, (p_{m}{\circ}f_{n})(x) \,=\, x^{m/2^{n}}$ is continuous on $[0,+{\infty})$ as well.
    We shall see later in this chapter that the exponent $m/2^{n}$ can be replaced by any rational exponent,
    and even later by any real number, with the result still continuous on~$[0,+{\infty})$.

\VV

%---------------
\StartSkip{
%%%
\begin{quotation}
{\footnotesize \underline{\Note}\IndB{\notes}{on a physical interpretation of Lemma~\Ref{LemmaD20.100}} (on a physical interpretation of Lemma~\Ref{LemmaD20.100})
        The truth of the preceding lemma becomes `obvious' by interpreting its statement properly.
    Indeed, think of the function $g$ as describing the motion of an object moving along the $y$-axis.
    That is, in the equation $y \,=\, g(x)$ think of $x$ as the `time' and think of $y$ as the `position on the $y$-axis' at the time~$x$.
    With this interpretation the quantity $s_{j}$ represents the velocity, and $|s_{j}|$ the speed, of the object during the time interval $[x_{j-1},x_{j}]$.
    Then $g(d)-g(c)$ represents the (net) change of position of the object along the $y$-axis during the time interval $c\,\,{\leq}\,\,x\,\,{\leq}\,\,d$.
    At no time during that time interval did the speed of the object exceed~$M$. Thus the magnitude of the change of position
    of the object during that time interval certainly cannot exceed the magnitude of the change of position, during the same interval,
    of a second object moving at {\em constant} speed~$M$ along the $y$-axis. That is, $|g(d)-g(c)|\,\,{\leq}\,\,M\,|d-c|$, as claimed.
}%EndFootNoteSize
\end{quotation}
%##
}%\EndSkip
%---------------

\VV

        Definition~\Ref{DefD20.30} seems to require that one verify that $\lim_{k \,{\rightarrow}\, {\infty}} f(x_{k}) \,=\, f(c)$ for {\em every} sequence ${\xi} \,=\, (x_{1},x_{2},\,{\ldots}\,)$ in $X$ which converges to $c$.
    The next result, however, shows that one can get by with checking it only for such sequences which are {\em monotonic}.

\V

            \subsection{\small{\bf Theorem}}
            \label{ThmD20.55}

\V

        Let $f:X \,{\rightarrow}\, {\RR}$ be a real-valued function whose domain is a (nonempty) subset $X$ of ${\RR}$.
    Let $c$ be a point of $X$.
    If $\lim_{k \,{\rightarrow}\, {\infty}} f(x_{k}) \,=\, f(c)$ for every monotonic sequence
    ${\xi} \,=\, (x_{1},x_{2},\,{\ldots}\,)$ in $X$ converging to $c$, then $f$ is continuous at $c$.

\V

        \underline{Proof} We actually prove the contrapositive of the given statement. That is:

\VA

        \h If $f$ is {\em not} continuous at $c$ then there exists a monotonic sequence ${{\zeta}} \,=\, (z_{1},z_{2},\,{\ldots}\,)$
    in $X$ converging to $c$ such that the corresponding sequence  $(f(z_{1}),f(z_{2}),\,{\ldots}\,)$ does {\em not} converge to $f(c)$.

\VA

        Indeed, suppose that $f$ is not continuous at $c$. Then there must exist a number ${\varepsilon}_{0}\,>\,0$ with the following property:
        \begin{displaymath}
        \mbox{For every $k$ in ${\NN}$ there exists a number $x_{k}$ in $X$ such that ${\displaystyle |x_{k}-c|\,<\,\frac{1}{k}}$ and $|f(x_{k})-f(c)|\,\,{\geq}\,\,{\varepsilon}_{0}$} \h ({\ast})
        \end{displaymath}
    It is clear from~$({\ast})$ that the sequence ${\xi} \,=\, (x_{1}, x_{2},\,{\ldots}\,)$ converges to~$c$.
    It then follows from Theorem~\Ref{ThmC30.15} that the sequence ${\xi}$ has a subsequence ${\zeta} \,=\, (z_{1}, z_{2},\,{\ldots}\,)$ which is monotonic.
    The sequence ${\zeta}$ also converges to~$c$, by Part~(b) of Theorem~\Ref{ThmC20.10A}.
    However, since $|f(x_{k}) - f(c)|\,\,{\geq}\,\,{\varepsilon}_{0}$ for all~$k$,
    it follows that $|f(z_{j}) - f(c)|\,\,{\geq}\,\,{\varepsilon}_{0}$ for all~$j$,
    and thus the sequence $(f(z_{1}), f(z_{2}),\,{\ldots}\,)$ does not converge to~$f(c)$. The desired result now follows.

\VV



        The preceding result is especially useful in determining whether a function $f$ is continuous
    at a point $c$ in the case the formula for $f$ to the left of $c$ differs from that to the right of~$c$.
    It leads naturally to the following generalization of the concept of `continuity'.

\V

            \subsection{\small{\bf Definition}}\IndBD{continuity}{one-sided continuity using sequences}
            \label{DefD20.55A}

\V

        Let $f:X \,{\rightarrow}\, {\RR}$ be a real-valued function defined on a nonempty subset $X$ of~${\RR}$, and let $c$ be a point of~$X$.
    One says that $f$ is {\bf continuous at $\Bfm{c}$ from the left}\IndB{continuity}{continnuity from the left, from the right} provided that $\lim_{k \,{\rightarrow}\, {\infty}} f(x_{k}) \,=\, f(c)$
    for every sequence ${\xi} \,=\, (x_{1}, x_{2},\,{\ldots}\,x_{k},\,{\ldots}\,)$ in $X$
    which is monotonic up and converges to~$c$. Likewise, one says that $f$ is {\bf continuous at $\Bfm{c}$ from the right} provided that $\lim_{k \,{\rightarrow}\, {\infty}} f(x_{k}) \,=\, f(c)$
    for every sequence ${\xi} \,=\, (x_{1}, x_{2},\,{\ldots}\,x_{k},\,{\ldots}\,)$ in $X$
    which is monotonic down and converges to~$c$. If either of these conditions holds, then one says that {\bf $\Bfm{f}$ has one-sided continuity at~$\Bfm{c}$}.


\V

        The preceding theorem can now be phrased more compactly.

\V

            \subsection{\small{\bf Corollary}}
            \label{CorD20.55A}

\V

        A function $f:X \,{\rightarrow}\, {\RR}$ is continuous at a point $c$ of a nonempty subset $X$ of ${\RR}$ if, and only if,
    $f$ has {\bf two-sided continuity at $\Bfm{c}$}\IndBD{continuity}{two-sided continuity};
    that is, $f$ is continuous at $c$ both from the left and from the right.

\V

        The simple proof is left as an exercise.

\V

    It is a useful exercise to use the preceding result to redo the analysis of the piecewise-linear function studied in Example~\Ref{ExampD20.53} above.

% EXERCISE `Compare with the earlier analysis. Which is easier? Which gives more information?

\V

        {\bf Remark} If $c \,=\, {\min}\,X$ then $f$ is automatically continuous from the left,
    since the only monotonic-up sequence in $X$ with limit $c$ is the constant sequence $(c,c,\,{\ldots}\,c,\,{\ldots}\,)$.
    In this situation one need only show that $f$ is continuous at $c$ from the right.
    Likewise, if $c \,=\, \max X$, then $f$ is automatically continuous from the right at~$c$, and one need only check continuity at $c$ from the left.

\VV


                \section{{\bf Continuity Theorems on Nonempty Subsets of~${\RR}$}}
                \label{SectD30}\IndB{ZZ Sections}{\Ref{SectD30} Continuity Theorems on Nonempty Subsets of~${\RR}$}

\VV

        The next theorem is always stated in elementary calculus, at least for functions defined on closed bounded intervals,
    and is used in a variety of important ways, both practical and theoretical; but it is almost never proved there.
    In contrast, the theorem which follows it is almost never even stated in elementary calculus.

\VV


        \subsection{\small{\bf Theorem} (The Extreme-Value Theorem in ${\RR}$)}
            \label{ThmD30.10}\IndBD{functions}{extreme-value theorem in ${\RR}$}

\V

        Suppose that $f:X \,{\rightarrow}\, {\RR}$ is a real-valued function which is continuous on a closed bounded set $X$ in ${\RR}$.
    Then the function $f$ assumes minimum and maximum values on the set~$X$. That is, there are numbers $c$ and $d$ in $X$ such that 
    $f(c)\,\,{\leq}\,\,f(x)\,\,{\leq}\,\,f(d)$ for all $x$ in $X$.

\V

        {\bf Proof} Let $S$ be the image $f[X]$ of the set $X$ under the function~$f$; that is,
        \begin{displaymath}
        S \,=\, \{y: y \,=\, f(x) \mbox{ for at least one $x$ in the set~$X$}\}.
        \end{displaymath}
     It is clear that {\em if} such points $c$ and $d$ in $X$ do exist,
    then one must have $f(c) \,=\, {\inf}\,S$ and $f(d) \,=\, {\sup}\,S$. Thus consider the quantities $m \,=\, {\inf}\,S$ and $M \,=\, {\sup}\,S$;
    we do not assume that either of these quantities is finite.

    First note that, by Theorem~\Ref{ThmC50.10}, there exists a sequence ${\zeta} \,=\, (z_{1}, z_{2},\,{\ldots}\,z_{k},\,{\ldots}\,)$
    in the set $S$ such that $\lim_{k \,{\rightarrow}\, {\infty}} z_{k} \,=\, m$.
    (The fact that this sequence can be chosen to be monotonic down is not needed here.)
    The definition of the set $S$ then implies that for each index $k$ there exists at least one point $x_{k}$ in $X$ such that $f(x_{k}) \,=\, z_{k}$.
    Of course there is no reason to expect that the sequence ${\xi} \,=\, (x_{1}, x_{2},\,{\ldots}\,x_{k},\,{\ldots}\,)$ has a limit.
    However, the hypothesis that $X$ is bounded allows one to conclude, from the Standard Bolzano-Weierstrass Theorem for sequences,
    that the sequence ${\xi}$ does have a subsequence ${\sigma} \,=\, (s_{1}, s_{2},\,{\ldots}\,s_{m},\,{\ldots}\,)$ which is convergent.
    Let $c \,=\, \lim_{m \,{\rightarrow}\, {\infty}} s_{m}$. The hypothesis that $X$ is closed in~${\RR}$ then implies that $c{\in}X$.
    Since, by hypothesis, $f$ is continuous at every point of the set~$X$, it is therefore continuous at~$c$.
    Thus $\lim_{m \,{\rightarrow}\, {\infty}} f(s_{m}) \,=\, f(c)$. However, the sequence $(f(s_{1}), f(s_{2}),\,{\ldots}\,f(s_{m}),\,{\ldots}\,)$
    is a subsequence of the sequence ${\zeta}$,
    so $\lim_{m \,{\rightarrow}\, {\infty}} f(s_{m}) \,=\, \lim_{k \,{\rightarrow}\, {\infty}} z_{k} \,=\, {\inf}\,S$.
    Combine these facts to get that ${\inf}\,S \,=\, f(c)$ with $c$ in~$X$, and thus ${\inf}\,S$, i.e., $f(c)$,
    is an element of~$S$, hence $f(c)$ is the minimum element of~$S$. That is, $f(c)$ is the minimum value of $f$ on the set~$X$.

        A similar proof works for the maximum value of $f$ on~$X$. Even simpler, just apply the preceding result for `minimum' to the function $g \,=\, -f$.
    

\V%%\\\

            \subsection{\small{\bf Corollary}}
            \label{CorD30.20}

\V

        Suppose that $X$ and $f:X \,{\rightarrow}\, {\RR}$ is as in the preceding theorem,
    and that for every $x$ in $X$ one has $f(x) \,\,{\neq}\,\, 0$. Then there exists a number $m\,>\,0$ such that $|f(x)|\,\,{\geq}\,\,m$ for all $x$ in~$X$.

\V

        \underline{Proof} Let $h:X \,=\, {\RR}$ be defined by $h(x) \,=\, 1/|f(x)|$ for each $x$ in~$X$; note that $h(x)\,>\,0$ for every $x$ in~$I$.
    Clearly $h$ is continuous on $I$, so the preceding theorem implies that $h$ is bounded.
    In particular, there exists $M\,>\,0$ such that $0\,<\,h(x)\,\,{\leq}\,\,M$ for all $x$ in~$I$; that is, $0\,<\,1/|f(x)|\,\,{\leq}\,\,M$.
    It follows that $|f(x)|\,\,{\geq}\,\,m\,>\,0$, where $m \,=\, 1/M$.

\VV


%%%
\begin{quotation}
{\footnotesize \underline{\Note}\IndB{\notes}{on motivating the concept of uniform continuity} (on motivating the next important concept)

\V

        Because of the length of this {\Note}, it seems approriate to remind you that the content of
    a {\Note} is not needed for the logical development of the material in {\ThisText}, and no exercises are based on it.
    Nevertheless, you are encouraged to read the {\Notes} for whatever extra insights they might provide.

\VV

        The next big definition, namely Definition~\Ref{DefD25.65} below, illustrates a major problem with the modern style of mathematical writing:
    in order to present topics in a `logical' order, these topics must often be introduced without explaining
    historically how their importance became clear to mathematicians.
    For example, the definition in question `logically' belong in the current chapter because it involves only continuity.
    In reality, however, it originally grew out of Cauchy's attempt to rigorously define the definite integral,
    a topic which appears -- quite `logically' -- much later in {\ThisText}, and thus whose historical origins cannot readily be discussed at this point.

        One common way out of this dilemma, certainly the easiest, and often the best, is for the textbook author to give the following advice:

\VA

       \h {\em `Trust me; you'll understand the true significance of the concept later on.'}

\VA

\noindent However, sometimes it is possible to follow a `pseudo-historical' approach to such a topic.
    This means considering a situation which makes sense at the current point of the logical treatment and which {\em could} have motivated the desired topic,
    but which historically did not play that role. This is the approach taken here to motivate Definition~\Ref{DefD25.65}.
    Of course, if this approach does not provide sufficient motivation for your needs, then follow the dictates of the advice given above.

\VV

        {\bf Preliminary Remark} 

\VV

        Suppose that $f:(a,b) \,{\rightarrow}\, {\RR}$ is a continuous function whose domain is an open interval $(a,b)$, where $a$ and $b$ are finite.
    Is it possible to `extend' $f$ to a continuous function on the closed interval $[a,b]$?
    That is, is there a function $g:[a,b] \,{\rightarrow}\, {\RR}$ which is continuous on $[a,b]$ and whose restriction to $(a,b)$ equals~$f$?

        The immediate answer is: it depends on the circumstances.
\V

        {\bf Example} Consider the following functions, each with domain the open interval $(0,1)$:
        \begin{displaymath}
        f_{1}(x) \,=\, x\,(1-x); \h f_{2}(x) \,=\, \frac{1}{x\,(1-x)}.
        \end{displaymath}
    Clearly $f_{1}$ extends to continuous $g_{1}:[0,1] \,{\rightarrow}\, {\RR}$, where $g_{1}(x) \,=\, x\,(1-x)$ for all $x$ in $[0,1]$.
    However, $f_{2}$ cannot extend to a continuous function on $[0,1]$, since such an extension would be bounded on $[0,1]$ (by the Extreme-Value Theorem), 
    hence its restriction to $(0,1)$ would also be bounded on $(0,1)$, while the given function $f$ is clearly {\em unbounded} on~$(0,1)$.

\V

        One way to analyse the issue is to predict what the values of such $g$ -- if it exists -- must be.
    Of course the value $g(x)$ of the hoped-for extension $g$ is clear when $0\,<\,x\,<\,1$: for such $x$ one must have $g(x) \,=\, f(x)$.
    Thus the issue becomes this: what about $g(0)$ and $g(1)$? Let us consider $g(1)$.

        Let ${\xi} \,=\, (x_{1}, x_{2},\,{\ldots}\,x_{k},\,{\ldots}\,)$ be any sequence in $(0,1)$
    such that $\lim_{k \,{\rightarrow}\, {\infty}} x_{k} \,=\, 1$. Then the only possible value for $g(1)$ is $\lim_{k \,{\rightarrow}\, {\infty}} f(x_{k})$.
    This, in turn, would require that for every such sequence, the corresponding sequence
    $f{\circ}{\xi} \,=\, (f(x_{1}),f(x_{2}),\,{\ldots}\,f(x_{k}),\,{\ldots}\,)$ must be convergent to some number; that is, it must be a Cauchy sequence.

\VV

        The issue, of extending a given continuous function $f:X \,{\rightarrow}\, {\RR}$
    from its original domain $X$ to some superset of that domain, arises with domains that are much more complicated than simple intervals;
    and in those more general situations the connection with Cauchy sequences is still crucial. This connection leads to the following question:
    If ${\xi} \,=\, {(x_{1},x_{2}},\,{\ldots}\,x_{k},\,{\ldots}\,)$ is a Cauchy sequence in the set~$X$,
    under what circumstances must the corresponding sequence $f{\circ}{\xi} \,=\, (f(x_{1}),f(x_{2}),\,{\ldots}\,f(x_{k}),\,{\ldots}\,)$
    also be a Cauchy sequence in~${\RR}$?

\V

        To get a feel for the situation, it helps to consider what must hold for $f$ to {\em not} not have this property.
    The answer is that there must exist a Cauchy sequence ${\xi} \,=\, (x_{1}, x_{2},\,{\ldots}\,x_{k},\,{\ldots}\,)$ in $X$
    for which the corresponding sequence $f{\circ}{\xi}\,=\, (f(x_{1}),f(x_{2}),\,{\ldots}\,f(x_{k})\,{\ldots}\,)$ is not a Cauchy sequence.
    The latter fact would imply that there must exist ${\varepsilon}_{0}\,>\,0$ such that for every $N$ in ${\NN}$, no matter how large,
    there exists $m$ in ${\NN}$ for which $|f(x_{N+m})-f(x_{N})|\,\,{\geq}\,\,{\varepsilon}_{0}$.
    The hypothesis that ${\xi}$ is a Cauchy sequence, however, implies that $x_{N+m}$ and $x_{N}$
    can be made arbitrarily close to each other by simply choosing $N$ sufficiently large. Thus one can conclude that
    in this situation, then the following condition must hold:

\VA

    \underline{Condition A} There exists ${\varepsilon}_{0}\,>\,0$ such that for every ${\delta}\,>\,0$, no matter how small,
    there exist points $c$ and $d$ in $X$ such that $|d-c|\,<\,{\delta}$ but $|f(d)-f(c)|\,\,{\geq}\,\,{\varepsilon}_{0}$.

\VA

\hspace*{\parindent}It follows that the negation of Condition~A is {\em sufficient} to guarantee that if ${\xi}$ is Cauchy then so is~$f{\circ}{\xi}$.
    This negation can be written as follows:

\VA

        \underline{Condition B} For every ${\varepsilon}\,>\,0$ there exists ${\delta}\,>\,0$
    such that for every pair of points $c$ and $d$ in $X$ which satisfy $|d-c|\,<\,{\delta}$ one has $|f(d)-f(c)|\,<\,{\varepsilon}$.

\VA

        If one accepts that it is important to know about such issues, then it makes sense to give a name to Condition~B.
    This name is given in Definition~\Ref{DefD25.65}.


}%EndFootNoteSize
\end{quotation}
%##

\VV


            \subsection{\small{\bf Definition}}\IndBD{functions}{uniform continuity on a set}
            \label{DefD25.65}

\V

\hspace*{\parindent}(1) Let $f:X \,{\rightarrow}\, {\RR}$ be a real-valued function whose domain is  a nonempty subset $X$ of~${\RR}$.
    Suppose that $f$ enjoys the following property on the set~$X$:

\VA

    \h For every ${\varepsilon}\,>\,0$ there exists ${\delta}\,>\,0$ such that for each $c$ in $X$ if $d$ in $X$ satisfies $|d-c|\,<\,{\delta}$,
    then $|f(d)-f(c)|\,<\,{\varepsilon}$.

\VA

\noindent Then one says that the function $f$ is {\bf uniformly continuous on the set~$X$}.

\V

        (2) More generally, suppose as above that $f:X \,{\rightarrow}\, {\RR}$ is a function with domain~$X$, and that $Y$ is a nonempty subset of~$X$.
    One says that $f$ is {\bf uniformly continuous on the subset~$Y$} provided the restriction $f|_{Y}:Y \,{\rightarrow}\, {\RR}$, with domain~$Y$,
    is uniformly continuous on~$Y$ in the sense of Part~(a).

\VV

            \subsection{\small{\bf Examples}}
            \label{ExampD25.65A}

\V

        Let $f:X \,{\rightarrow}\, {\RR}$, with domain $X \,=\, {\RR}\,{\setminus}\,\{0\}$, be given by the formula $f(x) \,=\, 1/x$.
    Note that $f$ is continuous on the set $X$; that is, for each $c \,\,{\neq}\,\, 0$ the function $f$ is continuous at~$c$.

\V

        (1) The function $f$ is uniformly continuous on the interval $[1,+{\infty})$.
    Indeed, note that if $c,d\,\,{\geq}\,\,1$ then
        \begin{displaymath}
        |f(d) - f(c)| \,=\, \left|\frac{1}{d} - \frac{1}{c}\right| \,=\, \left|\frac{c-d}{d\,c}\right|
    \,\,{\leq}\,\,
        |c-d| \,=\, |d-c| \mbox{ since $|d\,c|\,\,{\geq}\,\,1$}.
        \end{displaymath}
    Let ${\varepsilon}\,>\,0$ be given, and let ${\delta} \,=\, {\varepsilon}$. Then the preceding inequality implies that
    if $d$ and $c$ are in the set $[1,+{\infty})$ and satisfy $|d-c|\,<\,{\delta}$, then $|f(d)-f(c)|\,<\,{\varepsilon}$.

\V

        (2) More generally, let $a\,>\,0$ be any positive real number. Then $f$ is uniformly continuous on the interval $[a,+{\infty})$.
    The simple verification is left as an exercise.
    %% EXERCISE (Reduce it to (1).)

\V

        (3) The continuous function $f$ is {\em not} uniformly continuous on the subset $(0,1]$.
    Indeed, one calculates as above that if $0\,<\,c,d\,\,{\leq}\,\,1$ then, as before,
        \begin{displaymath}
        |f(d) - f(c)| \,=\, \left|\frac{1}{d} - \frac{1}{c}\right| \,=\, \left|\frac{c-d}{d\,c}\right|.
        \end{displaymath}
    But now one has $1/(d\,c)\,\,{\geq}\,\,1/c$. Thus,
        \begin{displaymath}
        |f(d)-f(c)|\,\,{\geq}\,\,\frac{|d-c|}{c} \mbox{ for all $d$ in $(0,1]$}
        \end{displaymath}
    In particular, let $d \,=\, c/2$, so that $|d-c| \,=\, c/2$ and $|f(d)-f(c)| \,=\, |f(c/2)-f(c)| \,\,{\geq}\,\,(c/2)/c \,=\, 1/2$.

        Now let ${\varepsilon} \,=\, 1/2$. There is no single ${\delta}\,>\,0$ such that $|d-c|\,<\,{\delta}$
    implies that $|f(d)-f(c)|\,<\,1/2$ for {\em all} choices of $d$ and $c$ such that $0\,<\,d,c\,\,{\leq}\,\,1$.
    Indeed , suppose that such ${\delta}$ did exist. Choose $c \,=\, \min\,\{1, {\delta}\}$ and choose $d \,=\, c/2$.
    Then by the preceding analysis one would have $0\,<\,c\,\,{\leq}\,\,1$ and $0\,<\,d\,<\,1$, so that $|d-c| \,=\, c/2\,<\,{\delta}$.
    By the (purported) construction of ${\delta}$, this would imply that $|f(c/2)-f(c)|\,<\,1/2$, contrary to what was obtained above.

\V

        (4) Let $g:{\RR} \,{\rightarrow}\, {\RR}$ be the `squaring function'; that is, $g(x) \,=\, x^{2}$ for all $x$ in~${\RR}$.
    It is an instructive exercise to show that $g$ is not uniformly continuous on~$[1,+{\infty})$.

\VV


            \subsection{\small{\bf Remark}}
            \label{RemrkD25.20}

\V

        The statement of the preceding definition looks very much like the formulation, in Remark~\Ref{RemrkD20.50}~(2) above,
    of the condition for $f$ to be continuous on~$X$. Indeed, many students think the two formulations are equivalent ways of stating this condition.

        In reality, they are {\em not} equivalent. It is true that being uniformly continuous on a set does imply being continuous on that set.
    However, Example~(3) above illustrates the fact that the converse is not true.

        The key to this difference in the meanings of these concepts is the location of the phrase `there exists ${\delta}\,>\,0$':
    in the formulation of Remark~\Ref{RemrkD20.50}, this phrase comes after the phrase
    `for every ${\varepsilon}\,>\,0$ {\em and} after the phrase `for every~$c$'. This means that the choice of ${\delta}$ depends on
    the previous choices of both ${\varepsilon}$ and~$c$. In the current definition, however,
    the phrase `there exists ${\delta}\,>\,0$' comes after the phrase `for every ${\varepsilon}\,>\,0$' but {\em before} the phrase `for every~$c$'.
    This means that, given a choice of ${\varepsilon}$ one needs a choice ${\delta}$,
    still depending on~${\varepsilon}$, but one which works {\em simultanously} for all~$c$; that is,
    given the choice of ${\varepsilon}$, it requires that a `uniform' choice of ${\delta}$ can be made which works for all~$c$.
    This is why the word `uniform' is included.

\VV

        For future reference it is worth writing down more formally a fact which is alluded to in the preceding remark; its trivial proof is omitted.

\V


            \subsection{\small{\bf Theorem}}
            \label{ThmD25.65C}

\V

        If $f:X \,{\rightarrow}\, {\RR}$ is uniformly continuous on~$X$, then $f$ is continuous on~$X$.
    Otherwise stated: Uniform continuity on a set implies continuity on that set.

\VV

%%%
\begin{quotation}
{\footnotesize \underline{\Note}\IndB{\notes}{on students' confusion over Theorem~\Ref{ThmD25.65C}} (on students' confusion over Theorem~\Ref{ThmD25.65C})
        Some students get confused over the fact that Theorem~\Ref{ThmD25.65C} needs to be stated at all, much less proved.
    They reason, in effect, that the hypothesis `$f$ is uniformlly continuous' {\em means} that $f$ is continuous
    -- so it is pointless to even state this conclusion -- but continuous in a special way, namely `uniformly'.
    From this point of view it is not even necessary to know the meaning of `continuous' to agree that the conclusion is correct.
    In the context of ordinary language, it is similar to saying `The flower is bright pink';
    one does not need to then assert `Therefore the flower is pink', and one need not even know what `bright' or `pink' means,
    just their grammatical roles as adverb and adjective.

    The cause of the confusion, however, is that mathematical usage is often technical, and does not follow the usage of ordinary language.
    For example, the phrase `uniformly continuous' has the same grammatical  {\em format} as the phrase `bright pink': an adverb modifying an adjective.
    However, the former phrase is defined technically, as a single unit, by Definition~\Ref{DefD25.65},
    and not as the grammatical juxtaposition of an adverb (uniformly) and an adjective (continuous), each with their own previously defined meaning.

        One encounters similar confusion in elementary calculus; a good example appears there in the treatment of `infinite series'.
    (If you are not familiar with this topic, simply skip this paragraph.)
    Indeed, one learns to distinguish between a series which is `absolutely convergent' and a series which is `conditionally convergent'.
    In the former case one proves a theorem which says that an absolutely convergent series is convergent;
    this leads calculus students to the same grammatical type of confusion as was just discussed concerning `uniformly continuous'.
    In contrast, one does {\em not} prove that a conditionally convergent series is convergent,
    because the very {\em definition} of `conditionally convergent' includes the hypothesis that the series in question is convergent.
    That is, the phrase `conditionally convergent' {\em does} refer to a series which is convergent, but convergent in a special way.
    (What actually confuses students here instead is why the adverb `conditionally' is chosen to describe this special type of convergence.)
}%EndFootNoteSize
\end{quotation}
%##

\VV

            \subsection{\small{\bf Theorem} (The Uniform-Continuity Theorem in ${\RR}$)}\IndBD{continuity}{uniform-continuity theorem in ${\RR}$}
            \label{ThmD25.70}

\V

        Suppose that $f:X \,{\rightarrow}\, {\infty}$ is a continuous real-valued function on a nonempty closed and bounded subset $X$ of ${\RR}$.
    Then $f$ is uniformly continuous on~$X$.

\V

        {\bf Proof} Suppose that the statement is {\em not} true for some $f$ and $X$ satisfying the hypotheses.
    Then there exists ${\varepsilon}_{0}\,>\,0$ such that for every ${\delta}\,>\,0$ there exist $c$ and $d$ in $X$
    with the property that $|d-c|\,<\,{\delta}$ but $|f(d)-f(c)|\,\,{\geq}\,\,{\varepsilon}_{0}$;
    of course both $c$ and~$d$ depend on~${\delta}$. In particular, for each $k$ in ${\NN}$ let ${\delta}_{k} \,=\, 1/k$.
    Then there must exist $c_{k}$ and $d_{k}$ in $X$ such that
        \begin{displaymath}
        |d_{k}-c_{k}|\,<\,1/k \mbox{ but }|f(d_{k})-f(c_{k})|\,\,{\geq}\,\,{\varepsilon}_{0} \h ({\ast})
        \end{displaymath}
    Since $X$ is bounded, the standard Bolzano-Weierstrass Theorem implies that the sequence
    ${\psi} \,=\, (c_{1},c_{2},\,{\ldots}\,c_{k},\,{\ldots}\,)$ must have a subsequence
    $(c_{k_{1}}, c_{k_{2}},\,{\ldots}\,c_{k_{m}},\,{\ldots}\,)$ which converges to some point $p$, which must be in~$X$ by the hypothesis that $X$ is closed.
    Since $|d_{k_{m}}-c_{k_{m}}|\,<\,1/k_{m}$ for each $m$, it follows easily from the Squeeze Theorem
    for real sequences that the sequence $(d_{k_{1}}, d_{k_{2}},\,{\ldots}\,d_{k_{m}},\,{\ldots}\,)$ also converges to~$p$.
    It now follows from the continuity that $\lim_{m \,{\rightarrow}\, {\infty}} f(d_{k_{m}}) \,=\, \lim_{m \,{\rightarrow}\, {\infty}} f(c_{k_{m}}) \,=\, f(p)$.
    However, this contradicts Statement~$({\ast})$ above, so the desired result follows.

\VV

        {\bf Remark} Many texts attribute the first treatment of `Uniform Continuity' and the `Uniform Continuity Theorem in~${\RR}$'
    to the German mathematician E.~Heine around~$1871$, and refer to Theorem~\Ref{ThmD25.70} as `Heine's Theorem'.
    At various times, however, other names have been associated with these results, including Cantor, Borel, Schoenflies, Cousin and~Lebesgue.

        Historians now seem to agree that the concept and theorem were first clearly presented by Dirichlet in his lecture notes from around $1854$.
    (These lecture notes were first published, however, only in~$1904$, so the confusion is understandable.)
    The name `Uniform Continuity Theorem', given above to Theorem~\Ref{ThmD25.70}, is not used uniformly in analysis (no pun intended),
    but it does have the advantage of being more descriptive than, say, the `Dirichlet-Heine-Cantor-Borel-Schoenflies-Cousin Theorem'.

\VV

        Most of the applications of the Uniform Continuity Theorem appear later in {\ThisText}, but here is a simple one.



            \subsection{\small{\bf Theorem}}
            \label{ThmD25.80}

\V

        Suppose that $f:X \,{\rightarrow}\, {\RR}$ is a uniformly continuous function on a nonempty subset $X$ of~${\RR}$.

\V


        (a) If ${\xi} \,=\, (x_{1}, x_{2},\,{\ldots}\,x_{k},\,{\ldots}\,)$ is a Cauchy sequence in $X$,
    then $f{\circ}{\xi} \,=\, (f(x_{1}), f(x_{2}),\,{\ldots}\,f(x_{k}),\,{\ldots}\,)$ is a Cauchy sequence in~${\RR}$.

\V

        (b) Suppose that ${\xi} \,=\, (x_{1}, x_{2},\,{\ldots}\,x_{k},\,{\ldots}\,)$
    and ${\tau} \,=\, (t_{1}, t_{2}, \,{\ldots}\,t_{k},\,{\ldots}\,)$ are Cauchy sequences in $X$ which both converge to the same number~$L$ in~${\RR}$.
    Then the corresponding sequences $f{\circ}{\xi}$ and $f{\circ}{\tau}$ of values both converge to the same value.

\V

        {\bf Proof}\, (a) Let ${\varepsilon}\,>\,0$ be given, and let ${\delta}\,>\,0$
    be small enough that if $c$ and $d$ are in $X$ such that $|d-c|\,<\,{\delta}$, then $|f(d)-f(c)|\,<\,{\varepsilon}$.
    (Such ${\delta}$ exists by the hypothesis of uniform continuity.) Next, let $N$ in ${\NN}$
    be large enough that $|x_{N+m} - x_{N}|\,<\,{\delta}$ for all $m$ in ${\NN}$. (Such $N$ exists by the `Cauchy' hypothesis.)
    Combining all this then yields $|f(x_{N+m})-f(x_{N})|\,<\,{\varepsilon}$ for all $m$ in~${\NN}$. It follows that $(f{\circ}{\xi})$ is a Cauchy sequence,
    as required.

\V

        (b) It is given that ${\xi}$ and ${\tau}$ both converge to the number~$L$.
    Let ${\sigma} \,=\, (s_{1}, s_{2},\,{\ldots}\,s_{m}, \,{\ldots}\,)$ be the sequence $(x_{1}, t_{1}, x_{2},t_{2},\,{\ldots}\,)$.
    That is, $s_{2\,k-1} \,=\, x_{k}$ and $s_{2\,k} \,=\, t_{k}$ for each index~$k$.
    It follows from the Odd/Even Limit Theorem (see Theorem~\Ref{ThmC20.90}) that the sequence ${\sigma}$ is also convergent to~$L$,
    and thus, by the Cauchy Convergence Theorem, Theorem~\Ref{ThmC70.20}, ${\sigma}$ is a Cauchy sequence.
    By Part~(a) of the present theorem it then follows that $f{\circ}{\sigma}$ is a Cauchy sequence, and thus is convergent.
    By Theorem~\Ref{ThmC20.10A} it then follows that the subsequences $f{\circ}{\xi}$
    and~$f{\circ}{\tau}$ of $f{\circ}{\sigma}$ also converge to the same limit, as required.

\VV

                \section{{\bf Continuity Theorems on Intervals in ${\RR}$}}
                \label{SectD40}\IndB{ZZ Sections}{\Ref{SectD40} Continuity Theorems on Intervals in ${\RR}$}

\V

        {\bf Preliminary Remark} In elementary calculus one considers, almost exclusively, real-valued functions defined on intervals in~${\RR}$.
    This contrasts with the situations considered so far in this chapter, in which the domains can be arbitrary nonempty subsets of ${\RR}$.
    The present section focuses on theorems which make sense only for functions defined on real intervals.

\VV


            \subsection{\small{\bf Theorem} (The Intermediate-Value Theorem -- Standard Form)}
            \label{ThmD25.30}\IndBD{functions}{intermediate-value theorem -- standard form}

\V

        Suppose that $f:[a,b] \,{\rightarrow}\, {\RR}$ is a continuous function on a closed bounded interval $[a,b]$ in~${\RR}$, and $f(a) \,\,{\neq}\,\, f(b)$.
    Then for every number $y_{0}$ strictly between $f(a)$ and $f(b)$ there exists at least one number $x_{0}$, with $a\,<\,x_{0}\,<\,b$,
    such that $f(x_{0}) \,=\, y_{0}$.

        Otherwise stated: If a number $y_{0}$ is \underline{intermediate} between $f(a)$ and $f(b)$,
    then $y_{0}$ is a \underline{value} of $f$ on the interval~$[a,b]$. (This explains the name of the theorem.)

\V
        {\bf Proof} Without loss of generality assume that $f(a)\,<\,f(b)$, and define a bisection sequence
    ${\cal A} \,=\, ([a_{1},b_{1}], [a_{2},b_{2}],\,{\ldots}\,[a_{k},b_{k}],\,{\ldots}\,)$, 
    with initial interval $[a_{1},b_{1}] \,=\, [a,b]$, as follows:

\VA

        \h \underline{Step 1}\, Let $c_{1}$ be the midpoint of the interval $[a,b] \,=\, [a_{1},b_{1}]$.
    If $f(c_{1})\,\,{\leq}\,\,y_{0}$, then set $a_{2} \,=\, c_{1}$ and $b_{2} \,=\, b_{1}$;
    if, instead, $f(c_{1})\,>\,y_{0}$, then set $a_{2} \,=\, a_{1}$ and $b_{2} \,=\, c_{1}$.
    In either case, one has $f(a_{2})\,<\,f(b_{2})$.

        \h \underline{General Step}\,Suppose that $[a_{1},b_{1}]$,\,{\ldots}\, $[a_{m},b_{m}]$, with $m\,\,{\geq}\,\,2$,
    have been constructed so that for each $j \,=\, 1,\,{\ldots}\,m-1$ the interval
    $[a_{j+1},b_{j+1}]$ is one of the halves of the interval $[a_{j}, b_{j}]$ and one has $f(a_{m})\,<\,f(b_{m})$.
    Now let $c_{m}$ be the midpoint of $[a_{m},b_{m}]$, and define $a_{m+1} \,=\, c_{m}$ and $b_{m+1} \,=\, b_{m}$ if $f(c_{m})\,\,{\leq}\,\,y_{0}$,
    while $a_{m+1} \,=\, a_{m}$ and $b_{m+1} \,=\, c_{m}$ if $f(c_{m})\,>\,y_{0}$.

\VA

\noindent The resulting bisection sequence
    $[a_{1},b_{1}] \,{\supseteq}\, [a_{2},b_{2}] \,{\supseteq}\, \,{\ldots}\, \,{\supseteq}\, [a_{m},b_{m}] \,{\supseteq}\, $
    then satisfies the condition $f(a_{m})\,\,{\leq}\,\,y_{0}\,<\,b_{m}$ for each index~$m$.
    The Bisection Principle then implies that $\lim_{m \,{\rightarrow}\, {\infty}} a_{m} \,=\, \lim_{m \,{\rightarrow}\, {\infty}} b_{m} \,=\, c$,
    where $c$ is the unique number which lies in each of the intervals of the bisection principle.
    It follows that $\lim_{m \,{\rightarrow}\, {\infty}} f(a_{m}) \,=\, f(c) \,=\, \lim_{m \,{\rightarrow}\, {\infty}} f(b_{m})$.
    Since $f(x_{m})\,\,{\leq}\,\,y_{0}\,\,{\leq}\,\,f(b_{m})$ for all~$m$, it follows that $f(c)\,\,{\leq}\,\,y_{0}\,\,{\leq}\,\,f(c)$;
    that is, $f(c) \,=\, y_{0}$, and the desired result follows.

\V

        {\bf Remark} The proof given here is in essence the same as the proofs of Bolzano and Cauchy; see [BOLZANO~1817] and [CAUCHY~1821].

\V

        {\bf Example} If $f(x) \,=\, x^{2}$ and $y_{0} \,=\, 2$, then the procedure described above
    is essentially the same as that used to compute $\sqrt{2}$ in Example~\Ref{ExampB25.90}.

\VV


            \subsection{\small{\bf Corollary} (Intermediate-Value Theorem, Extended Form)}\IndBD{functions}{intermediate-value theorem, extended form}
            \label{CorD25.40}

\V

        Suppose that $f:I \,{\rightarrow}\, {\RR}$ is a continuous real-valued function defined on an interval $I$ in ${\RR}$;
    the interval $I$ need not be bounded or closed. Let $S \,=\, f[I]$ be the image of $I$ under $f$.

\V
        Let $A \,=\, {\inf}\,S$ and $B \,=\, {\sup}\,S$. If $y_{0}$ is any number such that $A\,<\,y_{0}\,<\,B$,
    then there exists a number $x_{0}$ in $I$ such that $f(x_{0} \,=\, y_{0})$.

        Equivalently, the image $f[I]$ of $f$ on $I$ is either a singleton set (if $f$ is constant on~$I$), or an interval (if $f$ is nonconstant on~$I$).

\V

        {\bf Proof} If $A \,=\, B$, so that $f$ is constant on~$I$, then $I$ is the singleton $\{A\}$.
    In this case the hypothesis $A\,<\,y_{0}\,<\,B$ is never satisfied, so the conclusion is automatically true.
    Thus, suppose, instead, that $A\,<\,B$, so that $f$ is not constant on~$I$. Let $y_{0}$ be such that $A\,<\,y_{0}\,<\,B$.
    It follows from the approximation properties for infima and suprema that there exist numbers $c$ and $d$
    in $I$ such that $A\,<\,c\,<\,y_{0}\,<\,d\,<\,B$. Since $c$ and $d$ are in $I$, it follows that
    there exist numbers $a$ and $b$ in $I$ such that $c \,=\, f(a)$ and $d \,=\, f(b)$. Since $c \,\,{\neq}\,\, d$ it follows,
    from the basic definition of `function' that $a \,\,{\neq}\,\, b$. It now follows from the preceding theorem that
    there exists $x_{0}$ between $a$ and $b$, so that $x_{0}$ is an element of the interval~$I$, such that $f(x_{0}) \,=\, y_{0}$, as required.

        The fact that $f[I]$ is either a singleton set or an interval now follows easily.
    

\VV

        The property described in the previous results is given a name.

\V

            \subsection{\small{\bf Definition}}
            \label{DefD25.50A}

\V

        Let $f:I \,{\rightarrow}\, {\RR}$ be a real-valued function whose domain is an interval~$I$.

\V

        (1) One says that $f$ has the {\bf weak intermediate-value property on~$I$}\IndBD{functions}{intermediate-value property, weak}
    provided that if $a$ and $b$ are any numbers in $I$ such that $f(a) \,\,{\neq}\,\, f(b)$,
    then for each number $y_{0}$ strictly between $f(a)$ and $f(b)$ there exists $x_{0}$ in $I$ such that $f(x_{0}) \,=\, y_{0}$.

\V

        (2) One says that $f$ has the {\bf strong intermediate-value property on~$I$}\IndBD{functions}{intermediate-value property, strong}
    provided that if $a$ and $b$ are any numbers in $I$ such that $f(a) \,\,{\neq}\,\, f(b)$,
    then for each number $y_{0}$ strictly between $f(a)$ and $f(b)$ there exists $x_{0}$, strictly between $a$ and $b$, such that $f(x_{0}) \,=\, y_{0}$.

\VV

        {\bf Remarks}(1) To say that $f:I \,{\rightarrow}\, {\RR}$ has the weak intermediate-value property on an interval~$I$
    is equivalent to saying that the image set $f[I]$ is either a singleton set (if $f$ is constant on~$I$), or else $f[I]$ is is an interval in~${\RR}$.
    Likewise, to say that $f:I \,{\rightarrow}\, {\RR}$ has the strong intermediate-value property on~$I$
    is equivalent to saying that for each pair of numbers $a$ and $b$ in $I$ with $a\,<\,b$
    the image of $[a,b]$ under $f$ is either a singleton or an interval.

\V

        (2) It is clear that if a function has the strong intermediate-value property on an interval,
    then it certainly also has the weak intermediate-value property on that interval.

\VV

        {\bf Examples} (1) The Intermediate-Value Theorem implies that every function
    which is continuous on an interval $I$ has the strong intermediate-value property on~$I$.

\V

        (2) Consider the function $f:[0,1] \,{\rightarrow}\, {\RR}$ given by the rule
        \begin{displaymath}
        f(x) \,=\, x \mbox{ if $0\,<\,x\,<\,1$};\, f(0) \,=\, 1,\,f(1) \,=\, 0.
        \end{displaymath}
    It is easy to verify that this function has the weak intermediate-value property on the interval~$[0,1]$, but not the strong.

\VV


        As late as the mid-nineteenth century many mathematicians believed that if a function had the strong intermediate-value property on an interval,
    then it must be continuous on that interval. In $1875$ the French mathematician G.~Darboux gave an example of a function
    $f:[0,1] \,{\rightarrow}\, {\RR}$ which has the strong intermediate-value property on~$[0,1]$ but is discontinuous at~$0$.
    Much later the English mathematician J.~Conway gave an example of a function $g:[0,1] \,{\rightarrow}\, {\RR}$ which also has the strong intermediate-value property but is discontinuous at every point of~$[0,1]$.

        The next definition and theorem suggest that such examples must be fairly complicated.

\V

            \subsection{\small{\bf Definition}}
            \label{DefD25.50B}

\V

        A function $f:[a,b] \,{\rightarrow}\, {\RR}$ defined on a closed bounded interval $[a,b]$ is said to be
    {\bf piecewise monotonic on~$[a,b]$}\IndBD{functions}{piecewise monotonic on an interval} provided either

\VA

        \h (i)\,$f$ is monotonic on~$[a,b]$; or

        \h (ii) there exists a partition $a \,=\, x_{0}\,<\,x_{1}\,<\,x_{2}\,<\,\,{\ldots}\,\,<\,x_{n} \,=\, b$ of $[a,b]$
    such that $f$ is monotonic on each closed subinterval of the form $[x_{j-1},x_{j}]$ for $j \,=\, 1,2,\,{\ldots}\,n$.

\VA

\noindent Roughly speaking, $f$ can be `patched together' with finitely many monotonic functions.
    It is allowed that some of these functions be monotonic up, while the others are monotonic down.

\V

        {\bf Remark} Note that Example~(2) above is {\em not} piecewise monotonic on~$[0,1]$.

\V

            \subsection{\small{\bf Theorem}}
            \label{ThmD25.55A}

\V

        Let $I \,=\, [a,b]$ be a closed bounded interval in~${\RR}$.

\V

        (a) Suppose that $f:I \,{\rightarrow}\, {\RR}$ is a monotonic function on the interval~$I$.
    Then $f$ is continuous on $I$ if, and only if, it has the weak intermediate-value property on~$I$.
    The result remains true if `weak' is replaced by `strong'.

\V

        (b) Suppose that $g:I \,{\rightarrow}\, {\RR}$ is a function such that, for each $a$ and $b$ in $I$ with $a\,<\,b$,
    $g$ is piecewise monotonic on $[a,b]$. Then $g$ is continuous on~$I$ if, and only if, it has the strong intermediate-value property on~$I$.

\V

        The simple proof is left as an exercise. %% EXERCISE

\V

        {\bf Example} For each $n$ in ${\NN}$ let $P_{n}$ be the ordered pair ${\displaystyle \left(\frac{1}{n}, (-1)^{n-1}\right)}$.
    Thus, $P_{1} \,=\, (1,1)$, $P_{2} \,=\, (1/2,-1)$, and so on; speaking geometrically,
    if $n$ is odd then $P_{n}$ is the point in the Euclidean plane with abscissa $1/n$ and ordinate~$1$,
    while if $n$ is even then $P_{n}$ is the point with abscissa $1/n$ and ordinate~$-1$.

        Let $I \,=\, (0,1]$, and define a function $g:I \,{\rightarrow}\, {\RR}$ as follows:
    if $0\,<\,x\,\,{\leq}\,\,1$, let $n$ be the largest natural number such that ${\displaystyle \frac{1}{n+1}\,<\,x\,\,{\leq}\,\,\frac{1}{n}}$.
    Then set $g(x)$ equal to the value of the linear function which interpolates the points $P_{n+1}$ and $P_{n}$.
    It is easy to see that if $0\,<\,a\,<\,b\,\,{\leq}\,\,1$, then the function $g$ is continuous, and thus has the strong intermediate-value property, 
    on~$[a,b]$. It follows from the preceding theorem that $g$ is also continuous on~$I$, and thus has the strong intermediate-value property on~$I$.
    Now let $f:[0,1] \,{\rightarrow}\, {\RR}$ be given by $f(x) \,=\, g(x)$ if $0\,<\,x\,\,{\leq}\,\,1$, and $f(0) \,=\, 0$.
    It is easy to see that $f$ also has the strong intermediate-value property on~$[0,1]$, but is {\em not} continuous at~$0$.

\V

        {\bf Remarks} (1) The preceding example is a simplified version of Darboux's original example.

\V

        (2) For those whose French pronunciation is weak: the `boux' in `Darboux' is pronounced like the English word `boo'.

\VV


        There are several other important results which hold for continuous monotonic functions.

\V

            \subsection{\small{\bf Theorem}}
            \label{ThmD25.55B}

\V

\hspace*{\parindent}(a) Suppose that $f:I \,{\rightarrow}\, {\RR}$ is a function which is continuous on an interval~$I$ which has at least one endpoint.
    Let $J$ be the set of interior points of~$I$. If the restriction of $f$ to $J$ is monotonic on~$J$, then $f$ is monotonic on~$I$.

\V

        (b) If the word `monotonic' in Part~(a) is replaced throughout by the phrase `strictly monotonic', the resulting statement remain true.

\V

        {\bf Proof}\, The simple proof using the preceding theorem is left as an exercise. %% EXERCISE

\VV


        If the function $f:I \,{\rightarrow}\, {\RR}$ is known to be both continuous {\em and} strictly monotonic, then even more can be proved.

\V

            \subsection{\small{\bf Theorem}}
            \label{ThmD25.55C}

\V

        Suppose that $f:I \,{\rightarrow}\, {\RR}$ is a continuous strictly monotonic function whose domain is an interval $I$.
    Let $J \,=\, f[I]$ be the image of $f$; note that the `strictly monotonic' hypothesis implies that $f$ is a bijection of $I$ onto~$J$,
    so by Part~(b) of the preceding corollary the image set $J$ is some type of interval in~${\RR}$. Then:

\V

        (a) The intervals $I$ and $J$ contain the same number of endpoints as each other.
    That is, they both contain two endpoints, or both contain one endpoint, or both contain no endpoints.

\V

        (b) The function $f$ is a bijection of $I$ onto $J$, and the inverse function $f^{-1}:J \,{\rightarrow}\, I$ is continuous on $J$.

\V

        \underline{Proof} 

\V

        For brevity consider only the case in which $I$ is an open interval $(a,b)$ and $f$ is strictly increasing.
    Only small changes of the argument, which are left as exercises, are need to handle the other cases.

\V

        (a)\,Let $A \,=\, {\inf}\,\{f(x): a\,<\,x\,<\,b\}$, and let $B \,=\, {\sup}\,\{f(x): a\,<\,x\,<\,b\}$.
    Then $J$ is the open interval $(A,B)$. Indeed, Part~(b) of Corollary~\Ref{CorD25.40}
    implies that the open interval $(A,B)$ is a subset of $J$, and that the only way these intervals can not be equal is if $J$ contains either $A$ or~$B$.
    However, it is impossible to have $A$ be in the set $J$. Indeed, if this were to happen,
    then $A$ would equal $f(x)$ for some $x$ such that $a\,<\,x\,<\,b$. Let $x'$ be a number such that $a\,<\,x'\,<\,x$, so that $x'$ is also in~$I$.
    Then the strict monotonicity would imply that $f(x')\,<\,f(x) \,=\, A$, which would contradict $A$ being a lower bound for the values of~$f$.
    Similarly, it is impossible for $B$ to be in~$J$, so $J \,=\, (A,B)$, as claimed.

\V

        (b) Since $f:(a,b) \,{\rightarrow}\, (A,B)$ is a bijection, it follows that the inverse function $f^{-1}:(A,B) \,{\rightarrow}\, (a,b)$ exists.
    It is clear that $g$ is strictly increasing on $(A,B)$, and its image is the interval~$(a,b)$.
    It then follows from Part~(a) of Theorem~\Ref{ThmD25.55A}, appied to the function~$g$, that $g$ is continuous on $(A,B)$, as claimed.

\VV

%-------------
\StartSkip{
            \subsection{\small{\bf Theorem}}
            \label{ThmD25.55D}

\V

        Suppose that $f:I \,{\rightarrow}\, {\RR}$ is a continuous function on an interval~$I$, and $f$ is one-to-one on~$I$.
    Then $f$ is strictly monotonic in~$I$.
}%\EndSkip
%----------------- GIVE AS AN EXERCISE


            \subsection{\small{\bf Examples}}
            \label{ExampD30.110}

\V

\hspace*{\parindent}(1) Let $n$ be a natural number, and let $f:(0,+{\infty}) \,{\rightarrow}\, {\RR}$
    be the function, with domain $(0,{\infty})$, given by the rule $f(x) \,=\, x^{n}$.
    Clearly $f$ is continuous and strictly increasing on $(0,+{\infty})$, and its inverse function
    $g:(0,+{\infty}) \,{\rightarrow}\, (0,+{\infty})$ is the corresponding `positive $n$-th root' function:
        \begin{displaymath}
        g(x) \,=\, \sqrt[n]{x} \mbox{ for $0\,<\,x\,<\,+{\infty}$}.
        \end{displaymath}
    It follows from the preceding theorem that this function is continuous on $(0,+{\infty})$.

\V

        (2) More generally,  suppose that $r \,=\, m/n$ is a rational number, with $m$ and $n$ in ${\ZZ}$ and $n\,\,{\geq}\,\,1$.
    Define the function $h:(0,+{\infty}) \,{\rightarrow}\, {\RR}$ by the rule
        \begin{displaymath}
        h(x) \,=\, x^{r} \,=\, (\sqrt[n]{x})^{m} \mbox{ for $0\,<\,x\,<\,+{\infty}$}.
        \end{displaymath}
    It is clear that is also continuous. Indeed, it is the composition of the `$m$-th power' function with the function discussed in Example~(1).

        {\bf Remark} It is a simple exercise to show that if the rational number $r$ is expressed in a second way in the form $r \,=\, m'/n'$,
    with $m'$ and $n'$ in ${\ZZ}$ and $n'\,\,{\geq}\,\,1$, then the resulting value of $h(x)$ is the same, at least for $0\,<\,x\,<\,+{\infty}$.
    %% EXERCISE

\VV

        The next result is almost never included in a course in elementary calculus; but we see in Chapter~\Ref{ChaptE} that it is useful in that context.

\VV

            \subsection{\small{\bf Theorem} (The Piecewise-Linear Interpolation Theorem)}\IndBD{functions}{piecewise-linear interpolation}
            \label{ThmD25.60}
\V

        Suppose that $f:[a,b] \,{\rightarrow}\, {\RR}$ is a real-valued function which is continuous on the closed bounded interval~$[a,b]$.
    Let ${\varepsilon}\,>\,0$ be given. Then there exists a continuous piecewise-linear function
    $g:{\RR} \,{\rightarrow}\, {\RR}$ on ${\RR}$ such that $|f(x)-g(x)|\,<\,{\varepsilon}$ for every $x$ in~$[a,b]$.

\V

        {\bf Proof} Let $a \,=\, x_{0}\,<\,x_{1}\,<\,\,{\ldots}\,\,<\,x_{k-1}\,<\,x_{k} \,=\, b$ be any partition of the interval~$[a,b]$ into $k$ subintervals; see Definition~\Ref{DefB20.130}.
    Let $g:{\RR} \,{\rightarrow}\, {\RR}$ be the continuous piecewise-linear function such that $g(x_{j}) \,=\, f(x_{j})$ for each $j \,=\, 0,1,2,\,{\ldots}\,k$; see Example~\Ref{ExampD20.53}~(6).
    If $x$ is any element of the interval $[a,b]$ we need to relate the quantity of interest, namely $|f(x)-g(x)|$, to the given information about $f$ and $g$.

        First note that there exists an index $j$ such that $x_{j-1}\,\,{\leq}\,\,x\,\,{\leq}\,\,x_{j}$.
    In this interval one knows the formula for $g(x)$, given in Example~\Ref{ExampA30.25}~(6).
    Furthermore, the fact that $f$ is continuous on the closed bounded set $[a,b]$, hence {\em uniformly} continuous there,
    implies that one can make $|f(u)-f(x_{j-1})|$ as small as one needs for each $u$ in $[x_{j-1},x_{j}]$ by having $|x_{j}-x_{j-1}|$ sufficently small.
    The connection between this data can be obtained, once again, by a clever use of the `Add-and-Subtract Trick', combined with the Triangle Inequality.
    More precisely, if $x_{j-1}\,\,{\leq}\,\,x\,\,{\leq}\,\,x_{j}$, then one has
        \begin{displaymath}
        f(x) - g(x) \,=\, (f(x) - f(x_{j-1})) + (f(x_{j-1}) - g(x))
     \,=\, 
        (f(x) - f(x_{j-1})) - \left(\frac{f(x_{j}) - f(x_{j-1})}{x_{j} - x_{j-1}}\right)\,(x - x_{j-1}).
        \end{displaymath}
    (Note that this calculation uses the formula for linear interpolation between two points,
    together with the fact that $g(x_{i}) \,=\, f(x_{i})$ for each~$i \,=\, 0,1,\,{\ldots}\,k$.)
   Thus
        \begin{displaymath}
        |f(x) - g(x)| \,\,{\leq}\,\,
        |f(x) - f(x_{j-1})| + \left|\frac{f(x_{j}) - f(x_{j-1})}{x_{j} - x_{j-1}}\right|\,|x - x_{j-1}|\,\,{\leq}\,\,|f(x) - f(x_{j-1})| + |f(x_{j})-f(x_{j-1})| \h ({\ast})
        \end{displaymath}
    since $|x-x_{j-1}|\,\,{\leq}\,\,|x_{j}-x_{j-1}|$.

\V

        Now let ${\varepsilon}\,>\,0$ be given. Let ${\delta}\,>\,0$ be small enough that if $c$ and $d$ are any points of $[a,b]$ such that
    $|d-c|\,<\,{\delta}$, then $|f(d)-f(c)|\,<\,{\varepsilon}/2$; such ${\delta}$ exists because $f$ is uniformly continuous on~$[a,b]$.
    Choose the partition points $x_{j}$ so that $|x_{j}-x_{j-1}|\,<\,{\delta}$ for each~$j$;
    for example, choose $k$ large enough that $|b-a|/k\,<\,{\delta}$, and for that $k$ set ${\displaystyle x_{j} \,=\, a + j\,\left(\frac{b-a}{k}\right)}$.
    It then follows from Inequality~$({\ast})$  above that if $x{\in}[a,b]$, then, as required,
        \begin{displaymath}
        |f(x)-g(x)|\,<\,\frac{{\varepsilon}}{2} + \frac{{\varepsilon}}{2} \,=\, {\varepsilon}.
        \end{displaymath}
%--------------

\VV

                 \section{{\bf Limits and Continuity as in Elementary Calculus}}
                \label{SectD50}\IndB{ZZ Sections}{\Ref{SectD50} Limits and Continuity as in Elementary Calculus}

\VV

        As has been mentioned in the Introduction to this chapter, the way the topics of `limits' and `continuity'
    are treated in elementary calculus is somewhat different from the approach taken in more advanced analysis books.
    It is important to know both approaches, especially for the following chapter.

\VV


            \subsection{\small{\bf Definition}}
            \label{DefD50.10}\IndBD{functions}{limits of functions defined on intervals}\IndC{functions}{limits of functions defined on intervals}{one-sided limits}\IndC{functions}{limits of functions defined on intervals}{two-sided limits}

\V

        Let $c$ and $L$ be extended real numbers.

\V

        (1) Suppose that $f$ is a real-valued function which is defined on some open interval of the form $(a,c)$, where $-{\infty}\,\,{\leq}\,\,a\,<\,c\,\,{\leq}\,\,+{\infty}$.
    One says that {\bf $\Bfm{f}$ has left-hand limit $\Bfm{L}$ at $\Bfm{c}$} provided that for every strictly increasing sequence
    ${\xi} \,=\, (x_{1}, x_{2},\,{\ldots}\,x_{k},\,{\ldots}\,)$ of reals such that $\lim_{k \,{\rightarrow}\, {\infty}} x_{k} \,=\, c$
    one has $\lim_{k \,{\rightarrow}\, {\infty}} f(x_{k}) \,=\, L$. Then one writes $L \,=\, \lim_{x \,{\rightarrow}\, c-} f(x)$
    or $\lim_{x \,{\nearrow}\, c} f(x) \,=\, L$; in spoken form, `$f(x)$ approaches $L$ as $x$ approaches $c$ from below',
    or `$f(x)$ approaches $L$ as $x$ approaches $c$ from the left'.

\V

        (2) Suppose, instead, that $f$ is a real-valued function which is defined on some open interval of the form $(c,b)$, where $-{\infty}\,\,{\leq}\,\,c\,<\,b\,\,{\leq}\,\,+{\infty}$.
    One says that {\bf $\Bfm{f}$ has right-hand limit $\Bfm{L}$ at $\Bfm{c}$} provided that for every strictly decreasing sequence
    ${\xi} \,=\, (x_{1}, x_{2},\,{\ldots}\,x_{k},\,{\ldots}\,)$ of reals such that $\lim_{k \,{\rightarrow}\, {\infty}} x_{k} \,=\, c$
    one has $\lim_{k \,{\rightarrow}\, {\infty}} f(x_{k}) \,=\, L$. Then one writes $L \,=\, \lim_{x \,{\rightarrow}\, c+} f(x)$
    or $\lim_{x \,{\searrow}\, c} f(x) \,=\, L$; in spoken form, `$f(x)$ approaches $L$ as $x$ approaches $c$ from above',
    or `$f(x)$ approaches $L$ as $x$ approaches $c$ from the right'.

\V

    (3) Finally, suppose that $c$ is finite and that $f$ is defined on open intervals of the form $(a,c)$ and $(c,b)$, where $-{\infty}\,\,{\leq}\,\,a\,<\,c\,<\,b\,\,{\leq}\,\,+{\infty}$.
    If both of the equations $\lim_{x  \,{\rightarrow}\, c-} f(x) \,=\, L$ and $\lim_{x \,{\rightarrow}\, c+} \,=\, L$ are true, as described above,
    then one says that {\bf $\Bfm{f}$ has two-sided limit~$\Bfm{L}$ at~$\Bfm{c}$}, or, more simply,
    that {\bf $\Bfm{f}$ has limit~$\Bfm{L}$ at~$\Bfm{c}$}, and one writes $L \,=\, \lim_{x \,{\rightarrow}\, c} f(x)$.

\V


        \underline{Special Cases} If $c$ is the supremum of the domain of $f$, so that the expression $\lim_{x \,{\rightarrow}\, c+} f(x)$ makes no sense,
    then one often abbreviates the equation $\lim_{x \,{\rightarrow}\, c-} f(x) \,=\, L$ to $\lim_{x \,{\rightarrow}\, c} f(x) \,=\, L$.
    Likewise, if $c$ is the infimum of the domain of~$f$, then one often abbreviates the equation
    $\lim_{x \,{\rightarrow}\, c+} f(x) \,=\, L$ to $\lim_{x \,{\rightarrow}\, c} f(x) \,=\, L$.
    In particular, these conventions hold if $c \,=\, \,{\pm}\, {\infty}$. These `abuses of notation' generally cause no problems.

\VV

            \subsection{\small{\bf Remarks}}
            \label{RemrkD50.20}

\V

\hspace*{\parindent}(1) We often write $f(c-)$ for the quantity $\lim_{x \,{\rightarrow}\, c-} f(x)$ provided both $c$ and this limit are finite.
    Likewise, we often write $f(c+)$ for the quantity $\lim_{x \,{\rightarrow}\, c+}$ provided both $c$ and this second limit are finite.
    Of course, using either expression does not require that $f$ b defined at~$c$.

\V

        (2) In Part~(1) of the preceding definition we do not require that the open interval  $(a,c)$ be the full domain of $f$ or that $x_{k}$ be in $(a,c)$ for each index~$k$;
    it suffices that this hold for all but a finite number of the indices. In effect,
    we are using the `Extended Definition' of limits given in Definition~\Ref{DefC10.40}. A similar comment holds for Part~(2).

\V

        (3) In Part (1) of the preceding definition it is required that $f$ be defined on {\em some} open interval of the form $(a,c)$.
    Of course if that is true, then $f$ is also defined on any open interval of the form $(a',c)$, where $a\,\,{\leq}\,\,a'\,<\,c$.
    It may also happen that $f$ is defined on an interval $(a'',c)$ where $a''\,<\,a$.
    It is clear from the preceding remark, however, combined with Part~(b) of Theorem~\Ref{ThmD20.60A},
    that neither the existence of the purported limit $L$, nor its value, depends on the particular choice of this open interval.
    A similar comment holds for the open interval $(c,b)$ in Part~(2) of the definition.

\V

        (4) If $L$ is finite, then one often expresses the equation $\lim_{x \,{\rightarrow}\, c-} f(x) \,=\, L$
    as something like `the function $f$ {\bf converges} to $L$ from the left'. Similar remarks hold for the equations
    $\lim_{x \,{\rightarrow}\, c+} f(x) \,=\, L$ and $\lim_{x \,{\rightarrow}\, c} f(x) \,=\, L$ when $L$ is finite.
    This practice agrees with the usage with sequences, for which one uses the word `converge' only when the limiting value is finite.

\VV

        The restriction to strictly monotonic sequences in the preceding definition is mainly to aid the intuition dealing with one-sides limits.
    The following variation is often used in place of Parts~(1) and~(2).

\V

            \subsection{\small{\bf Theorem}}
            \label{ThmD50.25}

\V

        Let $c$ and $L$ be extended real numbers.

\V

        (a) Suppose that $f$ is a real-valued function defined on some open interval of the form $(a,c)$.
    A necessary and sufficient condition that the equation $L \,=\, \lim_{x \,{\rightarrow}\, c-} f(x)$ hold, in the sense of the preceding definition,
    is that for every sequence ${\xi} \,=\, (x_{1}, x_{2},\,{\ldots}\,x_{k},\,{\ldots}\,)$ of reals such that
    $x_{k}\,<\,c$ for each $k$ and  $\lim_{k \,{\rightarrow}\, {\infty}} x_{k} \,=\, c$ one has $\lim_{k \,{\rightarrow}\, {\infty}} f(x_{k}) \,=\, L$.

\V

        (b) Suppose, instead, that $f$ is a real-valued function which is defined on some open interval of the form $(c,b)$.
    A necessary and sufficient condition that the equation $L \,=\, \lim_{x \,{\rightarrow}\, c+} f(x)$ hold, in the sense of the preceding definition,
    is that for every sequence ${\xi} \,=\, (x_{1}, x_{2},\,{\ldots}\,x_{k},\,{\ldots}\,)$ of reals such that
    $x_{k}\,>\,c$ for each $k$ and  $\lim_{k \,{\rightarrow}\, {\infty}} x_{k} \,=\, c$ one has $\lim_{k \,{\rightarrow}\, {\infty}} f(x_{k}) \,=\, L$.

\V

        (c) Finally, suppose that $c$ is finite $f$ is a real-valued function defined on some open interval containing~$c$, except possibly at~$c$ itself.
    A necessary and sufficient condition that the equation $L \,=\, \lim_{x \,{\rightarrow}\, c} f(x)$ hold, in the sense of the preceding definition,
    is that for every sequence ${\xi} \,=\, (x_{1}, x_{2},\,{\ldots}\,x_{k},\,{\ldots}\,)$ of reals such that for each~$k$
    $x_{k} \,\,{\neq}\,\, c$ and  $\lim_{k \,{\rightarrow}\, {\infty}} x_{k} \,=\, c$ one has $\lim_{k \,{\rightarrow}\, {\infty}} f(x_{k}) \,=\, L$.

\V

        The proof is left as an exercise.

\VV

        In elementary calculus the `limit' concept is normally formulated without the uses of sequences. The next result summarizes that approach.

\V


            \subsection{\small{\bf Theorem}}
            \label{ThmD50.30}

\V

\hspace*{\parindent}(a) The equation $\lim_{x \,{\rightarrow}\, c-} f(x) \,=\, L$ in Part~(a)
    of the preceding definition is equivalent to the following pair of statements being true:

\VA

        \h \underline{Statement A--}\,For each real number $y$ such that $y\,<\,L$, there exists a number $a$, with $a\,<\,c$,
    such that $y\,<\,f(x)$ for all $x$ such that $a\,<\,x\,<\,c$.

        \h \underline{Staement B--}\, For each real number $z$ such that $L\,<\,z$, there exists a number~$a$, with $a\,<\,c$,
    such that $f(x)\,<\,z$ for all $x$ such that $a\,<\,x\,<\,c$.

\V

        (b) The equation $\lim_{x \,{\rightarrow}\, c+} f(x) \,=\, L$ in Part~(b)
    of the preceding definition is equivalent to the following pair of statements being true:

\VA

        \h \underline{Statement A+}\,For each real number $y$ such that $y\,<\,L$, there exists a number $b$, with $c\,<\,b$,
    such that $y\,<\,f(x)$ for all $x$ such that $c\,<\,x\,<\,a$.

        \h \underline{Statement B+} For each real number $z$ such that $L\,<\,z$, there exists a number~$b$, with $c\,<\,b$,
    such that $f(x)\,<\,z$ for all $x$ such that $c\,<\,x\,<\,b$.

\V

        The simple proof is left as a simple application of Theorem~\Ref{ThmC40.80}. %% EXERCISE

\V


            \subsection{\small{\bf Remarks}}
            \label{RemrkD50.35}

\V

\hspace*{\parindent}(1) The phrasing of this theorem is slightly complex because it includes all the possibilities for the extended reals $c$ and~$L$:
    $c$ can be finite or one of the infinities, as can~$L$. Definition~\Ref{DefD50.10}
    hides these complexities under the previously defined concept of a sequence of real numbers having a (possibly infinite) limit; see Theorem~\Ref{ThmC40.80}.

\V

        (2) If $L$ is an upper bound for the image of $f$, then Statements~B-- and~B+ are automatically true.
    Likewise, if $L$ is a lower bound for the image of $f$, then Statements~A-- and~A+ are automatically true.
    (Compare with Remark~\Ref{RemrkC40.80A}.)

\V

        (3) In elementary calculus the normal procedure is to formulate the meaning of the two-sided limit
    `$\lim_{x \,{\rightarrow}\, c} f(x) \,=\, L$' directly, without first defining one-sided limits.

\V

        (4) Let ${\xi} \,=\, (x_{1},x_{2},\,{\ldots}\,x_{k},\,{\ldots}\,)$ be a sequence of real numbers.
    Recall that in the `official' definition of `sequence', ${\xi}$ is actually a real-valued function with domain~${\NN}$;
    with that viewpoint, $x_{k}$ is simply an alternate notation for the function value ${\xi}(k)$.
    The limit of ${\xi}$, if it has one, then can be written as $\lim_{k \,{\rightarrow}\, {\infty}} {\xi}(k)$.

        Suppose now that for some strange reason we decide to label the terms of this sequence using the letter $x$ instead of the letter~$k$,
    and to use the English letter $f$ instead of the Greek letter ${\xi}$ as the `name' of this sequence.
    Then the notation for the sequential limit just written would change to $\lim_{x \,{\rightarrow}\, {\infty}} f(x)$.

        Would this change of notation be legal? Absolutely yes. Would it be wise? Absolutely no, because it would be confusing to the reader:
    the latter notation looks like that introduced in Definition~\Ref{DefD50.10} above.
    In {\ThisText} we shall try to make it clear which version of `limit' is meant; in particular, we shall avoid such `unwise' changes of notation.

\V

            \subsection{\small{\bf Corollary}}
            \label{CorD50.40}


\V

        Let $c$ and $L$ be real (i.e., finite) numbers.

\V

        (a) The equation $\lim_{x \,{\rightarrow}\, c-} f(x) \,=\, L$ in Part~(a) of Definition~\Ref{DefD50.10}
    is equivalent to the following condition: for every ${\varepsilon}\,>\,0$ there is ${\delta}\,>\,0$ such that if $c-{\delta}\,<\,x\,<\,c$,
    then $|f(x)-L|\,<\,{\varepsilon}$.

\V

        (b) The equation $\lim_{x \,{\rightarrow}\, c+} f(x) \,=\, L$ in Part~(b) of Definition~\Ref{DefD50.10}
    is equivalent to the following condition: for every ${\varepsilon}\,>\,0$ there is ${\delta}\,>\,0$ such that if $c\,<\,x\,<\,c+{\delta}$,
    then $|f(x)-L|\,<\,{\varepsilon}$.

\V

        The simple proof is left as an exercise. %% EXERCISE

\V

        {\bf Remark} Note that in this formulation one still does not require that the function $f$ be defined at the point~$c$.
    Indeed, the requirements on $x$ include $x\,<\,c$ in Part~(a) and $x\,>\,c$ in Part~(b).

\VV


        The fact that we have already proved numerous results about limits of sequences
    makes it much easier to prove analogous results for limits of functions on intervals.
    The next three theorems illustrate this statement. In order to simplify the exposition,
    these results focus on `two-sided' limits, which is the case that arises most frequently in practice;
    the reader is given the task to formulate the corresponding `one-sided' results, as well as to prove most portions of these results.
    That is, we consider limits of the form $\lim_{x \,{\rightarrow}\, c} f(x)$, where $c$ is finite,
    and the real-valued function $f$ is defined at every point of some open interval $(a,b)$ containing $c$ except, possibly, at $c$ itself.
    For convenience, write $X_{c} \,=\, (a,b)\,{\setminus}\,\{c\}$, so that $f$ is defined on~$X_{c}$.

\V

            \subsection{\small{\bf Theorem}}
            \label{ThmC90.50}

        Let $f:X_{c} \,{\rightarrow}\, {\RR}$ be as above.

\V

        (a) If the function $f$ has a limit at $c$, then this limit is unique.
    That is, if the statements ${\displaystyle \lim_{x \,{\rightarrow}\, c} f(x) \,=\, L_{1}}$ and ${\displaystyle \lim_{x \,{\rightarrow}\, c} f(x) \,=\, L_{2}}$ are both true, then $L_{1} \,=\, L_{2}$.

\V

        (b) If there is a real number $A$ such that $f(x) \,=\, A$ for all $x$ in $X_{c}$, then ${\displaystyle \lim_{x \,{\rightarrow}\, c} f(x)} \,=\, A$.

\V

        (c) Suppose that there exist real numbers $m$ and $M$, with $m\,<\,M$, such that $f(x){\in}[m,M]$ for all $x$ in the set $X_{c}$.
    If $f$ has a limit at $c$, then ${\displaystyle \lim_{x \,{\rightarrow}\, c} f(x)}$ is also in the set $[m,M]$.

\V

        (d) (Squeeze Property for Limits on Intervals) Let $f$, $g$ and $h$ be real-valued functions defined on the set $X_{c}$,
    and let $L$ be an extended real number. Suppose that the following conditions hold:

        \h (i) $g(x){\in}\mbox{Seg}\,[f(x),h(x)]$ for all $x$ in $X_{c}$.

        \h (ii) ${\displaystyle \lim_{x \,{\rightarrow}\, c} f(x)
     \,=\, \lim_{x \,{\rightarrow}\, c} h(x) \,=\, L}$.

    Then ${\displaystyle \lim_{x \,{\rightarrow}\, c} g(x) \,=\, L}$.

\V

        (e) Suppose that the quantity $L$ is a real number (i.e., $L$ is finite).
    Then the following statements are equivalent:

        \h (i)\,\, ${\displaystyle \lim_{x \,{\rightarrow}\, c} f(x) \,=\, L}$;

        \h (ii)\, ${\displaystyle \lim_{x \,{\rightarrow}\, c} |L-f(x)| \,=\, 0}$;

        \h (iii) ${\displaystyle \lim_{x \,{\rightarrow}\, c} (L-f(x)) \,=\, 0}$.

\V

        (f) Suppose that ${\displaystyle \lim_{x \,{\rightarrow}\, c} f(x) \,=\, 0}$.
    If $g$ is real-valued function which is bounded on the set $X_{c}$, then ${\displaystyle \lim_{x \,{\rightarrow}\, c} f(x){\cdot}g(x) \,=\, 0}$.

\VV

            \subsection{\small{\bf Theorem}}
            \label{ThmC90.60}

\V

        Using the same notation as above, let $f:X_{c} \,{\rightarrow}\, {\RR}$ be a real-valued function defined on the set $X_{c} \,=\, (a,b)\,{\setminus}\,\{c\}$.

\V

        (a) (Constant Factor Rule) Suppose that ${\displaystyle \lim_{x \,{\rightarrow}\, c} f(x) \,=\, L}$ for some real number $L$.
    Then for every real number $A$ one has ${\displaystyle \lim_{x \,{\rightarrow}\, c} (A{\cdot}f)(x) \,=\, A{\cdot}L}$.

\V

        (b) (Absolute-value Rule) Suppose that ${\displaystyle \lim_{x \,{\rightarrow}\, c}} f(x) \,=\, L$, where $L$ is an extended real number.
    Then ${\displaystyle \lim_{x \,{\rightarrow}\, c}} |f(x)| \,=\, |L|$.

\V

        (c) Suppose that ${\displaystyle \lim_{x \,{\rightarrow}\, c} f(x) \,=\, L}$,
    where $L$ is an extended real number. Also, assume that $L \,\,{\neq}\,\, 0$.
    Then there exists ${\delta}\,>\,0$ such that if $0\,<\,|c-x|\,\,{\leq}\,\,{\delta}$ then $f(x) \,\,{\neq}\,\, 0$.

        More precisely:

        \underline{Case (i)} If $L\,>\,0$ and $m$ is any number such that $0\,<\,m\,<\,L$, then there exists ${\delta}\,>\,0$ so that $f(x)\,\,{\geq}\,\,m$ for all $x$ such that $0\,<\,|x-c|\,\,{\leq}\,\,{\delta}$.

        \underline{Case (ii)} If $L\,<\,0$ and $m\,>\,0$ is any number such that $L\,<\,-m\,<\,0$, then there exists ${\delta}\,>\,0$ so that $f(x)\,\,{\leq}\,\,-m$ for all $x$ such that $0\,<\,|x-c|\,\,{\leq}\,\,{\delta}$.

\V

        \underline{Proof of Case~(i) of Part~(c)} Suppose that $L\,>\,0$ and that the conclusion in~(c) is {\em not} true for a given $f$.
    Then there would have to exist a number $m$, with $0\,<\,m\,<\,L$, such that the following holds:
    for every $k$ in ${\NN}$ there exists a number $x_{k}$ in the set $X_{c}$ such that $0\,<\,|c-x_{k}|\,<\,1/k$ and $f(x_{k})\,<\,m$.
    Let ${\xi} \,=\, (x_{1},x_{2},\,{\ldots}\,)$ be the corresponding infinite sequence.
    Since, by construction, one has $|c-x_{k}|\,<\,1/k$, it is clear that the sequence ${\xi}$ converges to $c$.
    Since $x_{k}$ is never equal to~$c$, it follows that ${\xi}$ does not have a constant subsequence.
    Theorem~\Ref{ThmC30.15}, the `Monotonic-Subsequences Theorem', implies that ${\xi}$ has a strictly monotonic subsequence
    ${\zeta} \,=\, (z_{1},z_{2},\,{\ldots}\,z_{n},\,{\ldots}\,)$ which converges to~$c$.
    It follows from Definition~\Ref{DefD50.10} that $\lim_{n \,{\rightarrow}\, {\infty}} f(z_{m})\,=\,L$.
    However, one also has, by construction, $f(x_{k})\,<\,m$ for each $k$, hence $f(z_{n})\,<\,m$ for each~$n$.
    Thus by Part~(c) of Theorem~\Ref{ThmC20.10B} one also has $L\,\,{\leq}\,\,m$, which contradicts the hypothesis that $0\,<\,m\,<\,L$.

\VV

            \subsection{\small{\bf Theorem}}
            \label{ThmC90.70}

        Using the same notation as above, let $f$ and $g$ be real-valued functions defined on the set $X_{c} \,=\, (a,b)\,{\setminus}\,\{c\}$.
    Assume that ${\displaystyle \lim_{x \,{\rightarrow}\, c} f(x) \,=\, A}$ and ${\displaystyle \lim_{x \,{\rightarrow}\, c} g(x) \,=\, B}$,
    where $A$ and $B$ are real numbers. Then:

\V
        (a) (Sum/Difference Rules)\IndBD{functions}{sum and difference limit rules for functions on intervals}
    ${\displaystyle \lim_{x \,{\rightarrow}\, c} (f(x)+g(x)) \,=\, A+B}$ and ${\displaystyle \lim_{x \,{\rightarrow}\, c} (f(x)-g(x)) \,=\, A-B}$.

\V

        (c) (Product Rule)\IndBD{functions}{product limit rule for functions on intervals}
    ${\displaystyle \lim_{x \,{\rightarrow}\, c} (f(x){\cdot}g(x)) \,=\, A{\cdot}B}$.

\V

        (d) (Quotient Rule)\IndBD{functions}{quotient limit rule for functions on intervals}
    Suppose that, in addition, $g(x) \,\,{\neq}\,\, 0$ when $x{\in}X_{c}$, and that $B \,\,{\neq}\,\, 0$.
    Then ${\displaystyle \lim_{x \,{\rightarrow}\, c} (f(x)/g(x)) \,=\, A/B}$.

    Note: The requirement that $g(x) \,\,{\neq}\,\, 0$ when $x \,\,{\neq}\,\, c$ is included just to simplify the statement of the result.
    In light of Part~(c) of Theorem~\Ref{ThmC90.60}, however, it is clear that this condition can be dropped since $B \,\,{\neq}\,\, 0$.

\VV


            \subsection{\small{\bf Remark}}
            \label{RemrkC90.80}

        The preceding result provides `Sum, `Difference', `Product' and `Quotient' rules for limits of functions on intervals.
    It is natural to ask about a correponding `Composition' rule for limits of such functions.
    The obvious statement would be something like this:

        `If ${\displaystyle \lim_{x \,{\rightarrow}\, c}} f(x) \,=\, b$ and ${\displaystyle \lim_{y \,{\rightarrow}\, b} g(y) \,=\, L}$, then
    ${\displaystyle \lim_{x \,{\rightarrow}\, c} g(f(x)) \,=\, L}$.'

\noindent Intuitively, this equation seems quite plausible:  As $y$ gets close to $b$, $g(y)$ gets close to $L$,
    and as $x$ gets close to $a$, the quantity $y \,=\, f(x)$ gets close to $b$;
    thus combining these facts seems to imply the `law' stated above.

    Unfortunately, this `law' is \underline{not} true in general. The underlying reason is the fact that
    in the expression $\lim_{y \,{\rightarrow}\, b} g(y)$ the value of $g$ at $b$ -- or even whether $g$ is defined at $b$ -- is irrelevent.

\V

        \underline{Example}: Suppose that $f(x) \,=\, 1$ for all $x$ in ${\RR}$.
    Likewise, suppose that $g:{\RR} \,{\rightarrow}\, {\RR}$ is given by the rule that $g(y) \,=\, 4$ if $y \,\,{\neq}\,\, 1$, but $g(1) \,=\, 0$.
    It is clear that
        \begin{displaymath}
        \lim_{x \,{\rightarrow}\, 0} f(x) \,=\, 1 \mbox{ and } 
        \lim_{y \,{\rightarrow}\, 1} g(y) \,=\, 4
        \end{displaymath}
    However, $g(f(x)) \,=\, g(1) \,=\, 0$ for all $x$, so that ${\displaystyle \lim_{x \,{\rightarrow}\, 0} g(f(x))} \,=\, 0$, not $4$ as the `law' would have predicted.

\VV

        There are also analogs for `limits on intervals' of the Monotonic-Sequences Principle (i.e., Part~(a) of Theorem~\Ref{ThmC20.10B}).
    Of necessity, these analogs involve one-sided limits.

\V

            \subsection{\small{\bf Theorem (Limits for Monotonic Functions)}}
            \label{ThmC90.90}
\V

        Let $(a,b)$ be an open interval in ${\RR}$, so that $-{\infty}\,\,{\leq}\,\,a\,<\,b\,\,{\leq}\,\,+{\infty}$.

\V

       (a) Suppose that a function $f:(a,b) \,{\rightarrow}\, {\RR}$ is monotonic up on $(a,b)$. Then the one-sided limits $\lim_{x{\nearrow}b} f(x)$ and $\lim_{x{\searrow}a} f(x)$ both exist.
    More precisely,
        \begin{displaymath}
        (i)\,\,\lim_{x{\nearrow}b} f(x) \,=\, {\sup}\,\{f(x): a\,<\,x\,<\,b\}
    \mbox{ and }
        (ii)\,\lim_{x{\searrow}a} f(x) \,=\, {\inf}\,\{f(x): a\,<\,x\,<\,b\} \h
        ({\ast})
        \end{displaymath}
    In particular, 
        \begin{displaymath}
        \lim_{x{\searrow}a} f(x)\,\,{\leq}\,\,f(u)\,\,{\leq}\,\,\lim_{x{\nearrow}b} f(x) \mbox{ for all $u$ in $(a,b)$}.
        \end{displaymath}

\V

        (b) Suppose, instead, that the function $f:(a,b) \,{\rightarrow}\, {\RR}$ is monotonic down on $(a,b)$.
    Then the one-sided limits $\lim_{x{\nearrow}b} f(x)$ and $\lim_{x{\searrow}a} f(x)$ both exist.
    More precisely,
        \begin{displaymath}
        (i)\,\,\lim_{x{\nearrow}b} f(x) \,=\, {\inf}\,\{f(x): a\,<\,x\,<\,b\}
    \mbox{ and }
        (ii)\,\lim_{x{\searrow}a} f(x) \,=\, {\sup}\,\{f(x): a\,<\,x\,<\,b\} \h
        ({\ast}{\ast})
        \end{displaymath}
    In particular, 
        \begin{displaymath}
        \lim_{x{\searrow}a} f(x)\,\,{\geq}\,\,f(u)\,\,{\geq}\,\,\lim_{x{\nearrow}b} f(x) \mbox{ for all $u$ in $(a,b)$}.
        \end{displaymath}

    

\V

        \underline{Partial Proof} We prove here only Equation~(i) of~$({\ast})$ in Part~(a). The rest of the proof is quite similar and is left as an exercise.
    % EXERCISE

\V

        Let $L \,=\, {\sup}\,\{f(x): a\,<\,x\,<\,b\}$. It suffices to show that Statement~A+ of Part~(b) of Theorem~\Ref{ThmD50.30} holds;
    indeed, by Remark~\Ref{RemrkD50.35}~(2), Statement~B+ of that theorem is true automatically since $L$ is an upper bound of the image of~$f$.
    Thus, let $y$ be a number such that $y\,<\,L$. By the Approximation Property for Suprema, there exists $u$ in $(a,b)$ such that $y\,<\,f(u)\,\,{\leq}\,\,L$.
    Since $f$ is monotonic up, it follows that if $u\,<\,x\,<\,b$ then $f(u)\,\,{\leq}\,\,f(x)\,\,{\leq}\,\,L$.
    That is, Statement~A+ of Part~(b) of Theorem~\Ref{ThmD50.30} holds, and the desired result follows.

\VV

        {\bf Continuity in Elementary Calculus} The definition of `continuity' used in elementary calculus follows the approach to `limits' studied here.
    
\V

            \subsection{\small{\bf Definition} (Continuity as Defined in Elementary Calculus)}
            \label{DefD50.50}\IndBD{continuity}{one-sided, two-sided continuity as in calculus}

\V

\hspace*{\parindent}(a) A real-valued function $f$ is said to {\bf continuous from the left at a number $\Bfm{c}$ in the sense of calculus}
     provided that the following conditions hold:

\VA

        \h (i)\,\, The function $f$ is defined at $c$;

        \h (ii)\, The limit $\lim_{x \,{\rightarrow}\, c-} f(x)$ exists;

        \h (iii) $\lim_{x \,{\rightarrow}\, c-} f(x) \,=\, f(c)$.

\VA

        (b) If, instead Conditions (i), (ii) and (iii) hold if one replaces $c-$ by $c+$ throughout,
    then one says that {\bf continuous from the right at $\Bfm{c}$ in the sense of calculus}.

\V

        (c) If $f$ is continuous both from the left and from the right at~$c$, then one says that {\bf $\Bfm{f}$ is continuous at~$\Bfm{c}$}.

\V

        {\bf Remarks} (1) Recall that in {\ThisText} the symbols $f(c-)$ and $f(c+)$ are allowed only when the corresponding one-sided limits are finite;
    see Remark~\Ref{RemrkD50.20}~(1).

\V

        (2)\,In Part (a) of the current definition the use of the left-hand limit $f(c-)$ in (ii)
    tacitly assumes that $f$ is defined on some open interval of the form $(a,c)$. Also, our convention
    Likewise, in Part~(b) there is the tacit assumption that $f$ is defined on an open interval of the form $(c,b)$.
    Finally, in Part~(c) there is the tacit assumption that $f$ is defined at every point of an open interval of the form $(a,b)$ such that $a\,<\,c\,<\,b$,
    with the possible exception of $c$ itself. The hypothesis that $f$ is actually defined at $c$ is treated separately by assuming Condition~(i).

\VV

        It is easy to see that the preceding definition agrees with Definition~\Ref{DefD20.30} in the case $X$ is an interval.
    Thus the general results about continuity obtained earlier in this chapter are valid in this context as well.

\VV

        In Part~(a) of the preceding definition, if Conditions (i) and~(ii) hold,
    then the quantity $f(c) - f(c-)$ measures the extent to which Condition~(iii) holds -- or fails to hold:
    $f$ is continuous from the left at~$c$ if, and only if, this difference is~$0$.
    A similar comment holds for right-hand limits: $f$ is continuous from the right at $c$ if, and only if, $f(c+)-f(c) \,=\, 0$.
    These quantities are given names.

\VV

            \subsection{\small{\bf Definition} (Jump Discontinuities)}
            \label{DefD50.55}\IndBD{continuity}{jump discontinuities}

\V

\hspace*{\parindent}(1) Suppose that a function $f$ is defined at a real number $c$ and has (finite) left-hand limit $f(c-)$ at~$c$.
    Then the quantity $f(c) - f(c-)$ is called the {\bf left-hand jump of $\Bfm{f}$ at~$c$}.

        If the left-hand jump of $f$ is not zero, then one says that $f$ has a {\bf left-hand jump discontinuity} at~$c$.

\V

        (2) Likewise, if $f$ is defined at a number $c$ and has (finite) right-hand limit $f(c+)$ at~$c$,
    then the quantity $f(c+) - f(c)$ is called the {\bf right-hand jump of $\Bfm{f}$ at~$c$}.

        If the right-hand jump of $f$ is not zero, then one says that $f$ has a {\bf right-hand jump discontinuity} at~$c$.

\V

        (3) If both (1) and (2) are applicable, then the sum of the left-hand jump and the right-hand jump of $f$ at $c$
    is called the {\bf total jump of $\Bfm{f}$ at $\Bfm{c}$}. That is, it is the quantity
        \begin{displaymath}
        (f(c) - f(c-)) + (f(c+) - f(c)); \mbox{ that is, } f(c+) - f(c-)
        \end{displaymath}
    If either the left-hand jump or the right-hand jump of $f$ is not zero, then. one says that $f$ has a {\bf jump discontinuity} at~$c$.

        \underline{Note} Example~(1) below shows why the definition of `jump discontinuity' here is {\em not} simply `total jump equals zero'.

\VV

            \subsection{\small{\bf Examples}}
            \label{ExampD50.55A}

\V

        (1) Let $f:{\RR} \,{\rightarrow}\, {\RR}$ be given by the rule
        \begin{displaymath}
        f(x) \,=\, 1 \mbox{ if $x \,\,{\neq}\,\, 0$}, f(0) \,=\, A \mbox{ for some real number~$A$}.
        \end{displaymath}
    It is clear that $f(0-) \,=\, f(0+) \,=\, 1$, so that the left-hand jump of $f$ at~$0$ is $A-1$ while the corresponding right-hand jump is~$1-A$.
    The total jump at $0$ is then $(A-1) + (1-A) \,=\, 0$. It is clear that $f$ is continuous at~$0$ if $A \,=\, 1$,
    discontinuous there if $A \,\,{\neq}\,\, 1$. In particular, if one redefines $f$ at the one point $x \,=\, 0$ by resetting $f(0) \,=\, 0$,
    then the `new' $f$ is continuous there. More generally,  if $f$ satisfies $f(c-) \,=\, f(c+) \,=\, L$,
    but $f(c) \,\,{\neq}\,\, L$,
then one says that $f$ has a {\bf removable discontinuity at~$\Bfm{c}$}. It is `removable' in the sense that by merely redefining the value of $f(c)$,
    and at not other values of~$x$, the resulting function is continuous at~$c$.
 

\V

        (2) Let $f:{\RR} \,{\rightarrow}\, {\RR}$ be given by the rule
        \begin{displaymath}
        f(x) \,=\, \frac{|x|}{x} \mbox{ if $x \,\,{\neq}\,\, 0$}; \h
        f(0) \,=\, A \mbox{ for some real number~$A$}.
        \end{displaymath}
        Note that $f(x) \,=\, -1$ if $x\,<\,0$, $f(x) \,=\, +1$ if $x\,>\,0$. It is clear that
    $\lim_{x \,{\rightarrow}\, 0-} f(x) \,=\, -1$ and $\lim_{x \,{\rightarrow}\, 0+} f(x) \,=\, +1$.
    Thus the left-hand jump of $f$ at $0$ is $A-(-1) \,=\, A+1$, while the corresponding right-hand jump is $1-A$. The total jump is then~$2$.
    Because the left-side and right-side limits of $f$ at~$0$ are not equal, the function $f$ is discontinuous at $x \,=\, 0$,
    regardless of the value $A$ of $f(0)$. In other words, the discontinuity of this $f$ at $c$ cannot be removed by merely changing the value of~$f(0)$.

\V

        (3) It is easy to see that if $f:{\RR} \,{\rightarrow}\, {\RR}$ is the Dirichlet function (see Example~\Ref{ExampA30.25}),
    then for every $c$ in ${\RR}$ neither one-sided limit exists. In contrast, if $g:{\RR} \,{\rightarrow}\, {\RR}$
    is the Thomae function (see the same definition), then $g$ does have both left-sided and right-sided limits at every $c$ in~${\RR}$;
    indeed, both equal $0$ at each~$c$. The discontinuities of this function occur precisely at the rational numbers, and each one is removable.

\VV

        There is an important class of functions for which the nature of any discontinuities is easy to analyse.

\V


            \subsection{\small{\bf Theorem}}
            \label{ThmD60.60}
\V

        Suppose that $f:I \,{\rightarrow}\, {\RR}$ is monotonic on an interval $I$, and let $c$ be a point of~$I$.


        (a) If $c$ is an interior point of $I$, then both of the one-sided limits $\lim_{x \,{\rightarrow}\, c-} f(x)$ and
    $\lim_{x \,{\rightarrow}\, c+} f(x)$ exist and are finite. If, instead, $c$ is an endpoint of~$I$,
    then the corresponding one-sided limit exists and is finite.

\V

        (b) If $c$ is an endpoint of $I$, then $f$ is continuous at $c$ if, and only if, the corresponding one-sided jump of $f$ at $c$ equals~$0$.
    Likewise, if $c$ is an interior point of $I$, then $f$ is continuous at $c$ if, and only if, the total jump $f(c+)-f(c-)$ of $f$ at $c$ equals~$0$.

\V



        {\bf Proof} (a) Assume first that $f$ is monotonic up on~$I$ and that $c$ is an interior point of~$I$; the other cases can be handled similarly.
    Let $a$ and $b$ be numbers in $I$ such that $a\,<\,c\,<\,b$. By the monotonicity hypothesis one has
    $f(a)\,\,{\leq}\,\,f(x)\,\,{\leq}\,\,f(c)$ for all $x$ in $(a,c)$. Thus, by properties of `supremum', one has
        \begin{displaymath}
        f(a)\,\,{\leq}\,\,{\sup}\,\{f(x): a\,<\,x\,<\,c\}\,\,{\leq}\,\,f(c)
        \end{displaymath}
    Part~(a) of the preceding theorem then implies that $f(a)\,\,{\leq}\,\,f(c-)\,\,{\leq}\,\,f(c)$.
    Similarly, apply Part~(a) of the preceding theorem to the interval $(c,b)$ to get $f(c)\,\,{\leq}\,\,f(c+)\,\,{\leq}\,\,f(b)$. Combine these to get
        \begin{displaymath}
        f(a)\,\,{\leq}\,\,f(c-)\,\,{\leq}\,\,f(c)\,\,{\leq}\,\,f(c+)\,\,{\leq}\,\,f(b).
        \end{displaymath}
    In particular, both $f(c-)$ and $f(c+)$ exist and are finite, as claimed.

        If $f$ is monotonic down on~$I$ and $c$ is an interior point of~$I$, then apply what was just obtained to the function $g \,=\, -f$.
    Finally, the modifications needed in the preceding argument to cover the case in which $c$ is an endpoint of~$I$ are left as an easy exercise. % EXERCISE

\V

        (b) The case in which $c$ is an endpoint of $I$ reduces to Definitions~\Ref{DefD50.50} and~\Ref{DefD50.55}.
    Thus, suppose that $c$ is an interior point of~$I$. Note that the `only if' portion of the claim also follows from the same definitions.

    To prove the `if' portion when $c$ is an interior point of~$I$, first assume that $f$ is monotonic up on~$I$.
    From the definition of `total jump' one has
        \begin{displaymath}
        f(c+) - f(c-) \,=\, (f(c+) - f(c)) + (f(c)-f(c-)).
        \end{displaymath}
    Since $f$ is monotonic up on $I$, one has $f(c+)-f(c)\,\,{\geq}\,\,0$ and $f(c)-f(c-)\,\,{\geq}\,\,0$.
    It follows that $f(c+) - f(c-) \,=\, 0$ if, and only if, $f(c+)-f(c) \,=\, 0$ and $f(c)-f(c-) \,=\, 0$.

        If $f$ is montonic down on~$I$, apply the result just obtained to $g \,=\, -f$.

\VV

        The next theorem states that a function which is monotonic on an interval has, at worst,
    a countable number of discontinuities, all of which are jump discontinuities.

\VV

            \subsection{\small{\bf Theorem}}
            \label{ThmD50.60}

\VV

        Suppose that $f:I \,{\rightarrow}\, {\RR}$ is a function which is monotonic on an interval~$I$.
    Let $S$ be the set of discontinuities of $f$ on the interval~$I$. Then:

\V

        (a) Every discontinuity of $f$ in $I$ is a jump discontinuity.
    (At any endpoint of $I$ one uses the appropriate concept of one-sided continuity.)

\V

        (b) The set $S$ is countable (possibly empty).

\V

        {\bf Proof} (a) This follows directly from Theorem~\Ref{ThmD60.60} and the definition of `continuity as in calculus'.

\V

        (b) Let $S_{0}$ be the set of interior points of $I$ at which $f$ is discontinuous.
    Since $I$ has at most two boundary points, the problem reduces to showing that the set $S_{0}$ is countable.

    As usual, first assume that $f$ is monotonic up, and suppose that $c{\in}S_{0}$. (The case of $f$ being monotonic down then follows easily.)
    Associate with $c$ the set $J_{c} \,=\, \{y: f(c-)\,\,{\leq}\,\,y\,\,{\leq}\,\,f(c+)\}$.
    Since $f(c-)\,<\,f(c+)$ (because $f$ is monotonic up and discontinuous at~$c$),
    it follows that $J_{c}$ is a true interval in~${\RR}$. In particular, $J_{c}$ contains infinitely many rational numbers.

        Now suppose that $S_{0} \,\,{\neq}\,\, {\emptyset}$, and define a function $h:S_{0} \,{\rightarrow}\, {\QQ}$
    by the rule that for each $c$ in $S_{0}$, $h(c)$ is one of the rational numbers in the interval $J_{c}$.

        \underline{Claim} The function $h$ is one-to-one on $S_{0}$.

        \underline{Proof of Claim} Suppose that $c_{1}$ and $c_{2}$ are points of $S_{0}$ such that $c_{1} \,\,{\neq}\,\, c_{2}$;
    without loss of generality, assume that $c_{1}\,<\,c_{2}$.
    It follows from Theorem~\Ref{ThmC90.90} that $f(c_{1}+)\,\,{\leq}\,\,f(c_{2}-)$. Since, by definition, 
    $h(c_{1})\,<\,f(c_{1}+)$ and  $h(c_{2})\,>\,f(c_{2}-)$, it then follows that $h(c_{1})\,<\,h(c_{2})$, and the claim follows.

        It now follows that $h:S_{0} \,{\rightarrow}\, {\QQ}$ is a  bijection of the set $S_{0}$ onto a subset of the countable set~${\QQ}$, and thus is countable, as asserted.

\V

        {\bf Remark} The approach followed here, which is based on the countability of the set of rational numbers, is standard.
    In the exercises an alternate approach to this result, also standard, is outlined.
    It relates the monotonicity to the countability in a more direct manner.



%--------------------
\StartSkip{
                \section{{\bf Uniform Continuity of a Function on a Set}}
                \label{SectD60}\IndB{ZZ Sections}{\Ref{SectD60} Uniform Continuity for a Function on a Set}

\V
\V



        One of the major advances in the foundations of analysis was made by E.~Heine in the $1870$s,
    when he clarified the significance of, and gave a name to, a certain property possessed by many real-valued functions; in English, his name for this property is translated as `Uniform Continuity'.
    (As often happens in mathematics, the actual history of this property is somewhat complicated, and involves more than one person; but Heine published first, so his name is usually associated with it.)
    Here is the precise definition, stated somewhat more generally than in Heine's original formulation.

\V


             \subsection{\small{\bf Definition} (Uniform Continuity)}
            \label{DefD60.20}

\V

        Let $f:X \,{\rightarrow}\, {\RR}$ be a function whose domain is a set $D \,{\subseteq}\, {\RR}$, and let $X$ be a nonempty subset of~$D$.
     The function $f$ is said to be {\bf uniformly continuous on $X$} if the following holds:

\VA

        \h for every ${\varepsilon}\,>\,0$ there exists ${\delta}\,>\,0$ such that if $x$ and $y$ are in $X$ and $|y-x|\,<\,{\delta}$, then $|f(y) - f(x)|\,<\,{\varepsilon}$.

\VHalf



\V

             \subsection{\small{\bf Remarks}}
            \label{RemrkD60.25}

\V

\hspace*{\parindent}(1) To understand the meaning of this definition, it helps to compare it with the more familiar idea of `continuity on the set $X$'.
    To make this comparison more transparent, it helps to reformulate the condition for uniform continuity on $X$ in the following equivalent -- albeit more awkward -- form:

\VHalf

        \h (I) `if ${\varepsilon}\,>\,0$ then there exists ${\delta}\,>\,0$ such that for every $x$ in $X$ if $y$ is an element of $X$ such that $|y-x|\,<\,{\delta}$
    then $|f(y)-f(x)|\,<\,{\varepsilon}$'

\VHalf


     Recall that a function $f$ is said to be {\bf continuous on $X$} provided it is continuous at every $x$ in $X$; that is, provided the following condition holds:

\VHalf

        \h (II) `for every $x$ in $X$ if ${\varepsilon}\,>\,0$ then there exists ${\delta}\,>\,0$ such that if $y$ is an element of $X$ such that
    $|y-x|\,<\,{\delta}$ then $|f(y)-f(x)|\,<\,{\varepsilon}$'

\VHalf


\noindent The only difference between Condition $(I)$ and  Condition~$(II)$ is in the placement of the phrase `for every $x$ in $X$'; but this difference changes the meanings of the statements.
    Indeed, the placement of this phrase in the first condition, namely {\em after} the determination of ${\delta}$,
    implies that while the choice of ${\delta}$ may depend on ${\varepsilon}$, this choice can be made independently of the choice of $x$.
    In contrast, the placement of this phrase in the second condition, namely {\em before} the determination of ${\delta}$,
    implies that the choice of ${\delta}$ may depend, in an {\em essential} way, on both $x$ and ${\varepsilon}$.
    This is a rather subtle distinction, and it confuses many students.

\V

        (2) Heine's formulation of `uniform continuity' illustrates why it can be important in mathematics to have more than one approach to a subject.
    The major results obtained earlier in this chapter -- the Extreme-Value Theorem, the Intermediate-Value Theorem -- are most easily proved using the `sequence characterization' of continuity.
    However, that characterization does not lend itself nearly so well to describing uniform continuity as does the `${\varepsilon}{\delta}$ characterization'.


             \subsection{\small{\bf Examples}}
            \label{ExampD60.30}

\V

\hspace*{\parindent}(1) Here are some examples of functions which {\em are} uniformly continuous on suitable subsets of their domains.


        \h (i)\,\, If $f$ is a linear function on ${\RR}$, then $f$ is uniformly continuous on ${\RR}$.
    Indeed, suppose that $a$ and $b$ are constants such that $f(x) \,=\, ax+b$ for all $x$ in ${\RR}$.
    Then one has
        \begin{displaymath}
        |f(y) - f(x)| \,=\, |(ay+b) - (ax+b)| \,=\, |a||y-x| \h ({\ast})
        \end{displaymath}
    If $a \,=\, 0$, i.e., if $f$ is a constant function, then it is clear that for every ${\varepsilon}\,>\,0$ one has
    $|f(y)-f(x)| \,=\, 0\,<\,{\varepsilon}$ for {\em all} $x$ and $y$ in ${\RR}$.
    In particular, one can choose ${\delta}$ to be any positive number. If, instead, $a \,\,{\neq}\,\, 0$, then choose ${\delta} \,=\, {\varepsilon}/|a|$. For such ${\delta}$ Equation~$({\ast})$ shows that if $|y-x|\,<\,{\delta}$ then 
    $|f(y)-f(x)|\,<\,|a||y-x|\,<\,|a|\left({\varepsilon}/|a|\right)\,<\,{\varepsilon}$, as required.


        \h (ii)\, Let $f:{\RR} \,{\rightarrow}\, {\RR}$ be given by the rule $f(x) \,=\, x^{2}$.
    Then $f$ is uniformly continuous on every closed bounded interval $[a,b]$.
    Indeed, note that for all $x$ and $y$ in ${\RR}$ one has
        \begin{displaymath}
        |f(y)-f(x)| \,=\, |y^{2} - x^{2}| \,=\, |y+x||y-x|
        \end{displaymath}
    Now let $B$ be any positive number such that $|a|\,\,{\leq}\,\,B$ and $|b|\,\,{\leq}\,\,B$;
    for instance, let $B \,=\, \max{|a|,|b|}$. Then clearly one has $-B\,\,{\leq}\,\,-|a|\,\,{\leq}\,\,a\,\,{\leq}\,\,b\,\,{\leq}\,\,|b|\,\,{\leq}\,\,B$. In particular, for all $x$ and $y$ in $[a,b]$ one has $|y+x|\,\,{\leq}\,\,|y| + |x|\,\,{\leq}\,\,2B$.
    Now let ${\varepsilon}\,>\,0$ be given, and let ${\delta} \,=\, {\varepsilon}/2B$.
    If $|y-x|\,<\,{\delta}$, then one has
        \begin{displaymath}
        |f(y) - f(x)| \,=\, |y+x||y-x|\,<\,2B\frac{{\varepsilon}}{2B} \,=\, {\varepsilon}.
        \end{displaymath}
    The uniform continuity of $f$on $[a,b]$ now follows.


\V

        (2) The function $f(x) \,=\, x^{2}$ discussed in Part~(ii) of the preceding example is {\em not} uniformly continuous on ${\RR}$.
    Note that for every ${\delta}\,>\,0$ one can find $x$ and $y$ so that $|y-x|\,<\,{\delta}$ but $|y^{2}-x^{2}|\,\,{\geq}\,\,1$.
    Indeed, let $x \,=\, 2/{\delta}$ and $y \,=\, x+{\delta}/2$, then one has $|y-x| \,=\, {\delta}/2\,<\,{\delta}$ and $|x+y|\,>\,2/{\delta}$. Thus
        \begin{displaymath}
        |f(y) - f(x)| \,=\, |y-x|{\cdot}|y+x|\,\,{\geq}\,\,\left(\frac{{\delta}}{2}\right){\cdot}\left(\frac{2}{{\delta}}\right) \,=\, 1.
        \end{displaymath}
    Thus $f$ is not uniformly continuous on ${\RR}$.

\V
%
        (3) Let $f:(0,+{\infty}) \,{\rightarrow}\, {\RR}$ be given by $f(x) \,=\, 1/x$ for all $x\,>\,0$. Let $a\,>\,0$ be a fixed positive number.

        \h (i)\, It is clear that $f$ is uniformly continuous on each unbounded interval of the form $[a,+{\infty})$.
    Indeed, if $x\,\,{\geq}\,\,a$ and $y\,\,{\geq}\,\,a$ then
        \begin{displaymath}
        \left|\frac{1}{y} - \frac{1}{x}\right| \,=\, \left|\frac{x-y}{xy}\right|\,\,{\leq}\,\,\frac{|y-x|}{a^{2}}
        \end{displaymath}
    Thus if ${\varepsilon}\,>\,0$ is given, let ${\delta} \,=\, a^{2}/{\varepsilon}$. It follows that if $x$ and $y$ are in $[a,+{\infty})$ and $|y-x|\,<\,{\delta}$ then $|f(y) - f(x)|\,<\,{\varepsilon}$, as required.

        \h (ii) It is perhaps less obvious, but still true, that $f$ {\em fails} to be uniformly continuous on the bounded interval $(0,a]$.
    To see this, note that
        \begin{displaymath}
        |f(y)-f(x)| \,=\, \left|\frac{1}{y} - \frac{1}{x}\right| \,=\, \left|\frac{x-y}{xy}\right| \,=\, \frac{|y-x|}{xy}.
        \end{displaymath}
    In particular, if $0\,<\,y \,=\, 2x\,\,{\leq}\,\,a$, then one gets
        \begin{displaymath}
        |f(y) - f(x)| \,=\, |f(2x) - f(x)| \,=\, \frac{|2x-x|}{(2x)x} \,=\, \frac{1}{2x}.
        \end{displaymath}
    If ${\delta}\,>\,0$ is given, let $x$ satisfy $0\,<\,x\,<\,{\min}\,\{{\delta}, a/2\}$.
    Then $x$ and $y \,=\, 2x$ are both in $(0,a]$, and $|y-x| \,=\, |2x-x| \,=\, x\,<\,{\delta}$,
    but ${\displaystyle |f(y)-f(x)| \,=\, \frac{1}{2x}\,\,{\geq}\,\,\frac{1}{a}}$.
    In other words, if one choose ${\varepsilon} \,=\, 1/a$, then it is not the case that there is a single ${\delta}\,\,{\geq}\,\,0$ such that $|f(y)-f(x)|\,<\,{\varepsilon}$ whenever $x$ and $y$ in $(0,a]$ satisfy $|y-x|\,<\,{\delta}$.


\V
\V

        The first result Heine proved after defining the concept of `uniform convergence' is a generalization of Example~(1)~(ii) above.
    It is convenient for future applications to formulate it in two steps.

\V

            \subsection{\small{\bf Lemma} (The Difference-of-Powers Inequality)}\index{difference-of-powers inequality}
            \label{LemmaD60.35A}

        Let $m$ be a natural number. If $x$ and $y$ are real numbers, and $R\,>\,0$ is a real number such that $|x|\,\,{\leq}\,\,R$ and $|y|\,\,{\leq}\,\,R$, then
        \begin{equation}
        \label{IneqD.30A} %% Formerly E.20A
        |y^{m}-x^{m}|\,\,{\leq}\,\,mR^{m-1}|y-x|
        \end{equation}

\V

        \underline{Proof} Note that, from basic algebra, one has
        \begin{equation}
        \label{EqnD.30B}
        y^{m}-x^{m} \,=\, (y^{m-1}+y^{m-2}x+\,{\ldots}\,+\,{\ldots}\, +yx^{m-2}+x^{m-1})(y-x) \,=\, \left(\sum_{j=0}^{m-1} y^{m-1-j}x^{j}\right)(y-x).
        \end{equation}
    Now apply the Extended Triangle Inequality, and the hypotheses on $|x|$ and $|y|$, to obtain
        \begin{displaymath}
        |y^{m}-x^{m}|\,\,{\leq}\,\,\left(\sum_{j=0}^{m-1} |y|^{m-1-j}{\cdot}|x|^{j}\right)|y-x|
    \,\,{\leq}\,\,
       \end{displaymath}
        \begin{displaymath}
    \left(\sum_{j=0}^{m-1} R^{m-1-j}R^{j}\right)|y-x| \,=\, \left(\sum_{j=0}^{m-1} R^{m-1}\right)|y-x| \,=\, mR^{m-1}|y-x|.
        \end{displaymath}
    (The final equation reflects the fact that the expression $\sum_{j=0}^{m-1} R^{m-1}$ consists of the sum of $m$ terms, each equal to $R^{m-1}$.)
    In particular, Inequality~\Ref{IneqD.30A} holds. \Q

\V

            \subsection{\small{\bf Corollary}} 
            \label{CorD60.35B}

\V

        Let $m$ be a positive integer and let $f:{\RR} \,{\rightarrow}\, {\RR}$ be given by the rule $f(x) \,=\, x^{m}$ for all $x$ in ${\RR}$.
    Then $f$ is uniformly continuous on every nonempty compact subset $X$ if ${\RR}$.

\V

        \underline{Proof} It clearly suffices to show the result whenever $X$ is of the form $[-R,R]$ for some positive number~$R$.
    In this situation, if $x$ and $y$ are in $[-R,R]$ then $|x|\,\,{\leq}\,\,R$ and $|y|\,\,{\leq}\,\,R$.
    Now let ${\varepsilon}\,>\,0$ be given, and let ${\delta} \,=\, {\varepsilon}/(mR^{m-1})$.
    Then the preceding lemma implies that if $|y-x|\,<\,{\delta}$ then
        \begin{displaymath}
        |f(y)-f(x)| \,=\, |y^{m} - x^{m}|\,\,{\leq}\,\,mR^{m-1}{\delta}\,<\,{\varepsilon},
        \end{displaymath}
    as required. \Q


\V
\V

        The preceding examples demonstrate uniform continuity by obtaining explicit relations between ${\varepsilon}$ and ${\delta}$.
    The next result, which is one of the major theorems in analysis, proves uniform continuity `abstractly'.

\V


             \subsection{\small{\bf Theorem} (Heine's Theorem on Uniform Continuity in ${\RR}$)}
            \label{ThmD60.40}

        Suppose that $f:X \,{\rightarrow}\, {\RR}$ is a continuous function whose domain $X$ is a compact nonempty subset of ${\RR}$.
    Then $f$ is uniformly continuous on $X$.

\V

        \underline{Proof} Suppose that $f$ is {\em not} uniformly continuous on $X$.
    Then there exists ${\varepsilon}_{0}\,>\,0$ such that for every ${\delta}\,>\,0$ there exist numbers $x$ and $y$ in $X$
    with $|y-x|\,<\,{\delta}$ but $|f(y)-f(x)|\,\,{\geq}\,\,{\varepsilon}_{0}$.
    In particular, for each $k$ in ${\NN}$ there exist numbers $x_{k}$ and $y_{k}$ in $X$
    such that $|y_{k}-x_{k}|\,<\,1/k$ but $|f(y_{k})-f(x_{k})|\,\,{\geq}\,\,{\varepsilon}_{0}$.
    Note, in particular, that one has $\lim_{k \,{\rightarrow}\, {\infty}} |y_{k}-x_{k}| \,=\, 0$.
    Let $(y_{k_{1}}, y_{k_{2}},\,{\ldots}\,)$ be a subsequence of the sequence $(y_{1},y_{2},\,{\ldots}\,)$ which is convergent;
    the existence of such a convergent subsequence follows from the Bolzano-Weierstrass Theorem for Sequences,
    together with the fact that (by definition of `compactness') $X$ is a bounded subset of ${\RR}$.
    Let $c \,=\, \lim_{j \,{\rightarrow}\, {\infty}} y_{k_{j}}$. Then, since (by definition of `compactness') the set $X$ is closed in ${\RR}$, it follows that $c{\in}X$.
    In addition, since $\lim_{k \,{\rightarrow}\, {\infty}} (y_{k}-x_{k}) \,=\, 0$,
    and thus $\lim_{j \,{\rightarrow}\, {\infty}} (y_{k_{j}}-x_{k_{j}}) \,=\, 0$,
    it follows easily that the subsequence $(x_{k_{1}},x_{k_{2}},\,{\ldots}\,)$ also converges to~$c$.
    Since, by hypothesis, $f$ is continuous at $c$, it follows that $\lim_{j \,{\rightarrow}\, {\infty}} f(y_{k_{j}}) \,=\, f(c)$ and $\lim_{j \,{\rightarrow}\, {\infty}} f(x_{k_{j}}) \,=\, f(c)$.
    Thus, $\lim_{j \,{\rightarrow}\, {\infty}} |f(y_{k_{j}}) - f(x_{k_{j}})| \,=\, 0$.
    However, this contradicts the hypothesis that $|f(y_{k})-f(x_{k})|\,\,{\geq}\,\,{\varepsilon}_{0}$ for {\em all} $k$; in particular $k \,=\, k_{j}$.
    Thus, $f$ is uniformally continuous on $X$, as claimed.

\V
\V

             \subsection{\small{\bf Remarks}}
            \label{RemrkD60.50}

\V

\hspace*{\parindent}(1) Heine actually stated and proved his `Uniform Continuity' theorem only in the case $X$ is a closed interval $[a,b]$ in ${\RR}$.
    His proof, which is tailored to that context, is outlined in the exercises.

\V

        (2) Some texts refer to Theorem~\Ref{ThmD60.40} as the {\bf Heine-Cantor Theorem};
    a few refer to it as the {\bf Cantor Theorem}, leaving poor Heine out in the cold.

\V

        (3) Heine is pronounced as `high nuh', with accent on the first syllabel.
    His full name was Heinrich Eduard Heine, but he dropped the `Heinrich' in his publications;
    perhaps he wanted to avoid confusion with the famous German poet Heinrich Heine.


\V
\V

        The next theorem shows that if a function $f:[a,b] \,{\rightarrow}\, {\RR}$ is continuous on a closed interval $[a,b]$,
    then $f$ can be well approximated throughout $[a,b]$ by a piecewise-linear function.
    This result is used in the next chapter to prove that every continuous function on an interval has an antiderivative.


}%\EndSkip
%--------------

%% POSSIBLE EXERCISES
%%
%% True or False: If f is cont on X, and X=Y {\cup} Z, then unif cont of f|_{Y} %% and f|_{Z} implies unif cont of f on X
%% Show true if X = [a,b], Y = [a,c], Z = [c,b] with a\,<\,c\,<\,b

%-----------------
\StartSkip{ % HEINE'S PROOF -- REDO FOR INTERVAL


        \underline{Heine's Original Proof (more or less)} For convenience, let $a \,=\, \min\,X$ and $b \,=\, \max\,X$;
    these numbers exist (and are elements of $X$) because $X$ is compact and nonempty.
    To avoid trivial cases, assume that $X$ has more than one element, so that $a\,<\,b$.

        Suppose that $g$ is not uniformly continuous on $X$.
    Then there must exist ${\varepsilon}_{0}\,>\,0$ for which {\em no} ${\delta}\,>\,0$ is small enough to ensure that $|g(x_{2})-g(x_{1})|\,<\,{\varepsilon}_{0}$ whenever $x_{1},x_{2}$ in $X$ satisfy $|x_{2}-x_{1}|\,<\,{\delta}$.

        The continuity hypothesis for $g$, applied at $x \,=\, a$, implies that there exists a number ${\xi}$, with $a\,<\,{\xi}\,\,{\leq}\,\,b$, such that if $a\,\,{\leq}\,\,x\,\,{\leq}\,\,{\xi}$ and $x{\in}X$ then $|g(a)-g(x)|\,\,{\leq}\,\,{\varepsilon}_{0}/3$.
    Let $x_{1}$ denote the supremum of the set of all such ${\xi}$.

        \underline{Case 1} Suppose that $x_{1} \,=\, b$. It is then clear that $|g(a)-g(x)|\,\,{\leq}\,\,{\varepsilon}_{0}/3$ for all $x$ in $X$.
    In particular, if $x_{1}$ and $x_{2}$ are any elements of $X$ then one has
        \begin{displaymath}
        |g(x_{2})-g(x_{1})|\,\,{\leq}\,\,|g(x_{2})-g(a)| + |g(a)-g(x_{2})|
\,\,{\leq}\,\, \frac{{\varepsilon}_{0}}{3} + \frac{{\varepsilon}_{0}}{3}\,<\,{\varepsilon}_{0}
        \end{displaymath}
    In this case the number ${\delta} \,=\, b-a$ satisfies $|g(x_{2})-g(x_{1})|\,<\,{\varepsilon}_{0}$ for all $x_{1}$, $x_{2}$ in $X$ with $|x_{2}-x_{1}|\,<\,{\delta}$, contrary to the definition of the number ${\varepsilon}_{0}$.
    Thus $x_{1} \,=\, b$ cannot hold.

        \underline{Case 2} Suppose that $x_{1}\,<\,b$. It is still true that $x_{1}{\in}X$.
    Indeed, by the definition of `supremum', together with the fact that $b{\in}X$ and the hypothesis in this case that $x_{1}\,<\,b$, it follows that there is a sequence of points $z_{1}$, $z_{2}$,\,{\ldots}\,in $X$ with $x_{1}\,<\,z_{k}\,<\,b$, such that $x_{1} \,=\, \lim_{k \,{\rightarrow}\, {\infty}} z_{k}$ and $|g(z_{k})-g(a)|\,>\,{\varepsilon}_{0}/3$.
    Since $X$ is closed, it follows from the equation $x_{1} \,=\, \lim_{k \,{\rightarrow}\, {\infty}} (z_{k}$ that $x_{1}{\in}X$.
    It then follows from the continuity of $g$ on $X$ that $g(x_{1}) \,=\, \lim_{k \,{\rightarrow}\, {\infty}} g(z_{k})$.
    Thus
        \begin{displaymath}
        |g(a)-g(x_{1})| \,=\, \lim_{k \,{\rightarrow}\, {\infty}} |g(a)-g(z_{k})|\,\,{\geq}\,\,\frac{{\varepsilon}}{3}.
        \end{displaymath}
    However, it also follows from the definition of $x_{1}$ that $|g(a)-g(x_{1})|\,\,{\leq}\,\,{\varepsilon}/3$.
    Thus $|g(a)-g(x_{1})| \,=\, {\varepsilon}_{0}/3$, and
}%\EndSkip
%---------------------

\newpage

% Exercises_M140AB_D.TeX   Exercises for Chapter D

%
% Revised: 06/26/09
%

%% NOTE: Copy the 52 lines below to each chapter, and change the Chapter letters.

%\thispagestyle{myheadings}


%\markboth{Exercises for Chapter~\ref{ChaptD} -}{Exercises for Chapter~\ref{ChaptD} -}

\newcommand{\ExDa}{{\bf \ref{ChaptD} - \,1} }
\newcommand{\ExDb}{{\bf \ref{ChaptD} - \,2} }
\newcommand{\ExDc}{{\bf \ref{ChaptD} - \,3} }
\newcommand{\ExDd}{{\bf \ref{ChaptD} - \,4} }
\newcommand{\ExDe}{{\bf \ref{ChaptD} - \,5} }
\newcommand{\ExDf}{{\bf \ref{ChaptD} - \,6} }
\newcommand{\ExDg}{{\bf \ref{ChaptD} - \,7} }
\newcommand{\ExDh}{{\bf \ref{ChaptD} - \,8} }
\newcommand{\ExDi}{{\bf \ref{ChaptD} - \,9} }
\newcommand{\ExDj}{{\bf \ref{ChaptD} -  10} }
\newcommand{\ExDk}{{\bf \ref{ChaptD} -  11} }
\newcommand{\ExDl}{{\bf \ref{ChaptD} -  12} }
\newcommand{\ExDm}{{\bf \ref{ChaptD} -  13} }
\newcommand{\ExDn}{{\bf \ref{ChaptD} -  14} }
\newcommand{\ExDo}{{\bf \ref{ChaptD} -  15} }
\newcommand{\ExDp}{{\bf \ref{ChaptD} -  16} }
\newcommand{\ExDq}{{\bf \ref{ChaptD} -  17} }
\newcommand{\ExDr}{{\bf \ref{ChaptD} -  18} }
\newcommand{\ExDs}{{\bf \ref{ChaptD} -  19} }
\newcommand{\ExDt}{{\bf \ref{ChaptD} -  20} }
\newcommand{\ExDu}{{\bf \ref{ChaptD} -  21} }
\newcommand{\ExDv}{{\bf \ref{ChaptD} -  22} }
\newcommand{\ExDw}{{\bf \ref{ChaptD} -  23} }
\newcommand{\ExDx}{{\bf \ref{ChaptD} -  24} }
\newcommand{\ExDy}{{\bf \ref{ChaptD} -  25} }
\newcommand{\ExDz}{{\bf \ref{ChaptD} -  26} }


\newcommand{\ExDaa}{{\bf \ref{ChaptD} - 27} }
\newcommand{\ExDab}{{\bf \ref{ChaptD} - 28} }
\newcommand{\ExDac}{{\bf \ref{ChaptD} - 29} }
\newcommand{\ExDad}{{\bf \ref{ChaptD} - 30} }
\newcommand{\ExDae}{{\bf \ref{ChaptD} - 31} }
\newcommand{\ExDaf}{{\bf \ref{ChaptD} - 32} }
\newcommand{\ExDag}{{\bf \ref{ChaptD} - 33} }
\newcommand{\ExDah}{{\bf \ref{ChaptD} - 34} }
\newcommand{\ExDai}{{\bf \ref{ChaptD} - 35} }
\newcommand{\ExDaj}{{\bf \ref{ChaptD} - 36} }
\newcommand{\ExDak}{{\bf \ref{ChaptD} - 37} }
\newcommand{\ExDal}{{\bf \ref{ChaptD} - 38} }
\newcommand{\ExDam}{{\bf \ref{ChaptD} - 39} }
\newcommand{\ExDan}{{\bf \ref{ChaptD} - 40} }
\newcommand{\ExDao}{{\bf \ref{ChaptD} - 41} }
\newcommand{\ExDap}{{\bf \ref{ChaptD} - 42} }
\newcommand{\ExDaq}{{\bf \ref{ChaptD} - 43} }
\newcommand{\ExDar}{{\bf \ref{ChaptD} - 44} }
\newcommand{\ExDas}{{\bf \ref{ChaptD} - 45} }
\newcommand{\ExDat}{{\bf \ref{ChaptD} - 46} }
\newcommand{\ExDau}{{\bf \ref{ChaptD} - 47} }
\newcommand{\ExDav}{{\bf \ref{ChaptD} - 48} }
\newcommand{\ExDaw}{{\bf \ref{ChaptD} - 49} }
\newcommand{\ExDax}{{\bf \ref{ChaptD} - 50} }
\newcommand{\ExDay}{{\bf \ref{ChaptD} - 51} }
\newcommand{\ExDaz}{{\bf \ref{ChaptD} - 52} }



                       \section{EXERCISES FOR CHAPTER~\ref{ChaptD}}
                        \label{SectDEX}

\V
\V
\V
\V

\noindent  \ExDa Prove Part~(a) of Theorem~\Ref{ThmD20.85}.

\V
\V

\noindent \ExDb Let $X$ be a nonempty subset of ${\RR}$. Define $d_{X}:{\RR} \,{\rightarrow}\, {\RR}$ by the rule
        \begin{displaymath}
        d_{X}(y) \,=\, {\inf}\,\{t: t \,=\, |y-x| \mbox{ for some $x$ in $X$}\}
        \end{displaymath}

\V

        (a) \underline{Prove or Disprove}: The function $d_{X}$ is continuous at each point of its domain ${\RR}$.

\V

        (b) Give an example of a nonempty set $X$ in ${\RR}$ such that $d_{X}$ does not have a minimum value.


\V
\V

\noindent \ExDc Let $f:(a,b) \,{\rightarrow}\, {\RR}$ be a real-valued function defined on an open interval $I \,=\, (a,b)$.

\V

        Let $c$ be a point of $I$. Show that $f$ is continuous at $c$ if, and only if, the following condition holds:

        \h For every open subset $U$ of ${\RR}$ containing the point $f(c)$, the inverse image $f^{-1}[U]$ is an open subset of ${\RR}$ containing~$c$.

\V
\V

\noindent \ExDd Let $X$ be a nonempty subset of ${\RR}$, and suppose that $f_{1}$, $f_{2}$,\,{\ldots}\,$f_{k}$ are real-valued functions with domain $X$.
    Let $c$ be a point of $X$.

\V

    (a) Prove that the function ${\max}\,\{f_{1},\,{\ldots}\,f_{k}\}:X \,{\rightarrow}\, {\RR}$ is continuous at $c$.

\V

    (b) Prove that the function ${\min}\,\{f_{1},\,{\ldots}\,f_{k}\}:X \,{\rightarrow}\, {\RR}$ is continuous at $c$.


\V
\V

\noindent \ExDe Let $f:{\RR} \,{\rightarrow}\, {\RR}$ be a real-valued function whose domain is ${\RR}$.
    Prove that $f$ is continuous on ${\RR}$ if, and only if, for every bounded open interval $I$ in ${\RR}$ the restriction of $f$ to $I$ is continuous on $I$.
    Be sure to make clear which part of your solution proves the `if' portion of the statement, and which proves the `only if' portion.

\V
\V

\noindent \ExDf The proof of the Intermediate-Value Theorem for Continuous Functions (Theorem~\Ref{ThmD30.70}) %% Thm D.2.5 page 232
    given in the {\em Notes} is based on the Bisection Principle.

        \underline{Problem} Give an alternate proof that is based on the Supremum Principle.

\V
\V

\noindent \ExDg \underline{Prove or Disprove} If $f:[a,b] \,{\rightarrow}\, {\RR}$ is a continuous function,
    and if $g:[a,b] \,{\rightarrow}\, {\RR}$ is defined by the rule
        \begin{displaymath}
        g(x) \,=\, {\sup}\,\{f(u):a\,\,{\leq}\,\,u\,\,{\leq}\,\,x\} \mbox{ for $x$ in $[a,b]$},
        \end{displaymath}
    then $g$ is also continuous.

\V
\V

\noindent \ExDh Let $f:{\RR}{\setminus}\{1\} \,{\rightarrow}\, {\RR}$ and $g:{\RR}{\setminus}\{1\} \,{\rightarrow}\, {\RR}$ be given by
        \begin{displaymath}
        f(x) \,=\, \frac{x^{3} - 1}{x-1} \mbox{ if $x \,\,{\neq}\,\, 1$},
        \end{displaymath}
    and
        \begin{displaymath}
        g(x) \,=\, \frac{x^{2} + x - 2}{x-1} \mbox{ if $x \,\,{\neq}\,\, 1$}
        \end{displaymath}
    \underline{Problem} Determine whether there is a choice of $C$ in ${\RR}$ such that the concatenation $f\&_{(1,C)}g$ is continuous on ${\RR}$.

\V
\V

\noindent \ExDi \underline{Prove or Disprove} If $f:{\RR} \,{\rightarrow}\, {\RR}$ is a function such that $f^{2}$ is continuous on ${\RR}$, then $f$ is also continuous on ${\RR}$.

\V
\V

\noindent \ExDj \underline{Prove or Disprove} If $f$ and $g$ are real-valued functions defined on ${\RR}$,
    and $f$ and $f{\circ}g$ are continuous, then $g$ is continuous.

\V
\V

\noindent \ExDk {\bf Definition} (1) A real-valued function $f$ defined on an interval $I$ in ${\RR}$ is said to be a {\bf convex function on $I$} provided the following condition holds:
        \begin{displaymath}
        \mbox{If $x$ and $y$ are in $I$, then }
        f(tx + (1-t)y)\,\,{\leq}\,\,tf(x) + (1-t)f(y) \mbox{ for all $t$ such that $0\,\,{\leq}\,\,t\,\,{\leq}\,\,1$}.
        \end{displaymath}
    The function $f$ is said to be {\bf strictly convex on $I$} provided one has, in addition, when $x\,<\,y$ and $0\,<\,t\,<\,1$ one has
       % \begin{displaymath}
        $f(tx + (1-t)y)\,<\,tf(x) + (1-t)f(y)$.
       % \end{displaymath}

        (2) A real-valued function $g$ on an interval $I$ in ${\RR}$ is said to be a {\bf concave function on $I$} provided $-g$ is a convex function on $I$.
    Likewise, $g$ is said to be {\bf strictly concave on $I$} provided $-g$ is  a strictly convex function on $I$.

\V

        \underline{Problem}

        (a) Give geometric interpretations, in terms of the relation between the graph of a function $f$ on an interval $I$ and the chords joining pairs of points on that graph,
    of the statements `$f$ is a convex function on $I$' and `$f$ is a strictly convex function on $I$'.

\V

        (b) Determine the convexity/concavity properties the `absolute-value' function $\mbox{abs}:{\RR} \,{\rightarrow}\, {\RR}$, given by the rule
        %\begin{displaymath}
        $\mbox{abs}\,(x) \,=\, |x| \mbox{ for all $x$ in ${\RR}$}$.
        %\end{displaymath}
    (By `determine the convexity/concavity properties' is meant: determine the largest intervals on which the function is convex, strictly convex, concave or strictly concave.)

\V

        \underline{Warning} The meanings of `convex' and `concave' given here are quite standard in advanced texts in analysis.
    However, most textbooks on elementary calculus use a different terminology. 
    Specifically, they say `concave up' instead of `convex function' and `concave down' instead of `concave function';
    a similar usage holds for `strict' convexity/concavity.
    The use of the word `concave' in the phrase `concave up' to describe `convexity' can sometimes cause confusion, so be careful.

\V
\V

\noindent \ExDl (a) Show that if a real-valued function is convex on an open interval $(a,b)$, then it is continuous on $(a,b)$.

\V

        (b) Give an example of a function $f$, whose domain is a {\em closed} interval $[a,b]$, such that $f$ is convex on $[a,b]$, but is not continuous on $[a,b]$.

\V
\V

\noindent \ExDm Suppose that $f$ is a continuous real-valued function defined on an interval $I$ in ${\RR}$ and that $f$ satisfies the condition
        \begin{displaymath}
        f\left(\frac{x+y}{2}\right)\,\,{\leq}\,\,\frac{f(x) + f(y)}{2} \mbox{ for all $x,y$ in $I$}.
        \end{displaymath}
    Prove that $f$ is convex on $I$.

\V
\V

\noindent \ExDn {\bf Definition} A function $f:{\RR} \,{\rightarrow}\, {\RR}$ is said to be an {\bf additive function} provided $f(x+y) \,=\, f(x)+f(y)$ for all $x$ and $y$ in ${\RR}$.

        Prove that if $f$ is an additive function which is continuous at~$0$ then $f$ is a linear function;
    more precisely show that there exists a real number $c$ such that $f(x) \,=\, cx$ for all $x$ in ${\RR}$.

\V
\V

\noindent \ExDo \underline{Prove or Disprove} If $f:{\RR} \,{\rightarrow}\, {\RR}$ is a continuous function such that $|f(y)-f(x)|\,\,{\geq}\,\,|y-x|$ for all $x$ and $y$ in ${\RR}$, then $f$ maps ${\RR}$ \underline{onto} ${\RR}$.

\V
\V

\noindent \ExDp \underline{Remark} In the proof of the Extreme-Value Theorem found in the {\em Notes}, one obtains the result that if $f$ is a continuous real-valued function defined on a closed bounded interval $I$, then $f$ is bounded above on $I$.
    However, this fact comes almost as an after-thought: one first shows that the supremum of $f$ over $I$ equals $f(c)$ for some $c$ in $I$,
    and thus ${\sup}\,f[I]$ must be finite; in particular, $f$ must be bounded above on $I$.
    In other words, one shows that $f$ is bounded by {\em first} showing that $f$ assumes a maximum value on $I$.

        The goal of the present exercise is to show directly that such a function $f$ must be bounded above on $I$;
    indeed, {\em two} approaches to this result are provided.
    The following execise then asks one to use this boundedness to prove that $f$ assumes a maximum on $I$.

\V

        Suppose that $f:I \,{\rightarrow}\, {\RR}$ is a real-valued function whose domain is a closed bounded interval $I \,=\, [a,b]$ in ${\RR}$,
    and assume that $f$ is continuous at each point of $I$.

\V

        (a) Give a direct proof of the fact that $f$ is bounded on $I$ by considering the set $X$ of all $x\,>\,a$ in $[a,b]$ such that $f$ is bounded on the subinterval $[a,x]$.
    In the course of events you may need to show that the set $X$ is nonempty, and you may wish to consider the supremum of this set.

\V

        (b) Give a direct proof of the fact that $f$ is bounded on $I$ by noting that for each $x$ in $I$ there exists ${\delta}_{x}\,>\,0$ such that
    $f$ is bounded on the subset $U_{x} \,=\, (x-{\delta}_{x},x+{\delta})\,{\cap}\,I$. Then prove that there is a finite subset of the family ${\cal F} \,=\, \{U_{x}:x{\in}I\}$ whose union has $I$ as a subset.

\V
\V

\noindent \ExDq Give a proof, of the Extreme-Value Theorem for continuous real-valued functions defined on closed bounded intervals, which is based on the fact that continuous real-valued functions defined on closed bounded intervals are bounded functions.

\V
\V

\noindent \ExDr {\bf Definition} Let $f:X \,{\rightarrow}\, {\RR}$ be a real-valued function whose domain is a nonempty subset $X$ of ${\RR}$,
    and let $c$ be a point of $X$.

        \h (i)\, One says that $f$ is {\bf upper-semicontinuous at $c$} provided that for every ${\varepsilon}\,>\,0$ there exists ${\delta}\,>\,0$
    such that if $x$ in $X$ satisfies $|x-c|\,<\,{\delta}$ then $f(x)\,\,{\leq}\,\,f(c) + {\varepsilon}$.

        \h (ii) One says that $f$ is {\bf lower-semicontinuous at $c$} provided that for every ${\varepsilon}\,>\,0$ there exists ${\delta}\,>\,0$
    such that if $x$ in $X$ satisfies $|x-c|\,<\,{\delta}$ then $f(x)\,\,{\geq}\,\,f(c) - {\varepsilon}$.



\V

        (a) Give an example of a function which is upper-semicontinuous, but not continuous, at a point.

\V

        (b) Prove that the function $f$ is continuous at the point $c$ if, and only if, $f$ is both upper-semicontinuous and lower-semicontinuous at $c$.

\V

        (c) Prove that if $f:[a,b] \,{\rightarrow}\, {\RR}$ is upper-semicontinuous at each point of a closed interval $[a,b]$, then $f$ assumes a maximum value for $[a,b]$ at some point of $[a,b]$.

\V
\V

        
\noindent \ExDs {\bf Definition} Let $r$ be a rational number and let $c$ be a positive real number.
    Then the number $c^{r}$, which is pronounced `the $r$-th power of $c$', is defined as follows:
        \begin{displaymath}
        c^{r} \,=\, \left\{
        \begin{array}{ll}
        1 & \mbox{if $r \,=\, 0$} \\
        \left(\sqrt[n]{c}\right)^{m} & \mbox{if $r\,>\,0$ and $r \,=\, m/n$ for natural numbers $m$ and $n$ in lowest terms} \\
        c^{-r} & \mbox{if $r\,<\,0$}
        \end{array}
                                \right.
        \end{displaymath}
    As usual, the symbol $\sqrt[n]{c}$ denotes the unique positive real number $u$ such that $u^{n} \,=\, c$.

\V

        (a) Suppose that $c\,>\,0$ and that $r$ is a rational number. Show that if one writes $r$ in the form $r \,=\, p/q$,
    where $p$ and $q$ are integers and $q\,>\,0$, then $c^{r} \,=\, \left(\sqrt[q]{c}\right)^{p}$;
    that is, one does not need to express the rational number $r$ in `lowest terms'.

\V

        (b) Suppose that $c\,>\,0$. Show that $c^{-r} \,=\, 1/c^{r}$ for every rational $r$.

\V

        (c) Suppose that $c\,>\,0$. Show that $c^{r+s} \,=\, c^{r}c^{s}$ and $\left(c^{r}\right)^{s} \,=\, c^{(rs)}$  for all rationals $r$ and $s$.
 %% NOTE: Automatically starts on a new page

%\newpage

