% Exercises_M140AB_E.TeX   Exercises for Chapter E

%
% Revised: 08/06/09
%

%% NOTE: Copy the 52 lines below to each chapter, and change the Chapter letters.

%\thispagestyle{myheadings}


%\markboth{Exercises for Chapter~\ref{ChaptE} -}{Exercises for Chapter~\ref{ChaptE} -}

\newcommand{\ExEa}{{\bf \ref{ChaptE} - \,1} }
\newcommand{\ExEb}{{\bf \ref{ChaptE} - \,2} }
\newcommand{\ExEc}{{\bf \ref{ChaptE} - \,3} }
\newcommand{\ExEd}{{\bf \ref{ChaptE} - \,4} }
\newcommand{\ExEe}{{\bf \ref{ChaptE} - \,5} }
\newcommand{\ExEf}{{\bf \ref{ChaptE} - \,6} }
\newcommand{\ExEg}{{\bf \ref{ChaptE} - \,7} }
\newcommand{\ExEh}{{\bf \ref{ChaptE} - \,8} }
\newcommand{\ExEi}{{\bf \ref{ChaptE} - \,9} }
\newcommand{\ExEj}{{\bf \ref{ChaptE} -  10} }
\newcommand{\ExEk}{{\bf \ref{ChaptE} -  11} }
\newcommand{\ExEl}{{\bf \ref{ChaptE} -  12} }
\newcommand{\ExEm}{{\bf \ref{ChaptE} -  13} }
\newcommand{\ExEn}{{\bf \ref{ChaptE} -  14} }
\newcommand{\ExEo}{{\bf \ref{ChaptE} -  15} }
\newcommand{\ExEp}{{\bf \ref{ChaptE} -  16} }
\newcommand{\ExEq}{{\bf \ref{ChaptE} -  17} }
\newcommand{\ExEr}{{\bf \ref{ChaptE} -  18} }
\newcommand{\ExEs}{{\bf \ref{ChaptE} -  19} }
\newcommand{\ExEt}{{\bf \ref{ChaptE} -  20} }
\newcommand{\ExEu}{{\bf \ref{ChaptE} -  21} }
\newcommand{\ExEv}{{\bf \ref{ChaptE} -  22} }
\newcommand{\ExEw}{{\bf \ref{ChaptE} -  23} }
\newcommand{\ExEx}{{\bf \ref{ChaptE} -  24} }
\newcommand{\ExEy}{{\bf \ref{ChaptE} -  25} }
\newcommand{\ExEz}{{\bf \ref{ChaptE} -  26} }


\newcommand{\ExEaa}{{\bf \ref{ChaptE} - 27} }
\newcommand{\ExEab}{{\bf \ref{ChaptE} - 28} }
\newcommand{\ExEac}{{\bf \ref{ChaptE} - 29} }
\newcommand{\ExEad}{{\bf \ref{ChaptE} - 30} }
\newcommand{\ExEae}{{\bf \ref{ChaptE} - 31} }
\newcommand{\ExEaf}{{\bf \ref{ChaptE} - 32} }
\newcommand{\ExEag}{{\bf \ref{ChaptE} - 33} }
\newcommand{\ExEah}{{\bf \ref{ChaptE} - 34} }
\newcommand{\ExEai}{{\bf \ref{ChaptE} - 35} }
\newcommand{\ExEaj}{{\bf \ref{ChaptE} - 36} }
\newcommand{\ExEak}{{\bf \ref{ChaptE} - 37} }
\newcommand{\ExEal}{{\bf \ref{ChaptE} - 38} }
\newcommand{\ExEam}{{\bf \ref{ChaptE} - 39} }
\newcommand{\ExEan}{{\bf \ref{ChaptE} - 40} }
\newcommand{\ExEao}{{\bf \ref{ChaptE} - 41} }
\newcommand{\ExEap}{{\bf \ref{ChaptE} - 42} }
\newcommand{\ExEaq}{{\bf \ref{ChaptE} - 43} }
\newcommand{\ExEar}{{\bf \ref{ChaptE} - 44} }
\newcommand{\ExEas}{{\bf \ref{ChaptE} - 45} }
\newcommand{\ExEat}{{\bf \ref{ChaptE} - 46} }
\newcommand{\ExEau}{{\bf \ref{ChaptE} - 47} }
\newcommand{\ExEav}{{\bf \ref{ChaptE} - 48} }
\newcommand{\ExEaw}{{\bf \ref{ChaptE} - 49} }
\newcommand{\ExEax}{{\bf \ref{ChaptE} - 50} }
\newcommand{\ExEay}{{\bf \ref{ChaptE} - 51} }
\newcommand{\ExEaz}{{\bf \ref{ChaptE} - 52} }


                       \section{EXERCISES FOR CHAPTER~\ref{ChaptE}}
                        \label{SectEEX}

\V
\V
\V
\V

\noindent  \ExEa Prove the claim in Example~(E.1.5)~(2) that the function $F$ in that example is differentiable precisely at the points $r_{1}$, $r_{2}$,\,{\ldots}\,$r_{m}$.

\V
\V

\noindent  \ExEb Let $f:[0,+{\infty}) \,{\rightarrow}\, {\RR}$ be the square-root function, so that $f(x) \,=\, \sqrt{x}$ for all $x\,>\,0$.
    In Example (E.1.9)~(2) on Page~240 it is shown that $f'(x)$ exists and equals $1/(2\sqrt{x})$ for every $x\,>\,0$.
    However, the case $x \,=\, 0$ is not discussed there; of course in that case one would have to consider a one-sided derivative.

        Consider the following argument:

\V

       \h `{\em From the formula $f'(x) \,=\, 1/(2\sqrt{x})$ for $x\,>\,0$ one sees that $\lim_{x{\searrow} 0} f'(x) \,=\, +{\infty}$. This implies that $f'(0)$ cannot exists, since we require the values of $f'$ to be finite.}

\V

        (a) Explain why this argument, although intuitively convincing, is not valid.

\V

        (b) Give a correct proof of the fact that the square root function is not differentiable (from the right) at~$0$.

\V
\V

\noindent  \ExEc The proof of the Product Rule for Differentiation given in the {\em Notes} (see Page~247) is a little different from the `standard proof' that one finds in most texts.

        \underline{Problem} Carry out the standard proof.

        (Hint: This proof starts by using the ever-popular `Add-and-Subtract' trick:
        \begin{displaymath}
        f(x)g(x) - f(c)g(c) \,=\, f(x)g(x) - f(x)g(c) + f(x)g(c) - f(c)g(c).
        \end{displaymath}
    That is, one adds and subtracts the (cleverly chosen) quantity $f(x)g(c)$.)

\V
\V

    \noindent  \ExEd Prove the Quotient Rule for Differentiation (Part~(b) of Theorem~E.2.2 on Page~247). % ThmE30.40

\V
\V

\noindent  \ExEe In the `Remark' on Page~252, right after the proof of the Chain Rule,
    it is pointed out that if one assumes {\em continuous} first derivatives then one can get a much simpler proof of this result.
    Here is a more precise statement:

        \underline{The Weakish Chain Rule} Suppose that $f:I \,{\rightarrow}\, {\RR}$ and $g:J \,{\rightarrow}\, {\RR}$
    are functions defined on open intervals $I$ and $J$ in ${\RR}$. Assume that $f(x){\in}J$ for all $x$ in $I$,
    so that the composition $h \,=\, g{\circ}f:I \,{\rightarrow}\, {\RR}$ is defined.
    Let $c$ be a number in $I$ and let $d \,=\, f(c)$, so $d{\in}J$. If $f$ is differentiable at $c$ and $g$ is {\em continuously} differentiable at $d$,
    then $h$ is differentiable at $c$, and $h'(c) \,=\, g'(d){\cdot}f'(c)$.
    (Note that the hypothesis of $g$ being differentiable at $d$ tacitly requires that $g'$ be defined on some open subinterval of $J$ containing~$d$.)

        \underline{Problem} Give a short -- but rigorous -- direct proof of the Weakish Chain Rule.
    (`Direct Proof': You are not allowed to simply say `it's a special case of the regular Chain Rule'.
    However, you are free to use results, such as the Lagrange Mean-Value Theorem, which are not themselves  based on the regular Chain Rule.)

\StartSkip{
        \underline{Remarks} (i) In elementary single-variable calculus nearly all the functions one encounters are, in fact, $C^{{\infty}}$ on the interiors of their domains.
    Thus, the Weakish version is more than adequate for the students' needs. In addition,
    a simple proof that the students can understand is more useful to them than a complicated one which they don't understand.
        (ii) When one goes on to multivariable calculus, one finds that it is the Weakish version of the Chain Rule which generalizes to partial derivatives, not the regular version.
        }%\EndSkip

\V
\V

\noindent  \ExEf Let $f:I \,{\rightarrow}\, {\RR}$ be a function defined on an open interval $I$.

\V

        (a) If $f''(c)\,>\,0$ at some point $c$ in $I$, then $f(x)\,>\,f(c) + f'(c)(x-c)$ for all $x$ sufficiently near~$c$.

\V

        (b) If $f''(x)\,>\,0$ for all $x$ in $I$, then for every $c$ in $I$ one has $f(x)\,>\,f(c) + f'(c)(x-c)$ for all $x$ in $I$ with $x \,\,{\neq}\,\, c$.

\V

        (c) Give a geometric interpretation of these results in terms of `tangent lines'.
        
\V
\V

        (d) Suppose that $f'(x)$ exists for each $x$ in $I$, and assume that $f$ has exactly one critical point in $I$; that is, there is exactly one value of $x$ in $I$ such that $f'(x)\,=\,0$.
    Let $c$ be that unique critical point.

        \underline{Problem}: Show that if $f''(c)$ exists and $f''(c)\,>\,0$, then $f$ has a strict minimum for $I$ at $c$.
    That is, if $x{\in}I$ and $x\,\,{\neq}\,\,c$, then $f(x) \,>\, f(c)$.

\V
\V

\noindent  \ExEg Let $f:I \,{\rightarrow}\, {\RR}$ be a function defined on an open interval $I$.
    Assume that $f''(x)\,>\,0$ for all $x$ in $I$.

\V

        (a) Show that if $a$ and $b$ are in $I$, with $a\,<\,b$, then
        \begin{displaymath}
        f((1-t)a+tb)\,<\,(1-t)f(a) + tf(b) \mbox{ for all $t$ such that $0\,<\,t\,<\,1$}.
        \end{displaymath}

\V

        (b) Give a geometric interpretation of this result in terms of the graph of $f$.


\V
\V

\noindent  \ExEh The standard proof of Rolle's Theorem (Corollary~E.4.2 on Page~266), as taught in courses on elementary calculus,
    does {\em not} assume that one has already proved the Lagrange Mean-Value Theorem (Theorem~E.4.1 on Page~264).
    Indeed, the standard proof of the Mean-Value Theorem taught in such courses uses Rolle's Theorem.

\V

        (a) Give the standard proof of Rolle's Theorem. (Hint: What can you say about the location of the maximum and minimum values of $f$ on $[a,b]$ in light of the hypothesis that $f(b) \,=\, f(a)$?)

\V

        (b) Use Rolle's Theorem to give the standard proof of the Lagrange Mean-Value Theorem.
    (Hint: Let $g:{\RR} \,{\rightarrow}\, {\RR}$ be the linear function whose graph passes through the endpoints,
    $(a,f(a))$ and $(b,f(b))$, of the graph of $f$ on $[a,b]$.
    What does Rolle's Theorem say about $h \,=\, f-g$?)

\V
\V

\noindent \ExEi Suppose that $f:[0,1] \,{\rightarrow}\, {\RR}$ is continuous on $[0,1]$ and differentiable on $(0,1)$. Assume that $f(0) \,=\, 0$.
    Prove that if the derivative $f'$ is monotonic up on the open interval $(0,1)$, then so is the function $g:(0,1) \,{\rightarrow}\, {\RR}$ given by $g(x) \,=\, f(x)/x$.

%% Apostol Ex 5.17; M205B F04 HW #3 (II)(b)

\V
\V

\StartSkip{
\noindent \ExEj Find an example of a function $f:[0,+{\infty}) \,{\rightarrow}\, {\RR}$ such that $f'(0) \,=\, 0$ and $f'(x)\,\,{\geq}\,\,1$ for all $x\,>\,0$,
    or prove that no such function exists.

%% Apostol 5.23 Bad problem
}%EndSkip

\V
\V

\noindent \ExEj Suppose that $f$ is differentiable on an open interval $(c-{\delta},c+{\delta})$, where ${\delta}\,>\,0$.
    Let $h$ satisfy $0\,<\,h\,<\,{\delta}$.

\V

        (a) Show that there exists $t$ with $0\,<\,t\,<\,h$ such that
        \begin{displaymath}
        \frac{f(c+h) - f(c-h)}{h}  \,=\, f'(c+t) + f'(c-t)
        \end{displaymath}

\V

        (b) Show that there exists ${\tau}$ with $0\,<\,{\tau}\,<\,h$ such that
        \begin{displaymath}
        \frac{f(c+h) - 2f(c) + f(c-h)}{h} \,=\, f'(c+{\tau}) - f'(c-{\tau}).
        \end{displaymath}

\V

        (c) Show that if $f''(c)$ exists then
        \begin{displaymath}
        f''(c) \,=\, \lim_{h \,{\rightarrow}\, 0} \frac{f(c+h) - 2f(c) + f(c-h)}{h^{2}} \h ({\ast})
        \end{displaymath}

\V

        (d) Give an example of differentiable $f$ such that the limit on the right side of Equation~$({\ast})$ exists and is finite, but where $f''(c)$ does not exist.

%% Apostol 5.24

\V
\V

\noindent  \ExEk (a) Prove Parts (d) and (e) of Theorem~E.6.14 (see Pages~287-288). % ThmE45.127C

\V

        (b) Prove Part (g) of Theorem~E.6.16 (see Pages~288-289. % ThmE45.127E

\V
\V

\noindent  \ExEl Prove that, for each $x$ in ${\RR}$, $e^{x} \,=\, \lim_{k \,{\rightarrow}\, {\infty}} {\displaystyle \left(1+ \frac{x}{k}\right)^{k}}$.

        Hint: Consider the quantity ${\displaystyle {\ln}\,\left(\left(1+ \frac{x}{k}\right)^{k}\right)}$.

        \underline{Note}: Part of the solution of this exercise should include a proof of the fact that the limit in question actually exists;
    you cannot simply assume it.

\V
\V

\noindent  \ExEm Let $g:I \,{\rightarrow}\, {\RR}$ be a function which has an antiderivative on the open interval $I$.

        \underline{Problem} Determine the functions $f:I \,{\rightarrow}\, {\RR}$, if any, such that $f'(x) \,=\, g(x)f(x)$ for all $x$ in $I$.

\V
\V

\noindent  \ExEn Suppose that $f$ and $g$ are $C^{2}$ functions on an open interval $I$. Let $c$ be a point of $I$.

\V

        (a) Prove that the product functions $f{\cdot}g'$ and $g{\cdot}f'$ both have antiderivatives on $I$, and that
        \begin{displaymath}
        D^{-1}_{c} (f{\cdot}g') \,=\, f{\cdot}g - f(c){\cdot}g(c) - D^{-1}_{c} (g{\cdot}f').
        \end{displaymath}

\V

        (b) Suppose that $f$ is a real-valued function such that $f'$ is defined at each point of an open interval $I$ in ${\RR}$.
    Likewise, suppose that $g$ is a function which has an antiderivative on an open interval $J$ in ${\RR}$; let $G:J \,{\rightarrow}\, {\RR}$ be such an antiderivative.
    Assume that $f[I] \,{\subseteq}\, J$, so the composition $h \,=\, g{\circ}f:I \,{\rightarrow}\, {\RR}$ is defined.

    \underline{Problem}: Show that the function $h{\cdot}f'$ has an antiderivative on $I$, and that if $c$ is a point of $I$ then
$D^{-1}_{c}( h{\cdot}f') \,=\, G{\circ}f - G(f(c))$.

        \underline{Remark} In elementary calculus the result stated in Part~(a) is called the {\bf Law of Integration-by-Parts},
    while the result in Part~(b) is called the {\bf Substitution Law}.

\V
\V

\noindent  \ExEo \underline{Prove or Disprove}: If $f:{\RR} \,{\rightarrow}\, {\RR}$ satisfies $|f(y)-f(x)|\,\,{\leq}\,\,(y-x)^{2}$ for all $x$ and $y$ in ${\RR}$, then $f$ is constant.

\V
\V

\noindent  \ExEp Let $f:[a,b] \,{\rightarrow}\, {\RR}$ be a function which is differentiable at each point of a closed bounded interval $[a,b]$.
    (At each endpoint one uses the appropriate one-sided derivative.)
    Consider the following statements about $f$:

        \h (i)\, The function $f$ is $C^{1}$ on $[a,b]$.

        \h (ii) For every ${\varepsilon}\,>\,0$ there exists ${\delta}\,>\,0$ so that if $t$ and $x$ are in $[a,b]$ and $0\,<\,|t-x|\,<\,{\delta}$,
    then ${\displaystyle \left|\frac{f(t)-f(x)}{t-x} - f'(x)\right|\,<\,{\varepsilon}}$.

        \underline{Problem}:

\V

        (a) Prove or disprove that Statement (i) implies Statement (ii).

\V

        (b) Prove or disprove that Statement (ii) implies Statement (i).

\V
\V

\noindent  \ExEq Let $f$ and $g$ be functions which are continuous on a closed bounded interval $[a,b]$ and differentiable on the open interval $(a,b)$.
    Prove that there exists $c$ in $(a,b)$ such that
        \begin{displaymath}
        f'(c)(g(b) - g(a)) \,=\, g'(c)(f(b) - f(a)).
        \end{displaymath}

\V
\V

\noindent  \ExEr (a) Show that the sine function satisfies the equation ${\sin}\,''(x) \,=\, -{\sin}\,(x)$ except {\em possibly} at points of the form ${\pi}/2 - k{\pi}$ with $k$ in ${\bf Z}$.

\V

        (b) Show that the sine function is continuous at each point of the form ${\pi}/2-k{\pi}$ with $k$ in ${\bf Z}$.

\V

        (c) Prove Part~(a) of Theorem~E.6.24 (see Page $293$). %% ThmE.45.125C
    (Hint: At points of the form $x \,=\, {\pi}/2-k{\pi}$ consider using L'H\^{o}pital's Rule.)

\V

        (d) Prove Part~(b) of Theorem~E.6.24. %% ThmE.45.125C

\V
\V

\noindent  \ExEs Note: In this problem you may assume Parts (a) and (b) of Theorem~E.6.24 (see page~$293$). %% ThmE45.128C

\V

        (a) Prove Parts~(c) and~(d) of Theorem E.6.24. %% ThmE.45.125C

\V

        (b) Prove Parts~(e) and~(f) of Theorem E.6.24. %% ThmE.45.125C
    (Hint: You may wish to prove (f) first.)

\V
\V

\noindent \ExEt (a) Define the four remaining basic trigonometric functions ($\tan$, $\cot$, $\sec$ and $\csc$) in terms of the sine and cosine functions.
    Make it clear at which points of ${\RR}$ these functions fail to be defined.

\V

        (b) Derive the standard formulas for the derivatives of the functions $\tan$, $\cot$, $\sec$ and $\csc$.
    (`Standard formula': the derivatives of $\tan$ and $\sec$ should be expressed in terms of $\tan$ and $\sec$;
    likewise, the derivatives of $\cot$ and $\csc$ should be expressed in terms of $\cot$ and $\csc$.)

\V

        (c) Prove that the tangent function maps the open interval $(-{\pi}/2, {\pi}/2)$ bijectively onto ${\RR}$.

 \V
        (d) Let ${\arctan}:{\RR} \,{\rightarrow}\, (-{\pi}/2, {\pi}/2)$ be the inverse (relative to the interval $(-{\pi}/2, {\pi}/2)$) of the tangent function.
    Derive the formula for the derivative of the function ${\arctan}$.

\V
\V

\noindent  \ExEu Suppose that $f:I \,{\rightarrow}\, {\RR}$ is a $C^{k}$ function on the open interval $I$, and that $c$ and $x$ are in~$I$.
    Throughout this exercise let  $p_{k-1}$ denotes the Taylor polynomial of order $k-1$ for $f$ about the center $c$.

\V

        (a) Let $H_{k}:I \,{\rightarrow}\, {\RR}$ be given by the rule $H_{k}(t) \,=\, \frac{1}{(k-1)!} f^{(k)}(t)(x-t)^{k-1} \mbox{ for all $t$ in $I$}$.
    Show that
\[
        f(x) \,=\, p_{k-1}(x) + \left(D^{-1}_{c} H_{k}\right)(x) \h ({\ast})
\]
        Hint: Define $h:I \,{\rightarrow}\, {\RR}$ by the rule (REVISED)
        \begin{displaymath}
        h(t) \,=\, f(t) + f'(t)(x-t) + \frac{1}{2}f''(t)(x-t)^{2} + \,{\ldots}\,
    + \frac{f^{(k-1)}(t)}{(k-1)!}(x-t)^{k-1} \mbox{ for all $t$ in $I$}.
        \end{displaymath}
    Compute $h'$ and notice the considerable simplification that occurs.

        \underline{Remark} Equation~$({\ast})$ is often called the {\bf Integral Form of the Taylor Formula with Remainder};
    compare it with the `Derivative Form' presented in Theorem~E.7.9. (The word `integral' here is used in the classical sense of `indefinite integral;
    that is, `antiderivative'.)

\V

        (b) Show that if $m$ in ${\NN}$ satisfies $1\,\,{\leq}\,\,m\,\,{\leq}\,\,k$,
    then there exists ${\tau}$ in $\mbox{Seg}[c,x]$ such that
        \begin{displaymath}
        f(x) \,=\, p_{k-1}(x) + \frac{f^{(k)}{({\tau})}}{m(k-1)!}(x-{\tau})^{k-m}(x-c)^{m} \h ({\ast}{\ast})
        \end{displaymath}
    Note: The case $m \,=\, k$ in Equation~$({\ast}{\ast})$ is the `Derivative Form' from Theorem~E.7.9.
    The case $m \,=\, 1$ is called the {\bf Cauchy Form} of the remainder.

\V

        (c) Let $f(x) \,=\, {\ln}\,(1+x)$ for $x\,>\,-1$, with $c\,=\,0$.

        \underline{Problem} Use the Cauchy Form described above to show that $\lim_{k \,{\rightarrow}\, {\infty}} p_{k-1}(x) \,=\, f(x)$ for all $x$ such that $-1\,<\,x\,<\,1$.


