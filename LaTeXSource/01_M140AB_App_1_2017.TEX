% M140AB_A.TeX  Notes for `Single-Variable Analysis': Appendix A

%
% Revised: 08/29/2016 %% Encoding: Western ASCII
%


\appendix
                  \chapter{The Dedekind-Peano Axioms for ${\NN}$}

        \underline{Quotes for Appendix~A}\IndB{chapter quotes}{for Appendix~A}:

\V

\begin{quotation}
{\footnotesize

              (1) `Die ganzen Zahlen hat der liebe Gott gemacht, alles andere ist Menschenwerk.'

        (`The dear God has made the integers; all the rest is man's work.')

        Statement quoted of Leopold Kronecker.

}%EndFootNoteSize
\end{quotation}


\V
\V


        In this appendix we present an axiomatic approach to the concept of `counting'.
     The usual custom is to refer to the axioms discussed here as the `Peano Axioms'.
    However, since they apparently were discovered earlier (and independently) by Dedekind,
    in {\TheseNotes} we follow the usage of several authors and refer to them as the {\em Dedekind-Peano axioms}.
    The approach in this Appendix largely follows that of Dedekind in his famous essay {\em Was sind und was sollen die Zahlen};
    of course, Dedekind's terminology and notation have been upgraded to reflect the more modern usage.


        \underline{Note} If you have never worked through the `Dedekind-Peano' axiomatic approach to the natural numbers,
    it is worth your time to at least skim through the material in this appendix;
    however, doing so is not required for reading the preceding chapters.


\V
\V

        In Definition~\Ref{DefA12.70} we characterize what it means for a finite nonempty set $A$ to have (exactly) $k$ elements, for some $k{\in}{\NN}$,
    in terms of a complete pairing of $A$ with the subset~${\NN}_{k}$ of the `standard counting set~${\NN}$; that is, the set of standard counting numbers.
    Of course other sets could have been chosen to play the role of `standard counting set';
    for example, in many computer spreadsheets the columns are labeled by letters, not natural numbers:
        \begin{displaymath}
        A, B, \,{\ldots}\,Z, A\,A, A\,B,\,{\ldots}\,A\,Z, B\,A, B\,B\,{\ldots}\,
        \end{displaymath}
    In this appendix we characterize axiomatically the properties which any such comparison set ought to enjoy.


\V

        {\bf Add1-\Ref{ChaptA} 1: Definition} A {\bf counting structure} is a pair of objects $(X,{\sigma})$ consisting of a nonempty set $X$ together with a function ${\sigma}:X \,{\rightarrow}\, X$ which satisfies the following {\bf Dedekind-Peano Axioms}:

\V

        (a) The function ${\sigma}$ is one-to-one on $X$, and there is exactly one element of $X$ which is {\em not} in the image ${\sigma}(X)$.
    This element is denoted $u_{{\sigma}}$; the letter $u$ stands for `unit'.

\V

        (b) Suppose that $A$ is any subset of $X$ with the following properties:

            \h (i)\, The element $u_{{\sigma}}$ is in the set $A$;

            \h (ii)  For every element $x$ in $A$ the element ${\sigma}(x)$ is also in $A$;
    that is, ${\sigma}[A] \,{\subseteq}\, A$.

\noindent Then $A \,=\, X$.

\V


        {\bf Add1-\Ref{ChaptA} 2: Remarks}

        (1) \underline{Concerning Axiom (a)} The function ${\sigma}$ associated with the counting structure $(X,{\sigma})$ is usually called the {\bf successor function} of that structure,
    and if $x$ is any element of $X$ then ${\sigma}(x)$ is called the {\bf successor of $x$}.
    The requirement that ${\sigma}$ be one-to-one then can be phrased as `No  element of $X$ can be the successor of more than one element'.
    Likewise, the requirement on the special element $u_{{\sigma}}$ can be phrased as `The special element $u_{{\sigma}}$ is not the successor of any element, and it is the {\em unique} element of $X$ which is not a successor'.
    Because of this last fact, one often refers to $u_{{\sigma}}$ as the {\bf initial element of $X$}.
    When there can be no confusion, one normally writes the simpler $u$ instead of $u_{{\sigma}}$.

        Note that most authors assume the existence, but not the uniqueness, of an element $u$ in $X$ which is not the successor of any element.
    They can do this because the uniqueness can be trivially deduced later on using Axiom~(b).
    The choice, whether to treat the `uniqueness' as part of an axiom or as a theorem to be proved later on, is thus largely a matter of taste.

\V

        (2) \underline{Concerning Axiom (b)} For obvious reasons, this axiom is called the {\bf Induction Axiom},
    and any nonempty subset of $X$ which satisfies Part~(ii) of this condition is called an {\bf inductive subset of $X$}.
    Peano uses essentially this axiom in his formulation; that is, he {\em assumes} as an axiom the `Principle of Mathematical Induction'.

        In contrast, the analogous axiom in Dedekind's formulation can be phrased as follows:

        \h `The set $X$ is the intersection of the family of all the inductive subsets of $X$ containing the element $u$.'

\noindent (Note that $X$ itself is an inductive subset of $X$ containing $u$, so the family in question is not empty.)
    In particular, Dedekind does {\em not} assume the Principle of Mathematical Induction as an axiom;
    instead he {\em proves} that Principle as a consequence of his axioms.
    Likewise, one can prove that the Peano axiom implies the Dedekind version. (Both proofs are trivial.)
    In any event, one needs both the `Principle of Mathematical Induction' and Dedekind's `intersection of inductive subsets' construction to work out the theory;
    the choice, of which is to be an axiom and which is to be a theorem, is thus largely a matter of taste.

\V
\V

        {\bf Add1-\Ref{ChaptA} 3: Examples}

\V

        (1) Let $X$ be the standard set ${\NN} \,=\, \{1,2,3,\,{\ldots}\,\}$, whose elements are described by Arabic (decimal) numerals.
    Let the function ${\sigma}:{\NN} \,{\rightarrow}\, {\NN}$ be given by the usual rule for finding the successor of a natural number $k$:
    if the right-most (decimal) numeral of a number $k$ is one of the digits $0$, $1$,\,{\ldots}\,$8$, then replace that digit by the next higher one,
    and leave the other digits alone.
    If the decimal expression for $k$ ends with a string of one or more `$9$'s on the right, to get ${\sigma}(k+1)$ replace each such $9$ with the digit $0$,
    and increase the right-most non-$9$ digit to the next higher digit; if {\em all} the numerals of $k$ are $9$,
    then ${\sigma}(k+1)$ has initial numeral $1$ followed by as many $0$s as $k$ has $9$s.
    The initial element for this counting structure is~$1$.

        \underline{Note} A faster way of describing the successor function in this example would be to simply say ${\sigma}(k) \,=\, k+1$.
    The reason for using the wordier description given above is to emphasize that one does not have to know about `addition' --
    not even the special case of `addition with $1$' -- to characterize the successor function.
    Because of this, when we later {\em define} addition for arbitrary counting structures in terms of the successor function,
    we can avoid the sensation that the definition is somehow `circular' -- we define addition in terms of the successor function,
        which itself is often defined in terms of `addition with $1$'.
    Having done so in this key example, however, we take the easy way out in some of the following examples
    and describe the successor function there in terms of addition.

\V

        (2) The set $X$ is $\hat{{\NN}} \,=\, \{0,1,2,\,{\ldots}\,\}$, the function ${\sigma}:\hat{{\NN}} \,{\rightarrow}\, \hat{{\NN}}$ is given by ${\sigma}(k) \,=\, k+1$.
    Then $u \,=\, 0$.

\V

        (3) The set $X$ consists of all elements of ${\NN}$ starting with $7$:
    $X \,=\, \{7, 8, 9, \,{\ldots}\,\}$.
    The function ${\sigma}$ is given as usual by ${\sigma}(k) \,=\, k+1$.
    In this case the initial element is $u \,=\, 7$.

\V

        (4) The set $X$ consists of all the {\em even} natural numbers; that is, $X \,=\, \{2,4,6,\,{\ldots}\,\}$.
    The function ${\sigma}$ is given by the rule ${\sigma}(k) \,=\, k+2$, so that $u \,=\, 2$.

\V

        (5) The set $X$ consists of the standard Roman numerals
        \begin{displaymath}
        I, II, III, IV, V,\,{\ldots}\,X,\,{\ldots}\,C,\,{\ldots}\,M,\,{\ldots}\,
        \end{displaymath}
    Since we shall not be using these numerals extensively, we shall leave to the reader the happy task of determining the corresponding successor function~${\sigma}$.

    The reader should be able to easily figure out the pattern and from that to determine the corresponding `successsor function'~${\sigma}$.

\V
\V

        {\bf Add1-\Ref{ChaptA} 4: Theorem} Suppose that $(X,{\sigma})$ is a counting structure with initial element~$u$.
    Then for all $x$ in $X$ one has ${\sigma}(x) \,\,{\neq}\,\, x$.

\V

        {\bf Proof}\,  Let $A$ be the set of all $x$ in $X$ such that ${\sigma}(x) \,\,{\neq}\,\, x$.
    Certainly the initial element $u$ is in $A$; for if not then one would have ${\sigma}(u) \,=\, u$, contrary to the fact that the initial element is not in the image of the function ${\sigma}$.

        Next, suppose that $x{\in}A$. If ${\sigma}(x)$ were not in $A$, then one would have ${\sigma}({\sigma}(x)) \,=\, {\sigma}(x)$.
    But by the fact that ${\sigma}$ is one-to-one, one would then also have ${\sigma}(x) \,=\, x$, contrary to the induction hypothesis that $x{\in}A$.
    Thus, if $x{\in}A$ then ${\sigma}(x){\in}A$ as well.
    By the Induction Axiom it follows that $A \,=\, X$, and the desired result follows.

\V
\V


        Example~(3) above illustrates a way of obtaining new counting structures from a given such structure.

\V

        {\bf Add1-\Ref{ChaptA} 5: Theorem} Suppose that $(X,{\sigma})$ is a counting structure with initial element~$u$.
    Let $x_{1}$ be an element of $X$, and let $X_{1}$ denote the intersection of the family ${\cal F}_{1}$ of all the inductive subsets of $X$ which contain the element $x_{1}$.
    Then $X_{1}$ is a nonempty subset of $X$.
    Furthermore, if ${\sigma}_{1}$ denotes the restriction of the function ${\sigma}$ to the subset $X_{1}$,
    then $(X_{1},{\sigma}_{1})$ is a counting structure, and its initial element is $x_{1}$.

\V

        {\bf Proof}\,  Let $A$ be the set of all $x_{1}$ in $X$ for which the conclusion of the theorem is true.
    It suffices to show that $A \,=\, X$.

        \underline{Initial Step} Clearly $u{\in}A$, since by the Induction Axiom the only inductive subset of $X$ containing $u$ is $X$ itself, and thus when $x_{1} \,=\, u$ one has $X_{1} \,=\, X$ and ${\sigma}_{1} \,=\, {\sigma}$.

        \underline{Inductive Step} Now suppose that $w{\in}A$, and let $W$ denote the intersection of all inductive subsets of $X$ containing $w$,
    and let ${\sigma}_{w}$ denote the restriction of ${\sigma}$ to $W$. Then, by the definition of the set~$A$,
    the pair $(W,{\sigma}_{w})$ is a counting structure with initial element~$w$.
    Let $x_{1} \,=\, {\sigma}(w)$, and let ${\cal F}_{1}$, $X_{1}$ and ${\sigma}_{1}$ be as in the statement of the theorem.
    Note that the family ${\cal F}_{1}$ is nonempty, since $X$ itself is an inductive set containing $x_{1}$.
    In adddition, by definition of ${\cal F}_{1}$, every set in the family ${\cal F}_{1}$ contains $x_{1}$, hence so does the intersection~$X_{1}$.
    Thus, $x_{1}{\in}X_{1}$, so in particular $X_{1} \,\,{\neq}\,\, {\emptyset}$.
    Next, suppose that $y{\in}X_{1}$. Then for every set $Y$ in the family ${\cal F}_{1}$ one has $y{\in}Y$;
    and since each such set $Y$ is an inductive subset of $X$, it follows that ${\sigma}(y){\in}Y$ as well.
    Thus, ${\sigma}(y)$ is also in the intersection $X_{1}$, hence $X_{1}$ is an inductive set.
    In particular, the restriction ${\sigma}_{1}$ of ${\sigma}$ to $X_{1}$ maps $X_{1}$ into itself.
    It is clear that ${\sigma}_{1}$ is one-to-one on $X_{1}$, since it is the resriction to $X_{1}$ of the one-to-one function ${\sigma}$ on $X$.
    Furthermore, if $z$ is a point of $X_{1}$ such that $z \,\,{\neq}\,\, x_{1}$, then $z$ must be of the form ${\sigma}_{1}(x)$ for some $x$ in $X_{1}$.
    Indeed, if this were not the case, consider the set $Y \,=\, X_{1}{\setminus}\{z\}$.
    Since $z \,\,{\neq}\,\, x_{1}$ it is clear that $x_{1}{\in}Y$.
    And since $z$ is not in ${\sigma}_{1}(X_{1})$, and $X_{1}$ is an inductive subset of $X$,
    it is clear that $Y$ is a nonempty inductive subset of $X$ containing $x_{1}$.
    Thus, $X_{1}$, being the intersection of all such sets, must be a subset of $Y$.
    However, this is impossible since $z{\in}X_{1}$ but $z$ is {\em not} in $Y$.

        All that is left to show is that $x_{1}$ is not in ${\sigma}_{1}(X_{1})$.
    Since, by definition, $x_{1} \,=\, {\sigma}(w)$, and ${\sigma}$ is one-to-one on $X$, this means one need only show that $w$ is not in $X_{1}$.
    But if $w$ were in $X_{1}$, then $X_{1}$ would be an inductive subset of $X$ containing $w$,
    hence one would have $W \,{\subseteq}\, X_{1}$, since $W$ is the intersection of all such subsets.
    On the other hand, $w$ is in $W$, and $W$ is an inductive subset of $X$, so $x_{1} \,=\, {\sigma}(w)$ is also in $W$, so $W$ is an inductive subset of $X$ containing $x_{1}$.
    Since $X_{1}$ is the intersection of all such subsets of $X$, it follows that $X_{1} \,{\subseteq}\, W$.
    Combining these results, one sees that if $w{\in}X_{1}$, then $X_{1} \,=\, W$.
    However, since $w{\in}A$ it follows that $w$ is the initial element of $W$, hence if one removes $w$ from $W \,=\, X_{1}$ one would get a proper subset of $X_{1}$ which contains $x_{1} \,=\, {\sigma}(w)$ and is inductive.
    (Note that ${\sigma}(w) \,\,{\neq}\,\, w$ because $w$ is the initial element of $W$; that is, $x_{1} \,\,{\neq}\,\, w$. Thus, removing $w$ from $W$ does not remove $x_{1}$ from $W$.)
    This would contradict the definition of $X_{1}$ as the intersection of all such subsets of $X$.

\V

        {\bf Add1-\Ref{ChaptA} 6: Definition} Let $(X,{\sigma})$ be a counting structure with initial element $u$.
    Let $x_{1}$ be an element of $X$. Then the counting structure $(X_{1},{\sigma}_{1})$ with initial element $x_{1}$
    described in the preceding theorem is called the {\bf counting substructure of $(X,{\sigma})$ determined by $x_{1}$}.
    The set $X_{1}$ so described is denoted $<x_{1}>_{{\sigma}}$, and the correspondng function ${\sigma}_{1}$ is denoted ${\sigma}_{x_{1}}$.

\V
\V

        By a similar line of reasoning one can prove the following result; the details are left as an exercise.

\V

        {\bf Add1-\Ref{ChaptA} 7: Corollary} Let $(X,{\sigma})$ be a counting structure with initial element $u$.
    Let $x_{1}$ be an element of $X$, and let $(X_{1},{\sigma}_{1})$ be the counting substructure of $(X,{\sigma})$ determined by $x_{1}$.
    Then the counting substructure of $(X,{\sigma})$ determined by $x_{2} \,=\, {\sigma}(x_{1})$ is of the form $(X_{2},{\sigma}_{2})$,
    where $X_{2} \,=\, {\sigma}(X_{1}) \,=\, X_{1}{\setminus}\{x_{1}\}$.

\V
\V

        {\bf Add1-\Ref{ChaptA} 8: Theorem} Let $(X,{\sigma})$ be a counting structure with initial element $u$.
    Let $x_{1}$ and $x_{2}$ be elements of $X$ such that $x_{1} \,\,{\neq}\,\, x_{2}$, and let $(X_{1},{\sigma}_{1})$ and $(X_{2},{\sigma}_{2})$ be the corresponding counting substructures of $(X,{\sigma})$.
    Then exactly one of the following statements is true:

        \h (i)\, $x_{1}$ is in $X_{2}$

        \h (ii)  $x_{2}$ is in $X_{1}$

\V

        {\bf Proof}\,  To see that {\em at most} one of these properties can hold for such $x_{1}$ and $x_{2}$,
    suppose that $x_{1}$ is in $X_{2}$. Then $X_{2}$ is an inductive subset of $X$ which contain $x_{1}$.
    Since $x_{2} \,\,{\neq}\,\, x_{1}$, and (by the preceding theorem) $x_{2}$ is the initial element of the set $X_{2}$,
    it follows that the set $Y \,=\, X_{2}{\setminus}\{x_{2}\}$ obtained by removing $x_{2}$ from $X_{2}$ is also an inductive subset of $X$ which contains $x_{1}$.
    Since $X_{1}$ is, by definition, the intersection of such subsets, it follows that $X_{1} \,{\subseteq}\, Y$.
    But $Y$ does not contain $x_{2}$, hence neither does $X_{1}$, as claimed.
    A similar argument shows that if $x_{2}$ is in $X_{1}$ then $x_{1}$ cannot be an element of $X_{2}$.

        To see that {\em at least} one of these properties must hold, let $A$ be the set of all $x_{1}$ in $X$ such that if $x_{2}$ is an element of $X$ not equal to $x_{1}$, then either (i) or (ii) holds.

        \underline{Initial Step} Certainly $x_{1} \,=\, u$ is in $A$, since in this case $X_{1} \,=\, X$ and thus (ii) holds for every $x_{2}$.

        \underline{Induction Step} Suppose that $w{\in}A$, and let $(W,{\sigma}_{w})$ denote the counting substructure of $(X,{\sigma})$ determined by $w$.
    Let $x_{1} \,=\, {\sigma}(w)$, and suppose that $x_{2}$ is an element of $X$ with $x_{2} \,\,{\neq}\,\, x_{1}$.
    Let $X_{1}$ and $X_{2}$ be as in the statement of the theorem. Note that, by the preceding corollary, one has $X_{1} \,=\, {\sigma}[W]$.
    If $x_{2} \,=\, w$ then $x_{1} \,=\, {\sigma}(x_{2})$, and thus $x_{1}$ is an element of $W_{2}$, so (i) holds.
    Thus, suppose $x_{2} \,\,{\neq}\,\, w$.
    Then, by the induction hypothesis that $w$ is in $A$, it follows that either $w$ is in $X_{2}$ or $x_{2}$ is in $W$.
    In the former case it follows (from the fact that $X_{2}$ is an inductive set) that $x_{1} \,=\, {\sigma}(w)$ is also in $X_{2}$, and thus (i) holds.
    In the latter case, since $x_{2} \,\,{\neq}\,\, w$, it follows from the fact that ${\sigma}[W] \,=\, W{\setminus}\{w\}$, that $x_{2}{\in}{\sigma}[W]$; that is, since ${\sigma}[W] \,=\, X_{1}$, $x_{2}{\in}X_{1}$, and thus in this case (ii) holds.
    Thus, in all the cases either (i) or (ii) holds, hence ${\sigma}(w)$ is also in $A$.

        It now follows from the Induction Axiom that $A \,=\, X$, and the desired theorem follows.

\V
\V

        In light of the preceding theorem, it is now easy to introduce the concept of `greater than' (or equivalently, `less than') for any counting structure.

\V

        {\bf Add1-\Ref{ChaptA} 9: Definition} Let $(X,{\sigma})$ be a counting structure. Let $x_{1}$ and $x_{2}$ be elements of $X$ with $x_{1} \,\,{\neq}\,\, x_{2}$,
    and let $(X_{1},{\sigma}_{1})$ and $(X_{2},{\sigma}_{2})$ denote the counting substructures of $(X,{\sigma})$ determined by $x_{1}$ and $x_{2}$, respectively.
    One says that {\bf $x_{2}$ is greater than $x_{1}$} or, equivalently, {\bf $x_{1}$ is less than $x_{2}$}, with respect to the given structure $(X,{\sigma})$, provided $x_{2}$ is in $X_{1}$.
    In this case one writes $x_{2}\,>_{(X,{\sigma})}\,x_{1}$ (equivalently, $x_{1}\,<_{(X,{\sigma})}\,x_{2}$);
    or, if the choice of counting structure $(X,{\sigma})$ is understood from the context, simply $x_{2}\,>\,x_{1}$ (equivalently, $x_{1}\,<\,x_{2}$).

        More generally, one writes $x_{2}\,\,{\geq}\,\,x_{1}$, or, equivalently, $x_{1}\,\,{\leq}\,\,x_{2}$, if either $x_{1} \,=\, x_{2}$ or $x_{2}\,>\,x_{1}$.

        If $w$ is any element of $X$, then $X_{w}$ denotes the set of all elements $x$ of $X$ such that $x\,\,{\leq}\,\,w$;
    this set is called the {\bf initial section of $X$}.
    More generally, if $w_{1}$ and $w_{2}$ are elements of $X$ such that $w_{1}\,\,{\leq}\,\,w_{2}$,
    then $X_{[x_{1},x_{2}]}$ denotes the set of all $x$ in $X$ such that $w_{1}\,\,{\leq}\,\,x\,\,{\leq}\,\,w_{2}$.
    Note that $X_{w} \,=\, X_{[u,w]}$.

\V

        {\bf Add1-\Ref{ChaptA} 10: Examples} $X_{u} \,=\, \{u\}$; $X_{{\sigma}(u)} \,=\, \{u,{\sigma}(u)\}$.
    More generally, for each $x$ in $X$ one has $X_{{\sigma}(x)} \,=\, X_{x}\,{\cup}\,\{{\sigma}(x)\}$.

\V

        The next result follows easily from what precedes, so the proof is left as an exercise.

\V

        {\bf Add1-\Ref{ChaptA} 11: Theorem} Let $(X,{\sigma})$ be a counting structure, and let $\,<\,$ denote the corresponding `less than' relation.
    Then:

\V

        (a) (`The Trichotomy Law') If $x_{1}$ and $x_{2}$ are elements of $X$, then exactly one of the following statements is true:

        \h (i)\,\,$x_{1} \,=\, x_{2}$

        \h (ii)\, $x_{1}\,<\,x_{2}$

        \h (iii) $x_{2}\,<\,x_{1}$

\V

        (b) (`The Transitivity Law') If $x_{1}$, $x_{2}$ and $x_{3}$ are elements of $X$ such that $x_{1}\,<\,x_{2}$ and $x_{2}\,<\,x_{3}$, then $x_{1}\,<\,x_{3}$.

\V
\V


        {\bf Add1-\Ref{ChaptA} 12: Theorem} Let $(X,{\sigma})$ be a counting structure. If $x_{1}$ is any element of $X$, then $x_{1}\,<\,{\sigma}(x_{1})$.
    Moreover, then there is no element $w$ of $X$ such that $x_{1}\,<\,w\,<\,{\sigma}(x_{1})$.
    (This is often phrased: `There is no element of $X$ between two consecutive elements of $X$'.)

\V

        {\bf Proof}\,  Let $x_{2} \,=\, {\sigma}(x_{1})$, and as usual denote the counting substructures of $(X,{\sigma})$ determined by $x_{1}$ and $x_{2}$, respectively, by $(X_{1},{\sigma}_{1})$ and $(X_{2},{\sigma}_{2})$.
    It follows from the corollary above that $X_{2} \,=\, X_{1}{\setminus}\{x_{1}\}$; that is, $X_{2}$ is obtained by removing from $X_{1}$ the single element $x_{1}$.
    Now suppose that there exists $w$ in $X$ such that $x_{1}\,<\,w\,<\,x_{2}$.
    Denote the counting substructure of $(X,{\sigma})$ determined by $w$ as $(X_{w},{\sigma}_{w})$.
    Then, by the definition of the relation `$\,<\,$', the element $w$ must be in the set $X_{1}$ but not in the set $X_{2}$.
    However, the only element of $X_{1}$ which is not in $X_{2}$ is $x_{1}$, which would imply $w \,=\, x_{1}$.
    This would contradict the Trichotomy Law, since it is given that $x_{1}\,<\,w$.
    Thus, no such $w$ can exist, which proves the desired result.

\V


        {\bf Add1-\Ref{ChaptA} 13: Theorem} Let $(X,{\sigma})$ be a counting structure. Then every nonempty subset $Y$ of $X$ has a least element;
    that is, there is an element $y_{0}$ in $X$ such that $y_{0}{\in}Y$ and $y_{0}\,\,{\leq}\,\,y$ for all $y$ in~$Y$.

\V

        This is essentially just the `Least-Natural-Number Principle',
    but expressed in the more general context of counting structures.
    The proof is left as an exercise.

\V
\V


        The next result says, in effect, that all counting structures are equivalent in a natural sense,
    and thus it does not matter which specific example one uses as one's `standard counting structure'.

\V

        {\bf Add1-\Ref{ChaptA} 14: Theorem} Let $(X,{\sigma})$ and $(Y,{\tau})$ be counting structures, with initial elements $u$ and $v$, respectively.
    Then there exists a unique bijection $F:X \,{\rightarrow}\, Y$ of $X$ onto $Y$ such that

        \h (i)\, $F(u) \,=\, v$;

        \h (ii) $F({\sigma}(x)) \,=\, {\tau}\left(F(x)\right)$ for all $x$ in~$X$.

\V

        {\bf Proof}\,  The proof of the stated result follows directly from the following.

        \underline{Claim} For each $w$ in $X$ there exists a unique function $F_{w}: X_{{\sigma}(w)} \,{\rightarrow}\, Y$
    such that $F_{w}(u) \,=\, v$ and $F_{w}({\sigma}(x)) \,=\, {\tau}\left(F_{w}(x)\right)$ for all $x$ in $X_{w}$.

        \underline{Proof (`by Contradiction') of the Claim} Let $C$ denote the set of all $w$ in $X$ for which it is {\em not}
    the case that there exists a unique function with the indicated properties.
    If $C \,\,{\neq}\,\, {\emptyset}$, then by the Theorem of the Least Element the set $C$ must have a least element; call it $m$.
    Clearly $m \,\,{\neq}\,\, u$, since it is obvious that the function $F_{u}:X_{{\sigma}(u)} \,{\rightarrow}\, Y$ given by the rule $F_{u}(u) \,=\, v$, $F_{u}({\sigma}(u)) \,=\, {\tau}(v)$ has the desired properties, and is the only such function.
    Thus, $m$ must be of the form $m \,=\, {\sigma}(w_{1})$ for some (unique) $w_{1}$ in $X$; clearly $w_{1}\,<\,m$.
    Since, by hypothesis, $m$ is the {\em least} element of $C$, it follows that $w_{1}$ is not in $C$,
    and thus there is exactly one function $F_{w_{1}}:X_{{\sigma}(w_{1})} \,{\rightarrow}\, Y$ which has the desired properties.
    Note that the domain of $F_{w_{1}}$ can also be written as $X_{m}$ since $m \,=\, {\sigma}(w_{1})$.
    Now define $F_{m}:X_{{\sigma}(m)} \,{\rightarrow}\, Y$ by the rule
        \begin{displaymath}
        F_{m}(x) \,=\, F_{w_{1}}(x) \mbox{ if $x{\in}X_{m}$}; \h F_{m}({\sigma}(m)) \,=\, {\tau}\left(F_{m}(m)\right).
        \end{displaymath}
    It is clear that this function also satisfies the conditions of the `Claim', and thus $m$ is not in $C$.
    That is, assuming that $C \,\,{\neq}\,\, {\emptyset}$ leads to an element which is simultaneously in $C$ and not in $C$, which is impossible.
    Thus, $C \,=\, {\emptyset}$, and the claim follows.

        The theorem now follows easily.
    Indeed, it is clear that the set $X$ is the union of the nonempty subsets of the form $X_{x}$ for $x$ in $X$.
    Moreover, since the intersection of sets $X_{x_{1}}$ and $X_{x_{2}}$ is of the form $X_{x}$, with $x$ being the larger of $x_{1}$ and $x_{2}$,
    it follows from the uniqueness properties enjoyed by the functions $F_{w}$ described in the Claim that Theorem~\Ref{ThmA30.27} can be applied to conclude that there is a unique function $F:X \,{\rightarrow}\, Y$ whose restriction to each set $X_{{\sigma}(w)}$ equals $F_{w}$.
    In light of the results of the `Claim', this function clearly satisfies Conditions~(i) and~(ii) of the theorem, and is the only function that does.
    To see that $F$ is a bijection of $X$ onto $Y$, first note that in the special case $(Y,{\tau}) \,=\, (X,{\sigma})$ one gets that there is a unique function $H:X \,{\rightarrow}\, X$ such that $H(u) \,=\, u$ and $H({\sigma}(x)) \,=\, {\sigma}\left(H(x)\right)$ for all $x$ in $X$.
    However, it is clear that the identity map $I_{X}$ on $X$ has these properties, so in this case $H \,=\, I_{X}$.
    Next, note that by reversing the roles of $(X,{\sigma})$ and $(Y,{\tau})$ in the theorem,
    it follows that there exists a unique function $G:Y \,=\, X$ such that $G(v) \,=\, u$ and $G({\tau}(z)) \,=\, {\sigma}\left(G(z)\right)$ for all $z$ in $V$.
    Now let $H:X \,{\rightarrow}\, X$ be the composition $H \,=\, G{\circ}F$. Then $H(u) \,=\, G(F(u)) \,=\, G(v) \,=\, u$, and $H({\sigma}(x)) \,=\, G((F({\sigma}(x)))) \,=\, G({\tau}\left(F(x))\right) \,=\, {\sigma}(G(F(x))) \,=\, {\sigma}(H(x))$ for all $x$.
    Thus by what was observed above, it follows that $G \,=\, I_{X}$. In a similar way one gets $F{\circ}G \,=\, I_{Y}$.
    It now follows from Part~(c) of Theorem~\Ref{ThmA30.160} that $F$ is a bijection of $X$ onto $Y$, and that $G \,=\, F^{-1}$.

\V
\V

        The Peano Axioms are strong enough to characterize all the main features of the standard comparison sets for counting.
    For example, here is how to define `addition'.

\V

        {\bf Add1-\Ref{ChaptA} 15: Theorem} Let $(X,{\sigma})$ be a counting structure with initial element~$u$.
    Then there is a unique function $S_{{\sigma}}:X{\times}X \,{\rightarrow}\, X$, called the {\bf sum function associated with the structure $(X,{\sigma})$},
    with the following properties: If $x$ is any element of $X$, then

        (a) $S_{{\sigma}}(x,u) \,=\, S_{{\sigma}}(u,x) \,=\, {\sigma}(x)$;

        (b) $S_{{\sigma}}(x,{\sigma}(y)) \,=\, {\sigma}(S_{{\sigma}}(x,y))$ for all $y$ in $X$.

\noindent If the context makes clear which counting structure $(X,{\sigma})$ is under discussion,
    one normally drops the explicit reference to ${\sigma}$ and writes $S$ instead of the more proper $S_{{\sigma}}$.

\V

        \underline{NOTE} To understand how these formulas arise, first think of the expression $S(x,y)$ as a shorthand for `the {\em sum} of $x$ and $y$'.
    Then in the context of  ${\NN}$, for which $u \,=\, 1$ and ${\sigma}(x) \,=\, x+1$, the formulas become
        \begin{displaymath}
        \mbox{(i) } S(x,1) \,=\, x+1 \mbox{ and } \mbox{(ii) } S(x,y+1) \,=\, S(x,y)+1.
        \end{displaymath}
    Equation~(i) says that the `sum' $S$ agrees with the usual addition~$+$ in ${\NN}$ when applied to $x$ and $1$;
    that is, the `sum of $x$ and $1$', as defined by the function $S$, equals the usual $x+1$ in ${\NN}$.
    Equation~(ii) says, in effect, that if $S(x,y)$ agrees with the usual $x+y$ in ${\NN}$ for a particular second summand $y$,
    then it continues to agree with the usual addition for the next larger second summand $y+1$.
    Indeed, if $S(x,y) \,=\, x+y$, then $S(x,y)+1 \,=\, (x+y)+1 \,=\, x+(y+1)$, where the last equation reflects the associative law for ordinary addition of  natural numbers.
    Thus, Equation~(ii) becomes $S(x,y+1) \,=\, x+(y+1)$. Combining these observations with induction on the second summand $y$ then yields the fact that, 
    in the case of ${\NN}$, the sum $S$ satisfies $S(x,y) \,=\, x+y$ for {\em all} $x$ and $y$.

\V

        {\bf Proof of Theorem}

\V

        \underline{Uniqueness of $S$} Suppose that $S_{1}$ and $S_{2}$ are both functions which satisfy the conditions stated in the theorem.
    Then certainly $S_{1}(x,u) \,=\, S_{2}(x,u)$ for all $x$ in $X$, since, by~(a), both quantities equal ${\sigma}(x)$.

        Next, let $A$ denote the set of $y$ in $X$ such that $S_{1}(x,y) \,=\, S_{2}(x,y)$ for all $x$ in $X$.
    By what was just proved, it is clear that $u{\in}A$.
    Furthermore, if $y{\in}A$, then by~(b) one has
        \begin{displaymath}
        S_{1}(x,{\sigma}(y)) \,=\, {\sigma}(S_{1}(x,y)) \mbox{ and }
        S_{2}(x,{\sigma}(y)) \,=\, {\sigma}(S_{2}(x,y)) \mbox{ for all $x$ in $X$}.
        \end{displaymath}
    However, by the hypothesis that $y{\in}A$ one has $S_{1}(x,y) \,=\, S_{2}(x,y)$, so that ${\sigma}(y)$  is also in $A$.
    Now the Induction Axiom implies that $A \,=\, X$.
    Thus, $S_{1}(x,y) \,=\, S_{2}(x,y)$ for all $(x,y)$ in $X{\times}X$, so $S_{1} \,=\, S_{2}$, as claimed.

\V

        \underline{Existence of $S$} According to Definition~\Ref{DefA30.10}, a function with domain $X{\times}X$ and values in $X$ is a subset of the
    Cartesian product $Z \,=\, (X{\times}X){\times}X$ which satisfies certain properties.
    The approach followed here is to construct the subset of $Z$ corresponding to the desired function $S$ -- viewed as a  subset of $Z$ --
    as the disjoint union of smaller subsets of $Z$. In effect, we use the fact that $Z$  can be expressed as the disjoint union
        \begin{displaymath}
        Z \,=\, {\bigcup}_{x{\in}X} \left(\{x\}{\times}X\right){\times}X
        \end{displaymath}

        Let $A$ be the set of all $x$ in $X$ such that the following holds:
    There exists a nonempty subset $Y_{x}$ of $Z$ such that

        \h (i)\,\, Every element of $Y_{x}$ is of the form $((x,y),z)$ for some $y$ and $z$ in $X$.
    That is, $Y_{x}$ is a subset of $\left(\{x\}{\times}X\right){\times}X$.

        \h (ii)\, For every $y$ in $X$ there is exactly one element of the form $((x,y),z)$ in $Y_{x}$.
    That is, $Y_{x}$ is a function (in the sense of Definition~\Ref{DefA30.10}) with domain $\{x\}{\times}X$ and values in $X$.

        \h (iii) The points $((x,u),{\sigma}(x))$ and $((u,x),{\sigma}(x))$ are in $Y_{x}$.
    Likewise, for every $y$ in $X$ if $((x,y),z)$ is in $Y_{x}$ then $((x,{\sigma}(y)),{\sigma}(z))$ is also in $Y_{x}$.
    That is, $Y_{x}$ satisfies Conditions~(a) and~(b) for the desired function~$S$.

\noindent It is easy to show, by Mathematical Induction, that $A \,=\, X$:


        \underline{Initial Step} To see that $u{\in}A$, define $Y_{u}$ to be the set of all $((x,y),z)$ of the form $((u,y),{\sigma}(y))$ for $y$ in $X$.
    Since ${\sigma}$ is a function with domain $X$ and with values in $X$, it is clear that Conditions~(i) and~(ii) for the set $Y_{u}$ hold.
    As for Condition~(iii), note that when $y \,=\, u$ then the element $((u,y),{\sigma}(y))$ becomes $((u,u),{\sigma}(u))$, so that the first portion of the condition is satisfied.
    Likewise, if $((u,y),z)$ is in $Y_{u}$, then (by definition of $Y_{u}$) one has $z \,=\, {\sigma}(y)$.
    But (again be the definition of $Y_{u}$) one also has $((u,{\sigma}(y)),{\sigma}({\sigma}(y)))$ in $Y_{u}$;
    that is, the second part of Condition~(iii) also holds, so $u{\in}A$.

        \underline{Induction Step} Suppose that $x{\in}A$, and let $Y_{x}$ be a set which satisfies Conditions~(i), (ii) and~(iii).
    Then for each $y$ in $X$ there is a unique element $y'$ in $X$ such that $((x,y),y')$ is in $Y_{x}$.
    (Of course $y'$  depends on both $x$ and $y$.)
    Using this notation, define $Y_{{\sigma}(x)}$ to be the set of all elements  of $Z$ of the form $(({\sigma}(x),y),{\sigma}(y'))$ with $y$ in $X$.
    It is easy to see, from the properties of $Y_{x}$, that $Y_{{\sigma}(x)}$ also satisfies Conditions~(i), (ii) and~(iii), and thus ${\sigma}(x)$ is in $A$.

        Finally, define $S$ to be the union of the sets $Y_{x}$, with $x$ in $X$, constructed above.
    The sets $Y_{x}$ are clearly disjoint, so it is easy to see that the set $S$ is a function with domain $X{\times}X$ and with values in $X$.

    And since each set $Y_{x}$ satisfies Conditions~(i), (ii) and~(iii), it is clear that the function $S$ satisfies Conditions~(a) and~(b) of the theorem.

\V
\V

        In a like manner one can define `multiplication'.

\V

        {\bf Add1-\Ref{ChaptA} 16: Theorem} Let $(X,{\sigma})$ be a counting structure with initial element~$u$.
    Then there is a unique function $P_{{\sigma}}:X{\times}X \,{\rightarrow}\, X$, called the {\bf product function associated with the structure $(X,{\sigma})$},
    with the following properties: If $x$ is any element of $X$, then

        (a) $P_{{\sigma}}(x,u) \,=\, P_{{\sigma}}(u,x) \,=\, x$;

        (b) $P_{{\sigma}}(x,{\sigma}(y)) \,=\, S(P_{{\sigma}}(x,y),x)$ for all $y$ in $X$, where $S$ denotes the sum function $S_{{\sigma}}$ associated with $(X,{\sigma})$.

\noindent As usual, if there is no possibility of confusion one normally writes $P$ instead of $P_{{\sigma}}$.

\V

        {\bf Proof}\,  Left to the reader.

\V
\V


        {\bf Add1-\Ref{ChaptA} 17: Examples}

\V

        (1) Let $X \,=\, {\NN} \,=\, \{1,2,3,\,{\ldots}\,\}$, with the usual rule for the successor function ${\sigma}$; see Example~P3~(1) above.
    Then the successor operation can be written
        \begin{displaymath}
        {\sigma}(x) \,=\, x+1 \mbox{ for all $x$ in ${\NN}$},
        \end{displaymath}
    where throughout this example the symbol $+$ denote the usual addition of natural numbers.
    Likewise, the operations $S$ and $P$ associated with this counting structure are clearly the usual addition and multiplication $+$ and ${\cdot}$ on ${\NN}$.
    For example, the condition $S(x,1) \,=\, {\sigma}(x)$ then takes the form
        \begin{displaymath}
        S(x,1) \,=\, x+1.
        \end{displaymath}
    Thus, by induction on $y$, if $x$ and $y$ are elements of ${\NN}$ such that $S(x,y) \,=\, x+y$,
then the condition $S(x,{\sigma}(y)) \,=\, {\sigma}\left(S(x,y)\right)$ takes the form
        \begin{displaymath}
        S(x,y+1) \,=\, (x+y)+1 \,=\, x+(y+1),
        \end{displaymath}
    in which the final equation reflects the usual properties of arithmetic in ${\NN}$.
    That is, $S(x,y) \,=\, x+y$ holds for all $x$, $y$  in ${\NN}$.
    A similar argument shows that $P(x,y) \,=\, x{\cdot}y$ for all $x$, $y$ in ${\NN}$.

\V

        (2) Now let $X \,=\, \hat{{\NN}} \,=\, \{0,1,2,\,{\ldots}\,\}$, so that the role of initial element $u$ is now played by the number~$0$.
    In this case the `addition' and `multiplication' functions associated with the given counting structure do {\em not} agree with the usual $+$ and ${\cdot}$ of numbers.
    For example, $x+u \,=\, x+0 \,=\, x \,\,{\neq}\,\, {\sigma}(x)$, since in $\hat{{\NN}}$ one has ${\sigma}(x) \,=\, x+1$.
    Thus, it is not the case that $S(x,y) \,=\, x+y$ for all $x$, $y$.
    Likewise, $x{\cdot}0 \,=\, 0 \,\,{\neq}\,\,x$ when $x \,\,{\neq}\,\, 0$,  so it is not the case that $P(x,y) \,=\, x{\cdot}y$ for all $x$, $y$.
    Indeed, one can easily show that in $\hat{{\NN}}$ one has
        \begin{displaymath}
        S(x,y) \,=\, x+y+1 \mbox{ and } P(x,y) \,=\, x(y+1) + y \mbox{ for all $x$, $y$ in $\hat{{\NN}}$}
        \end{displaymath}

        The fact that in the example of $\hat{{\NN}}$ the operations $S$ and $P$ as define above do not agree with the usual addition and multiplication on $\hat{{\NN}}$ is not a flaw.
    It simply reflects the fact that, in any choice of addition and multiplication on a counting structure $(X,{\sigma})$,
    one must decide the role to be played by the initial element $u$.
    The choice taken in {\TheseNotes} reflects the fact that we shall use ${\NN}$, not $\hat{{\NN}}$,
    as our primary model of a counting structure.
    It is an easy exercise to provide an alternative formulation of `addition'  and `multiplication' axioms for those who prefer that the initial element $u$ behave like $0$.

\V
\V

        {\bf Add1-\Ref{ChaptA} 18: Modified Notation} It is customary to use the more familiar notation $1_{{\sigma}}$, $x+_{{\sigma}}y$, $x{\cdot}_{{\sigma}}y$
    instead of $u_{{\sigma}}$, $S_{{\sigma}}(x,y)$ and $P_{{\sigma}}(x,y)$ when dealing with a general counting structure $(X,{\sigma})$.
    And, as usual, if the context makes clear which ${\sigma}$ is under consideration, the subscript ${\sigma}$ is normally omitted and one writes $1$, $x+y$ and $x{\cdot}y$ instead.
    With this notation, the basic defining properties of addition and multiplication take the following more familiar form for all $x$ (or, where appropriate, all $x$ and $y$) of $X$:
        \begin{displaymath}
        (i)\, {\sigma}(x) \,=\, x+1 \,=\, 1+x;\, (ii)\, x+(y+1) \,=\, (x+y)+1; \, (iii)\, x{\cdot}1 \,=\, 1{\cdot}x \,=\, x; \, (iv)\, x{\cdot}(x+1) \,=\, x{\cdot}y+x \h ({\ast})
        \end{displaymath}

\V
\V

        The operations of addition and multiplication for a counting structure obey the usual rules of grade-school arithmetic.

\V

        {\bf Add1-\Ref{ChaptA} 19: Theorem} Let $(X,{\sigma})$ be a counting structure, with initial element $1$, addition operation $+$ and multiplication ${\cdot}$.
    Then the following facts hold.

\V

        (a) (Commutative Laws) If $x$ and $y$  are in $X$  then
        \begin{displaymath}
        (i)\, x+y \,=\, y+x \mbox{ and } (ii)\, x{\cdot}y \,=\, y{\cdot}x.
        \end{displaymath}


\V

        (b) (Associative Laws) If $x$, $y$ and $z$ are in $X$  then
        \begin{displaymath}
        (i)\, (x+y)+z \,=\, x+(y+z) \mbox{ and } (ii)\, (x{\cdot}y){\cdot}z \,=\, x{\cdot}(y{\cdot}z).
        \end{displaymath}


\V

        (c) (Distributive Laws) If $x$, $y$ and $z$ are in $X$ then
        \begin{displaymath}
        (i)\, x{\cdot}(y+z) \,=\, x{\cdot}y + x{\cdot}z \mbox{ and } (ii)\, (x+y){\cdot}z \,=\, x{\cdot}y + x{\cdot}z.
        \end{displaymath}



\V

        {\bf Partial Proof} The laws above are grouped to emphasize the similarities shared by addition and multiplication, and to ease the task of learning their statements;
    for example, the two commutative laws are joined together, and are stated before the two (more complicated) associative laws.
    The proofs of the laws as given here, however, require handling them in a different order.
    For instance, we use the Commutative and Associative Laws for Addition, together with the Distributive Laws,
    in the proof of the Associative Law for Multiplication.
    As usual, since the structure of the proofs of the various laws are so similar, we leave some of the proofs as exercises.

\V

        \underline{Part $(i)$ of (b)} (Associative Law for Addition) Let $A_{1}$ denote the set of $z$ in $X$ such that $(x+y)+z \,=\, x+(y+z)$ for all $x$ and $y$ in $X$.
    It is clear that it suffices to show that $A_{1} \,=\, X$.

        \underline{Initial Step} Property~$(ii)$ of $({\ast})$ above states that $1{\in}A_{1}$.

        \underline{Inductive Step} Suppose that $z{\in}A_{1}$. Then for all $x$ and $y$ in $X$ one has
        \begin{displaymath}
        x+(y+(z+1)) \stackrel{(1)}{ \,=\, } x+((y+z)+1) \stackrel{(2)}{ \,=\, } (x+(y+z))+1  \stackrel{(3)}{ \,=\, } ((x+y)+z)+1  \stackrel{(4)}{ \,=\, }
        (x+y)+(z+1)
        \end{displaymath}
    That is, $(z+1)$ is also in $A_{1}$, so by mathematical induction $A_{2} \,=\, X$, as desired.

        Here are the justifications for the numbered equations above:

        \h Equation~(1): This comes by applying Property~(ii) of $({\ast})$ to the expression $y+(z+1)$.

        \h Equation~(2): This follows from the fact, proved above, that $1{\in}A_{1}$.

        \h Equation~(3): This reflects the induction hypothesis that $z{\in}A_{1}$.

        \h Equation~(4): This comes by applying Property~$(ii)$ of $({\ast})$ to the expression $((x+y)+z)+1$.

\V

        \underline{Part $(i)$ of $(a)$} (Commutative Law for Addition) Let $A_{2}$ be the set of all $y$ in $X$ such that $x+y \,=\, y+x$ for all $x$ in $X$.


        \underline{Initial Step} It is clear, from the defining proeprties of `addition',
    that $x+1 \,=\, 1+x \,=\, {\sigma}(x)$ for all $x$ in $X$, so certainly $1{\in}A_{2}$.

        \underline{Inductive Step} Suppose that $y{\in}A_{2}$. Then for all $x$ in $X$ one has
        \begin{displaymath}
        x+(y+1) \stackrel{(1)}{ \,=\, }
        (x+y)+1 \stackrel{(2)}{ \,=\, }
        (y+x)+1 \stackrel{(3)}{ \,=\, }
        y+(x+1) \stackrel{(4)}{ \,=\, }
        y+(1+x) \stackrel{(5)}{ \,=\, }
        (y+1)+x.
        \end{displaymath}
    That is, $(y+1){\in}A_{2}$, so $A_{2} \,=\, X$, and the desired result follows.

        Here are the justifications of the preceding equations:

        \h Equations $(1)$, $(3)$ and $(5)$: the Associative Law for Addition

        \h Equations $(2)$ and $(4)$: the fact that $1$ and (by the Induction Hypothesis) $y$ are both in $A_{2}$.


\V


        \underline{Part $(i)$ of(c)} (The First Distributive Law) Let $A_{3}$ be the set of all $z$ in $X$ such that $x{\cdot}(y+z) \,=\, x{\cdot}y + x{\cdot}z$ for all $x,y$ in $X$.

        \underline{Initial Step} Note that, from the defining properties of `multiplication', one has
        \begin{displaymath}
        x{\cdot}(y+1) \,=\, x{\cdot}y + x \,=\, x{\cdot}y+x{\cdot}1.
        \end{displaymath}
    That is, $x{\cdot}(y+1) \,=\, x{\cdot}y+x{\cdot}1$ for all $x$ and $y$ in $X$, hence $1{\in}A_{3}$.

        \underline{Induction Step} Suppose that $z{\in}A_{3}$. Then for all $x$ and $y$ in $X$ one has
        \begin{displaymath}
        x{\cdot}(y+(z+1)) \stackrel{(1)}{ \,=\, }
        x{\cdot}((y+z)+1) \stackrel{(2)}{ \,=\, }
        x{\cdot}(y+z) + x \stackrel{(3)}{ \,=\, }
        (x{\cdot}y + x{\cdot}z)+x \stackrel{(4)}{ \,=\, }
        x{\cdot}y+(x{\cdot}z+x) \stackrel{(5)}{ \,=\, }
        x{\cdot}y+x{\cdot}(z+1).
        \end{displaymath}
    Thus, $z+1$ is also in $A_{3}$. It now follows that $A_{3} \,=\, X$, so the First Distributative Law, $x{\cdot}(y+z) \,=\, x{\cdot}y + x{\cdot}z$, follows.

    The justifications of the preceding equations are as follows:

        \h Equations $(1)$ and $(4)$: The Associative Law for Addition

        \h Equations $(2)$ and $(5)$: Defining properties of `multiplication'

        \h Equation $(3)$: The induction hypothesis that $z{\in}A_{3}$.

\V

        \underline{Part $(ii)$ of(c)} (The Second Distributive Law) Let $A_{4}$ be the set of all $z$ in $X$ such that $(x+y){\cdot}z \,=\, x{\cdot}z + y{\cdot}z$ for all $x,y$ in $X$.

        \underline{Initial Step} Since $(x+y){\cdot}1 \,=\, x+y \,=\, x{\cdot}1 + y{\cdot}1$ by one of the defining properties of `multiplication',
    it follows that $1{\in}A_{4}$.

        \underline{Induction Step} Suppose that $z{\in}A_{4}$. Then for all $x$ and $y$ in $X$ one has
        \begin{displaymath}
        (x+y){\cdot}(z+1) \stackrel{(1)}{ \,=\, }
        (x+y){\cdot}z + (x+y) \stackrel{(2)}{ \,=\, }
        (x{\cdot}z+y{\cdot}z) + (x+y) \stackrel{(3)}{ \,=\, }
        (x{\cdot}z + x) + (y{\cdot}z + y) \stackrel{(4)}{ \,=\, }
        x{\cdot}(z+1) + y{\cdot}(z+1).
        \end{displaymath}
    Thus $z+1$ is also in $A_{4}$. It follows that $A_{4} \,=\, X$, so the desired result holds.

        The reader should be able to figure out the justifications for the preceding equations.
    Note that Equation~$(3)$ uses several applications of the commutative and associative laws for addition;
    be sure you can break it down to the individual applications of those laws.
    Also, be sure you can determine where the induction hypothesis gets used.

\V

        \underline{Part $(ii)$ of (a)} (The Commutative Law for Multiplication) Let $A_{5}$ be the set of all $y$ in $X$ such that $x{\cdot}y \,=\, y{\cdot}x$ for all $x$ in $X$.

        \underline{Initial Step} One of the defining proerties of  'multiplication' is that $x{\cdot}1 \,=\, 1{\cdot}x \,=\, x$ for all $x$ in $X$,
    so clearly $1{\in}A_{5}$.

        \underline{Induction Step} Suppose that $y{\in}X$. Then for all $x$ in $X$ one has
        \begin{displaymath}
        x{\cdot}(y+1) \,=\, x{\cdot}y+x \,=\, y{\cdot}x + x \,=\, y{\cdot}x + 1{\cdot}x \,=\, (y+1){\cdot}x.
        \end{displaymath}
    (The reader is invited to explain why each of these equations is valid.)
    In particular, $y+1$ is also in $A_{5}$, and thus $A_{5} \,=\, X$, as desired.

\V

        The proofs of the remaining laws are left as exercises.


\V
\V

        There is a simple relation between the operation of `addition' and the concept of `greater than'.

\V

        {\bf Add1-\Ref{ChaptA} 20: Theorem} Let $(X,{\sigma})$ be a counting structure, with associated initial element $1$,
    addition operation $+$ and `greater than' ordering $\,>\,$.
    Then a pair of elements $x$ and $y$ in $X$ satisfy the condition $y\,>\,x$ if, and only if, there exists $z$ in $X$ such that $y \,=\, x+z$.

\V

        {\bf Proof}  

\V

        (The `If' Part) Let $A$ be the set of all $z$ in $X$ such that for all $x$ in $X$ the elements $x$ and $x+z$ satisfy $x+z\,>\,x$.
    Certainly $1{\in}A_{1}$,  by Theorem~Add1-\Ref{ChaptA}~12.
    Next, suppose that $z{\in}A$.
    Then, for all $x$ in $X$, one has $(x+z)+1\,>\,x+z$ (since $1{\in}A$), and $x+z\,>\,x$ (since, by the induction hypothesis, $z{\in}A_{1}$).
    Then, from the Transitive Law one obtains $(x+z)+1\,>\,x$; that is, in light of the Associative Law, $x+(z+1)\,>\,x$.
    It follows that $z+1$ is also in $A$.
    Now apply the Principle of Mathematical Induction to conclude that $A \,=\, X$, as required.

\V

        (The `Only if' Part) Suppose that $x$ and $y$ are elements of $X$ such that $y\,>\,x$.
    Let $C$ be the set of all $w$ in $X$ such that $x+w\,\,{\geq}\,\,y$.
    By what was just proved, it is clear that $y{\in}C$, since $x+y\,>\,y$.
    In particular, $C$ is a nonempty subset of $X$, so, by Theorem~Add1-\Ref{ChaptA}~13, there is a smallest element of $C$; call it $z$.
    Clearly $x+z\,\,{\geq}\,\,y$, since $z$ is in $C$. If $z \,=\, 1$ then one has $x+1\,\,{\geq}\,\,y\,>\,x$,
    which implies $x+1 \,=\, y$, since there  are no elements of $X$ which lie between $x$ and $x+1$.
    Thus, suppose $z \,\,{\neq}\,\, 1$, so that $z \,=\, v+1$ for some $v$ in $X$.
    Since $z$ is the {\em smallest} element of $C$, and clearly $v\,<\,z$, it follows that $v$ is {\em not} in $C$.
    That is, it is not the case that $x+v\,\,{\geq}\,\,y$, hence one has $x+v\,<\,y$.
    Thus, one has $x+v\,<\,y\,\,{\leq}\,\,(x+v)+1$. Since there are no elements of $X$ between $(x+v)$ and $(x+v)+1$,
    it must be the case that $y \,=\, (x+v)+1 \,=\, x+(v+1) \,=\, x+z$, as claimed.

\V
\V

        There is a similar theorem involving multiplication and the ordering `greater than'.

\V

        {\bf Add1-\Ref{ChaptA} 21: Theorem} Let $(X,{\sigma})$ be a counting structure, with associated initial element $1$,
    multiplication operation ${\cdot}$ and `greater than' ordering $\,>\,$.
    If $x$ and $y$  are elements of $X$ such that $y\,>\,x$,
    then $y{\cdot}z\,>\,x{\cdot}z$ for all $x$ in $X$.

\V

        The simple proof is left as an exercise.

\V
\V


        There is a result, {\em not} one  of the standard facts from grade-school arithmetic, that is worth mentioning here.

\V
\V

        {\bf Add1-\Ref{ChaptA} 22: Theorem} Let $(X,{\sigma})$ and $(Y,{\tau})$ be counting structures with initial elements $1_{{\sigma}}$ and $1_{{\tau}}$,
    addition operations $+_{{\sigma}}$ and $+_{{\tau}}$,
    multiplication operations ${\cdot}_{{\sigma}}$ and ${\cdot}_{{\tau}}$, and orderings $\,>_{{\sigma}}$ and $\,>_{{\tau}}$, respectively.
    Furthemore, let $F:X \,{\rightarrow}\, Y$ be the bijection described in Theorem~Add1-\Ref{ChaptA}~14.
  Then the bijection $F$ preserves addition and multiplication, in the following sense:
        \begin{displaymath}
        F(x_{1}+_{{\sigma}}x_{2}) \,=\, F(x_{1}) +_{{\tau}} F(x_{2}) \mbox{ and }
        F(x_{1}{\cdot}_{{\sigma}}x_{2}) \,=\, F(x_{1}) {\cdot}_{{\tau}} F(x_{2})
    \mbox{ for all $x_{1},x_{2}$ in $X$}.
        \end{displaymath}
    Likewise, $F$ preserves the ordering, in the following sense:
        \begin{displaymath}
        \mbox{ If $x_{2}\,>_{{\sigma}}x_{1}$ then $F(x_{2})\,>_{{\tau}}F(x_{1})$}.
        \end{displaymath}
    (For readers with a background in modern algebra: the map $F$ is an {\em isomorphism} between the two counting structures.)


    The simple proof of this theorem is left as an exercise; it boils down to noticing that the definitions of
    addition, multiplication and the order all come from the properties of  the successor functions ${\sigma}$ and ${\tau}$;
    and the bijection $F$ `preserves' those functions because of the condition $F({\sigma}(x)) \,=\, {\tau}(F(x))$.

        The import of this result is that it does not matter which counting structure one elects to use: they are equivalent.
    From now on we shall use the standard example ${\NN}$ from grade-school arithmetic, for which the successor function is `addition by~$1$'.

\V
\V

        It is useful to prove some of the results that were accepted without proof earlier in {\TheseNotes}.
    For example, here is a more complete treatment of Theorem~\Ref{ThmA15.30}


        {\bf Add1-\Ref{ChaptA} 23: Theorem}
            %\subsection{\small{\bf Theorem}}
            %\label{ThmA15.30}

\hspace*{\parindent}(a) Let $X$ be a nonempty finite set. If $X$ has the same cardinality as ${\NN}_{k}$ and the same cardinality as ${\NN}_{m}$ for natural numbers $k$ and $m$, then $k \,=\, m$.

\V

        (b) If $Y$ is a subset of a finite set $X$, then $Y$ is a finite set, and $\#(Y)\,\,{\leq}\,\,\#(X)$.
    Moreover, the only time one gets $\#(Y) \,=\, \#(X)$ is when $Y \,=\, X$.
    In particular, $X$ cannot have the same cardinality as one of its proper subsets (i.e., there is no `Galileo Paradox' for finite sets.)


\V

        (c) Suppose that $\{X_{1},X_{2},\,{\ldots}\,X_{n}\}$ is a finite collection of finite sets.
    Then the union $X_{1}\,{\cup}\,X_{2}\,{\cup}\,\,{\ldots}\,\,{\cup}\,X_{n}$ is also a finite set. More precisely,
        \begin{displaymath}
         \#(X_{1}\,{\cup}\,X_{2}\,{\cup}\,\,{\ldots}\,\,{\cup}\,X_{n})
    \,\,{\leq}\,\,
        \#(X_{1}) + \#(X_{2}) + \,{\ldots}\,+\#(X_{n}).
        \end{displaymath}
    One gets equality in this last relation if, and only if, the sets are mutually disjoint,
    in the sense that $X_{i}\,{\cap}\,X_{j} \,=\, {\emptyset}$ whenever $i \,\,{\neq}\,\, j$.

\V


        {\bf Proof}\, 

\V

        (a) It follows easily from Part~(c) of Theorem~\Ref{ThmA15.15} that the issue to be proved reduces to this:
    if $k$ and $m$ are natural numbers such that ${\NN}_{k}$ has the same cardinality as ${\NN}_{m}$, then $k \,=\, m$;
    equivalently: if $k \,\,{\neq}\,\, m$, then ${\NN}_{k}$ does {\em not} have the same cardinality as ${\NN}_{m}$.
    Then in light of Part~(b) of the same theorem, the problem reduces to proving the following statement:

\V

        \h For every $j$ in ${\NN}$, if $k$ in ${\NN}$ satisfies the condition $k\,>\,j$, then ${\NN}_{j}$ does not have the same cardinality as ${\NN}_{k}$.

\V

\noindent We shall prove a slightly more precise result: For every $j$ in ${\NN}$. if $k{\in}{\NN}$ satisfies $k\,>\,j$,
    then there is no surjection of ${\NN}_{j}$ onto ${\NN}_{k}$.

        Indeed, let $A$ be the set of $j$ in ${\NN}$ for which this last statement is true.


        \underline{Initial Step} It is clear that $1{\in}A$.
    Indeed, suppose that $k\,>\,1$, so that $k\,\,{\geq}\,\,2$ and thus $1$ and $2$ are both elements of ${\NN}_{k}$.
    If $F:{\NN}_{1} \,{\rightarrow}\, {\NN}_{k}$ were a surjection onto ${\NN}_{k}$,
    then there would have to exist $x_{1}$ and $x_{2}$ in ${\NN}_{1}$ such that $F(x_{1}) \,=\, 1$ and $F(x_{2}) \,=\, 2$.
    However, the only element of ${\NN}_{1}$ is~$1$, so this would require $x_{1} \,=\, x_{1} \,=\, 1$, and thus $F(1) \,=\, 1$ and $F(1) \,=\, 2$.
    Viewing $F$ as a set of ordered pairs, this would mean that $(1,1){\in}F$ and $(1,2){\in}F$, contrary to the definition of `function'.

        \underline{Inductive Step} Suppose that $j{\in}A$. If $j+1$ were {\em not} in $A$,
    then there would have to exist a surjection $F$ from ${\NN}_{j+1}$ onto ${\NN}_{m}$ for some $m\,>\,j+1$.
    Note that in this case one would certainly have $k \,=\, m-1\,>\,j$. There are two cases to consider:

        \h \underline{Case 1}: Suppose that $F(j+1) \,=\, m$.
    Define $G:{\NN}_{j} \,{\rightarrow}\, {\NN}_{k}$ by the rule
        \begin{displaymath}
        G(i) \,=\, \left\{
        \begin{array}{cl}
        F(i) & \mbox{if $F(i) \,\,{\neq}\,\, m$} \\
           1 & \mbox{if $F(i) \,=\, m$}
        \end{array}
        \right.
        \end{displaymath}
    It is easy to see that $G$ would have to map ${\NN}_{j}$ onto ${\NN}_{k}$ with $k\,>\,j$.
    This would contradict the induction hypothesis that $j{\in}A$.

        \h \underline{Case 2}: Suppose that $F(j+1) \,\,{\neq}\,\, m$. Let $p \,=\, F(j+1)$, so that $1\,\,{\leq}\,\,p\,\,{\leq}\,\,k$.
    Now define $G:{\NN}_{j} \,{\rightarrow}\, {\NN}_{k}$ by the rule
        \begin{displaymath}
        G(i) \,=\,
        \left\{
        \begin{array}{cl}
        F(i) & \mbox{if $F(i) \,\,{\neq}\,\, m$} \\
          p  & \mbox{if $F(i) \,=\, m$}
        \end{array}
        \right.
        \end{displaymath}
    It is easy to see that $G$ maps ${\NN}_{j}$ onto ${\NN}_{k}$, contrary to the hypothesis that $j{\in}A$.

    This argument shows that $j+1$ is also in $A$. Thus, by the Principle of Mathematical Induction one concludes that $A \,=\, {\NN}$. The desired result now follows.

\V

        (b) and (c): These follow easily from Part~(a) combined with Mathematical Induction. The details are left as exercises.

\VV

        There are many more results about ${\NN}$ (or, if you prefer, about counting structures) that one could list.
    However, the main purpose of this appendix is to convince the reader  that the standard properties of ${\NN}$ can be derived from the Dedekind-Peano axioms.
    The results already given here are sufficient for that purpose.

