% Exercises_M140AB_D.TeX   Exercises for Chapter D

%
% Revised: 06/26/09
%

%% NOTE: Copy the 52 lines below to each chapter, and change the Chapter letters.

%\thispagestyle{myheadings}


%\markboth{Exercises for Chapter~\ref{ChaptD} -}{Exercises for Chapter~\ref{ChaptD} -}

\newcommand{\ExDa}{{\bf \ref{ChaptD} - \,1} }
\newcommand{\ExDb}{{\bf \ref{ChaptD} - \,2} }
\newcommand{\ExDc}{{\bf \ref{ChaptD} - \,3} }
\newcommand{\ExDd}{{\bf \ref{ChaptD} - \,4} }
\newcommand{\ExDe}{{\bf \ref{ChaptD} - \,5} }
\newcommand{\ExDf}{{\bf \ref{ChaptD} - \,6} }
\newcommand{\ExDg}{{\bf \ref{ChaptD} - \,7} }
\newcommand{\ExDh}{{\bf \ref{ChaptD} - \,8} }
\newcommand{\ExDi}{{\bf \ref{ChaptD} - \,9} }
\newcommand{\ExDj}{{\bf \ref{ChaptD} -  10} }
\newcommand{\ExDk}{{\bf \ref{ChaptD} -  11} }
\newcommand{\ExDl}{{\bf \ref{ChaptD} -  12} }
\newcommand{\ExDm}{{\bf \ref{ChaptD} -  13} }
\newcommand{\ExDn}{{\bf \ref{ChaptD} -  14} }
\newcommand{\ExDo}{{\bf \ref{ChaptD} -  15} }
\newcommand{\ExDp}{{\bf \ref{ChaptD} -  16} }
\newcommand{\ExDq}{{\bf \ref{ChaptD} -  17} }
\newcommand{\ExDr}{{\bf \ref{ChaptD} -  18} }
\newcommand{\ExDs}{{\bf \ref{ChaptD} -  19} }
\newcommand{\ExDt}{{\bf \ref{ChaptD} -  20} }
\newcommand{\ExDu}{{\bf \ref{ChaptD} -  21} }
\newcommand{\ExDv}{{\bf \ref{ChaptD} -  22} }
\newcommand{\ExDw}{{\bf \ref{ChaptD} -  23} }
\newcommand{\ExDx}{{\bf \ref{ChaptD} -  24} }
\newcommand{\ExDy}{{\bf \ref{ChaptD} -  25} }
\newcommand{\ExDz}{{\bf \ref{ChaptD} -  26} }


\newcommand{\ExDaa}{{\bf \ref{ChaptD} - 27} }
\newcommand{\ExDab}{{\bf \ref{ChaptD} - 28} }
\newcommand{\ExDac}{{\bf \ref{ChaptD} - 29} }
\newcommand{\ExDad}{{\bf \ref{ChaptD} - 30} }
\newcommand{\ExDae}{{\bf \ref{ChaptD} - 31} }
\newcommand{\ExDaf}{{\bf \ref{ChaptD} - 32} }
\newcommand{\ExDag}{{\bf \ref{ChaptD} - 33} }
\newcommand{\ExDah}{{\bf \ref{ChaptD} - 34} }
\newcommand{\ExDai}{{\bf \ref{ChaptD} - 35} }
\newcommand{\ExDaj}{{\bf \ref{ChaptD} - 36} }
\newcommand{\ExDak}{{\bf \ref{ChaptD} - 37} }
\newcommand{\ExDal}{{\bf \ref{ChaptD} - 38} }
\newcommand{\ExDam}{{\bf \ref{ChaptD} - 39} }
\newcommand{\ExDan}{{\bf \ref{ChaptD} - 40} }
\newcommand{\ExDao}{{\bf \ref{ChaptD} - 41} }
\newcommand{\ExDap}{{\bf \ref{ChaptD} - 42} }
\newcommand{\ExDaq}{{\bf \ref{ChaptD} - 43} }
\newcommand{\ExDar}{{\bf \ref{ChaptD} - 44} }
\newcommand{\ExDas}{{\bf \ref{ChaptD} - 45} }
\newcommand{\ExDat}{{\bf \ref{ChaptD} - 46} }
\newcommand{\ExDau}{{\bf \ref{ChaptD} - 47} }
\newcommand{\ExDav}{{\bf \ref{ChaptD} - 48} }
\newcommand{\ExDaw}{{\bf \ref{ChaptD} - 49} }
\newcommand{\ExDax}{{\bf \ref{ChaptD} - 50} }
\newcommand{\ExDay}{{\bf \ref{ChaptD} - 51} }
\newcommand{\ExDaz}{{\bf \ref{ChaptD} - 52} }



                       \section{EXERCISES FOR CHAPTER~\ref{ChaptD}}
                        \label{SectDEX}

\V
\V
\V
\V

\noindent  \ExDa Prove Part~(a) of Theorem~\Ref{ThmD20.85}.

\V
\V

\noindent \ExDb Let $X$ be a nonempty subset of ${\RR}$. Define $d_{X}:{\RR} \,{\rightarrow}\, {\RR}$ by the rule
        \begin{displaymath}
        d_{X}(y) \,=\, {\inf}\,\{t: t \,=\, |y-x| \mbox{ for some $x$ in $X$}\}
        \end{displaymath}

\V

        (a) \underline{Prove or Disprove}: The function $d_{X}$ is continuous at each point of its domain ${\RR}$.

\V

        (b) Give an example of a nonempty set $X$ in ${\RR}$ such that $d_{X}$ does not have a minimum value.


\V
\V

\noindent \ExDc Let $f:(a,b) \,{\rightarrow}\, {\RR}$ be a real-valued function defined on an open interval $I \,=\, (a,b)$.

\V

        Let $c$ be a point of $I$. Show that $f$ is continuous at $c$ if, and only if, the following condition holds:

        \h For every open subset $U$ of ${\RR}$ containing the point $f(c)$, the inverse image $f^{-1}[U]$ is an open subset of ${\RR}$ containing~$c$.

\V
\V

\noindent \ExDd Let $X$ be a nonempty subset of ${\RR}$, and suppose that $f_{1}$, $f_{2}$,\,{\ldots}\,$f_{k}$ are real-valued functions with domain $X$.
    Let $c$ be a point of $X$.

\V

    (a) Prove that the function ${\max}\,\{f_{1},\,{\ldots}\,f_{k}\}:X \,{\rightarrow}\, {\RR}$ is continuous at $c$.

\V

    (b) Prove that the function ${\min}\,\{f_{1},\,{\ldots}\,f_{k}\}:X \,{\rightarrow}\, {\RR}$ is continuous at $c$.


\V
\V

\noindent \ExDe Let $f:{\RR} \,{\rightarrow}\, {\RR}$ be a real-valued function whose domain is ${\RR}$.
    Prove that $f$ is continuous on ${\RR}$ if, and only if, for every bounded open interval $I$ in ${\RR}$ the restriction of $f$ to $I$ is continuous on $I$.
    Be sure to make clear which part of your solution proves the `if' portion of the statement, and which proves the `only if' portion.

\V
\V

\noindent \ExDf The proof of the Intermediate-Value Theorem for Continuous Functions (Theorem~\Ref{ThmD30.70}) %% Thm D.2.5 page 232
    given in the {\em Notes} is based on the Bisection Principle.

        \underline{Problem} Give an alternate proof that is based on the Supremum Principle.

\V
\V

\noindent \ExDg \underline{Prove or Disprove} If $f:[a,b] \,{\rightarrow}\, {\RR}$ is a continuous function,
    and if $g:[a,b] \,{\rightarrow}\, {\RR}$ is defined by the rule
        \begin{displaymath}
        g(x) \,=\, {\sup}\,\{f(u):a\,\,{\leq}\,\,u\,\,{\leq}\,\,x\} \mbox{ for $x$ in $[a,b]$},
        \end{displaymath}
    then $g$ is also continuous.

\V
\V

\noindent \ExDh Let $f:{\RR}{\setminus}\{1\} \,{\rightarrow}\, {\RR}$ and $g:{\RR}{\setminus}\{1\} \,{\rightarrow}\, {\RR}$ be given by
        \begin{displaymath}
        f(x) \,=\, \frac{x^{3} - 1}{x-1} \mbox{ if $x \,\,{\neq}\,\, 1$},
        \end{displaymath}
    and
        \begin{displaymath}
        g(x) \,=\, \frac{x^{2} + x - 2}{x-1} \mbox{ if $x \,\,{\neq}\,\, 1$}
        \end{displaymath}
    \underline{Problem} Determine whether there is a choice of $C$ in ${\RR}$ such that the concatenation $f\&_{(1,C)}g$ is continuous on ${\RR}$.

\V
\V

\noindent \ExDi \underline{Prove or Disprove} If $f:{\RR} \,{\rightarrow}\, {\RR}$ is a function such that $f^{2}$ is continuous on ${\RR}$, then $f$ is also continuous on ${\RR}$.

\V
\V

\noindent \ExDj \underline{Prove or Disprove} If $f$ and $g$ are real-valued functions defined on ${\RR}$,
    and $f$ and $f{\circ}g$ are continuous, then $g$ is continuous.

\V
\V

\noindent \ExDk {\bf Definition} (1) A real-valued function $f$ defined on an interval $I$ in ${\RR}$ is said to be a {\bf convex function on $I$} provided the following condition holds:
        \begin{displaymath}
        \mbox{If $x$ and $y$ are in $I$, then }
        f(tx + (1-t)y)\,\,{\leq}\,\,tf(x) + (1-t)f(y) \mbox{ for all $t$ such that $0\,\,{\leq}\,\,t\,\,{\leq}\,\,1$}.
        \end{displaymath}
    The function $f$ is said to be {\bf strictly convex on $I$} provided one has, in addition, when $x\,<\,y$ and $0\,<\,t\,<\,1$ one has
       % \begin{displaymath}
        $f(tx + (1-t)y)\,<\,tf(x) + (1-t)f(y)$.
       % \end{displaymath}

        (2) A real-valued function $g$ on an interval $I$ in ${\RR}$ is said to be a {\bf concave function on $I$} provided $-g$ is a convex function on $I$.
    Likewise, $g$ is said to be {\bf strictly concave on $I$} provided $-g$ is  a strictly convex function on $I$.

\V

        \underline{Problem}

        (a) Give geometric interpretations, in terms of the relation between the graph of a function $f$ on an interval $I$ and the chords joining pairs of points on that graph,
    of the statements `$f$ is a convex function on $I$' and `$f$ is a strictly convex function on $I$'.

\V

        (b) Determine the convexity/concavity properties the `absolute-value' function $\mbox{abs}:{\RR} \,{\rightarrow}\, {\RR}$, given by the rule
        %\begin{displaymath}
        $\mbox{abs}\,(x) \,=\, |x| \mbox{ for all $x$ in ${\RR}$}$.
        %\end{displaymath}
    (By `determine the convexity/concavity properties' is meant: determine the largest intervals on which the function is convex, strictly convex, concave or strictly concave.)

\V

        \underline{Warning} The meanings of `convex' and `concave' given here are quite standard in advanced texts in analysis.
    However, most textbooks on elementary calculus use a different terminology. 
    Specifically, they say `concave up' instead of `convex function' and `concave down' instead of `concave function';
    a similar usage holds for `strict' convexity/concavity.
    The use of the word `concave' in the phrase `concave up' to describe `convexity' can sometimes cause confusion, so be careful.

\V
\V

\noindent \ExDl (a) Show that if a real-valued function is convex on an open interval $(a,b)$, then it is continuous on $(a,b)$.

\V

        (b) Give an example of a function $f$, whose domain is a {\em closed} interval $[a,b]$, such that $f$ is convex on $[a,b]$, but is not continuous on $[a,b]$.

\V
\V

\noindent \ExDm Suppose that $f$ is a continuous real-valued function defined on an interval $I$ in ${\RR}$ and that $f$ satisfies the condition
        \begin{displaymath}
        f\left(\frac{x+y}{2}\right)\,\,{\leq}\,\,\frac{f(x) + f(y)}{2} \mbox{ for all $x,y$ in $I$}.
        \end{displaymath}
    Prove that $f$ is convex on $I$.

\V
\V

\noindent \ExDn {\bf Definition} A function $f:{\RR} \,{\rightarrow}\, {\RR}$ is said to be an {\bf additive function} provided $f(x+y) \,=\, f(x)+f(y)$ for all $x$ and $y$ in ${\RR}$.

        Prove that if $f$ is an additive function which is continuous at~$0$ then $f$ is a linear function;
    more precisely show that there exists a real number $c$ such that $f(x) \,=\, cx$ for all $x$ in ${\RR}$.

\V
\V

\noindent \ExDo \underline{Prove or Disprove} If $f:{\RR} \,{\rightarrow}\, {\RR}$ is a continuous function such that $|f(y)-f(x)|\,\,{\geq}\,\,|y-x|$ for all $x$ and $y$ in ${\RR}$, then $f$ maps ${\RR}$ \underline{onto} ${\RR}$.

\V
\V

\noindent \ExDp \underline{Remark} In the proof of the Extreme-Value Theorem found in the {\em Notes}, one obtains the result that if $f$ is a continuous real-valued function defined on a closed bounded interval $I$, then $f$ is bounded above on $I$.
    However, this fact comes almost as an after-thought: one first shows that the supremum of $f$ over $I$ equals $f(c)$ for some $c$ in $I$,
    and thus ${\sup}\,f[I]$ must be finite; in particular, $f$ must be bounded above on $I$.
    In other words, one shows that $f$ is bounded by {\em first} showing that $f$ assumes a maximum value on $I$.

        The goal of the present exercise is to show directly that such a function $f$ must be bounded above on $I$;
    indeed, {\em two} approaches to this result are provided.
    The following execise then asks one to use this boundedness to prove that $f$ assumes a maximum on $I$.

\V

        Suppose that $f:I \,{\rightarrow}\, {\RR}$ is a real-valued function whose domain is a closed bounded interval $I \,=\, [a,b]$ in ${\RR}$,
    and assume that $f$ is continuous at each point of $I$.

\V

        (a) Give a direct proof of the fact that $f$ is bounded on $I$ by considering the set $X$ of all $x\,>\,a$ in $[a,b]$ such that $f$ is bounded on the subinterval $[a,x]$.
    In the course of events you may need to show that the set $X$ is nonempty, and you may wish to consider the supremum of this set.

\V

        (b) Give a direct proof of the fact that $f$ is bounded on $I$ by noting that for each $x$ in $I$ there exists ${\delta}_{x}\,>\,0$ such that
    $f$ is bounded on the subset $U_{x} \,=\, (x-{\delta}_{x},x+{\delta})\,{\cap}\,I$. Then prove that there is a finite subset of the family ${\cal F} \,=\, \{U_{x}:x{\in}I\}$ whose union has $I$ as a subset.

\V
\V

\noindent \ExDq Give a proof, of the Extreme-Value Theorem for continuous real-valued functions defined on closed bounded intervals, which is based on the fact that continuous real-valued functions defined on closed bounded intervals are bounded functions.

\V
\V

\noindent \ExDr {\bf Definition} Let $f:X \,{\rightarrow}\, {\RR}$ be a real-valued function whose domain is a nonempty subset $X$ of ${\RR}$,
    and let $c$ be a point of $X$.

        \h (i)\, One says that $f$ is {\bf upper-semicontinuous at $c$} provided that for every ${\varepsilon}\,>\,0$ there exists ${\delta}\,>\,0$
    such that if $x$ in $X$ satisfies $|x-c|\,<\,{\delta}$ then $f(x)\,\,{\leq}\,\,f(c) + {\varepsilon}$.

        \h (ii) One says that $f$ is {\bf lower-semicontinuous at $c$} provided that for every ${\varepsilon}\,>\,0$ there exists ${\delta}\,>\,0$
    such that if $x$ in $X$ satisfies $|x-c|\,<\,{\delta}$ then $f(x)\,\,{\geq}\,\,f(c) - {\varepsilon}$.



\V

        (a) Give an example of a function which is upper-semicontinuous, but not continuous, at a point.

\V

        (b) Prove that the function $f$ is continuous at the point $c$ if, and only if, $f$ is both upper-semicontinuous and lower-semicontinuous at $c$.

\V

        (c) Prove that if $f:[a,b] \,{\rightarrow}\, {\RR}$ is upper-semicontinuous at each point of a closed interval $[a,b]$, then $f$ assumes a maximum value for $[a,b]$ at some point of $[a,b]$.

\V
\V

        
\noindent \ExDs {\bf Definition} Let $r$ be a rational number and let $c$ be a positive real number.
    Then the number $c^{r}$, which is pronounced `the $r$-th power of $c$', is defined as follows:
        \begin{displaymath}
        c^{r} \,=\, \left\{
        \begin{array}{ll}
        1 & \mbox{if $r \,=\, 0$} \\
        \left(\sqrt[n]{c}\right)^{m} & \mbox{if $r\,>\,0$ and $r \,=\, m/n$ for natural numbers $m$ and $n$ in lowest terms} \\
        c^{-r} & \mbox{if $r\,<\,0$}
        \end{array}
                                \right.
        \end{displaymath}
    As usual, the symbol $\sqrt[n]{c}$ denotes the unique positive real number $u$ such that $u^{n} \,=\, c$.

\V

        (a) Suppose that $c\,>\,0$ and that $r$ is a rational number. Show that if one writes $r$ in the form $r \,=\, p/q$,
    where $p$ and $q$ are integers and $q\,>\,0$, then $c^{r} \,=\, \left(\sqrt[q]{c}\right)^{p}$;
    that is, one does not need to express the rational number $r$ in `lowest terms'.

\V

        (b) Suppose that $c\,>\,0$. Show that $c^{-r} \,=\, 1/c^{r}$ for every rational $r$.

\V

        (c) Suppose that $c\,>\,0$. Show that $c^{r+s} \,=\, c^{r}c^{s}$ and $\left(c^{r}\right)^{s} \,=\, c^{(rs)}$  for all rationals $r$ and $s$.
