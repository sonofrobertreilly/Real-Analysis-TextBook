% Exercises_M140AB_A.TeX   Exercises for Chapter A

%
% Revised: 02/17/2012
%

%% NOTE: Copy the 52 lines below to each chapter, and change the Chapter letters.

%\thispagestyle{myheadings}


%\markboth{Exercises for Chapter~\ref{ChaptA} -}{Exercises for Chapter~\ref{ChaptA} -}

\newcommand{\ExAa}{{\bf \ref{ChaptA} - \,1} }
\newcommand{\ExAb}{{\bf \ref{ChaptA} - \,2} }
\newcommand{\ExAc}{{\bf \ref{ChaptA} - \,3} }
\newcommand{\ExAd}{{\bf \ref{ChaptA} - \,4} }
\newcommand{\ExAe}{{\bf \ref{ChaptA} - \,5} }
\newcommand{\ExAf}{{\bf \ref{ChaptA} - \,6} }
\newcommand{\ExAg}{{\bf \ref{ChaptA} - \,7} }
\newcommand{\ExAh}{{\bf \ref{ChaptA} - \,8} }
\newcommand{\ExAi}{{\bf \ref{ChaptA} - \,9} }
\newcommand{\ExAj}{{\bf \ref{ChaptA} -  10} }
\newcommand{\ExAk}{{\bf \ref{ChaptA} -  11} }
\newcommand{\ExAl}{{\bf \ref{ChaptA} -  12} }
\newcommand{\ExAm}{{\bf \ref{ChaptA} -  13} }
\newcommand{\ExAn}{{\bf \ref{ChaptA} -  14} }
\newcommand{\ExAo}{{\bf \ref{ChaptA} -  15} }
\newcommand{\ExAp}{{\bf \ref{ChaptA} -  16} }
\newcommand{\ExAq}{{\bf \ref{ChaptA} -  17} }
\newcommand{\ExAr}{{\bf \ref{ChaptA} -  18} }
\newcommand{\ExAs}{{\bf \ref{ChaptA} -  19} }
\newcommand{\ExAt}{{\bf \ref{ChaptA} -  20} }
\newcommand{\ExAu}{{\bf \ref{ChaptA} -  21} }
\newcommand{\ExAv}{{\bf \ref{ChaptA} -  22} }
\newcommand{\ExAw}{{\bf \ref{ChaptA} -  23} }
\newcommand{\ExAx}{{\bf \ref{ChaptA} -  24} }
\newcommand{\ExAy}{{\bf \ref{ChaptA} -  25} }
\newcommand{\ExAz}{{\bf \ref{ChaptA} -  26} }


\newcommand{\ExAaa}{{\bf \ref{ChaptA} - 27} }
\newcommand{\ExAab}{{\bf \ref{ChaptA} - 28} }
\newcommand{\ExAac}{{\bf \ref{ChaptA} - 29} }
\newcommand{\ExAad}{{\bf \ref{ChaptA} - 30} }
\newcommand{\ExAae}{{\bf \ref{ChaptA} - 31} }
\newcommand{\ExAaf}{{\bf \ref{ChaptA} - 32} }
\newcommand{\ExAag}{{\bf \ref{ChaptA} - 33} }
\newcommand{\ExAah}{{\bf \ref{ChaptA} - 34} }
\newcommand{\ExAai}{{\bf \ref{ChaptA} - 35} }
\newcommand{\ExAaj}{{\bf \ref{ChaptA} - 36} }
\newcommand{\ExAak}{{\bf \ref{ChaptA} - 37} }
\newcommand{\ExAal}{{\bf \ref{ChaptA} - 38} }
\newcommand{\ExAam}{{\bf \ref{ChaptA} - 39} }
\newcommand{\ExAan}{{\bf \ref{ChaptA} - 40} }
\newcommand{\ExAao}{{\bf \ref{ChaptA} - 41} }
\newcommand{\ExAap}{{\bf \ref{ChaptA} - 42} }
\newcommand{\ExAaq}{{\bf \ref{ChaptA} - 43} }
\newcommand{\ExAar}{{\bf \ref{ChaptA} - 44} }
\newcommand{\ExAas}{{\bf \ref{ChaptA} - 45} }
\newcommand{\ExAat}{{\bf \ref{ChaptA} - 46} }
\newcommand{\ExAau}{{\bf \ref{ChaptA} - 47} }
\newcommand{\ExAav}{{\bf \ref{ChaptA} - 48} }
\newcommand{\ExAaw}{{\bf \ref{ChaptA} - 49} }
\newcommand{\ExAax}{{\bf \ref{ChaptA} - 50} }
\newcommand{\ExAay}{{\bf \ref{ChaptA} - 51} }
\newcommand{\ExAaz}{{\bf \ref{ChaptA} - 52} }




                        \section{EXERCISES FOR CHAPTER~\ref{ChaptA}}
                        \label{SectAEX}

\V
\V
\V
\V

\noindent \ExAa Prove Parts~(b) and~(c) of Theorem~\Ref{ThmA10.25}.

\V
\V

\noindent \ExAb Prove Parts~(d) and~(e) of Theorem~\Ref{ThmA10.25}.

\V
\V

\noindent \ExAc Prove Parts~(f), (g) and~(h) of Theorem~\Ref{ThmA10.25}

\V
\V
 
 \noindent \ExAd Prove or disprove the following statements:

\V

        (a) For all sets $A$, $X$ and $Y$, one has
    $(A-X){\cap}(A-Y) \,=\, A-(X{\cap}Y)$.

\V

        (b) For all sets $A$, $X$ and $Y$, one has
    $(A-X){\cup}(A-Y) \,=\, A-(X{\cup}Y)$.

\V

        (c) Same as (a), but with `all sets' replaced by `some sets'.

\V


        (d) Same as (b), but with `all sets' replaced by `some sets'.

% NOTE Solutions in Math 205A F04 HW#1

\V
\V

\noindent \ExAe The Kuratowski definition of `ordered pair' (see Definition~\Ref{DefA12.10})
    is only one of several set-theoretic definitions of this concept which have been proposed over the years.
    Of course the goal of all these definitions is that they should allow one to distinguish between the `first' element $x$ and the `second' element $y$ of the pair $(x,y)$ purely in terms of set theory.
    Determine which, if any, of the following definitions of the ordered pair $(x,y)$ satisfies this goal.
    (Explain your answer.)

\V

        (a) $(x,y) \,=\, \{x,y\}$
        
        % Answer: No

\V

        (b) $(x,y) \,=\, \{\{x,1\},\{y,2\}\}$
        
        % Answer: Yes. Hausdorff's Def (more or less)

\V
 
        (c) $(x,y) \,=\, \{ \{\{x\},{\emptyset}\}, \{\{y\}\}\}$
        
        % Answer: Yes. Wiener's Def

\V
        (d) $(x,y) \,=\, \{x,\{y\}\}$
        
        % Answer: No

\V
\V

\noindent \ExAf There are two `obvious' ways to define the concept of `ordered triple' using the Kuratowski definition of `ordered pair':

    \h Method (1) Define $(x,y,z)$ to equal  $((x,y),z)$.

    \h Method (2) $(x,y,z)$ to equal $(x,(y,z))$.
    
\noindent Determine whether these definitions give the same value for $(x,y,z)$.

\V
\V

\noindent \ExAg Let $X$ be a nonempty set.
    Recall that a map $f:X{\times}X {\rightarrow} X$ is called a {\em binary operation on $X$}.
    Such an operation is said to be {\em commutative} if $f(x_{1},x_{2}) \,=\, f(x_{2},x_{1})$ for all $x_{1},x_{2}$ in $X$.
    It is said to be {\em associative} if $f(f(x_{1},x_{2}),x_{3}) \,=\, f(x_{1},f(x_{2},x_{3}))$ for all $x_{1},x_{2},x_{3}$ in $X$.

        \underline{Problem}: Show that if $f$ is an associative and commutative binary operation on $X$,
    then
        \begin{displaymath}
        f(f(x_{1},x_{2}),f(x_{3},x_{4})) \,=\, f(f(x_{3},x_{2}),f(x_{1},x_{4}))
    \mbox{ for all $x_{1},x_{2},x_{3},x_{4}$ in $X$}.
        \end{displaymath}
    (Hint: Try it out on some familiar case first.)

% NOTE Solutions in Math 205A F04 HW#1

\V
\V


\noindent \ExAh Suppose that $f$ is a function whose domain is a set $A$ and whose values lie in a set $B$.

\V

        (a) Prove the following statement: If $X$ is a subset of $A$ and $Y$ is a subset of $B$, then
        \begin{displaymath}
        X \,{\subseteq}\, f^{-1}[f[X]] \mbox{ and } Y \,{\supseteq}\, f[f^{-1}[Y]].
        \end{displaymath}
    Also, determine whether it could happen that $X$ is a {\em proper} subset of $f^{-1}[f[X]]$;
    likewise, determine whether it could happen that $Y$ is a {\em proper} superset of $f[f^{-1}[Y]]$.

\V

        (b) Let $X_{1}$ and $X_{2}$ be subsets of $A$. Prove the following:

        \h (i)\, $f[X_{1}\,{\cup}\,X_{2}] \,=\, f[X_{1}]\,{\cup}\,f[X_{2}]$ and
           $f[X_{1}\,{\cap}\,X_{2}] \,{\subseteq}\, f[X_{1}]\,{\cap}\,f[X_{2}]$.

        \h (ii) Either find an example of $f$, $A$, $B$, $X_{1}$ and $X_{2}$ for which one has
    $f[X_{1}\,{\cap}\,X_{2}]  \,\,{\neq}\,\, f[X_{1}]\,{\cap}\,f[X_{2}]$, or else prove that no such example exists.

% NO Solution in 205A F04 Apostol Chap 1 Ex 2.6

\V

        (c) Let $Y_{1}$ and $Y_{2}$ be subsets of $B$. Prove the following:

        \h (i)\, $f^{-1}[Y_{1}\,{\cup}\,Y_{2}] \,=\, f^{-1}[Y_{1}]\,{\cup}\,f^{-1}[Y_{2}]$ and $f^{-1}[Y_{1}\,{\cap}\,Y_{2}] \,=\, f^{-1}[Y_{1}]\,{\cap}\,f^{-1}[Y_{2}]$.

        \h (ii) $f^{-1}[B\,{\setminus}\,Y] \,=\, A\,{\setminus}\,f^{-1}[Y]$.

% NOTE Solutions in Math 205A F04 HW#1, but may need to change letters Apostol Chap 1, Ex 2.8

\V
\V

\noindent \ExAi Let $f:X \,{\rightarrow}\, Y$ be a function from a set $X$ into a set $Y$.
    Recall that one says that a function $g:Y\,{\rightarrow}\,X$ is a {\bf left inverse of $f$} provided $g{\circ}f\,=\, I_{X}$.
        Likewise, one says that a function $h:Y\,{\rightarrow}\,X$ is a {\bf right inverse of $f$} provided $f{\circ}h\,=\, I_{Y}$.

\V        
      (a) Show that a necessary and sufficient condition for $f$ to be a bijection of $X$ onto $Y$ is that $f$ admit both a left inverse $g:Y\,{\rightarrow}\,X$ and a right inverse $h:Y\,{\rightarrow}\,X$.
      In addition, if this occurs then $g\,=\,h\,=\,f^{-1}$.
      
\V

      (b) \underline{Prove or Disprove} If $f:X\,{\rightarrow}\,Y$ has at least one left inverse $g:Y\,{\rightarrow}\,X$ but no right inverse,  then $f$ has more than one  such left inverse.

\V
\V

\noindent \ExAj \underline{Prove or Disprove}: If $f$ and $g$ are maps from a set $X$ onto $X$,
    and if the composition $f{\circ}g:X {\rightarrow} X$ is one-to-one, then $f$ and $g$ are also one-to-one. (Compare with Part~(e) of Theorem~\Ref{ThmA30.160}.)

% NOTE Solutions in Math 205A F04 HW#1

\V
\V


\noindent \ExAk Let $X$ be a set which has exactly $k$ elements, where $k$ is a natural number.
    Let $Y$ be the set of all bijections of $X$ onto itself. Prove that $Y$ has exactly $k!$ elements.

\V
\V

\noindent \ExAl Let $W$ be a set, and let $A$ be the set of all subsets of $W$; that is, $A$ is the power set ${\cal P}(W)$ of $W$.

        \underline{Definition}: A function $f:A \,{\rightarrow}\, {\RR}$ is said to  be an {\bf additive function on $A$} provided
    $f(X_{1}\,{\cup}\,X_{2}) \,=\, f(X_{1}) + f(X_{2})$ for every pair of mutually disjoint subsets of $W$.

\V

        (a) Suppose that $W$ and $A$ are as above, and that $f:A \,{\rightarrow}\, {\RR}$ is an additive real-valued function defined on $A$.
    Prove that if $X_{1}$ and $X_{2}$ are {\em arbitrary} subsets of $W$ then
        \begin{displaymath}
        f(X_{1}\,{\cup}\,X_{2}) \,=\, f(X_{1}) + f(X_{2}) - f(X_{1}\,{\cap}\,X_{2}).
        \end{displaymath}

% NOTE Solutions in Math 205A F04 HW#1, but may need to change letters Apostol Chap 1 Ex 2.22

\V

        (b) Suppose that $W \,=\, \{1,2\}$, so that $W$ has exactly two elements.
    Determine all of the corresponding additive functions.


% NO solutions from 205A F04 HW#1


\V
\V

\noindent  \ExAm(a) Prove Part~(b) of Theorem~\Ref{ThmA15.30}. You may use the result of Part~(a) of that theorem.

\V

        (b) Prove Part~(c) of Theorem~\Ref{ThmA15.30}. You may use the results of Parts~(a) and~(b) of that theorem.

\V

       (c) Prove Part~(d) of Theorem~\Ref{ThmA20.10}. You may assume the results of Parts~(a), (b) and~(c).
       
\V
\V

\noindent \ExAn Determine whether there exists a nonempty binary relation $R$ on some set $X$ such that $R$ satisfies both symmetry and transitivity, but not reflexivity.

\V
\V

\noindent \ExAo Let $P_{1} \,=\, (-3,2)$ and $P_{2} \,=\, (2,5)$. Determine the value at $x \,=\, 1$ of the linear interpolation between $P_{1}$ and $P_{2}$.

\V
\V

\noindent \ExAp Prove or Disprove: Let $A$ and $B$ be nonempty sets. If $f:A \,{\rightarrow}\, B$ and $g:B \,{\rightarrow}\, A$ are surjections, then $A$ and $B$ have the same cardinality.

\V
\V

\noindent \ExAq In each part of this exercise, find an explicit example of a bijection of $X$ with $Y$.
    (`Explicit': Give a formula using standard functions that would be familiar to students in elementary calculus.
    You may use the well-know properties of such functions, but make it clear which such properties you are using.)

\V

        (a) $X \,=\, [3,5]$, $Y \,=\, [2,10]$.

\V

        (b) $X \,=\, {\RR}$, $Y \,=\, (-{\pi},+{\infty})$.

\V

        (c) $X \,=\, {\RR}$, $Y \,=\, (-1,1)$.
        
\V
\V


\noindent \ExAr Give a direct proof the following version of the `Amazing Grace' effect:
    If $X$ is an infinite set, and $B$ is any \underline{finite} subset of $X$, then the set $X{\setminus}B$ has the same cardinality as $X$.

    NOTE: By a `direct proof', it is meant that you are \underline{not} allowed to use the conclusions of Corollary~\Ref{CorA20.20} or Theorem~\Ref{ThmA20.112} in your solution.

\V
\V

\noindent \ExAs In Set Theory one often encounters the following notation: Let $X$ and $Y$ be nonempty sets.
    Then $Y^{X}$ is the set of all functions $f:X \,{\rightarrow}\, Y$ with domain $X$ and with values in $Y$.

\V

        (a) Show that if $Y \,=\, \{a,b\}$ is a doubleton set, so that $a \,\,{\neq}\,\, b$, then $Y^{X}$ has the same cardinality as ${\cal P}(X)$, the power set of $X$.

\V

        (b) \underline{Prove or Disprove} If $Y \,=\, \{a,b,c\}$ is a set with exactly three distinct elelments $a$, $b$ and $c$, and $X$ is an infinite set,
    then $Y^{X}$ has the same cardinality as ${\cal P}(X)$.

\V
\V

\noindent \ExAt Prove Parts (b) and (c) of Theorem~\Ref{ThmA15.30}.

\V
\V

\noindent \ExAu Let $X$ and $Y$ be nonempty sets, and suppose that $F:X \,{\rightarrow}\, Y$ is a bijection of $X$ onto $Y$.

\V

        (a) Let ${\beta}:Y{\times}Y \,{\rightarrow}\, Y$ is a binary operation on $Y$.
    Show that there exists a unique binary operation ${\alpha}:X{\times}X \,{\rightarrow}\, X$ such that
        \begin{displaymath}
        F({\alpha}(x_{1},x_{2})) \,=\, {\beta}(F(x_{1}),F(x_{2}))
        \end{displaymath}

\V

        (b) Let ${\alpha}$ and ${\beta}$ be as in Part~(a). Prove that ${\alpha}$ has an identity element if, and only if, ${\beta}$ has an identity element.

\V

       (c) Let ${\alpha}$ and ${\beta}$ be as in Part~(a). Prove that ${\alpha}$ satisfies the associative law if, and only if, ${\beta}$ satisfies the associative law.

\V
\V

\noindent \ExAv Suppose that $X$ and $Y$ are finite sets. Prove that $X{\times}Y$ is a finite set, and express $\#(X{\times}Y)$ in terms of $\#(X)$ and $\#(Y)$.

\V
\V

\noindent \ExAw {\bf Definition} A real number $c$ is said to be an{\bf algebraic number} provided there exists a polynomial function of the form $p(t) \,=\, t^{k} + a_{k-1}t^{k-1} + \,{\ldots}\,+ ta_{1} + a_{0}$, where $k$ is a natural number and the coefficients $a_{k-1}$, $a_{k-2}$,\,{\ldots}\,$a_{1}$, $a_{0}$ are rational numbers, such that $p(c) \,=\, 0$. A real number which is not algebraic is said to be a {\bf transcendental number}.

    {\bf Problem} Prove that the set of algebraic real numbers is a countably infinite set.
    You may use -- without needing to prove -- standard facts from high-school algebra concerning the roots of polynomials.
        NOTE: In light of Corollary~\Ref{CorA20.115}, it follows from the conclusion of this problem that in every interval $[a,b]$ in ${\RR}$ there exist uncountably many transcendental numbers.

\V
\V

\noindent \ExAx Let $X$ and $Y$ be sets. Let $S$ be a subset of $X$, and let $T$ a subset of $Y$.

        \underline{Prove or Disprove} If $X$ and $Y$ have the same cardinality as each other, and $S$ and $T$ have the same cardinality as each other,
    then $X{\setminus}S$ and $Y{\setminus}T$ have the same cardinality as each other.
% {\bf EXERCISES FOR THE ADDENDUM TO CHAPTER~\ref{ChaptA} -}











