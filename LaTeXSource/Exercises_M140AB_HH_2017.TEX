% Exercises_M140AB_H.TeX   Exercises for Chapter H

%
% Revised: 06/26/09
%

%% NOTE: Copy the 52 lines below to each chapter, and change the Chapter letters.

%\thispagestyle{myheadings}


%\markboth{Exercises for Chapter~\ref{ChaptH} -}{Exercises for Chapter~\ref{ChaptH} -}

\newcommand{\ExHa}{{\bf \ref{ChaptH} - \,1} }
\newcommand{\ExHb}{{\bf \ref{ChaptH} - \,2} }
\newcommand{\ExHc}{{\bf \ref{ChaptH} - \,3} }
\newcommand{\ExHd}{{\bf \ref{ChaptH} - \,4} }
\newcommand{\ExHe}{{\bf \ref{ChaptH} - \,5} }
\newcommand{\ExHf}{{\bf \ref{ChaptH} - \,6} }
\newcommand{\ExHg}{{\bf \ref{ChaptH} - \,7} }
\newcommand{\ExHh}{{\bf \ref{ChaptH} - \,8} }
\newcommand{\ExHi}{{\bf \ref{ChaptH} - \,9} }
\newcommand{\ExHj}{{\bf \ref{ChaptH} -  10} }
\newcommand{\ExHk}{{\bf \ref{ChaptH} -  11} }
\newcommand{\ExHl}{{\bf \ref{ChaptH} -  12} }
\newcommand{\ExHm}{{\bf \ref{ChaptH} -  13} }
\newcommand{\ExHn}{{\bf \ref{ChaptH} -  14} }
\newcommand{\ExHo}{{\bf \ref{ChaptH} -  15} }
\newcommand{\ExHp}{{\bf \ref{ChaptH} -  16} }
\newcommand{\ExHq}{{\bf \ref{ChaptH} -  17} }
\newcommand{\ExHr}{{\bf \ref{ChaptH} -  18} }
\newcommand{\ExHs}{{\bf \ref{ChaptH} -  19} }
\newcommand{\ExHt}{{\bf \ref{ChaptH} -  20} }
\newcommand{\ExHu}{{\bf \ref{ChaptH} -  21} }
\newcommand{\ExHv}{{\bf \ref{ChaptH} -  22} }
\newcommand{\ExHw}{{\bf \ref{ChaptH} -  23} }
\newcommand{\ExHx}{{\bf \ref{ChaptH} -  24} }
\newcommand{\ExHy}{{\bf \ref{ChaptH} -  25} }
\newcommand{\ExHz}{{\bf \ref{ChaptH} -  26} }


\newcommand{\ExHaa}{{\bf \ref{ChaptH} - 27} }
\newcommand{\ExHab}{{\bf \ref{ChaptH} - 28} }
\newcommand{\ExHac}{{\bf \ref{ChaptH} - 29} }
\newcommand{\ExHad}{{\bf \ref{ChaptH} - 30} }
\newcommand{\ExHae}{{\bf \ref{ChaptH} - 31} }
\newcommand{\ExHaf}{{\bf \ref{ChaptH} - 32} }
\newcommand{\ExHag}{{\bf \ref{ChaptH} - 33} }
\newcommand{\ExHah}{{\bf \ref{ChaptH} - 34} }
\newcommand{\ExHai}{{\bf \ref{ChaptH} - 35} }
\newcommand{\ExHaj}{{\bf \ref{ChaptH} - 36} }
\newcommand{\ExHak}{{\bf \ref{ChaptH} - 37} }
\newcommand{\ExHal}{{\bf \ref{ChaptH} - 38} }
\newcommand{\ExHam}{{\bf \ref{ChaptH} - 39} }
\newcommand{\ExHan}{{\bf \ref{ChaptH} - 40} }
\newcommand{\ExHao}{{\bf \ref{ChaptH} - 41} }
\newcommand{\ExHap}{{\bf \ref{ChaptH} - 42} }
\newcommand{\ExHaq}{{\bf \ref{ChaptH} - 43} }
\newcommand{\ExHar}{{\bf \ref{ChaptH} - 44} }
\newcommand{\ExHas}{{\bf \ref{ChaptH} - 45} }
\newcommand{\ExHat}{{\bf \ref{ChaptH} - 46} }
\newcommand{\ExHau}{{\bf \ref{ChaptH} - 47} }
\newcommand{\ExHav}{{\bf \ref{ChaptH} - 48} }
\newcommand{\ExHaw}{{\bf \ref{ChaptH} - 49} }
\newcommand{\ExHax}{{\bf \ref{ChaptH} - 50} }
\newcommand{\ExHay}{{\bf \ref{ChaptH} - 51} }
\newcommand{\ExHaz}{{\bf \ref{ChaptH} - 52} }



                       \section{EXERCISES FOR CHAPTER~\ref{ChaptH}}
                        \label{SectHEX}

\V
\V
\V
\V

\noindent \ExHa \underline{Preliminary Remarks} The definition of ${\displaystyle \int_{a}^{b} f\,d{\alpha}}$ given in the {\em Notes} is the same as the one found in,
    say, Rudin's `Principles of Mathematical Analysis'. Some authors use the following definition, which yields a somewhat different theory.

\V

        {\bf Alternate Definition} Let $[a,b]$ be a closed bounded interval in ${\RR}$, and as usual let ${\cal P}_{[a,b]}$ denote the set of partitions of the interval $[a,b]$.

\V

        (i)\, If ${\cal P} \,=\, \{a \,=\, x_{0}\,<\,x_{1}\,<\,\,{\ldots}\,\,<\,x_{k} \,=\, b\}$ is a partition of $[a,b]$ then the {\bf mesh of the partition ${\cal P}$} is the number $||{\cal P}||$ given by
        \begin{displaymath}
        ||{\cal P}|| \,=\, {\max}\,\{|x_{1}-x_{0}|, |x_{2}-x_{1}|,\,{\ldots}\,|x_{k}-x_{k-1}|\}.
        \end{displaymath}
    Speaking geometrically, $||{\cal P}||$ is the length of the longest subinterval of $[a,b]$ determined by the partition~${\cal P}$.

\V

        (ii) Suppose that $f:[a,b] \,{\rightarrow}\, {\RR}$ is a bounded function and ${\alpha}:[a,b] \,{\rightarrow}\, {\RR}$ is monotonic up.
    Using the notation and terminology of Lemma~H.1.7 and Definition~H.1.8, one says that $f$ is `Riemann-Stieltjes Integrable' in this alternate sense provided that the following condition holds:

       \h There exists a number $A$ such that for every ${\varepsilon}\,>\,0$ there exists ${\delta}\,>\,0$ such that if ${\cal P}$ is any partition of $[a,b]$ such that $||{\cal P}||\,<\,{\delta}$, then for every choice list ${\tau} \,=\, (t_{1},t_{2},\,{\ldots}\,t_{k})$ associated with ${\cal P}$ one has
        \begin{displaymath}
        \left|S({\cal P},f,{\alpha},{\tau}) - A\right|\,<\,{\varepsilon}.
        \end{displaymath}
    (Recall that $S({\cal P},f,{\alpha},{\tau}) \,=\, {\displaystyle \sum_{j=1}^{k}} f(t_{j}){\Delta}{\alpha}_{j}$.) The number $A$ is then the value of the integral.

\V

        \underline{Problem}

\V

        (a) Show that if $f$ is Riemann-Stieltjes integrable on $[a,b]$ with respect to ${\alpha}$ in this alternate sense,
    then it is also Riemann-Stieltjes integrable on $[a,b]$ with respect to ${\alpha}$ in the sense of Definition~H.1.4, and the integrals are equal.

\V

        (b) Give an example of $[a,b]$, $f$ and ${\alpha}$ for which the integral exists in the sense of Definition~H.1.4, but not in this alternate sense.

\V      (c) Show that in the case of the ordinary Riemann integral, i.e., when ${\alpha}(x) \,=\, x$ for all $x$,
    the two definitions are equivalent; that is, they produce the same class of integrable functions, and the same values for the integrals.

% Apostol # 7.3, 7.4 p. 174

\V
\V

\noindent \ExHb Prove Part~(a) of Theorem~H.2.4 (Hint: Use Theorem H.1.9)

\V
\V

\noindent \ExHc Prove Part~(b) of Theorem~H.2.4 (Hint: Use Theorem H.1.9)

\V
\V

\noindent \ExHd For each {\em even} natural number $m$, let ${\alpha}^{(m)}$ be the integrator described on Page~415,
    in Example~H.1.10~(6), the `Simpson's Rule' example.

\V

        (a) Determine the value ${\alpha}^{(m)}(x)$ for each $x$ in $[a,b]$.

\V

        (b) Show that if $f:[a,b] \,{\rightarrow}\, {\RR}$ is continuous, then ${\displaystyle \int_{a}^{b} f \,=\, \lim_{m \,{\rightarrow}\, {\infty}} \int_{a}^{b} f\,d{\alpha}^{(m)}}$.


\V
\V


\noindent \ExHe \underline{Prove or Disprove} If $f{\in}{\cal R}_{[0,1]}$, then ${\displaystyle \lim_{k \,{\rightarrow}\, {\infty}} \int_{0}^{1} x^{k}f(x)\,dx \,=\, 0}$.

