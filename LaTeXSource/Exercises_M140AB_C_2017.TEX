% Exercises_M140AB_C.TeX   Exercises for Chapter C

%
% Revised: 07/06/2016 Encoding: Western ASCII
%

%% NOTE: Copy the 52 lines below to each chapter, and change the Chapter letters.

%\thispagestyle{myheadings}


%\markboth{Exercises for Chapter~\ref{ChaptC} -}{Exercises for Chapter~\ref{ChaptC} -}

\newcommand{\ExCa}{{\bf \ref{ChaptC} - \,1} }
\newcommand{\ExCb}{{\bf \ref{ChaptC} - \,2} }
\newcommand{\ExCc}{{\bf \ref{ChaptC} - \,3} }
\newcommand{\ExCd}{{\bf \ref{ChaptC} - \,4} }
\newcommand{\ExCe}{{\bf \ref{ChaptC} - \,5} }
\newcommand{\ExCf}{{\bf \ref{ChaptC} - \,6} }
\newcommand{\ExCg}{{\bf \ref{ChaptC} - \,7} }
\newcommand{\ExCh}{{\bf \ref{ChaptC} - \,8} }
\newcommand{\ExCi}{{\bf \ref{ChaptC} - \,9} }
\newcommand{\ExCj}{{\bf \ref{ChaptC} -  10} }
\newcommand{\ExCk}{{\bf \ref{ChaptC} -  11} }
\newcommand{\ExCl}{{\bf \ref{ChaptC} -  12} }
\newcommand{\ExCm}{{\bf \ref{ChaptC} -  13} }
\newcommand{\ExCn}{{\bf \ref{ChaptC} -  14} }
\newcommand{\ExCo}{{\bf \ref{ChaptC} -  15} }
\newcommand{\ExCp}{{\bf \ref{ChaptC} -  16} }
\newcommand{\ExCq}{{\bf \ref{ChaptC} -  17} }
\newcommand{\ExCr}{{\bf \ref{ChaptC} -  18} }
\newcommand{\ExCs}{{\bf \ref{ChaptC} -  19} }
\newcommand{\ExCt}{{\bf \ref{ChaptC} -  20} }
\newcommand{\ExCu}{{\bf \ref{ChaptC} -  21} }
\newcommand{\ExCv}{{\bf \ref{ChaptC} -  22} }
\newcommand{\ExCw}{{\bf \ref{ChaptC} -  23} }
\newcommand{\ExCx}{{\bf \ref{ChaptC} -  24} }
\newcommand{\ExCy}{{\bf \ref{ChaptC} -  25} }
\newcommand{\ExCz}{{\bf \ref{ChaptC} -  26} }


\newcommand{\ExCaa}{{\bf \ref{ChaptC} - 27} }
\newcommand{\ExCab}{{\bf \ref{ChaptC} - 28} }
\newcommand{\ExCac}{{\bf \ref{ChaptC} - 29} }
\newcommand{\ExCad}{{\bf \ref{ChaptC} - 30} }
\newcommand{\ExCae}{{\bf \ref{ChaptC} - 31} }
\newcommand{\ExCaf}{{\bf \ref{ChaptC} - 32} }
\newcommand{\ExCag}{{\bf \ref{ChaptC} - 33} }
\newcommand{\ExCah}{{\bf \ref{ChaptC} - 34} }
\newcommand{\ExCai}{{\bf \ref{ChaptC} - 35} }
\newcommand{\ExCaj}{{\bf \ref{ChaptC} - 36} }
\newcommand{\ExCak}{{\bf \ref{ChaptC} - 37} }
\newcommand{\ExCal}{{\bf \ref{ChaptC} - 38} }
\newcommand{\ExCam}{{\bf \ref{ChaptC} - 39} }
\newcommand{\ExCan}{{\bf \ref{ChaptC} - 40} }
\newcommand{\ExCao}{{\bf \ref{ChaptC} - 41} }
\newcommand{\ExCap}{{\bf \ref{ChaptC} - 42} }
\newcommand{\ExCaq}{{\bf \ref{ChaptC} - 43} }
\newcommand{\ExCar}{{\bf \ref{ChaptC} - 44} }
\newcommand{\ExCas}{{\bf \ref{ChaptC} - 45} }
\newcommand{\ExCat}{{\bf \ref{ChaptC} - 46} }
\newcommand{\ExCau}{{\bf \ref{ChaptC} - 47} }
\newcommand{\ExCav}{{\bf \ref{ChaptC} - 48} }
\newcommand{\ExCaw}{{\bf \ref{ChaptC} - 49} }
\newcommand{\ExCax}{{\bf \ref{ChaptC} - 50} }
\newcommand{\ExCay}{{\bf \ref{ChaptC} - 51} }
\newcommand{\ExCaz}{{\bf \ref{ChaptC} - 52} }



                       \section{EXERCISES FOR CHAPTER~\ref{ChaptC}}
                        \label{SectCEX}

\V
\V
\V
\V

\noindent  \ExCa In each part of this exercise, determine -- directly from the definition of `limit' -- whether the given sequence is convergent or not.


\V

        (a) ${\xi} \,=\,(x_{1},x_{2},\,{\ldots}\,x_{k},\,{\ldots}\,)$, where ${\displaystyle x_{k} \,=\, \frac{3k+100}{k^{2} + 3}}$ for each $k$ in ${\NN}$.

\V

        (b) ${\tau} \,=\, (t_{1},t_{2},\,{\ldots}\,)$ where $t_{k} \,=\, {\displaystyle \frac{3k+17}{8k-17}}$ for each $k$ in ${\NN}$.

\V

        (c) ${\alpha} \,=\, (a_{1},a_{2},\,{\ldots}\,a_{k},\,{\ldots}\,)$, where ${\displaystyle a_{k} \,=\, \frac{3k^{2} + (-1)^{k}k}{k^{2}}}$.

\V

        (d) ${\beta} \,=\, (b_{1},b_{2},\,{\ldots}\,b_{k},\,{\ldots}\,)$, where $b_{k} \,=\, \sqrt{k+1}-\sqrt{k}$ for each $k$ in ${\NN}$.
    (You may assume that every positive real number has a unique positive square root.)

\V
\V                                                                        

 \noindent \ExCb (a) Prove that if $c$ is a real number such that $c \,>\, -1$, then $(1+c)^{k} \,\,{\geq}\,\, 1+kc$ for all natural numbers $k$.

        \V

        (b) Use the conclusion of Part~(a) to show that if $|r|\,<\,1$ then the sequence ${\xi} \,=\, (1, r, r^{2},\,{\ldots}\,)$, whose $k$-th term is $r^{k-1}$, coverges to~$0$.

\V
\V

\noindent \ExCc (a) \underline{Prove or Disprove}: Let ${\tau} \,=\, (t_{1}, t_{2}, \,{\ldots}\,)$ be a convergent sequence of real numbers,
    and suppose that ${\displaystyle \lim_{k {\rightarrow} {\infty}} t_{k} \,\,{\geq}\,\, a}$ for some number~$a$.
    Then there exists a number $N$ such that $k \,\,{\geq}\,\, N$ implies $t_{k} \,\,{\geq}\,\, a$.

\V

        (b) \underline{Prove or Disprove}: Let ${\xi} \,=\, (x_{1}, x_{2}, \,{\ldots}\,)$ be a convergent sequence of real numbers,
    and suppose that ${\displaystyle \lim_{k {\rightarrow} {\infty}} x_{k} \,>\, a}$ for some number~$a$.
    Then there exists a number $N$ such that $k \,\,{\geq}\,\, N$ implies $x_{k} \,>\,a$.

%% From (I) in HW #3 in W04 Math 140A. Notation changed.

\V
\V

\noindent \ExCd In both parts of this problem, ${\alpha} \,=\, (a_{1},a_{2},\,{\ldots}\,)$ and ${\beta} \,=\, (b_{1},b_{2},\,{\ldots}\,)$ are sequences of real numbers such that $a_{k} \,\,{\leq}\,\, b_{k}$ for all $k$ in ${\NN}$,
    and ${\displaystyle \lim_{k {\rightarrow} {\infty}} |b_{k}-a_{k}|} \,=\, 0$.

\V

        (a) \underline{Prove or Disprove}: Suppose that there exists a number $c$ such that $a_{k} \,\,{\leq}\,\, c \,\,{\leq}\,\, b_{k}$ for all $k{\in}{\NN}$.
    Then each of the sequences ${\alpha}$ and ${\beta}$ is convergent.

\V

        (b) \underline{Prove or Disprove}: Suppose that for each $k{\in}{\NN}$ there exists a number $c$ such that
    $a_{k} \,\,{\leq}\,\, c \,\,{\leq}\,\, b_{k}$.
    Then each of the sequences ${\alpha}$ and ${\beta}$ is convergent.

%% From (III) in HW #3 in W04 Math 140A. Notation changed.

\V
\V

\noindent \ExCe In each part of this problem, show that Statement~(i) implies Statement~(ii).
    Do this using only the axioms for an ordered field -- that is, without using `Completeness' of ${\RR}$.


\V

        (a) (i) The Bisection Principle; \, (ii) The Monotonic Sequences Principle

\V

        (b) (i) The Monotonic Sequences Principle; \, (ii) The Bisection Principle.

\V

        (c) (i) The Monotonic-Up Sequences Principle; \, (ii) The Supremum Principle.

\V
\V

\noindent \ExCf Prove Parts (a) and (b) of Theorem~\Ref{ThmC40.30} %% Thm C.5.4 on p.181


\V
\V

\noindent \ExCg Prove Parts (c) and (d) of Theorem~\Ref{ThmC40.30} %% Thm C.5.4 on p.181


\V
\V

\noindent \ExCh Prove Parts (e) and (f) of Theorem~\Ref{ThmC40.30} %% Thm C.5.4 on p.181

\V
\V

\noindent \ExCi Prove Parts (g) and (h) of Theorem~\Ref{ThmC40.30} %% Thm C.5.4 on p.181

\V
\V

\noindent \ExCj Let ${\xi} \,=\, (x_{1},x_{2},\,{\ldots}\,)$ be a sequence of {\em positive} real numbers,
    and let ${\rho} \,=\, (r_{1},r_{2},\,{\ldots}\,)$ be the corresponding sequence of ratios: $r_{k} \,=\, x_{k+1}/x_{k}$ for each $k$ in ${\NN}$.

\V

        \underline{Prove or Disprove} If the sequence ${\xi}$ is convergent, then so is the sequence ${\rho}$.

\V
\V

\noindent \ExCk {\bf Definition} Let $X$ be a nonempty subset of ${\RR}$. A point $c$ in ${\RR}$ is said to be an {\bf accumulation point of $X$} provided that there exists a sequence ${\xi} \,=\, (x_{1}, x_{2},\,{\ldots}\,x_{k},\,{\ldots}\,)$ of points in $X$ such that

        \h (i)\, for all $k$ in ${\NN}$ one has $x_{k} \,\,{\neq}\,\, c$;
        \h (ii) $\lim_{k \,{\rightarrow}\, {\infty}} x_{k} \,=\, c$.

\V

        (a) Prove that if a subset $X$ of ${\RR}$ has an accumulation point then $X$ is an infinite set.

\V

        (b) Give an example of a subset $X$ of ${\RR}$ which has exactly five accumulation points, three of which are elements of $X$, two which are not elements of $X$.

\V
\V

\noindent \ExCl (a) Prove the {\bf Bolzano-Weierstrass Theorem for Sets in ${\RR}$}:
    If $X$ is a bounded infinite subset of ${\RR}$, then $X$ has an accumulation point.
    (See Exercise~\ExCk for the definition of `accumulation point'.) 

\V

        (b) \underline{Prove or Disprove} If $X$ is an uncountable subset of ${\RR}$, then $X$ has an accumulation point.

\V
\V

\noindent \ExCm Give an alternate proof of Theorem~\Ref{ThmC30.30} along the lines of the proof of Theorem~\Ref{ThmC20.90}, the `Odd/Even Convergence Theorem'. %% i.e., Thm C.2.15 p.172

\V
\V

\noindent \ExCn Suppose that ${\xi}$ is a bounded infinite sequence of real numbers, and that ${\cal F} \,=\, (A_{1},A_{2},\,{\ldots}\,A_{n},\,{\ldots}\,)$ is a countably infinite family of infinite subsets $A_{n}$ of ${\NN}$ such that ${\NN} \,=\, {\bigcup}_{n=1}^{{\infty}} A_{n}$.

        \underline{Prove or Disprove}: If all of the subsequences of ${\xi}$ corresponding to the subsets $A_{n}$ in the family ${\cal F}$ converge to the same real number $L$, then the original sequence ${\xi}$ also converges to $L$.

        \underline{Note} See Remark~\Ref{RemrkC30.40}~(2) for the origin of this problem.


\V
\V

\noindent \ExCo Prove Parts (a), (b) and (c) of Theorem~\Ref{ThmC40.30}. %% C.5.4

\V
\V

\noindent \ExCp Prove Parts (d) and (e) of Theorem~\Ref{ThmC40.30}. %% C.5.4

\V
\V

\noindent \ExCq Prove Parts (f) and (g) of Theorem~\Ref{ThmC40.30}. %% C.5.4

\V
\V

\noindent \ExCr Prove Part (h) of Theorem~\Ref{ThmC40.30}. %% C.5.4

\V
\V

\noindent \ExCs Define a sequence ${\xi} \,=\, (x_{1},x_{2},\,{\ldots}\,x_{k},\,{\ldots}\,)$ by the rule
        \begin{displaymath}
        x_{1} \,=\, 1, \, x_{k+1} \,=\, \frac{1+x_{k}}{2+x_{k}} \mbox{ for each $k$ in ${\NN}$}
        \end{displaymath}

\V

        (a) Show that the sequence ${\xi}$ is monotonic down and bounded below by~$0$.

\V

        (b) Determine the limit of the sequence ${\xi}$.

%% See Example 2.11, p.39, of Fitzpatrick

\V
\V

\noindent \ExCt Suppose that ${\alpha} \,=\, (a_{1},a_{2},\,{\ldots}\,)$ and ${\beta} \,=\, (b_{1},b_{2},\,{\ldots}\,)$ are sequences such that

        \h (i)\, $b_{k} \,=\, {\displaystyle \frac{3+a_{k}}{5+a_{k}}}$ for all $k$ in ${\NN}$.
        \h (ii) $\lim_{k \,{\rightarrow}\, {\infty}} b_{k} \,=\, 10$.

\noindent Determine whether the sequence ${\alpha}$ is convergent; if it is, determine its limit.

\V
\V


\noindent \ExCu In the `Remark' immediately after the proof of Theorem~\Ref{ThmC30.10} %% Thm C.4.1, "Bolzano-Weierstrass p. 178
    it is indicated that the Bolzano-Weierstrass Theorem is equivalent to the following result:

        \h  Every bounded sequence has a monotonic convergent subsequence.

        \underline{Problem} Prove this last result using the Infinimum/Supremum Principles.

