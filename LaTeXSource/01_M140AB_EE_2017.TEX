% M140AB_EE.TeX  Notes for `Single-Variable Analysis'
%
% Revised: 10/25/2017 Encoding: Western ASCII ?
%


                  \chapter{Existence of Antiderivatives in ${\RR}$}
                  \label{ChaptEE}
     
     \label{SectEE20}\IndB{ZZ Sections}{\Ref{SectEE20} Preliminaries}
     
     In elementary calculus one learns that many important applied problems can be solved by determining antiderivatives of appropriate functions.
     For example, if $f:[a,b]{rightarrow}{\RR}$ is a continuous real-value function defined on the closed interval $[a,b]$ such that $f(x){\geq}0$ for all $x$ in $[a,b]$,
     and if $F:[a,b]{\rightarrow}[\RR]$ IS an antiderivative of $f$ on $[a,b]$, then the area under the graph of $f$ over $[a,b]$ equals $F(b) - F(a)$.
     
     Likewise, in the sciences often what can say about an important unknown quantity $Q(t)$ which depends on the elapsed time in some situation involves relations between 
     its derivatives $Q'(t)$, $Q''(t)$ and so on, and knowable physical quantities. For example, suppose that an object of mass $m$ is thrown vertically down from a high tower
     with initial height $H_{0}$ above the Earth and initial velocity $V_{0}$ at time $t=0$. Assume that air resistance can be ignored and that $H_{0}$ is not too large.
     Then  Newton's Law of Gravity says that the gravitational force $F(t)$ of the Earth on this object is a constant $F_{0}$ throughout its fall (i.e., until the object hits the Earth).
     That is, $F(t) \,=\,m\, A(t)$, where $A(t)$ is the acceleration of the object, so that $A(t)$ is a constant $A_{0}\,=\,F_{0}/m$ for all such $t$.
     But, by definition, $A(t)\,=\,f''(t)$, where $f(t)$ denotes the height of the object at time $t$. That is, one knows the second derivative of the function of the function $f$.
     Because the force of gravity pulls objects {\it down}, so $f(t)$ is {\it deacreasing}it is easy to see that the acceleration $A_{0}$ is a {\it negative} constant $-g$, where $g{\,>\,} 0$.
     
     