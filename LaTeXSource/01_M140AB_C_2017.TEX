% M140AB_C.TeX  Notes for `Single-Variable Analysis': 

%
% Revised:01/20/2017 Encoding: Western Ascii
%

                  \chapter{Limits of Real Sequences -- Basic Theory}
                  \label{ChaptC}


        \underline{Quotes for Chapter~\Ref{ChaptC}}: \IndB{chapter quotes}{for Chapter~\Ref{ChaptC} (Limits of Real Sequences -- Basic Theory)}

\V

\begin{quotation}
{\footnotesize
        (1) `A philosopher of imposing stature doesn't think in a vacuum. Even his most abstract ideas are, to some extent, conditioned by what is or is not known in the time when he lives.'

        (From `Dialogues of Alfred North Whitehead') %% Bartlett p. 585 #1

\V  

        (2) `In contrast, consider the following ancient algorithm, attributed to Heron of Alexandria, for approximating square roots:

        \h To approximate the square root of a positive number $X$,

        \h \h -- Make a guess for the square root of $X$

        \h \h -- Compute an improved guess as the average of the guess and $X$ divided by the guess.'

        (From the published majority opinion of the U.S. Ninth Circuit Court of Appeals in the case of Bernstein vs U.~S.~Department of Justice {\em et al}, 1999.)
}%EndFootNoteSize
\end{quotation}


\VV

            \small{\bf Reminder of `Limit' as Taught in Elementary Calculus}

\V

        In elementary calculus one learns that the concept of `limit' plays a vital role, especially in formulating the main concepts of the subject.
    Specifically, two types of limit processes appear in calculus:


        (1) Limits of the form ${\displaystyle \lim_{x \,{\rightarrow}\, c} g(x)}$;
    that is, limits of functions defined on an interval~$I$. The quantity $c$ can be either a real number or one of the symbols $+{\infty}$ or $-{\infty}$.
    The quantity $x$ is a real variable, and is allowed to vary over all numbers in $I$ except at $c$ itself (if $c$ is itself in~$I$).

        The most important example of such a limit in elementary calculus arises in the definition the definition of the `derivative' of a function $f$ at a number~$c$:
        \begin{displaymath}
        f'(c) \,=\, \lim_{x \,{\rightarrow}\, c} \frac{f(x)-f(c)}{x-c}.
        \end{displaymath}
    In this case, $g(x)$ is the ratio $(f(x)-f(c))/(x-c)$, which ratio, of course, is not defined at $c$.

        One also encounters such limits in L'H\^{o}pital's Rule. More precisely, under suitable circumstances one has
        \begin{displaymath}
        \lim_{x \,{\rightarrow}\, c} {\displaystyle \frac{f(x)}{g(x)}  \,=\, \lim_{x \,{\rightarrow}\, c} \frac{f'(x)}{g'(x)}}.
        \end{displaymath}

\V

        (2) Limits of the form $\lim_{k \,{\rightarrow}\, {\infty}} x_{k}$; that is, limits of sequences of real numbers.
    The numbers $x_{1}$, $x_{2}$, \,{\ldots}\,form an infinite sequence, and the variable $k$ -- usually called the {\em index} -- is allowed to vary over all natural numbers.

        The most important example of such a limit in elementary calculus occurs in the definition of the definite integral:
        \begin{displaymath}
        \int_{a}^{b} f(x)\,dx \,=\, \lim_{k \,{\rightarrow}\, {\infty}} R_{k},
        \end{displaymath}
    where $R_{k} \,=\, f(c_{1}){\Delta}x_{1} + f(c_{2}){\Delta}x_{2} + \,{\ldots}\, + f(c_{k}){\Delta}x_{k}$ is a typical Riemann sum for a partition of the interval $[a,b]$ into $k$ parts.

        In {\ThisText} the limits of the second type, i.e., limits of sequences, play the dominant role, so we study those limits first.


\V

        {\bf Remarks} (1) If you've forgotten the definitions of `derivative' and `definite integral', or if you have only misty memories of these concepts, fear not:
    they are not needed in the rest of this chapter, and we do study them in great detail later in {\ThisText}.

\VA

        (2) In typical courses of elementary calculus many of the results of this section do not appear at all.
    In particular, this is the case for results directly involving the completeness of~${\RR}$ -- `supremum', `infimum', and so on.
    Moreover, those that do appear in elementary calculus are often phrased more informally there and without full proofs.



\V
\V

                \section{{\bf The Limit of a Sequence of Real Numbers}}
                \label{SectC10}\IndB{ZZ Sections}{\Ref{SectC10} Limit of a Sequence of Real Numbers}

\V
\V
%-------------------------

        The intuitive idea behind the statement that `The number $L$ is the limit  of the sequence ${\xi} \,=\, (x_{1},x_{2},\,{\ldots}\,x_{k},\,{\ldots}\,)$' is this:
    if $k$ is large, then $x_{k}$ is close to $L$. More precisely, by merely choosing $k$ sufficiently large, one can be sure that $x_{k}$ is as close to $L$ as one wants, or even closer.
    Of course, qualitative phrases such as `as close as one wants' and `sufficiently large' are difficult to incorporate into rigorous proofs; one wants a more quantitative formulation.
    The next definition provides such a formulation.

\V
\V

            \subsection{\small{\bf Definition}}
            \label{DefC10.10}

\V 

        Let ${\xi} \,=\, (x_{1},x_{2},\,{\ldots}\,x_{k},\,{\ldots}\,)$ be a sequence of real numbers;
    that is, ${\xi}$ is a function with domain ${\NN}$ and values in ${\RR}$ (see Definition~\Ref{DefA40.20}).

\V

        (1) One says that the sequence ${\xi}$ {\bf converges to a real number $\Bfm{L}$} provided the following statement is true:

        For every real number ${\varepsilon}\,>\,0$ there exists a positive real number $B$ such that if $k$ is any index such that $k\,\,{\geq}\,\,B$, then $|x_{k}-L|\,<\,{\varepsilon}$.
    In this case one writes $\lim_{k \,{\rightarrow}\, {\infty}} x_{k} \,=\, L$.

\V

        (2) The sequence ${\xi}$ is said to be ${\bf convergent}$\IndBD{sequences}{convergent sequences}
    if there exists a real number $L$ such that $\lim_{k \,{\rightarrow}\, {\infty}} x_{k} \,=\, L$, in the sense of Part~(1).

\V

        (3) If the sequence ${\xi}$ fails to be convergent in the sense of Part~(2), then one says that {\bf ${\xi}$ is a divergent sequence},
    or, more briefly, that {\bf ${\xi}$ diverges}\IndBD{sequences}{divergent sequences}.


\VV

\begin{quotation}
{\footnotesize  \underline{\Note}\IndB{\notes}{on the limit definition and intuition} (on the limit definition and intuition)
        The relation between this formal definition and the intuitive view of `limit' stated above is not hard to see.
    Indeed, the intuitive version can be reformulated as follows: You tell me how close you want $x_{k}$ to be to~$L$,
    and I'll tell you how large is large enough to guarantee, from the size of $k$ alone, that $x_{k}$ is that close to~$L$, or even closer.

        In the formal statement of the definition, the measure of `how close $x_{k}$ is to~$L$' is, as usual, the nonnegative number $|x_{k}-L|$.
    The degree of closeness that you want is the positive number ${\varepsilon}$ that you give me, so what you want is that $|x_{k}-L|\,<\,{\varepsilon}$.
    The `large enough' refers to the number $B$, which depends on~${\varepsilon}$ (and, of course, on the choice of the sequence ${\xi}$ under discussion,
    but that is usually understood in the given context): if $k$ is at least as large as $B$, then $k$ is large enough to guarantee that $|x_{k}-L|\,<\,{\varepsilon}$.
    Note that it is irrelevant that there might also be values of $k$ {\em smaller} than $B$ for which one has $|x_{k}-L|\,<\,{\varepsilon}$.
}%EndFootNoteSize
\end{quotation}


\VV

       {\bf Remark} Some authors use minor variations of Definition~\Ref{DefC10.10}~(1) for the meaning of the statement that a given sequence ${\xi}$ converges to a given number~$L$.
    The next result gives some examples of such variations which are equivalent to Definition~\Ref{DefC10.10}~(1).
    It also provides an example of a similar variation which is {\em not} equivalent to that definition.
    

            \subsection{\small{\bf Theorem}}
            \label{ThmC10.12}

\V

        (1) (a) Supppose that in the statement of Definition~\Ref{DefC10.10}~(1) one makes one or more of the following changes:

\VA

      \h  (i) replace the hypothesis `positive real number $B$' by `positive integer $B$'

\VA

      \h  (ii) replace the hypothesis `$k\,\,{\geq}\,\,B$' by `$k\,>\,B$'

\VA

     \h   (iii) replace the hypothesis `$|x_{k}-L|\,<\,{\varepsilon}$' by `$|x_{k}-L|\,\,{\leq}\,\,{\varepsilon}$'.

\VA

\noindent Then the resulting modified definition is equivalent to Definition~\Ref{DefC10.10}~(1), in the sense that if the sequence ${\xi}$ converges to the number $L$ according to one definition, 
    then it does so according to the other.

\V

        (b) In contrast, if the hypothesis `${\varepsilon}\,>\,0$' in Definition~\Ref{DefC10.10}~(1) is replaced by the hypothesis `${\varepsilon}\,\,{\geq}\,\,0$',
    then the resulting definition is {\em not} equivalent to Definition~\Ref{DefC10.10}~(1).

\V

        The proofs of these assertions are left as exercises. In what follows we occasionally use the definitions modified as in Part~(a) of this theorem, often without further comment.
    %% EXERCISES


\VV

        The next result provides a more significant reformulation of Definition~\Ref{DefC10.10}~(1) which is often useful.

\V

            \subsection{\small{\bf Theorem}}
            \label{ThmC10.15}

\V

        The following statements are equivalent:

\V

        (i)\, The sequence ${\xi} \,=\, (x_{1},x_{2},\,{\ldots}\,)$ converges to the real number $L$, in the sense of Definition~\Ref{DefC10.10}~(1).

\V

        (ii) For every ${\varepsilon}\,>\,0$ there are only finitely many indices $j$ such that $|x_{j}-L|\,\,{\geq}\,\,{\varepsilon}$.
    That is, for each ${\varepsilon}\,>\,0$, the sequence $(|x_{1}-L|, |x_{2}-L|, \,{\ldots}\,|x_{k}-L|,\,{\ldots}\,)$ is eventually less than~${\varepsilon}$.

    
\V

        {\bf Proof} Suppose that Statement (i) is true, and let ${\varepsilon}\,>\,0$ be given.
    Let $N$ in ${\NN}$ be large enough that if the index $k$ satisfies $k\,\,{\geq}\,\,N$, then $|L-x_{k}|\,<\,{\varepsilon}$ (see Theorem~\Ref{ThmC10.12}~(a)~(i)).
    It follows that if $j$ is an index such that $|x_{j}-L|\,\,{\geq}\,\,{\varepsilon}$, then $1\,\,{\leq}\,\,j\,<\,N$.
    In particular, the set of all such indices $j$ is bounded above in ${\NN}$ (by $N$), and thus is a finite set (by Theorem~\Ref{ThmA20.10}).
    The reformulation in terms of the concept of `eventually' now follows from Definition~\Ref{DefA40.80}.

        Conversely, suppose that Statement (ii) is true, and let ${\varepsilon}\,>\,0$ be given.
    Let $A$ be the set of all indices $j$ such that $|x_{j}-L|\,\,{\geq}\,\,{\varepsilon}$.
    Then, by the hypothesis that Statement~(ii) holds, the set $A$ is a finite subset of ${\NN}$,
    and thus (by Theorem~\Ref{ThmA20.10} again) is bounded above by some natural number~$N$.
    It follows that if $k\,\,{\geq}\,\,N+1$ then $k$ is {\em not} in $A$ and thus $|x_{k} - L|\,<\,{\varepsilon}$. Statement~(1) now follows.

\VV

        The preceding result provides a useful method for proving that a given sequence is convergent.
    Also important, however, is the ability to prove that a given sequence is divergent. The next result is the analog to Theorem~\Ref{ThmC10.15} for that type of question.

\V

            \subsection{\small{\bf Theorem}}
            \label{ThmC10.16}

\V


        The following statements are equivalent:

\V

        (i)\, The sequence ${\xi} \,=\, (x_{1},x_{2},\,{\ldots}\,)$ is divergent, in the sense of Definition~\Ref{DefC10.10}~(3).

\V

        (ii) For every real number $L$ there exists ${\varepsilon}\,>\,0$ such that the inequality $|x_{j}-L|\,\,{\geq}\,\,{\varepsilon}$ holds for infinitely many indices $j$.

\V

        {\bf Proof}\, Suppose that Statement~(i) is true, so that ${\xi}$ is divergent. This means, by Definition~\Ref{DefC10.10}~(3),
    that for each real number $L$ the sequence ${\xi}$ does not converge to~$L$. That is, for each real $L$ Statement~(i) of Theorem~\Ref{ThmC10.15} is false.
    Since that statement is equivalent to Statement~(ii) of Theorem~\Ref{ThmC10.15}, this implies that for each real $L$ the latter statement is also false.
    It follows that there exists ${\varepsilon}\,>\,0$ such that $|x_{j}-L|\,\,{\geq}\,\,{\varepsilon}$ for infinitely many indices~$j$;
    that is, Statement~(ii) of the current theorem is also true. The proof that our Statement~(ii) implies our Statement~(i) follows similarly.
    
\V

        {\bf Remark}\,The argument given here is typical of what one finds in research papers and in texts at the level of {\ThisText}.
    In particular, it assumes that the reader is already used to dealing with the negations of mathematical statements.
    Some readers may not yet be comfortable dealing with such negations. The following {\Note} is intended for such readers.

\V

\begin{quotation}
{\footnotesize 
\underline{\Note} \IndB{\notes}{on negating mathematical statements} (on negating mathematical statements)
A common occurance in constructing logical arguments in mathematics (as well as in other areas of life) is the need to consider the negation of a given statement;
    that is, to consider the new statement which asserts that the given statement is not true. Such negations occur in so-called `Proofs by Contradiction':
    a given statement must be true because assuming that its negation is true leads to a contradiction to a known fact;
    see, for example, the proof of the Strong Principle of Mathematical induction (see~Theorem~\Ref{ThmA20.04A}).
    (Students often are queasy about using such arguments: they prefer to prove directly that a given statement is true,
    not indirectly by showing that it can't be false. This allows them to avoid considering the negation of the given statement.
    Once they get the hang of arguments `by contradiction', however, they sometimes go to the other extreme and use them in all cases, even when a direct proof might be easier and clearer.)
    Negations also arise as in Theorem~\Ref{ThmC10.16} above, where one tries to prove directly the truth of the negation of a given statement.

    In any event, the process of negating a given mathematical statement, and then using that negation in a useful manner, often confuses beginning math students.
    What follows illustrates, in excruciating detail, this process in the special case of proving that a given sequence is divergent.

        The statement to be proved is that a certain sequence is divergent. Thus, the first step would be to recall, or, if needed, to look up,
    the meaning of the phrase `divergent sequence' as it appears, say, in Definition~\Ref{DefC10.10}~(3).
    (Note: Experienced math teachers will tell you that surprisingly many students do {\em not} automatically think of
    `Look up its definition' as the obvious response to the statement `I don't know what this word means'.)

        The definition of `divergent sequence' is short and simple -- it means `it is not a convergent sequence'.
    That is, the statement to be proved presents itself as the negation of the statement that the given sequence {\em is} convergent.
    Beginners often encounter difficulties with formulating the negation of this last statement, mainly because the meaning of `convergent sequence' is complicated:

\VA

        \h `A sequence ${\xi}$ of real numbers is convergent if there exists a real number $L$ such that,
for every ${\varepsilon}\,>\,0$,
    there exists $N$ in ${\NN}$ such that if $k$ in ${\NN}$ such that $k\,\,{\geq}\,\,N$, then the inequality $|x_{k}-L|\,<\,{\varepsilon}$ holds.'

\VA

\noindent The definition of ${\xi} \,=\, (x_{1}, x_{2},\,{\ldots}\,x_{k},\,{\ldots}\,)$ being convergent then boils down to the following condition:

\VA

        \h \underline{Condition 0}\,`There exists a real number $L$ such that for every ${\varepsilon}\,>\,0$
    there exists $N$ in ${\NN}$ such that for every $k$ in ${\NN}$ such that $k\,\,{\geq}\,\,N$ then the inequality  $|x_{k}-L|\,<\,{\varepsilon}$ holds.'

\VA

\noindent Condition~0 is of the form `There exists a real number $L$ such that a certain complicated condition holds', where that condition is:

\VA

        \h \underline{Condition 1}\,`For every ${\varepsilon}\,>\,0$ there exists $N$ in ${\NN}$ such that
    if $k$ in ${\NN}$ satisfies the inequality  $k\,\,{\geq}\,\,N$, then the inequality $|x_{k}-L|\,<\,{\varepsilon}$ holds.'

\VA

\noindent The assertion that ${\xi}$ is {\em not} convergent is then the {\em negation} of Condition~0, which then takes the form

\VA

        \h `There does not exist a real number $L$ such that Condition~1 holds'.

\VA

\noindent Otherwise stated, `For every real number $L$, Condition~1 is not true'.
    This in turn requires understanding what it means for Condition~1 to not be true; 
    that is, what is the meaning of the negation of Condition~1. However, Condition~1 is of the form
    `For every ${\varepsilon}\,>\,0$ a certain condition holds', where this `certain condition' is

\VA

        \h \underline{Condition 2} `There exists $N$ in ${\NN}$ such that if $k$ in ${\NN}$
    satisfies $k\,\,{\geq}\,\,N$, then $|x_{k}-L|\,<\,{\varepsilon}$.'

\VA

\noindent The negation of Condition~1 then takes the form `It is not the case that for every ${\varepsilon}\,>\,0$ Condition~2 holds'.
    That is, there exists ${\varepsilon}\,>\,0$ for which Condition~2 fails to hold.
    Now one needs to understand the negation of Condition~2. This condition is of the form `There exists $N$ in ${\NN}$ such that a certain condition holds', where this next condition is

\VA

        \h \underline{Condition 3} `If $k$ in ${\NN}$
    satisfies $k\,\,{\geq}\,\,N$, then $|x_{k}-L|\,<\,{\varepsilon}$.'

\VA

\noindent The negation of Condition~2 then takes the form `It is not the case that there exists $N$ in ${\NN}$ such that Condition~3 holds'.
    That is, for every $N$ in ${\NN}$, Condition~3 must fail to hold. This leads one to consider the negation of Condition~3.
    But Condition~3 is of the form `If $k$ in ${\NN}$ satisfies $k\,\,{\geq}\,\,N$, then a certain condition holds'. This last condition is

\VA

        \h \underline{Condition 4} `$|x_{k}-L|\,<\,{\varepsilon}$.'

\VA

\noindent Thus, to say that Condition~3 fails to hold takes the form `It is not the case that if $k$ in ${\NN}$ satisfies $k\,\,{\geq}\,\,N$,
    then Condition~4 holds'. In other words, there exists $k$ in ${\NN}$ such that $k\,\,{\geq}\,\,N$ but Condition~4 fails to hold.
    This leads one, finally, to consider the negation of Condition~4, which is `It is not the case that $|x_{k}-L|\,<\,{\varepsilon}$'.

        Up to now, the analysis has been essentially all linguistic. Indeed, the analysis has consisted in two observations:

\VA

        \h (i)\,If a condition can be written in the form `For every object of a certain type,
    a certain simpler condition must be true',
    then the negation of the original condition can be written in the form `There exists an object of that type for which the simpler condition fails to be true'.

        \h (ii) If a condition can be written in the form `There exists an object of a certain type for which a certain simpler condition must be true', 
    then the negation of the original condition can be written in the form `For every object of that type the simpler condition fails to be true'.

\VA

\noindent One does not really need to understand the mathematics to carry out this linguistic analysis.

        The exception is Condition~4: It is not of Type~(i) or Type~(ii). In this example, Condition~4 is where mathmatics actually begins to play a role.
    Indeed, the statement `It is not the case that $|x_{k}-L|\,<\,{\varepsilon}$' does not admit further linguistic analysis along the lines followed above.
    However, this statement is known, from the order properties of ${\RR}$, to be equivalent to `$|x_{k}-L|\,\,{\geq}\,\,{\varepsilon}$'.

        If we combine all these results we get the following condition for the sequence
    ${\xi} \,=\, (x_{1}, x_{2},\,{\ldots}\,x_{k},\,{\ldots}\,)$ to {\em not} be convergent:

\VA

        \h `For every real number $L$, there exists ${\varepsilon}\,>\,0$ such that for every $N$ in ${\NN}$ there exists $k$ in ${\NN}$,
    with $k\,\,{\geq}\,\,N$, such that $|x_{k}-L|\,\,{\geq}\,\,{\varepsilon}$.'

\VA

        This last statement implies that the set of indices $k$ for which $|x_{k}-L|\,\,{\geq}\,\,{\varepsilon}$ is not bounded above by a number in~${\NN}$.
    By Theorem~\Ref{ThmA20.10}, this last statement can be reformulated as

\VA

        \h `For every real number $L$, there exists ${\varepsilon}\,>\,0$ such that there exist infinitely many indices $k$ in ${\NN}$ such that $|x_{k}-L|\,\,{\geq}\,\,{\varepsilon}$.'

\V

        The preceding analysis seems lengthy, but that is mainly because all the details are included.
    In mathematical writing it is assumed that the reader can carry out such an analysis without help, and can do it fairly rapidly (and accurately).
    With a little experience in carefully reading correct proofs in mathmatical texts, one finds that this assumption becomes reality.

        One reason for the complexity of this analysis is that it reverts to the original definition of `convergence'.
    Often, however, the same result can be proved more easily later on by using structural theorems; for instance, see Example~\Ref{ExampC20.11A}.

        {\bf Final Warning} Consider the following statement: \, `All automobiles are green'

        The correct negation of this statement is `There is at least one automobile which is not green.' (This is often written more informally as `Some automobiles are not green';
    but the use of the plural here may make it appear that one is saying that more than one automobile is not green, which is not what is intended.)

        Unfortunately, some beginners would state the negation of the original statement as `All automobiles are not green';
    equivalently, `No automobile is green'. One simple way to avoid this common error is to formulate the negation of a statement,
    as was done above, by simply adding the phrase `It is not the case that'. The correct negation is usually easy to formulate from that.
}%EndFootNoteSize
\end{quotation}


\VV

            \subsection{\small{\bf Examples}}
            \label{ExampC10.20}

\V

\hspace*{\parindent}(1) Let $c$ be a real number, and let ${\xi} \,=\, (c,c,c,\,{\ldots}\,)$ be the constant sequence with value $c$; that is, ${\xi}$ is the function from ${\NN}$ to ${\RR}$ such that ${\xi}(k) \,=\, c$ for all~$k$.
    It seems `obvious' that the sequence ${\xi}$ is convergent, and $\lim_{k \,{\rightarrow}\, {\infty}} x_{k} \,=\, c$.
    To verify this directly from Definition~\Ref{DefC10.10}, suppose that ${\varepsilon}\,>\,0$ is given.
    Clearly $|x_{k}-c| \,=\, 0$, so that $|x_{k}-c|\,<\,{\varepsilon}$, for every $k\,\,{\geq}\,\,1$. In particular,
    one can choose $B \,=\, 1$ in the definition of  `convergence'; although any larger value of $B$ would work just as well.

        More generally, suppose that ${\xi}:{\NN} \,{\rightarrow}\, {\RR}$ is {\em eventually} the constant $c$;
    that is, there exists $m$ in ${\NN}$ such that ${\xi}(k) \,=\, c$ for all $k\,\,{\geq}\,\,m$.
    Then it is easy to see that $\lim_{k \,{\rightarrow}\, {\infty}} x_{k} \,=\, c$.

\V

        (2) (a) Suppose that $x_{k} \,=\, 1/k$ for each $k$ in ${\NN}$. It seems `obvious' that this sequence converges to~$0$.
    To verify this directly from Definition~\Ref{DefC10.10}, let ${\varepsilon}\,>\,0$ be given,
    and let $N$ be a natural number such that $N\,>\,1/{\varepsilon}$. (That such $N$ exists follows from the Principle of Archimedes.)
    If $k\,\,{\geq}\,\,N$, so that $k\,>\,1/{\varepsilon}$, then clearly
        \begin{displaymath}
        |x_{k} - 0| \,=\, |x_{k}| \,=\, \frac{1}{k}\,<\,1/(1/{\varepsilon}); \mbox{ that is, }
        |x_{k}-0|\,<\,{\varepsilon} \mbox{ if $k\,\,{\geq}\,\,N$}.
        \end{displaymath}
    Thus, $L \,=\, 0$ satisfies the requirements for being the limit of the sequence ${\xi}$.

        (b) More generally, suppose that there is a real number $a$ such that $x_{k} \,=\, a/k$ for all $k$ in ${\NN}$.

        \h \underline{Case 1}: Suppose that $a \,=\, 0$. Then $x_{k} \,=\, 0$ for each $k$, so the result follows from Example~(1) above, with $c$ in that example taken to equal~$0$.

        \h \underline{Case 2}: Suppose, instead, that $a \,\,{\neq}\,\, 0$. Then by an argument similar to that used in Part~(a) it follows that
    if $N$ in ${\NN}$ satifies the inequality $N\,>\,|a|/{\varepsilon}$, then for each index $k$ such that $k\,\,{\geq}\,\,N$ one has
        \begin{displaymath}
        |x_{k} - 0| \,=\, \frac{|a|}{k}\,\,{\leq}\,\,\frac{|a|}{N}\,<\,
        |a|\,\frac{{\varepsilon}}{|a|} \,=\, {\varepsilon}.
        \end{displaymath}
    As before, this implies that $L \,=\, 0$ is the limit of the given sequence.

    \underline{Note} Handling the case $a \,=\, 0$ separately from the case $a \,\,{\neq}\,\, 0$ is not a waste of time:
    The proof in the case $a \,\,{\neq}\,\, 0$ involves division by $|a|$, a process that is not possible in the case $a \,=\, 0$.

\V

        (3) Consider the sequence ${\xi}$ whose $k$-th term $x_{k}$ is given by
        \begin{displaymath}
        x_{k} \,=\, \frac{3k^{2}+k+5}{6k^{2}+1} \mbox{ for $k$ in ${\NN}$}.
        \end{displaymath}
    In the previous examples there was an `obvious' candidate to play the role of the number~$L$ in Definition~\Ref{DefC10.10}.
    However, in the present example the candidate for $L$ is not at all obvious, especially for one inexperienced with such problems.
    Thus, let us compute $x_{k}$ for several values of $k$ to try to get some insight. One gets (after rounding off the decimals)
        \begin{displaymath}
        \begin{array}{||l|c|c|c|c|c|c|c||} \hline
        k & 1 & 5 & 10 & 20 & 50 & 100 & 1000 \\ \hline
          &   &   &   &    &     &     &      \\ 
        x_{k} & 1.28 & 0.5629 & 0.5241 & 0.5102 & 0.5036 & 0.50174 & 0.50017 \\
          &   &   &   &    &     &     & \\ \hline
        \end{array}
        \end{displaymath}
    As $k$ gets larger and larger, it seems plausible that $x_{k}$ is approaching $1/2$.
    Thus let us see what happens if we try $L \,=\, 1/2$.
    Using standard algebra, such as putting fractions over a common denominator and then simplifying the results, and some simple facts about inequalities, one computes:
        \begin{displaymath}
        \left|x_{k} - \frac{1}{2}\right| \,=\, \left|\left(\frac{3k^{2}+k+5}{6k^{2}+1}\right) - \frac{1}{2}\right|\,=\, \left|\frac{-(6k^{2}+1) + 2(3k^{2}+k+5)}{2(6k^{2}+1)}\right| \,=\, 
        \frac{2k+9}{12k^{2}+2}\,<\,
    \frac{2k+9}{12k^{2}}\,\,{\leq}\,\,\frac{2k+9k}{12k^{2}} \,=\, \frac{11}{12k}.
        \end{displaymath}
    Now let ${\varepsilon}\,>\,0$ be given. Then, using the same reasoning as in Example~(2\,b) above, let $N$ in ${\NN}$ be chosen so that $N\,>\,{\varepsilon}/a$, where $a \,=\, 11/12$.
    It is clear that if $k\,\,{\geq}\,\,N$ then $|x_{k}-1/2|\,<\,{\varepsilon}$.
    Thus, the sequence ${\xi}$ is convergent, and $\lim_{k \,{\rightarrow}\, {\infty}} x_{k} \,=\, 1/2$, as expected.

        \underline{Note} The approach taken in this example tries to determine convergence using Definition~\Ref{DefC10.10} directly.
    This requires knowing -- or at least guessing correctly -- the limit $L \,=\, 1/2$ in advance.
    Indeed, the sequence used here was chosen precisely because the `guessing' method would work.
    However, it is not hard to construct examples of sequences for which the exact value of $L$ can not be predicted, so some other method must be used.
    In Section~\Ref{SectC60} the present example is analysed again using such a method.

\V

        (4) Let ${\xi} \,=\, (2,-2,2,-2,\,{\ldots}\,)$ be the sequence whose $k$-th term $x_{k}$ equals $(-1)^{k-1}{\cdot}2$ for each $k$ in ${\NN}$. Then ${\xi}$ is divergent.

        Indeed, suppose that $L$ is any real number. If $L\,\,{\geq}\,\,0$, then clearly for all even values of $k$ one has $|x_{k}-L| \,=\, |-2-L| \,=\, L+2\,\,{\geq}\,\,2$;
    in particular, this holds for infinitely many values of the index~$k$. It follows from Theorem~\Ref{ThmC10.16},
    with ${\varepsilon} \,=\, 2$, that ${\xi}$ does not converge to $L$ if $L\,\,{\geq}\,\,0$. A similar argument shows that if $L\,\,{\leq}\,\,0$,
    then $|x_{k} - L| \,=\, 2+|L|\,\,{\geq}\,\,2$ for all the (infinitely many) {\em odd}~$k$, and thus the sequence still does not converge to~$L$.

\V

        (5) Let ${\zeta} \,=\, (1^{2}, 2^{2}, 3^{2},\,{\ldots}\,k^{2},\,{\ldots}\,)$ be the sequence whose $k$-th term $z_{k}$ is given by $z_{k} \,=\, k^{2}$ for each $k$ in~${\NN}$.
    Let $L$ be any real number.

        If $L\,\,{\leq}\,\,0$, then clearly $|x_{k} - L| \,=\, k^{2} + |L|\,\,{\geq}\,\,k^{2}\,\,{\geq}\,\,1$ for every index $k$.
    If, instead, $L\,>\,0$, then there exists a positive integer $N$ such that $L\,\,{\leq}\,\,N$. It follows that if $k$ is any index such that $k\,\,{\geq}\,\,N+1$,
    then $k^{2}\,\,{\geq}\,\,k\,\,{\geq}\,\,N+1\,\,{\geq}\,\,L+1$; that is, $|x_{k} - L| \,=\, x^{2}-L\,\,{\geq}\,\,1$.
    In particular, for every $L$ there are infinitely indices $k$ such that $|x_{k} - L|\,\,{\geq}\,\,1$, so it follows from Theorem~\Ref{ThmC10.16},
    with ${\varepsilon} \,=\, 1$ in that result, that ${\xi}$ is divergent.

\V

        {\bf Remark}\, The results obtained for Examples (4) and~(5) above can be verified much more quickly once we prove Theorem~\Ref{ThmC20.10A} below; indeed, somewhat more can be proved.


\VV

        {\bf Infinite Limits of Sequences} The sequences in Example~\Ref{ExampC10.20}~(4) and~(5) above are both divergent,
    but they are divergent in very different ways. Indeed, the sequence in Example~(4) `bounces around',
    with no definite `goal' as the index $k$ gets larger. In contrast, the sequence in Example~(5) does have,
    in an obvious sense, a `goal'; namely, its terms `tend towards'~$+{\infty}$. This type of divergence occurs so often that a special terminology has been 
    developed to describe it.

\V

           \subsection{\small{\bf Definition}}
            \label{DefC40.10}

        Let ${\xi} \,=\, (x_{1},x_{2},\,{\ldots}\,)$ be a sequence of real numbers.

\V

        (1) One says that the sequence ${\xi}$ {\bf diverges to $+{\infty}$},
\IndBD{sequences}{sequence diverges to $+{\infty}$ or to $-{\infty}$}
    and one writes $\lim_{k \,{\rightarrow}\, {\infty}} x_{k} \,=\,+{\infty}$, if, for every positive real number $M$,
    there is a real number $B$ such that if $k\,\,{\geq}\,\,B$ then one has $x_{k}\,\,{\geq}\,\,M$;
    equivalently: for every real $M\,>\,0$ the sequence is eventually greater than~$M$.

        Similarly, one says that ${\xi}$ {\bf diverges to $-{\infty}$}, and one writes $\lim_{k \,{\rightarrow}\, {\infty}} x_{k} \,=\,-{\infty}$,
    if for every negative real number $M$, there is a real number $B$ such that if $k\,\,{\geq}\,\,N$ then one has $x_{k}\,\,{\leq}\,\,M$;
    equivalently: for every real $M\,<\,0$, the sequence is eventually less than~$M$.

\V

        (2) One says that the sequence ${\xi}$ {\bf has a limit}, or that {\bf $\lim_{k  \,{\rightarrow}\, {\infty}} x_{k}$ is defined},
    \IndBD{sequences}{sequence has a limit, or its limit is defined, or its limit exists}
    or that {\bf $\lim_{k  \,{\rightarrow}\, {\infty}} x_{k}$ exists} if one of the following three possibilities holds:

        \h (i)\,\, ${\xi}$ is convergent to some real number~$L$;

        \h (ii)\, ${\xi}$ diverges to $+{\infty}$;

        \h (iii) ${\xi}$ diverges to $-{\infty}$.

\noindent If either (ii) or (iii) holds, one can say that ${\xi}$ {\bf has an infinite limit}.\IndBD{sequences}{sequence has an infinite limit}
    If, instead, case (i) holds, one can say that ${\xi}$ has a {\bf finite limit},\IndBD{sequences}{sequence has a finite limit}
    although that gives no more information than simply saying that ${\xi}$ is convergent, since (as we shall see) it may well be that the value of the liimit $L$ may not be known.

\V

        \subsection{\small{{\bf Remarks}}}
        \label{RemrkC40.15}

\V

\hspace*{\parindent}(1) The preceding definition allows one to make sense of an equation of the form $\lim_{k \,{\rightarrow}\, {\infty}} x_{k} \,=\, L$
    whenever $L$ is an extended real number;\IndBD{real numbers}{extended real numbers}
    that is, either $L$ is an ordinary real number, or $L$ is one of the infinities, $+{\infty}$ or $-{\infty}$.

\V

        (2) Sometimes, in place of the notation $\lim_{k \,{\rightarrow}\, {\infty}} x_{k} \,=\, L$, authors use the notation $x_{k} \,{\rightarrow}\, L$ as $k \,{\rightarrow}\, {\infty}$.
    The latter symbolism is read `$x_{k}$ approaches the limit $L$ as $k$ approaches infinity'.


\V

        (3) It is tempting to translate an equation such as $\lim_{k \,{\rightarrow}\, {\infty}} x_{k} \,=\, +{\infty}$ as `the sequence ${\xi}$ {\em converges} to $+{\infty}$'.
    This is improper usage; if ${\xi}$ has an infinite limit, one should say that ${\xi}$ {\em diverges} to that limit.

\V

        (4) It is common in calculus texts to write down expressions and only afterwards ask whether the expression makes sense.
    For instance, a typical `limit' problem in such a text mught take the form

        \h `Determine whether $\lim_{k \,{\rightarrow}\, {\infty}} x_{k}$ exists'

\noindent where $x_{k}$ is given by a specific formula. A better formulation would be

        \h `Determine whether the sequence $(x_{1},x_{2},\,{\ldots}\,x_{k},\,{\ldots}\,)$ has a limit.'

\noindent In {\ThisText} we attempt to follow the latter model. However,  be aware that many authors use the former model freely.
    Indeed, there are situations in which that mode of expression is so ingrained in standard usage that trying to avoid it would cause confusion.

\V

            \subsection{\small{\bf Examples}}
            \label{ExampC40.20}

\V

        Note: The statements made in the examples here follow easily from Definition~\Ref{DefC40.10}, and their proofs are left as simple exercises. % EXERCISE

\V

        (1) Let ${\xi} \,=\, (1,2,3,\,{\ldots}\,)$, so that $x_{k} \,=\, k$ for all $n$.
    Then $\lim_{k \,{\rightarrow}\, {\infty}} x_{k} \,=\, +{\infty}$. Likewise, the sequence $-{\xi} \,=\, (-1,-2,-3,\,{\ldots}\,)$ has the limit $-{\infty}$.

\V

        (2) The sequence $(1,-2,3,-4,\,{\ldots}\,)$ does not have a limit.

\V

        (3) The sequence ${\xi} \,=\, (0,4,0,8,0,12,\,{\ldots}\,)$ for which $x_{k} \,=\, k+(-1)^{k}k$.

\VV

%------------------------------
\StartSkip{

        The phrasings used in Definitions~\Ref{DefC10.10} and~\Ref{DefC40.10} above are standard.
    There are some minor variations of these phrasings, however, that are logically equivalent to them and which in some circumstances are simpler to use.
    The next result summarizes these variations for future reference.

\V

            \subsection{\small{\bf Theorem}}
            \label{ThmC10.15A}

\V

        In this theorem ${\xi} \,=\, (x_{1},x_{2},\,{\ldots}\,x_{k},\,{\ldots}\,)$ is a sequence of real numbers.

\V

        (a) Let $L$ be a real number. Then following statements are equivalent:

\VA

        \h (i) \,\, The sequence ${\xi}$ converges to the $L$, in the sense of Definition~\Ref{DefC10.10}.

\VA

        \h (ii)\, For every real number ${\varepsilon}\,>\,0$ there exists a real number $B$ such that if $k$ is an index which satisfies $k\,\,{\geq}\,\,B$,
    then $|x_{k}-L|\,<\,{\varepsilon}$.

\VA

        \h (iii) The same as the preceding statement except the inequality $k\,\,{\geq}\,\,B$ is replaced by $k\,>\,B$,
    or the inequality $|x_{k}-L|\,<\,{\varepsilon}$ is replaced by $|x_{k}-L|\,\,{\leq}\,\,{\varepsilon}$.
    (Recall that `or' means the `inclusive or', so that it is allowed to make both replacements.)

\V

        (b) The following statements are equivalent:

\VA

        \h (iv) The sequence ${\xi}$ diverges to $+{\infty}$, in the sense of Part~(a) of Definition~\Ref{DefC40.10}.

\VA

        \h (v)\, For every positive real number $M$ there is a real number $B$ such that if if $k$ is an index which satisfies $k\,\,{\geq}\,\,B$,
    then one has $x_{k}\,\,{\geq}\,\,M$.

\VA

        \h (vi) The same as the preceding statement except the inequality $k\,\,{\geq}\,\,B$ is replaced by $k\,>\,B$,
    or the inequality $x_{k}\,\,{\geq}\,\,M$ is replaced by $x_{k}\,>\,M$.

\V

        (c) The following statements are equivalent:

\V

        \h (ix) The sequence ${\xi}$ diverges to $-{\infty}$, in the sense of Part~(b) of Definition~\Ref{DefC40.10}.

\VA

        \h (x)\, For every negative real number $M$ there exists a real number $B$ such that if $k$ is an index which satisfies $k\,\,{\geq}\,\,B$,
    then $x_{k}\,\,{\leq}\,\,M$.

\VA

        \h (xi) The same as the preceding statement except the inequality $k\,\,{\geq}\,\,B$ is replaced by $k\,>\,B$,
    or the inequality $x_{k}\,\,{\leq}\,\,M$ is replaced by $x_{k}\,<\,M$.

\V

        The trival proofs are left as exercises.

\VV

%%%
\begin{quotation}
{\footnotesize \underline{\Note}\IndB{\notes}{on the various formulations of a sequence having a limit} (on the various formulations of a sequence having a limit)
        The various formulations given above differ only trivially from each other in the context of sequences of real numbers.
    However, something deeper arises if one replaces `real number' by `element of an ordered field' throughout.
    For example, the definition of a sequence ${\xi}$ of numbers converging to a number~$L$, as presented in Definition~\Ref{DefC10.10},
    makes sense if one replaces `real number' and ${\RR}$ by `rational number' and ${\QQ}$ throughout that definition.
    In particular, it makes sense to introduce the concept of `convergence of a sequence of numbers' before the introduction of the `Completeness' axiom.


        In any such `non-real' treatment of convergence, the various formulations of given in Theorem~\Ref{ThmC10.15A} are not equivalent.
    For instance, in any non-Archimedean ordered field there exists a number $B$ such that if $k$ is any natural number,
    viewed as an element of the ordered field as in Definition~\Ref{DefB25.40}, then $k\,\,{\leq}\,\,B$.
    For such $B$, the condition `$k\,>\,B$' appearing in Part~(ii) of Theorem~\Ref{ThmC10.15A} is satisfied by {\em no} index~$k$, and thus the statement
    `$|x_{k}-L|\,<\,{\varepsilon}$ if $k\,>\,B$' is automatically true for every ${\varepsilon}\,>\,0$ and every number~$L$.
    That is, in such a field every sequence converges to every $L$ if one uses Part~(ii) of Theorem~\Ref{ThmC10.15A} to define convergence.
}%EndFootNoteSize
\end{quotation}
%##
}%\EndSkip
%--------------------------------------------

        {\bf Remark} The careful reader will have noticed that, depending on whether $L$ is finite or one of the infinities,
    the description of `$\lim_{k \,{\rightarrow}\, {\infty}} x_{k} \,=\, L$' assumes different forms. In Definition~\Ref{DefC10.10} and what follows, one focuses on the expression $|x_{k}-L|$.
    This expression describes the distance between $x_{k}$ and $L$ if $L$ is finite, and thus it is adequate to describe the intuitive requirement that `$x_{k}$ is close to $L$ if $k$ is large'.
    In constrast, if $L \,=\,  \,{\pm}\, {\infty}$, the expression $|x_{k}-L|$ equals $+{\infty}$ for each index $k$ (see Remark~\Ref{RemrkB20.43}),
    and thus provides no information about how `close' the real number $x_{k}$ is to the (infinite) quantity $L$. Because of this dichotomy,
    one is frequently forced to provide separate proofs for results involving an equation $\lim_{k \,{\rightarrow}\, {\infty}} x_{k} \,=\, L$
    depending on whether $L$ is finite, $+{\infty}$ or $-{\infty}$. Although providing such separate proofs is normally not difficult, it can be annoying.
    The next result can sometimes be used to give unified proofs which work regardless of whether $L$ is finite or infinite.

\VV

            \subsection{\small{\bf Theorem}}
            \label{ThmC40.80}

\V

        Suppose that ${\xi} \,=\, (x_{1},x_{2},\,{\ldots}\,)$ is a sequence of real numbers, and that $L$ is an extended real number.
    Then a necessary and sufficient condition for the equation $\lim_{k \,{\rightarrow}\, {\infty}} x_{k} \,=\, L$ to hold is that both of the following statements are correct:

\VA

        \h \underline{Statement A}\, If $y$ is any real number such that $y\,<\,L$,
    then $y\,<\,x_{k}$ for all but finitely many indices~$k$.

\VA

        \h \underline{Statement B} If $z$ is any real number $z$ such that $z\,>\,L$,
        then $x_{k}\,<\,z$ for all but finitely many indices~$k$.

\V

        {\bf Proof} \underline{Case 1: $L$ is finite}

        Suppose that $\lim_{k \,{\rightarrow}\, {\infty}} x_{k} \,=\, L$.
    Let $y$ be any number such that $y\,<\,L$, and let ${\varepsilon} \,=\, L-y$, so that ${\varepsilon}\,>\,0$, and $y \,=\, L-{\varepsilon}$.
    The convergence hypothesis implies that $|L-x_{k}|\,<\,{\varepsilon}$, that is,
    $L-{\varepsilon}\,<\,x_{k}\,<\,L+{\varepsilon}$, for all but finitely many indices~$k$.
    In particular, since $L-{\varepsilon} \,=\, y$, Statement~A holds. A similar argument shows that Statement~B holds.

        Conversely, suppose that both Statement~A and Statement~B hold. Let ${\varepsilon}\,>\,0$ be given,
    and set $y \,=\, L-{\varepsilon}$ and $z \,=\, L+{\varepsilon}$. Then the truth of these statements implies that
    one has $L-{\varepsilon}\,<\,x_{k}$ and $x_{k}\,<\,L+{\varepsilon}$, that is, $|x_{k}-L|\,<\,{\varepsilon}$,  for all but finitely many indices~$k$.
    Thus, ${\xi}$ converges to~$L$, as required.

\V

        \underline{Case 2} $L$ is infinite; that is, $L \,=\, +{\infty}$ or $L \,=\, -{\infty}$.

    In this case a similar analysis works; indeed, the analysis is slightly easier. For example, if $L \,=\, +{\infty}$, then Statement~B is automatically true,
    because there are no real numbers $z$ such that $z\,>\,+{\infty}$. Thus in this case one needs to check only that Statement~A is true.
    Likewise, if $L \,=\, -{\infty}$, then one needs to check only that Statement~B is true.

\VV

        {\bf Extension of the Definition of Limit of a Sequence} This is a convenient place to discuss one feature of the definition of `infinite sequence' which is a source of mild irritation in practice.
    Specifically, Definition~\Ref{DefA40.20} defines an infinite sequence to be a function ${\alpha} \,=\, (x_{1},x_{2},\,{\ldots}\,x_{k},\,{\ldots}\,)$ whose domain is ${\NN}$,
    the set of {\em all} natural numbers. This form of the definition is widely used, and it corresponds exactly to the intuition that `sequence' is a fancy word for `ordered list'.

        The `mild irritation' referred to above is this: in dealing with an expression such as $\lim_{k \,{\rightarrow}\, {\infty}} x_{k}$,
    one really needs to consider only terms $x_{k}$ with $k$ large. More precisely, given any positive integer $N$, one can ignore all the terms $x_{k}$ with $k\,\,{\leq}\,\,N$.
    Whether the sequence has a limit $L$, and the value of $L$ if it exists, does not change even if one changes the values of $x_{k}$ for $k\,\,{\leq}\,\,N$.
    Indeed, it is not important for such `limit' questions whether $x_{k}$ is defined for $k\,\,{\leq}\,\,N$.

        For example, one would like to say that ${\displaystyle \lim_{k \,{\rightarrow}\, {\infty}} \frac{k}{k-2} \,=\, 1}$.
    Indeed, one computes that
        \begin{displaymath}
        \frac{k}{k-2} \,=\,\frac{k}{k\,(1-(2/k))} \,=\,  \frac{1}{1-(2/k)};
        \end{displaymath}
    and when $k$ is large then $-2/k$ is almost equal to $0$, and so on.

        Unfortunately, the fraction $k/(k-2)$ is not defined for {\em all} $k$ in ${\NN}$: there is a `division-by-zero' issue when $k \,=\, 2$.
    In this case the notation ${\displaystyle \lim_{k \,{\rightarrow}\, {\infty}} k/(k-2)}$ is technically not even defined,
    since the `limit' notation assumes that one is dealing with a true sequence. One can get around this
    by introducing a true sequence ${\sigma} \,=\, (s_{1},s_{2},\,{\ldots}\,)$ by the rule
        \begin{displaymath}
        s_{k} \,=\, 
        \left\{
        \begin{array}{cl}
     {\displaystyle \frac{k}{k-2}} & \mbox{if $k \,\,{\neq}\,\, 2$} \\
                                      &                                     \\
                b                     & \mbox{if $k \,=\, 2$}
        \end{array}
        \right.
        \end{displaymath}
    where the number $b$ can be any real number, and then study the limit of this modified (but `legal') sequence.

        The obvious problem with this solution is its artificiality, made evident by the fact that the choice of the value $b$ is irrelevant to the issue;
    it seems excessively fussy for what it accomplishes.

    The next definition uses the results of Theorem~\Ref{ThmC40.80} as motivation
    to extend the definition of `$\lim_{k \,{\rightarrow}\, {\infty}} x_{k} \,=\, L$.

\V

            \subsection{\small{\bf Extended Definition} of the Limit of a Real Sequence}
            \label{DefC10.40}\IndBD{sequences}{extended definition of limits}

\V

    Suppose that ${\xi}$ is a real-valued function whose domain is a subset of ${\NN}$ of the form ${\NN}{\setminus}S$,
    where $S$ is a finite subset of ${\NN}$. Let $x_{k} \,=\, {\xi}(k)$ for all $k$ in ${\NN}{\setminus}S$,
    and let $L$ be an extended real number.

    One says that $\lim_{k \,{\rightarrow}\, {\infty}} x_{k} \,=\, L$ (in the extended sense) provided both of the following statements hold:


        \h \underline{Statement~A}\, For each $y$ in ${\RR}$ such that $y\,<\,L$,
    there are only finitely many values of the index $k$ in ${\NN}\,{\setminus}\,S$ such that $x_{k}\,<\,y$.

        \h \underline{Statement~B} For each $z$ in ${\RR}$ such that $z\,>\,L$,
    there are only finitely many values of the index $k$ in ${\NN}\,{\setminus}\,S$ such that $x_{k}\,>\,z$.

\V

         \underline{Note} It is clear that this definition is equivalent to the usual one
    in the special case $S \,=\, {\emptyset}$ so that ${\NN}\,{\setminus}\,S \,=\, {\NN}$.
    Thus there is no harm in using it in that case as well.


                \section{{\bf Basic Theory of Limits of Sequences}}
                \label{SectC20}\IndB{ZZ Sections}{\Ref{SectC20} Basic Theory of Limits of Sequences}

\V

            \subsection{\small{\bf Theorem}}
            \label{ThmC20.10A}

\V

        Let ${\xi} \,=\, (x_{1},x_{2},\,{\ldots}\,x_{k},\,{\ldots}\,)$ be a sequence of real numbers.

\V

        (a) ({\bf Uniqueness-of-Limits Principle})\,\IndBD{sequences}{uniqueness-of-limits-principle for sequences}
    If the sequence ${\xi}$ has a limit, then its limit is unique. More precisely, suppose that $L$ is an extended real number such that the equation
    $\lim_{k \,{\rightarrow}\, {\infty}}  x_{k}\,=\, L$ is true. If $L'$ is an extended real number such that $L' \,\,{\neq}\,\, L$,
    then the equation $\lim_{k \,{\rightarrow}\, {\infty}} x_{k} \,=\, L'$ is {\em not}~true.


\V

        (b) Suppose that ${\xi}$ has limit $L$. Then:

        \h (i)\,\, If $L$ is a real number, so that ${\xi}$ is convergent, then ${\xi}$ is a bounded sequence.

        \h (ii)\, If $L \,=\, +{\infty}$, so that ${\xi}$ diverges to~$+{\infty}$,
    then the sequence ${\xi}$ is bounded below but unbounded above.

        \h (iii) If $L \,=\, -{\infty}$, so that ${\xi}$ diverges to~$-{\infty}$,
    then ${\xi}$ is unbounded below but bounded above.

\V

        (c) Suppose that ${\xi} \,=\, (x_{1},\,{\ldots}\,x_{k},\,{\ldots}\,)$ is
a sequence of real numbers with limit~$L$.
    Let ${\zeta} \,=\, (z_{1},z_{2},\,{\ldots}\,)$ be a subsequence of ${\xi}$.
    Then $\lim_{j \,{\rightarrow}\, {\infty}} z_{j} \,=\, \lim_{k \,{\rightarrow}\, {\infty}} x_{k} \,=\, L$.

        Alternate Phrasing: A sufficient condition for a subsequence ${\zeta}$ of a sequence ${\xi}$ to have the limit $L$ is that ${\xi}$ have the limit~$L$.

\V

        {\bf Proof}

\V

        (a) Suppose that $\lim_{k \,{\rightarrow}\, {\infty}} x_{k} \,=\, L$, and let $L'$ be any extended real number such that $L' \,\,{\neq}\,\, L$.
    Assume first that $L'\,<\,L$, and let $c$ be a real number such that $L'\,<\,c\,<\,L$.
    By Statement~A in Theorem~\Ref{ThmC40.80}, there are only finitely many values of the index $k$ such that $x_{k}\,<\,c$.
    In other words, $x_{k}\,\,{\geq}\,\,c\,>\,L'$ for all but finitely many values of $k$.
    Now Statement~B of the same theorem implies that the sequence ${\xi}$ does {\em not} have $L'$ as a limit.
    By a similar argument one proves that $L'\,>\,L$ implies that ${\xi}$ does not have $L'$ as a limit. The desired result follows.
    

\V

        (b) (i) Let $y$ be a real number such that $y\,<\,L$; for example, let $y \,=\, L-1$.
    (This subtraction makes sense because, by hypothesis, $L$ is a real number.)
    By Statement~A in Theorem~\Ref{ThmC40.80} one has $y\,<\,x_{k}$ for all but finitely many indices~$k$.
    By Part~(b) of Corollary~\Ref{CorB20.195C}, with $f \,=\, {\xi}:{\NN} \,{\rightarrow}\, {\RR}$ in that result,
    it then follows that the sequence ${\xi}$ is bounded below. A similar argument shows that ${\xi}$ is also bounded above, and thus ${\xi}$ is a bounded sequence, as claimed.


        \h (ii) Let $y$ be any real number; of course one automatically has $y\,<\,+{\infty}$.
    Then, by Statement~A of Theorem~\Ref{ThmC40.80} again, it follows that $y\,<\,x_{k}$ for all but finitely many indices~$k$;
    therefore, by Part~(b) of Corollary~\Ref{CorB20.195C} again, the sequence ${\xi}$ is bounded below. Furthermore, since $y$ can be any real number,
    and ${\NN}$ is an infinite set, it follows that the sequence ${\xi}$ is {\em un}bounded above.

        \h (iii) This follows much as in (ii) above.

\V

        (c)  Let $A \,=\, \{k_{1}, k_{2},\,{\ldots}\,k_{m},\,{\ldots}\,\}$ be an infinite subset of ${\NN}$, with $k_{1}\,<\,k_{2}\,<\,\,{\ldots}\,< k_{m}\,<\, \,{\ldots}\,$,
    such that ${\zeta}$ is the subsequence of ${\xi}$ determined by~$A$; see Definition~\Ref{DefA40.40}. Suppose that $y$ is any real number such that $y\,<\,L$. Then since $L$ is the limit of the sequence ${\xi}$,
    one has $x_{k}\,<\,y$ for at most finitely many indices $k$. It follows that $x_{k_{m}}\,<\,y$ for only finitely many~$m$.
     A similar proof shows that if $z$ is a real number such that $z\,>\,L$, then there are at most finitely many $m$ such that $x_{k_{m}}\,>\,z$.
        It now follows from Theorem~\Ref{ThmC40.80} that $L$ is also the limit of the subsequence~${\zeta}$, as claimed.

\V

            \subsection{\small{\bf Remark}}
            \label{RemrkC20.11B}

            We have already seen examples in which a subsequence ${\zeta}$ of a given ${\xi}$ has limit~$L$, but ${\xi}$ does not; that is,
    the  `sufficient condition' described above in the Alternate Phrasing of Part~(c) of Theorem~\Ref{ThmC20.10A} is not necessary for ${\zeta}$ to have limit~$L$.
    However, it is a simple exercise to show that if ${\zeta}$ is a {\em tail} of the sequence ${\xi}$ (see Example~\Ref{ExampA40.41A}), then this condition is also necessary. % EXERCISE

\VV

            \subsection{\small{\bf Example}}
            \label{ExampC20.11A}

\V%\\\\

        In Example~\Ref{ExampC10.20}~(4) it was proved, directly from the definition of `convergence', that the sequence ${\xi} \,=\, (2, -2, 2, \,{\ldots}\,(-1)^{k-1}\,2,\,{\ldots}\,)$ is divergent.
    In light of Part~(c) of the preceding theorem, this proof can be made simpler and more intuitive; actually, even more can be proved.
    Indeed, the odd-order terms of ${\xi}$ form a constant sequence with limit~$2$, while the even-order terms form a constant sequence with limit~$-2$.
    Since these subsequences do not have the same limit, the original sequence ${\xi}$ cannot have a limit, finite or infinite. In particular, ${\xi}$ is divergent, as previously claimed.

        Similarly, the preceding theorem can provide a simpler proof of the result in Example~\Ref{ExampC10.20}~(5) given earlier.
    Indeed, the sequence ${\zeta} \,=\, (1^{2}, 2^{2}, 3^{2},\,{\ldots}\,k^{2},\,{\ldots}\,)$ is clearly unbounded,
    so by Theorem~\Ref{ThmC20.10A}~(b)~(i) the sequence ${\zeta}$ is not convergent. Indeed, it is easy to prove that this sequence has $+{\infty}$ as its limit.

\VV

            \subsection{\small{\bf Remark}}
            \label{RemrkC20.11B}

\V

        The converses of Statements (i), (ii) and (iii) in Part~(b) of the previous theorem are not true.
    More precisely, knowing that a sequence is bounded on one or both sides does not imply that it has a limit. Examples illustrating this fact have already been presented.


\VV


            \subsection{\small{\bf Theorem}}
            \label{ThmC20.10B}

\V

        Recall from Chapter~\Ref{ChaptA} that if ${\zeta} \,=\, (z_{1}, z_{2},\,{\ldots}\,z_{k},\,{\ldots}\,)$ is a sequence in a nonempty set~$X$,
    then $S_{{\zeta}}$ denotes the `term-set' of ${\zeta}$; that is, the range $\{z_{1}, z_{2}, \,{\ldots}\,z_{k}, \,{\ldots}\,\}$
    of the function ${\zeta}:{\NN} \,{\rightarrow}\, X$.

\V

        (a) {\bf Monotonic-sequences Principle}\IndBD{sequences}{monotonic-sequences principle}
    Suppose that ${\alpha} \,=\, (a_{1},a_{2},\,{\ldots}\,a_{k},\,{\ldots}\,)$  is a real sequence which is monotonic up;
    that is, $a_{k+1}\,\,{\geq}\,\,a_{k}$ for all $k$ in ${\NN}$ (see Definition~\Ref{DefB20.180}).
    Then $\lim_{k \,{\rightarrow}\, {\infty}} a_{k}$ exists, and equals the supremum of the term-set $S_{{\alpha}}$.
    In particular, this limit is finite (and the sequence ${\alpha}$ is convergent)
    if ${\alpha}$ is bounded above, while the limit is $+{\infty}$ if ${\alpha}$ is unbounded above.

        Likewise, suppose that ${\beta} \,=\, (b_{1},b_{2},\,{\ldots}\,b_{k},\,{\ldots}\,)$
    is a real sequence which is monotonic down. Then ${\beta}$ has a limit. This limit is finite,
    and ${\beta}$ is convergent to it, if ${\beta}$ is bounded below; while the limit is  $-{\infty}$ if ${\beta}$ in unbounded below.

\V

        (b) {\bf Squeeze Theorem for Sequences}\IndBD{sequences}{squeeze theorem for sequences}
    Suppose that ${\alpha} \,=\, (a_{1},a_{2},\,{\ldots}\,)$ and ${\beta} \,=\, (b_{1},b_{2},\,{\ldots}\,)$
    are real sequences, each having the same limit~$L$. Suppose further that ${\xi} \,=\, (x_{1},x_{2},\,{\ldots}\,)$ is a real sequence such that 
    $x_{k}{\in}\mbox{Seg}\,[a_{k},b_{k}]$ for all but finitely many indices~$k$. Then ${\xi}$ also has the same limit~$L$.

\V

        (c) Let ${\xi} \,=\, (x_{1},x_{2},\,{\ldots}\,)$ be a real sequence, and let $S_{{\xi}}$ be the term-set of~${\xi}$.
    Assume that the sequence ${\xi}$ has limit~$L$. Then
        \begin{displaymath}
        {\inf}\,S_{{\xi}}\,\,{\leq}\,\,L\,\,{\leq}\,\,{\sup}\,S_{{\xi}} \h ({\ast})
        \end{displaymath}

        More generally, let $m$ be a lower bound for the sequence~${\xi}$, and let $M$ be an upper bound for~${\xi}$. Then
        \begin{displaymath}
        m\,\,{\leq}\,\,L\,\,{\leq}\,\,M \h ({\ast}{\ast})
        \end{displaymath}

\V

        {\bf Proof}
\V

        (a) Let $L \,=\, {\sup}\,S_{{\alpha}}$, and let $y$ be any real number such that $y\,<\,L$.
    By the Approximation Property for the supremum, there must exist an element $u$ of the set $S_{{\alpha}}$ such that $y\,<\,u\,\,{\leq}\,\,L$.
   By the definition of the set $S_{{\alpha}}$, this implies that there exists an index $m$ such that $u \,=\, x_{m}$ and thus $y\,<\,x_{m}\,\,{\leq}\,\,L$.
    By the `monotonic-up' hypothesis, this in turn implies that if $k\,\,{\geq}\,\,m$ then $y\,<\,x_{m}\,\,{\leq}\,\,x_{k}\,\,{\leq}\,\,L$.
    That is, Statement~A of Theorem~\Ref{ThmC40.80} is satisfied. Since $L$ is an upper bound for the set $S_{{\alpha}}$,
    it follows that if $z$ is a number such that $z\,>\,L$, then there are no indices $k$ such that $x_{k}\,>\,z$, and thus Statement~B of Theorem~\Ref{ThmC40.80} also holds,
    and thus $\lim_{k \,{\rightarrow}\, {\infty}} x_{k} \,=\, L$, as required.

        The proof in the case of monotonic-down sequences is similar.

\V

        (b) Let $y$ be any number such that $y\,<\,L$. Then by Theorem~\Ref{ThmC40.80}
    (and the hypothesis that the sequences ${\alpha}$ and ${\beta}$ both have limit~$L$),
    it follows that $a_{k}, b_{k}\,\,{\geq}\,\,y$ for all but finitely many indices~$k$.
    It follows that, for all but finitely many $k$, every element $x$ of $\mbox{Seg}\,[a_{k}, b_{k}]$ satisfies $x\,\,{\geq}\,\,y$.
    In particular, $x_{k}\,\,{\geq}\,\,y$ for all but finitely many indices~$k$. In a similar manner, if $z\,>\,L$ then one has $x_{k}\,\,{\leq}\,\,z$ for all but finite many indices~$k$.
    It follows from Theorem~\Ref{ThmC40.80} again that the sequence ${\xi}$ also has limit~$L$, as required.

\V

        (c) The simple proof is left as an exercise.


\VV

            \subsection{\small{\bf Remarks}}
            \label{RemrkC20.10BB}

\V

\hspace*{\parindent}(1) In Theorem~\Ref{ThmC20.40}, the `Nested-sequences/Nested-segments Theorem',
    Claim~2 of that theorem was phrased in terms of suprema, and proved using the Eudoxus Principle.
    The Monotonic-sequences Principle proved above allows an alternate approach to that result.
    Indeed, in Theorem~\Ref{ThmC20.40} one proves that $J \,=\, \mbox{Seg}\,[A,B]$,
    where $A \,=\, {\sup}\,S_{{\alpha}}$ and $B \,=\, {\inf}\,S_{{\beta}}$ (see Theorem~\Ref{ThmC20.40} for the definition of these quantities).
    The Monotonic-sequences Principle allow one to state that $A \,=\, \lim_{k \,{\rightarrow}\, {\infty}} a_{k}$,
    and $B \,=\, \lim_{k \,{\rightarrow}\, {\infty}} b_{k}$, which is the usual formulation of the Nested-intervals Theorem.
    A similar result holds for the Nested-segments Theorem.

\V

        (2) In Part (c) of the preceding theorem it is clear that Inequality~$({\ast}{\ast})$ never provides an estimate of $L$
    which is more precise than that given by Inequality~$({\ast})$. Thus, it may seem pointless to even mention Inequality~$({\ast}{\ast})$.
    In practical cases, however, computing the exact values of ${\inf}\,{\xi}$ and ${\sup}\,{\xi}$ may be very difficult,
    while determining values of $m$ and $M$ which are sufficiently accurate for the needs of the given situation may be easy.
    We encounter several examples of this phenomenon later.

\V

        (3) The word `squeeze' which appears in the name of Part~(b) of the preceding theorem comes from the fact that the hypotheis
    $x_{k}{\in}\mbox[a_{k},b_{k}]$ means that the number $x_{k}$ is `squeezed' between the numbers $a_{k}$ and $b_{k}$.

\V


        (4) In many texts the theorem called the `Squeeze Theorem for Sequences' would have an stronger hypothesis such as that $a_{k}\,\,{\leq}\,\,b_{k}$ for each index~$k$.
    It is a simple exercise to show that the resulting theorem is equivalent to the one obtained above. %EXERCISE

\VV

            \subsection{\small{\bf Theorem}}
            \label{ThmC20.25}

\V

        Suppose that $S$ be a nonempty set of real numbers. Then there exists a sequence
    ${\xi} \,=\, (x_{1},x_{2},\,{\ldots}\,x_{k},\,{\ldots}\,)$ of numbers in~$S$ such that
    ${\xi}$ is monotonic up and $\lim_{k \,{\rightarrow}\, {\infty}} x_{k} \,=\, {\sup}\,S$.
    More precisely, if ${\sup}\,S{\in}S$, so that $S$ has a maximum element, then ${\xi}$ can be chosen to be a constant sequence.
    If, instead, ${\sup}\,S\not \in S$, then ${\xi}$ can be chosen to be strictly increasing.
    Likewise, there exists a monotonic-down sequence ${\zeta} \,=\, (z_{1}, z_{2},\,{\ldots}\,z_{k},\,{\ldots}\,)$
    of numbers in $S$ such that ${\inf}\,S \,=\, \lim_{k \,{\rightarrow}\, {\infty}} z_{k}$.
    This sequence can be chosen to be constant if ${\inf}\,S{\in}S$, or chosen to be strictly decreasing if ${\inf}\,S\not \in S$.

\V

        The straight-forward proof is left as an exercise. %EXERCISE


\VV

            \subsection{\small{\bf Theorem}}
            \label{ThmC20.30}

        Let ${\xi} \,=\, (x_{1},x_{2},\,{\ldots}\,)$ be a sequence of real numbers,
    and suppose that there are are subsequences ${\zeta} \,=\, (z_{1},z_{2},\,{\ldots}\,)$ and ${\tau} \,=\, (t_{1},t_{2},\,{\ldots}\,)$ of ${\xi}$ which have limits $L_{1}$ and $L_{2}$, respectively, with $L_{1} \,\,{\neq}\,\, L_{2}$.
    Then the sequence ${\xi}$ does not have a limit.

\V

        \underline{Proof}: Suppose, to the contrary, that $\lim_{k \,{\rightarrow}\, {\infty}} x_{k} \,=\, L$ for some  extended real~$L$.
    Since $L_{1} \,\,{\neq}\,\, L_{2}$, then at least one of the quantities $L_{1}$ or $L_{2}$ must not equal $L$.
    Thus at least one of the subsequences ${\zeta}$ or ${\tau}$ must fail to have the limit~$L$.
    This would contradict Part~(c) of Theorem~\Ref{ThmC20.10A}.

\VV

        The next several results apply mainly to questions of convergence. 

\V


            \subsection{\small{\bf Theorem}}
            \label{ThmC20.10C}

\V

\hspace*{\parindent}(a) Suppose that ${\xi} \,=\, (x_{1},\,{\ldots}\,x_{k},\,{\ldots}\,)$
    is a real sequence, and $L$ is a real number. Let $z_{k} \,=\, x_{k}-L$ for each $k$ in ${\RR}$. Then the following statements are equivalent:

        \h (i) $\lim_{k \,{\rightarrow}\, {\infty}} x_{k} \,=\, L$.

        \h (ii) $\lim_{k \,{\rightarrow}\, {\infty}} z_{k} \,=\, 0$.

        \h (iii) $\lim_{k \,{\rightarrow}\, {\infty}} |z_{k}| \,=\, 0$.

\V

        (b) Suppose that $(x_{1},x_{2},\,{\ldots}\,x_{k}, \,{\ldots}\,)$ is
a real sequence such that $\lim_{k \,{\rightarrow}\, {\infty}} x_{k} \,=\, 0$.
    Let ${\zeta} \,=\, (z_{1},z_{2},\,{\ldots}\,)$ be a bounded real sequence, and define $t_{k} \,=\, z_{k}u_{k}$ for each $k$.
    Then the sequence ${\tau} \,=\, (t_{1},t_{2},\,{\ldots}\,t_{k},\,{\ldots}\,)$ also converges to $0$.

\V

        {\bf Proof}

\V

        (a) Suppose that Statement (i) holds; that is, $\lim_{k \,{\rightarrow}\, {\infty}} x_{k} \,=\, L$.
    If ${\varepsilon}\,>\,0$, then there is $B$ in ${\RR}$ such that $k\,\,{\geq}\,\,B$ implies that $|x_{k}-L|\,<\,{\varepsilon}$.
    Thus, since $z_{k}-0 \,=\, z_{k} \,=\, x_{k}-L$, it follows that $k\,\,{\geq}\,\,B$ implies $|z_{k}-0| \,=\, |x_{k}-L|\,<\,{\varepsilon}$.
    Thus, $\lim_{k \,{\rightarrow}\, {\infty}} z_{k} \,=\, 0$.
    That is, Statement~(i) implies Statement~(ii).

        Now suppose that Statement~(ii) holds, and let ${\varepsilon}\,>\,0$ be given.
    Then there exists $B$ in ${\RR}$ such that if $k\,\,{\geq}\,\,B$ then $|z_{k}| \,=\, |z_{k}-0|,<\,{\varepsilon}$.
    But $||z_{k}|-0| \,=\, ||z_{k}|| \,=\, |z_{k}|$, so if $k\,\,{\geq}\,\,B$ then $||z_{k}|-0|\,\,{\leq}\,\,{\varepsilon}$.
    Thus, $\lim_{k \,{\rightarrow}\, {\infty}} |z_{k}| \,=\, 0$.
    That is, Statement~(ii) implies Statement~(iii).

        Finally, suppose that Statement~(iii) holds, and let ${\varepsilon}\,>\,0$ be given.
    Then there exists $B$ in ${\RR}$ such that if $k\,\,{\geq}\,\,B$ then $||z_{k}|-0|\,<\,{\varepsilon}$;
    that is, $|z_{k}|\,<\,{\varepsilon}$, which can be written $|x_{k}-L|\,<\,{\varepsilon}$.
    This means that $\lim_{k \,{\rightarrow}\, {\infty}} x_{k} \,=\, L$, so that Statement~(iii) implies Statement~(i).

\V
        (b) Since ${\zeta}$ is bounded, there must exist a real number $M\,>\,0$ such that $|z_{k}|\,\,{\leq}\,\,M$ for all $k$.
    Let ${\varepsilon}\,>\,0$ be given.
    Then since ${\xi}$ converges to $0$, there must exist $B$ in ${\RR}$ such that if $k\,\,{\geq}\,\,B$ then $|x_{k}|\,<\,{\varepsilon}/M$.
    Then for $k\,\,{\geq}\,\,B$ one has
        \begin{displaymath} |z_{k}x_{k}|\,\,{\leq}\,\,M\,|x_{k}|\,\,{\leq}\,\,M{\cdot}\frac{{\varepsilon}}{M}
     \,=\, {\varepsilon}.
        \end{displaymath}
    Thus, the sequence ${\tau}$ converges to $0$, as claimed.

\VV

        The following simple corollary is of unexpected importance in the theory.

\V

        \subsection{\small{{\bf Corollary}}}
        \label{CorC20.15}



        \hspace*{\parindent}(a) Let $r$ be a number in the interval $[0,1)$,
    and let ${\rho}$ denote the geometric sequence with initial term $A \,=\, 1$ and common ratio~$r$;
    that is, ${\rho} \,=\, (1,r,r^{2},\,{\ldots}\,r^{k},\,{\ldots}\,)$ (see Example~\Ref{ExampA40.30}~(4)).
    Then ${\rho}$ is convergent, and its limit is~$0$. In symbols:
        \begin{displaymath}
        \lim_{k \,{\rightarrow}\, {\infty}} r^{k-1} \,=\, 0 \mbox{ if $0\,<\,r\,<\,1$};
        \end{displaymath}
    equivalently,
        \begin{displaymath}
        \lim_{k \,{\rightarrow}\, {\infty}} r^{k} \,=\, 0 \mbox{ if $0\,<\,r\,<\,1$};
        \end{displaymath}



\V

        (b) More generally,  let ${\xi} \,=\, (x_{1},x_{2},\,{\ldots}\,)$ be a of sequence of real numbers such that there exist a number $r$ such that $0\,\,{\leq}\,\,r\,<\,1$, 
    and a number $M\,\,{\geq}\,\,0$, such that $|x_{k}|\,\,{\leq}\,\,M r^{k}$ for all sufficiently large $k$.
    Then the sequence ${\xi}$ is convergent, and $\lim_{k \,{\rightarrow}\, {\infty}} x_{k} \,=\, 0$.

\V

        (c) The conclusion of Part~(a) remains true if the hypothesis $0\,<\,r\,<\,1$ is replaced by the weaker hypothesis $|r|\,<\,1$.

        \underline{Proof}

\V

        (a) The result is obviously true if $r \,=\, 0$, so assume now that $0\,<\,r\,<\,1$. It is clear, from the `order' properties of real numbers,
    that ${\rho}$ is a strictly decreasing sequence which is bounded below by~$0$. It follows from Part~(a) of Theorem~\Ref{ThmC20.10B},
    i.e., the Monotonic-sequences Principle, that the sequence ${\rho}$ converges to the number $L \,=\, {\inf}\,\{1, r, r^{2}, \,{\ldots}\,r^{k}, \,{\ldots}\,\}$.
    It is also clear that $0\,\,{\leq}\,\,L\,\,{\leq}\,\,r^{k-1}$ for each index~$k$; equivalently, $0\,\,{\leq}\,\,L\,\,{\leq}\,\,r^{k}$ for each index~$k$.

        \underline{Claim} One has $L \,=\, 0$.

        \underline{Proof of Claim} Since $0\,<\,r\,<\,1$, so it follows that $1/r\,>\,1$.
    If the claim were incorrect, then it would follow that $L\,>\,0$ as well, and thus $L/r\,>\,L$.
    Since $L$ is the {\em greatest} lower bound of the set $\{r,r^{2},\,{\ldots}\,\}$,
    it follows that $L/r$ cannot be a lower bound for this set, and thus there must exist $k$ in ${\NN}$ such that $r^{k}\,<\,L/r$.
    Multiply both of the terms of this inequality by the positive number $r$ to conclude that $r^{k+1}\,<\,L$,
    which contradicts the fact that $L$ is a lower bound for the set $\{r,r^{2},\,{\ldots}\,\}$. Thus, $L\,>\,0$ is impossible, and the claim follows.


        \underline{Note} An alternative proof is given in Example~\Ref{ExampC60.50} below.

\V

        (b) This statement follows easily from Part~(a) together with Part~(b) of the preceding theorem.

\V

        (c) This follows from Part~(b) by replacing $r$ in that statement by $|r|$, and letting $M \,=\, 1$.


%% POTENTIAL EXERCISE: OUTLINE THE STANDARD PROOF WHICH USES BERNOULLI'S
%% INEQUALITY; NAMELY (1+x)^{k}\,\,{\geq}\,\,1+kx 
\V

            \subsection{\small{\bf Examples}} 
            \label{ExampC20.20}

\hspace*{\parindent}(1) Let ${\xi} \,=\, (x_{1},x_{2},\,{\ldots}\,)$ be the sequence given by the rule
        \begin{displaymath}
        x_{k} \,=\, \frac{1}{1^{2}} + \frac{1}{2^{2}} + \frac{1}{3^{2}} + \,{\ldots}\, + \frac{1}{k^{2}} \mbox{ for each $k$ in ${\NN}$}.
        \end{displaymath}
    Note that for each $k$ in ${\NN}$ one has $x_{k+1} \,=\, x_{k} + 1/(k+1)^{2}$. One computes the first few terms easily:
        \begin{displaymath}
        x_{1} \,=\, 1;\, x_{2} \,=\, x_{1} + \frac{1}{2^{2}} \,=\, \frac{5}{4};\, x_{3} \,=\, x_{2} + \frac{1}{3^{2}} \,=\, \frac{49}{36};\,
        x_{4} \,=\, \frac{49}{36} + \frac{1}{4^{2}} \,=\, \frac{820}{576}; \,{\ldots}\,
        \end{displaymath}
    It is clear that the sequence ${\xi}$ is strictly increasing. Less obvious, but true, is that this sequence is bounded above.
    Indeed, note that for each index $k\,\,{\geq}\,\,2$ one easily computes that
        \begin{displaymath}
        \frac{1}{k^{2}}\,<\,\frac{1}{k(k-1)} \,=\, \frac{1}{k-1} - \frac{1}{k},
        \end{displaymath}
    and thus
        \begin{displaymath}
        x_{k}\,<\, 1+\left(\frac{1}{1} - \frac{1}{2}\right) + \left(\frac{1}{2} - \frac{1}{3}\right)
    +\,{\ldots}\, + \left(\frac{1}{k-1} - \frac{1}{k}\right)
        \end{displaymath}
    In the sum of the terms inside the parentheses on the right, all the terms cancels out with the exception of the terms $1/1$ and $-1/k$, leaving one with
        \begin{displaymath}
        x_{k}\,\,{\leq}\,\,1+1-\frac{1}{k} \,=\, 2-\frac{1}{k}.
        \end{displaymath}
    In particular, $x_{k}\,<\,2-1/k\,<\,2$ for all $k\,\,{\geq}\,\,2$. Of course the same holds for $k \,=\, 1$, so the sequence ${\xi}$ is bounded above, by~$2$.
    It follows from the Monotonic-squences Principle (see Theorem~\Ref{ThmC20.10B}~(a) above) that the sequence ${\xi}$ is convergent.

        {\bf Remark}\, This is a good example of proving that a sequence is convergent, without having even a guess in advance the exact value of its limit~$L$.
    The argument given above may appear to suggest that the desired limit is $L \,=\, 2$, but that is incorrect. Indeed, the correct value turns out to be $L \,=\, {\pi}^{2}/6$,
    where ${\pi}$ denotes the famous constant from high-school geometry, with value ${\pi} \,=\, 3.14159\,{\ldots}\,$, so that $L$ is approximately $1.644$, which is much less than~$2$.
    The proof that this is the correct value of $L$ here is well beyond the techniques we have available at this point,
    but you can (informally) convince yourself by computing $x_{k}$ for some large values of $k$ directly from the fomula for $x_{k}$.

% EXERCISE Show the convergence if the power 2 is replaced by p>2.

\V

        (2) Suppose that ${\xi} \,=\, (x_{1},x_{2},\,{\ldots}\,)$ is given by the rule
        \begin{displaymath}
        x_{k} \,=\, 1 + \frac{1}{2} + \,{\ldots}\, + \frac{1}{k} \mbox{ for all $k$ in ${\NN}$}.
        \end{displaymath}
    At first glance this sequence appears to be very similar to that considered in Example~(1) above;
    for instance, it is also strictly increasing, and one goes from one term to the next by  adding the reciprocal of some natural number.
    However, there is a major difference, because it turns out that the sequence in the current example is {\em not} bounded above.

        To see that ${\xi}$ is not bounded above, it suffices to find a subsequence which is unbounded.
    And in fact it is easy to show that the subsequence ${\zeta} \,=\, (z_{1},z_{2},\,{\ldots}\,)$ given by $z_{j} \,=\, x_{2^{j}}$ satisfies
        \begin{displaymath}
        z_{j}\,>\,\frac{j}{2}.
        \end{displaymath}
    To see why this is true, use Mathematical Induction, and let $A$ be the set of $j$ in ${\NN}$ such that $z_{j}\,>\,j/2$.
    Clearly $1{\in}A$, since $z_{1} \,=\, x_{2} \,=\, 1+1/2\,>\,1/2$.
    Next, suppose that $j{\in}A$, so that $z_{j}\,>\,j/2$.
    Note that
        \begin{displaymath}
        z_{j+1} \,=\, x_{2^{j+1}} \,=\, x_{2^{j}} + \frac{1}{2^{j}+1} + \frac{1}{2^{j}+2} + \,{\ldots}\, \frac{1}{2^{j+1}}
        \end{displaymath}
    Clearly the smallest of the fractions appearing on the right side of the last equation is the final fraction, $1/2^{j+1}$.
    Since there exactly $2^{j}$ such fractions, their sum is at least $2^{j}/2^{j+1} \,=\, 1/2$. Also, the term $x_{2^{j}}$ appearing there is the same as $z_{j}$.
    Thus, one gets
        \begin{displaymath}
        z_{j+1}\,>\,z_{j}+\frac{1}{2}\,>\,\frac{j}{2}+\frac{1}{2} \,=\, \frac{j+1}{2}.
        \end{displaymath}
    (The induction hypothesis, that $j{\in}A$, is used in the inequality $z_{j}\,>\,j/2$.)
    Thus $A \,=\, {\NN}$, so $z_{j}\,>\,j/2$ for each $j$ in ${\NN}$, as claimed.
    It now follows from the Archimedean Principle that the subsequence ${\zeta}$ is unbounded, and thus that the original sequence ${\xi}$ is also unbounded.
    Now Theorem~\Ref{ThmC20.10A}~(c) implies that the sequence ${\xi}$ is divergent; in fact, it has the limit~$+{\infty}$.

\V


        (3) Let $R$ be a positive number, and define a sequence ${\xi} \,=\, (x_{1},x_{2},\,{\ldots}\,)$ by the rule
        \begin{displaymath}
        x_{k} \,=\, \frac{R^{k}}{k!} \mbox{ for } k \,=\, 1,2,\,{\ldots}\,
        \end{displaymath}
    Then $\lim_{k \,{\rightarrow}\, {\infty}} x_{k} \,=\, 0$. To see this, let $N$ be a positive integer such that $N\,\,{\geq}\,\,R$; then set $r \,=\, R/(N+1)$, so that $0\,<\,r\,<\,1$.
    (Such $N$ exists by the Archimedean Property.) Note that
        \begin{displaymath}
        x_{N+1} \,=\, \frac{R^{N+1}}{(N+1)!} \,=\, \left(\frac{R^{N}}{N!}\right){\cdot}\left(\frac{R}{N+1}\right) \,=\, rx_{N}.
        \end{displaymath}
    By repeating this procedure, one gets $x_{N+2}\,<\,rx_{N+1}\,<\,r^{2}x_{N}$, and so on. The general result is
        \begin{displaymath}
        x_{N+k}\,< r^{k}x_{N} \mbox{ for $k \,=\, 1,2,\,{\ldots}\,$}
        \end{displaymath}
    It now follows from Part~(b) of the preceding corollary that the sequence ${\xi}$ converges to~$0$.

        \underline{Note} The sequence ${\xi}$ in this example is not as strange as may first seem. Indeed,
    such ratios appear frequently in the so-called `Power Series' one studied in elementary calculus.

\VV

%-----------------------
\StartSkip{
        The limit process can be used to justify the standard decimal representation of real numbers.
    We carry out this description only for numbers in the interval $[0,1]$, since our applications of this representation can be reduced to this case.
    Of course, the extension to other numbers is easy to perform without any further use of limits.
    Indeed, the extension to arbitrary bases, and not just base~10, is carried out below.

\V

            \subsection{\small{\bf Definition}}
            \label{DefC20.25A}

        Let $D \,=\, {\ZZ}_{10}$ denote, as usual, the set of numbers $\{0,1,2,3,4,5,6,7,8,9\}$, thought of as the {\bf decimal digits}.
    If $x$ is a real number such that $0\,<\,x\,\,{\leq}\,\,1$, associate to~$x$ the infinite sequence $(d_{1},d_{2},\,{\ldots}\,d_{n},\,{\ldots}\,)$
    defined as follows:

        \h $d_{1}$ is the maximum of the finitely many decimal digits $d$ such that $d/10\,<\,x$.
    Note that, by the hypothesis that $x\,>\,0$, at least one such $d$ exists, namely $d \,=\, 0$.
    Also note that $(d_{1}+1)/10\,\,{\geq}\,\,x$. Indeed, if $d_{1} \,=\, 9$ then the last inequality reduces to $1\,\,{\geq}\,\,x$, which is a given fact;
    while if $d_{1}\,<\,9$, then $d_{1}+1$ is also a decimal digit, so the claimed inequality follows from the maximality of~$d_{1}$.
    Thus we have
        \begin{displaymath}
        0\,<\,x-\frac{d_{1}}{10}\,\,{\leq}\,\,\frac{d_{1}+1}{10}-\frac{d_{1}}{10}\,\,{\leq}\,\,\frac{1}{10}.
        \end{displaymath}

        \h $d_{2}$ is the largest of the finitely many decimal digits $d$ such that $d/10^{2}\,<\,x - d_{1}/10$.
    As before such $d$ exists because, by the construction of $d_{1}$, one has $x-d_{1}/10\,>\,0$.
    Likewise, $(d_{2}+1)/10^{2}\,\,{\geq}\,\,x-d_{1}/10$. It follows that
        \begin{displaymath}
        0\,<\,x - \frac{d_{1}}{10} - \frac{d_{2}}{10^{2}}\,\,{\leq}\,\,\frac{1}{10^{2}}.
        \end{displaymath}

        Continuing this way one obtains an infinite sequence ${\delta} \,=\, (d_{1}, d_{2},\,{\ldots}\,d_{n},\,{\ldots}\,)$, of decimal digits,
    with the following property:
        \begin{displaymath}
        \mbox{for each index $k$ one has } 
        0\,<\,x - \frac{d_{1}}{10} - \frac{d_{2}}{10^{2}} - \,{\ldots}\, -
        \frac{d_{k}}{10^{k}}\,\,{\leq}\,\,\frac{1}{10^{k}}.
        \end{displaymath}
}% EndSkip

\VV


            \subsection{\small{\bf Example} -- Heron's Method}
            \label{ExampC20.50A}

\V

        \underline{Preliminary Remark} As has already been mentioned, it is not hard to prove, using the Bisection Procedure, that $\sqrt{2}$ exists;
    in other words, that there exists a real number~$x$ such that $x^{2}\,=\,2$.
    As a computational tool, however, the `Bisection' approach is rather inefficient.

        There is a second procedure for computing square roots, however, that is generally much more efficient than the Bisection Procedure,
    and which uses the Nested Segments Theorem (see Theorem~\Ref{ThmC20.40}).
    It is has long been called `Heron's Method', in honor of the ancient Greek geometer Heron, from around the year~100.
    It is still widely used; see, for instance, Chapter Quote~\#2 at the start of this chapter.

    \underline{Note} In the twentieth century it was discovered that apparently the ancient Babylonians
    were also familiar with this method centuries before Heron. Because of this,  it is sometimes called the `Babylonian Method'.

        The basic idea behind Heron's method is quite simple: Let $C$ be a positive real number.
    It is clear that if $u$ is a positive number such that $u^{2}\,>\,C$, then $(C/u)$ is a number such that $(C/u)^{2}\,<\,C$.
    Indeed,
        \begin{displaymath}
        \left(\frac{C}{u}\right)^{2} \,=\, \frac{C^{2}}{u^{2}}\,<\,\frac{C^{2}}{C} \,=\, C.
        \end{displaymath}
    Likewise, if $u^{2}\,<\,C$, then $(C/u)^{2}\,>\,C$; and if $u^{2} \,=\, C$, then $(C/u) \,=\, C$, and conversely.

        In other words, {\em if} the positive number $C$ has a positive square root $\sqrt{C}$,
    then for every number $u\,>\,0$ the desired square root must lie in $\mbox{Seg}\,[u,C/u]$.
    Heron's Method consists of using this basic idea repeatedly, together with one extra ingredient: since we know that the desired square root, if it exists, must lie in this segment, it makes sense to use the midpoint of this segment as an `improved' guess of the actual value of the square root.

\VV

            \subsection{\small{\bf Theorem} (Heron's `Divide-and-Average' Method for Square Roots)}\IndBD{sequences}{Heron/Babylonian method for square roots}
            \label{ThmC20.70}

        Let $C$ and $x_{1}$ be positive real numbers, and let ${\xi} \,=\, (x_{1},x_{2},\,{\ldots}\,)$ be the sequence of positive real numbers constructed recursively from $C$ and $x_{1}$ as follows:
        \begin{equation}
        \label{EqnC.45}
        x_{k+1} \,=\, \frac{1}{2}\left(x_{k} + \frac{C}{x_{k}}\right),
        \end{equation}
    so that $x_{k+1}$ is the midpoint of $\mbox{Seg}\,[x_{k},C/x_{k}]$. For each $k$ in ${\RR}$ let $J_{k}$ denote the segment $\mbox{Seg}\,[x_{k},C/x_{k}]$.
    Then the segments $J_{1}$, $J_{2}$,\,{\ldots}\, form a nested sequence such that $\lim_{k \,{\rightarrow}\, {\infty}} |x_{k}-C/x_{k}| \,=\, 0$.
    In  addition, the sequence ${\xi} \,=\, (x_{1},x_{2},\,{\ldots}\,)$ converges to a number $L\,>\,0$ such that $L^{2} \,=\, C$.

\V

        \underline{Proof} It is clear that at each stage of the construction the number $x_{k}$ is positive.
    Indeed, $x_{1}\,>\,0$ by hypothesis. Then $C/x_{1}\,>\,0$ since the ratio of positive numbers is also positive.
    Then $x_{2}\,>\,0$ since $x_{2}$ is the average of two positive numbers. A similar argument shows that
    if $x_{k}$ has been constructed and is positive, then $x_{k+1}$, as the average of positive numbers, is positive.

        Since $x_{k+1}$ is the midpoint of the segment $\mbox{Seg}\,[x_{k}, C/x_{k}]$, it follows that $x_{k+1}$ is an element of $J_{k}$.
    Likewise, it follows that $C/x_{k+1}$ is an element of $\mbox{Seg}\,[C/x_{k},C/(C/x_{k})]$.
    But $C/(C/x_{k}) \,=\, x_{k}$, and $\mbox{Seg}\,[C/x_{k},x_{k}] \,=\, \mbox{Seg}\,[x_{k}, C/x_{k}]$ (see Part~(c) of Theorem~\Ref{ThmB20.150});
    thus it follows that $C/x_{k+1}{\in}J_{k}$ as well.
    Since $x_{k+1}$ and $C/x_{k+1}$ are both in $J_{k}$, and segments are convex sets,
    it follows that $J_{k+1} \,{\subseteq}\, J_{k}$ for each $k$ in ${\NN}$; that is, the segments form a nested sequence, as claimed.

        Next, notice that since $x_{k+1}$ is the midpoint of the segment $J_{k}$,
    its distance from any point of $J_{k}$ is at most half the length of that segment.
    In particular,
        \begin{displaymath}
        \left|x_{k+1} - \frac{C}{x_{k+1}}\right|\,\,{\leq}\,\,\frac{1}{2}\,\left|x_{k}-\frac{C}{x_{k}}\right| \mbox{ for each $k$ in ${\NN}$}.
        \end{displaymath}
    By repeated use of this last fact, one gets
        \begin{displaymath}
        \left|x_{k+1} - \frac{C}{x_{k+1}}\right|\,\,{\leq}\,\,\frac{1}{2^{k}}\,\left|x_{1}-\frac{C}{x_{1}}\right| \mbox{ for each $k$ in ${\NN}$}.
        \end{displaymath}
    It follows easily that $\lim_{k \,{\rightarrow}\, {\infty}} \left|x_{k}-C/x_{k}\right| \,=\, 0$, as claimed.

        One sees now that the Nested-Segment Theorem, as clarified by Remark~\Ref{RemrkC20.10BB}, can be applied to conclude that the intersection
    ${\bigcap}_{k=1}^{{\infty}} J_{k}$ is a singleton set $\{L\}$ such that $L \,=\, \lim_{k \,{\rightarrow}\, {\infty}} x_{k}$.

    Finally, note that since $L$ is a member of each segment $J_{k}$, it follows as above that $C/L$ is also in each $J_{k}$,
    and thus $C/L$ is in the intersection of these segments. In particular, one gets $C/L \,=\, L$, hence $L^{2} \,=\, C$, as required.


\V

        {\bf Remark} Heron's `Divide-and-Average' Method does appear in the typical textbook for elementary calculus, often as an optional topic;
    but usually it falls under the topic of `Newton's Method for Calcuating Roots',
    and is treated as an application of the derivative, and not as part of the chapter on sequences.


\V
\V



%%% 
\begin{quotation}
{\footnotesize \underline{\Note}\IndB{\notes}{on computing $\sqrt{2}$: Bisection vs Heron's Method} (on computing $\sqrt{2}$: Bisection vs Heron's Method):

        Let us try to compute the irrational number $\sqrt{2}$ in two ways: using the Bisection Method, and using the Divide-and-Average Method.
    We have already carried out ten steps of the Bisection Method on $\sqrt{2}$  using the initial values $a_{1} \,=\, 1$, $b_{1} \,=\, 3$;
    see Example~\Ref{ExampB25.90}.
%------------
\StartSkip{
        \underline{Case 1 -- The Bisection Method} We begin the method by choosing $a_{1} \,=\, 1$ and $b_{1} \,=\, 2$.
    Note that this is appropriate, since $1^{2} \,=\, 1\,<\,2$ while $2^{2} \,=\, 4\,>\,2$.
    Here is a sampling of the results for the first few values of $k$:
        \begin{displaymath}
        \begin{array}{||l|c|c|c|c|c|c||} \hline
        k & 1   & 2   & 4     & 6       & 8         & 10            \\ \hline
          &     &     &       &         &           &               \\ 
    a_{k} & 1.0 & 1.0 & 1.375 & 1.40625 & 1.4140625 & 1.414062500   \\
          &     &     &       &         &           &               \\ \hline
          &     &     &       &         &           &               \\ 
    b_{k} & 2.0 & 1.5 & 1.500 & 1.43750 & 1.4218750 & 1.416015625   \\
          &     &     &       &         &           &               \\ \hline
        \end{array}
        \end{displaymath}
}%\EndSkip
%------------

        To carry out the Divide-and-Average Method one needs a choice of a single initial value $x_{1}$,
    compared to the need for two initial choices $a_{1}$ and $b_{1}$ in the Bisection Method.
    To make the current example comparable to Example~\Ref{ExampB25.90}, let us take $x_{1} \,=\, (a_{1}+b_{1})/2$; that is, $x_{1} \,=\, 2$.
    Here are the first five results:
        \begin{displaymath}
        \begin{array}{||l|c|c|c|c|c|c||} \hline
        k & 1   & 2   & 3       & 4          & 5          \\ \hline
          &     &     &         &            &            \\ 
    x_{k} & 2.0 & 1.5 & 1.41667 & 1.41421569 & 1.41421356 \\
          &     &     &         &            &            \\ \hline
        \end{array}
        \end{displaymath}
    Compare these values with those obtained in Example~\Ref{ExampB25.90}:
        \begin{displaymath}
        a_{10} \,=\, \frac{362}{256} \,=\, 1.4140625; \h
        b_{10} \,=\, \frac{363}{256} \,=\, 1.41796875
        \end{displaymath}
    Note that the actual value of $\sqrt{2}$, to nine places, is known to be $1.414213562$.
    Thus the error in approximating $\sqrt{2}$ by $x_{5}$ above is quite a bit smaller
    than the error in approximating $\sqrt{2}$ by either $a_{10}$ or $b_{10}$ in the Bisection Method.
    That is, Divide-and-Average gets much better accuracy than Bisection with much less computational work.
    This result is typical: Given comparable starting values $a_{1}$, $b_{1}$ and $x_{1}$ for the two methods, the Divide-and-Average Method converges to the desired square root much more rapidly than the Bisection method.
    However, the Bisection Method can be applied easily to many more situations, and the rapidity of convergence is normally not an issue in theoretical discussions.
}%EndFootnotesize
\end{quotation} 
%## 

\VV



        It frequently happens that a sequence is constructed by using one formula for the terms with odd index and a different formula for terms with even index.
    The next result is often useful in such a situtation.

\V

            \subsection{\small{\bf Theorem} (The Odd/Even Limit Theorem)}
            \label{ThmC20.90}\IndBD{sequences}{odd/even limit theorem}

\V

        Let ${\xi} \,=\, \{x_{1},x_{2},\,{\ldots}\,\}$ be a sequence of real numbers, and let ${\zeta}$ and ${\tau}$ be the subsequences of ${\xi}$ formed from the odd and even terms, respectively, of ${\xi}$.
    That is,
        \begin{displaymath}
        {\zeta} \,=\, (z_{1},z_{2},z_{3},\,{\ldots}\,z_{j},\,{\ldots}\,) \,=\, (x_{1},x_{3},x_{5},\,{\ldots}\,x_{2j-1},\,{\ldots}\,);
        \end{displaymath}
    likewise,
        \begin{displaymath}
        {\tau} \,=\, (t_{1},t_{2},t_{3},\,{\ldots}\,t_{j},\,{\ldots}\,) \,=\, (x_{2},x_{4},x_{6},\,{\ldots}\,x_{2j},\,{\ldots}\,).
        \end{displaymath}
    Otherwise stated:
        \begin{displaymath}
        {\xi} \,=\, (z_{1},t_{1},z_{2},t_{2},\,{\ldots}\,).
        \end{displaymath}
    Suppose that the subsequences ${\zeta}$ and ${\tau}$ both have the same limit $L$, where $L$ is an extended real number.
    Then the original sequence ${\xi}$ also has limit~$L$. In symbols: if $\lim_{j \,{\rightarrow}\, {\infty}} z_{j} \,=\, L$
    and $\lim_{j \,{\rightarrow}\, {\infty}} t_{j} \,=\, L$, then $\lim_{k \,{\rightarrow}\, {\infty}} x_{k} \,=\, L$.

\V


\V

        {\bf Proof}: Let $y$ be any real number such that $y\,<\,L$.
    By Statement~A of Theorem~\Ref{ThmC40.80}, applied to the sequence~${\zeta}$ and its limit~$L$, one has $y\,<\,z_{j}$ for all but finitely many indices~$j$;
    that is, $y\,<\,x_{k}$ for all but finitely many {\em odd} indices~$k$.
    Likewise, the same statement, but now applied to the sequence~${\tau}$ and its limit~$L$,
    implies that $y\,<\,x_{k}$ for all but finitely many {\em even} indices~$k$. Since every index is either even or odd,
    it follows that Statement~A of that theorem also applies to~${\xi}$. A similar argument shows that ${\xi}$ satisfies Statement~B of that theorem.  
    It follows that $\lim_{k \,{\rightarrow}\, {\infty}} x_{k} \,=\, L$, as claimed.

\V

        \subsection{\small{{\bf Example}}}
        \label{ExampC20.95}

\V

        Define a sequence ${\xi} \,=\, (x_{1},x_{2},\,{\ldots}\,)$ by the rule
        \begin{displaymath}
        x_{k} \,=\, 
        \left\{
        \begin{array}{ll}
        {\displaystyle \frac{1}{2}} & \mbox{if $k$ is odd} \\
                                            &                      \\
        {\displaystyle \frac{3k^{2}+k+5}{6k^{2}+1}}  & \mbox{if $k$ is even}
        \end{array}
                \right.
        \end{displaymath}
    Let ${\zeta} \,=\, (z_{1},z_{2},\,{\ldots}\,)$ and ${\tau} \,=\, (t_{1},t_{2},\,{\ldots}\,)$ be the corresponding subsequences of ${\xi}$ formed from the terms of odd index and even index, respectively.
    That is,
        \begin{displaymath}
        z_{j} \,=\, x_{2j-1} \,=\, \frac{1}{2} \mbox{ for each $j$ in ${\NN}$}
        \end{displaymath}
    and
        \begin{displaymath}
        t_{j} \,=\, x_{2j} \,=\, \frac{3(2j)^{2}+(2j)+5}{6(2j)^{2}+1} \,=\, 
    \frac{12j^{2}+2j+5}{24j^{2}+1}
 \mbox{ for all $j$ in ${\NN}$}.
        \end{displaymath}
    It is clear by inspection that the subsequence ${\zeta}$ converges to $1/2$.
    As for the subsequence ${\tau}$, one could go through the usual calculations to  determine its limit, but there is no need to do so.
    Indeed, the careful reader will notice that the formula for $x_{k}$ when $k$ is even agrees with the formula for $x_{j}$ in Example~\Ref{ExampC10.20}~(3).
    Thus, ${\zeta}$ is a subsequence of the sequence studied in that earlier example.
    Since it was proved in that example that the sequence in question converges to $1/2$, it follows from Part~(b) of Theorem~\Ref{ThmC20.10A} that the subsequence ${\tau}$ in the current problem also converges to $1/2$.

        Since both of the subsequences ${\zeta}$ and ${\tau}$ converge to the same limit, namely $L \,=\, 1/2$,
    the `Odd/Even Limit Theorem' can be used to conclude that the sequence ${\xi}$ given here also converges to $1/2$.

\VV

        There Odd/Even Convergence Theorem works because the every natural number is either odd or even.
    The following more general result works in much the same way.

\V

            \subsection{\small{\bf Theorem} (Generalized Odd/Even Limit Theorem)}
            \label{ThmC20.90A}\IndBD{sequences}{generalized odd/even limit theorem}

\V

        Let ${\xi} \,=\, (x_{1}, x_{2},\,{\ldots}\,x_{k},\,{\ldots}\,)$ be a sequence of real numbers.
    Suppose that there exists an extended real number $L$ and a finite collection of infinite subsets $A_{1}$, $A_{2}$,\,{\ldots}\,$A_{n}$ of ${\NN}$ 
    and a finite subset $S$ of~${\NN}$, possibly empty, such that $A_{1}\,{\cup}\,A_{2}\,{\cup}\,\,{\ldots}\,\,{\cup}\,A_{n} \,=\,
    {\NN}\,{\setminus}\,S$. Let ${\xi}_{A_{1}}$, ${\xi}_{A_{2}}$,\,{\ldots}\,${\xi}_{A_{n}}$
    be the subsequences of ${\xi}$ that correspond to these infinite subsets of~${\NN}$; 
    see Definition~\Ref{DefA40.40}. If each of these subsequences has limit~$L$, then the original sequence ${\xi}$ also has limit $L$.

\V

        The simple proof is left as an exercise. %% EXERCISE

\V

        \subsection{\small{{\bf Examples}}}
        \label{ExampC20.95A}

\V

\hspace*{\parindent}(1) The Odd/Even Convergence Theorem corresponds to the case in which $n \,=\, 2$,
    $A_{1}$ is the set of odd natural numbers and $A_{2}$ is the set of even natural numbers.

\V

        (2) Let $r_{1}$, $r_{2}$ and $r_{3}$ be real numbers in the open interval $(0,1)$. Form a sequence ${\xi}$ as follows:
        \begin{displaymath}
        {\xi} \,=\, (r_{1}, r_{2}^{2}, r_{3}^{3},r_{1}^{4},r_{2}^{5}, r_{3}^{6},r_{1}^{7},\,{\ldots}\,);
        \end{displaymath}
    the pattern should be clear. Let $A_{1} \,=\, \{1, 4, 7, 10,\,{\ldots}\,\}$, $A_{2} \,=\, \{2, 5, 8, 11, \,{\ldots}\,\}$
     and $A_{3} \,=\, \{3, 6, 9, 12, \,{\ldots}\,\}$. It is easy to see that $A_{1}\,{\cup}\,A_{2}\,{\cup}\,A_{3} \,=\, {\NN}$.
    The corresponding subsequences are easy to figure out:
        \begin{displaymath}
        {\xi}_{A_{1}} \,=\, (r_{1}, r_{1}^{4}, r_{1}^{7},\,{\ldots}\,); \h
        {\xi}_{A_{2}} \,=\, (r_{2}^{2}, r_{2}^{5}, r_{2}^{8},\,{\ldots}\,); \h
        {\xi}_{A_{3}} \,=\, (r_{3}^{3}, r_{3}^{6}, r_{3}^{9},\,{\ldots}\,)
        \end{displaymath}
    Each of these sequences is a subsequence of a geometric sequence with common ratio in the open interval~$(0,1)$.
    Since each such sequence converges to~$0$ (see Corollary~\Ref{CorC20.15}), it follows
    from the preceding theorem that the original sequence ${\xi}$ also converges to~$0$.

\V

        (3) Suppose that $A_{1}$, $A_{2}$,\,{\ldots}\,$A_{n}$,\,{\ldots}\, is an {\em infinite} collection
    of infinite subsets of ${\NN}$ whose union is~${\NN}$. One might conjecture that if ${\xi}$ is a real sequence for which each subsequence ${\xi}_{A_{n}}$
    has the same limit~$L$, then ${\xi}$ itself must have limit~$L$. Alas, this conjecture would be incorrect.
    For example, let $r$ be a fixed number in the open interval $(0,1)$; if you prefer to be specific, let $r \,=\, 1/2$.
    Define a real sequence ${\xi} \,=\, (x_{1}, x_{2},\,{\ldots}\,x_{k},\,{\ldots}\,)$ as follows:

\VA

        \h For each index $k$, express $k$ in the form $k \,=\, 2^{n-1}\,(2\,m-1)$ for $n,m$ in~${\NN}$; see Example~\Ref{ExampA20.60}.

        \h Using this notation, set $x_{k} \,=\, r^{m-1}$.

\VA

\noindent  For each $n$ in ${\NN}$ let $A_{n} \,=\, \{2^{n-1}{\cdot}1, 2^{n-1}{\cdot}3, 2^{n-1}{\cdot}5, \,{\ldots}\,2^{n-1}{\cdot}(2\,m-1),\,{\ldots}\,\}$.
    It is easy to see that for each $n$ one has ${\xi}_{A_{n}} \,=\, (1, r, r^{2},\,{\ldots}\,r^{m-1},\,{\ldots}\,)$.
    This is the geometric sequence with initial term $1$ and common ratio~$r$, which is known to converge to~$0$.
    However, the subsequence ${\zeta} \,=\, (z_{1}, z_{2},\,{\ldots}\,z_{n},\,{\ldots}\,)$ of ${\xi}$
    given by $z_{n} \,=\, x_{2^{n-1}}$ is the constant sequence $(1,1,\,{\ldots}\,1,\,{\ldots}\,)$, which does {\em not} have $0$ as limit.
    

\VV


        The Monotonic-Sequence Principle (see Theorem~\Ref{ThmC20.10B}) uses the concepts of `supremum' and `infimum' to compute limits of certain sequences.
    The next result reverses the roles and shows how to use limits of sequences to compute suprema and infima.

\V


            \subsection{\small{\bf Theorem}}
            \label{ThmC50.10}

        Let $X$ be a nonempty set of real numbers. Then there exist monotonic sequences ${\alpha} \,=\, (a_{1},a_{2},\,{\ldots}\,)$ and ${\beta} \,=\, (b_{1},b_{2},\,{\ldots}\,)$ of elements in $X$ such that $\lim_{k \,{\rightarrow}\, {\infty}} a_{k} \,=\, {\inf}\,X$ and $\lim_{k \,{\rightarrow}\, {\infty}} b_{k} \,=\, {\sup}\,X$.

        More precisely, if ${\inf}\,X$ is an element of $X$ then one can take ${\alpha}$ to be a constant real sequence;
    but if ${\inf}\,X$ is {\em not} in $X$, then ${\alpha}$ can be chosen to be a strictly decreasing sequence.

        Likewise, if ${\sup}\,X$ is an element of $X$ then one can take ${\beta}$ to be a constant real sequence;
    but if ${\sup}\,X$ is {\em not} in $X$, then ${\beta}$ can be chosen to be a strictly increasing sequence.

\V

        \underline{Proof} Let $A \,=\, {\inf}\,X$ and $B \,=\, {\sup}\,X$, 
    First, suppose that $A{\in}X$. Then it is clear that all the terms of the constant sequence
    $(A, A,\,{\ldots}\,A,\,{\ldots}\,)$ are elements of $X$, and the limit of this sequence is ${\inf}\,X$, as required.

        Next, suppose that ${\inf}\,X$ is {\em not} in $X$.
    There are two cases:

        \underline{Case 1} Suppose that ${\inf}\,X \,\,{\neq}\,\, -{\infty}$, so that ${\inf}\,X$ is a real number.
    Define a sequence ${\alpha} \,=\, (a_{1},a_{2},\,{\ldots}\,)$ by the rule:
        $a_{1}$ is an element of $X$ such that ${\inf}\,X\,<\,a_{1}\,<\,({\inf}\,X) + 1$.
    (The existence of such $a_{1}$ follows from the definition of `infimum'.)
    Likewise, if $a_{1}$, \,{\ldots}$\,a_{m}$ have been defined then let $a_{m+1}$ be any element of $X$ such that ${\inf}\,X\,<\,a_{m+1}\,<\,a_{m}$ and ${\displaystyle a_{m+1}\,<\,({\inf}\,X) + \frac{1}{m}}$.
    It is clear that ${\alpha} \,=\, (a_{1},a_{2},\,{\ldots}\,)$ is a strictly decreasing sequence of reals such that ${\inf}\,X\,<\,a_{m}\,<\,({\inf}\,X) + 1/m$ for each $m$ in ${\NN}$.
    Now apply the Squeeze Property to conclude that ${\inf}\,X \,=\, \lim_{k \,{\rightarrow}\, {\infty}} a_{k}$.

        \underline{Case 2} Suppose that ${\inf}\,X \,=\, -{\infty}$.
    Let $a_{1}$ be an element of $X$ such that $a_{1}\,<\,-1$.
    If $a_{1}$,\,{\ldots}\,$a_{m}$ have been defined, let $a_{m+1}$ be any element of $X$ such that $a_{m+1}\,<\,\min\,\{a_{m},-m\}$.
    Then it is clear that ${\alpha} \,=\, (a_{1},a_{2},\,{\ldots}\,)$ is strictly decreasing and $\lim_{k \,{\rightarrow}\, {\infty}} a_{k} \,=\, -{\infty} \,=\, {\inf}\,X$, as required.

        The analysis of ${\sup}\,X$ is similar, and is left to the reader.

\VV



                \section{{\bf Computational Rules for Limits of Real Sequences}}
                \label{SectC60}\IndB{ZZ Sections}{\Ref{SectC60} Limit Calculations}


        {\bf Remark} The preceding results are mainly of a theoretical nature. In contrast,  the results in this section are more practical and computational.
    Some of them are already probably already familiar from elementary calculus; but several are almost certainly not.

        Because the algebraic rules for real numbers differ in important ways from those for extended real numbers,
    in this section we separate the results for `convergence' from those involving `divergence to one of the infinities'.
    In particular, we often do the convergence proofs directly from Definition~\Ref{DefC10.10}, even if a slicker proof is available:
    it is important that one have experience using the definitions.

\V

        The next two theorems involve convergent sequences. The first theorem involves properties of a single convergent sequence,
    while the second involves pairs of convergent sequences.

\VV

            \subsection{\small{\bf Theorem}}
            \label{ThmC60.10}

        Suppose that ${\xi} \,=\, (x_{1},x_{2},\,{\ldots}\,x_{k}\,{\ldots}\,)$ is a convergent sequence of real numbers.
    Let $L \,=\, \lim_{k \,{\rightarrow}\, {\infty}} x_{k}$. Then:

\V

        (a) (The Constant-Factor Rule for Convergent Real Sequences)\IndC{sequences}{algebraic rules for convergent sequences}{constant-factor rule}
    One has $\lim_{k \,{\rightarrow}\, {\infty}} (c\,x_{k}) \,=\, c\,L$ for every real number~$c$.

\V

        (b) (The Absolute-Value Rule for Convergent Real Sequences)\IndC{sequences}{algebraic rules for convergent sequences}{absolute-value rule} 
    One has $\lim_{k \,{\rightarrow}\, {\infty}} |x_{k}| \,=\, |L|$.

\V


        \underline{Proof} 

\V

        (a) This proof is left as an easy exercise.

\V

        (b) First note that by Inequality~\Ref{IneqB.35C}, the `Reverse Triangle Inequality', one can write
        \begin{equation} 
        \label{IneqC.60}
        0\,\,{\leq}\,\,||x_{k}|-|L||\,\,{\leq}\,\,|x_{k}-L| \mbox{ for all $k$ in ${\NN}$}.
        \end{equation}
    Let ${\varepsilon}\,>\,0$ be given. Then by the convergence hypothesis, there exists $N$
    so that if $k\,\,{\geq}\,\,N$ then $|x_{k}-L|\,<\,{\varepsilon}$. Combine this with Inequality~\Ref{IneqC.60} above
    to get that $||x_{k}|-|L||\,<\,{\varepsilon}$ when $k\,\,{\geq}\,\,N$. The desired result follows.


\VV

            \subsection{\small{\bf Theorem}}
            \label{ThmC60.30}

        Suppose that ${\alpha} \,=\, (a_{1},a_{2},\,{\ldots}\,)$ and ${\beta} \,=\, (b_{1},b_{2},\,{\ldots}\,)$ are convergent sequences of real numbers.
    Let $A \,=\, \lim_{k \,{\rightarrow}\, {\infty}} a_{k}$ and $B \,=\, \lim_{k \,{\rightarrow}\, {\infty}} b_{k}$.


\V

        (a) (The Sum and Difference Rules for Convergent Real Sequences)\IndC{sequences}{algebraic rules for convergent sequences}{sum and difference rules} Let ${\sigma} \,=\,(s_{1},s_{2},\,{\ldots}\,)$, where $s_{k} \,=\,a_{k}+b_{k}$ for each $k$ in ${\NN}$;
    that is, ${\sigma}$ is the sequence of sums of corresponding terms from ${\alpha}$ and ${\beta}$.
    Likewise, let ${\delta} \,=\, (d_{1},d_{2},\,{\ldots}\,)$, where $d_{k} \,=\, a_{k}-b_{k}$ for every $k$ in ${\NN}$;
    that is, ${\delta}$ is the sequence of differences of corresponding terms from ${\alpha}$ and ${\beta}$.
    Then the sequences ${\sigma}$ and ${\delta}$ are also convergent.
    Moreover, $\lim_{k \,{\rightarrow}\, {\infty}} s_{k} \,=\, A+B$ and $\lim_{k \,{\rightarrow}\, {\infty}} d_{k} \,=\, A-B$.



\V

        (b) (The Product Rule for Convergent Real Sequences)\IndC{sequences}{algebraic rules for convergent sequences}{product rule} Let ${\varphi} \,=\,(p_{1},p_{2},\,{\ldots}\,)$, where $p_{k} \,=\,a_{k}{\cdot}b_{k}$ for each $k$ in ${\NN}$;
    that is, ${\varphi}$ is the sequence of products of corresponding terms from ${\alpha}$ and ${\beta}$.
    Then the sequence ${\varphi}$ is also convergent, and $\lim_{k \,{\rightarrow}\, {\infty}} p_{k} \,=\, A{\cdot}B$.

\V

        (c) (The Quotient Rule for Convergent Real Sequences)\IndC{sequences}{algebraic rules for convergent sequences}{quotient rule} Assume in addition that (i) all the terms $b_{k}$ are nonzero and (ii) the limit $B$ is also nonzero.
    Let ${\rho} \,=\,(r_{1},r_{2},\,{\ldots}\,)$, where $r_{k} \,=\,a_{k}/b_{k}$ for each $k$ in ${\NN}$;
    that is, ${\rho}$ is the sequence of ratios of corresponding terms from ${\alpha}$ and ${\beta}$.
    Then the sequence ${\rho}$ is also convergent, and $\lim_{k \,{\rightarrow}\, {\infty}} r_{k} \,=\, A/B$.

\V

        \underline{Proof}:

\V

        (a) First note that for all $k$ in ${\NN}$ one has
        \begin{equation} 
        \label{IneqC.70}
        |(a_{k}+b_{k})-(A+B)| \,=\, |(a_{k}-A) + (b_{k}-B)|\,\,{\leq}\,\,|a_{k}-A| + |b_{k}-B|
        \end{equation}
    (The final inequality comes by using the Triangle Inequality.)
    Now let ${\varepsilon}\,>\,0$ be given.
    Then, by the `convergence' hypotheses, there exist numbers $N_{1}$ and $N_{2}$ in ${\NN}$ such that

        \h (i) if $k\,\,{\geq}\,\,N_{1}$,  then $|a_{k}-A|\,<\,{\varepsilon}/2$;

        \h (ii) if $k\,\,{\geq}\,\,N_{2}$, then $|b_{k}-B|\,<\,{\varepsilon}/2$.

\noindent  Let $N \,=\, \max\,\{N_{1},N_{2}\}$. If $k\,\,{\geq}\,\,N$ then $k\,\,{\geq}\,\,N_{1}$ and $k\,\,{\geq}\,\,N_{2}$,
    so by combining this with Inequality~\Ref{IneqC.70} one gets
        \begin{displaymath}
        |(a_{k}+b_{k})-(A+B)|\,\,{\leq}\,\,|a_{k}-A| + |b_{k}-B|\,<\,\frac{{\varepsilon}}{2} + \frac{{\varepsilon}}{2} \,=\, {\varepsilon} \mbox{ if $k\,\,{\geq}\,\,N$}.
        \end{displaymath}
    Thus $\lim_{k \,{\rightarrow}\, {\infty}} (a_{k}+b_{k}) \,=\, A+B$, as claimed.

        The direct proof of the claim for the sequence ${\delta}$ is similar. Alternatively, one can note that $a_{k}-b_{k} \,=\, a_{k} + (-1)\,b_{k}$,
	and then combine the Sum Rule with the Constant-Factor Rule to get the desired result.

\V

        (b) Using the familiar `Add-and-Subtract Trick', for each index $k$ one can write
        \begin{equation} 
        \label{EqnC.80}
        a_{k}\,b_{k}-AB \,=\, a_{k}\,b_{k} - a_{k}\,B + a_{k}\,B - A\,B \,=\, (a_{k}-A)B + a_{k}(b_{k}-B).
        \end{equation}
    The hypotheses that $\lim_{k \,{\rightarrow}\, {\infty}} a_{k} \,=\, A$ and $\lim_{k \,{\rightarrow}\, {\infty}} b_{k} \,=\, B$ imply (by Part~(d) of Theorem~\Ref{ThmC20.10B}) that $\lim_{k \,{\rightarrow}\, {\infty}} (a_{k}-A) \,=\, 0$ and $\lim_{k \,{\rightarrow}\, {\infty}} (b_{k}-B) \,=\, 0$.
    Note that the convergence hypothesis implies (by Part~(b) of Theorem~\Ref{ThmC20.10A})
    that the factors $B$ and $a_{k}$ in the expressions $(a_{k}-A)B$ and $a_{k}(b_{k}-B)$ in Equation~\Ref{EqnC.80} are bounded.
    Thus one can apply Part~(f) of Theorem~\Ref{ThmC20.10B} to the terms $(a_{k}-A)B$ and $a_{k}(b_{k}-B)$ on the right side of Equation~\Ref{EqnC.80} to conclude that
        \begin{displaymath}
        \lim_{k \,{\rightarrow}\, {\infty}} (a_{k}-A)B \,=\, 0 \mbox{ and }
        \lim_{k \,{\rightarrow}\, {\infty}} a_{k}(b_{k}-B) \,=\, 0.
        \end{displaymath}
    Finally, apply Part~(a) of the current theorem to what has been shown to conclude that
    the left side of Equation~\Ref{EqnC.80} converges to $0+0 \,=\, 0$, and thus $\lim_{k \,{\rightarrow}\, {\infty}} a_{k}b_{k} \,=\, AB$, as claimed.

\V

        (c) Note that for all indices $k$ one has
        \begin{equation} 
        \label{EqnC.90}
        \left|\frac{1}{b_{k}}-\frac{1}{B}\right| \,=\, 
        \left|\frac{B-b_{k}}{Bb_{k}}\right| \,=\, \frac{|B-b_{k}|}{|Bb_{k}|}.
        \end{equation}
    Let $a_{k} \,=\, 1/(Bb_{k})$. By Corollary~\Ref{CorC60.20} there exists $c\,>\,0$ such that $|b_{k}|\,\,{\geq}\,\,c$ for all $k$.
    Thus $|a_{k}| \,=\, 1/|Bb_{k}|\,\,{\leq}\,\,c/|B|$, so the sequence $(a_{1},a_{2},\,{\ldots}\,)$ is bounded.
    Since $\lim_{k \,{\rightarrow}\, {\infty}} b_{k} \,=\, B$, it follows that $\lim_{k \,{\rightarrow}\, {\infty}} |B-b_{k}| \,=\, 0$.
    Thus by Part~(f) of Theorem~\Ref{ThmC20.10B} one has $\lim_{k \,{\rightarrow}\, {\infty}} |B-b_{k}|/|Bb_{k}| \,=\, 0$.
    Compare this with Equation~\Ref{EqnC.90} to conclude that $\lim_{k \,{\rightarrow}\, {\infty}} {\displaystyle \left(\frac{1}{B}-\frac{1}{b_{k}} \right)}\,=\, 0$, which implies the desired result.

\V


            \subsection{\small{\bf Corollary}}
            \label{CorC60.20}

\V

        Let ${\xi} \,=\, (x_{1},x_{2},\,{\ldots}\,)$ be a sequence of real numbers which converges to a nonzero number $L$.
    Assume, in addition, that none of the terms $x_{k}$ of the sequence equal $0$.
    Then there exists a number ${\delta}\,>\,0$ such that $|x_{k}|\,\,{\geq}\,\,{\delta}$ for {\em all} the indices $k$.


\V

        \underline{Proof} Let ${\zeta} \,=\, (z_{1}, z_{2},\,{\ldots}\,z_{k},\,{\ldots}\,)$
    be the sequence given by the rule $z_{k} \,=\, |1/x_{k}|$ for each index~$k$.
    It follows from the Quotient Rule for Convergent Real Sequences (see Part~(c) of the preceding theorem),
    combined with the Absolute-Value Rule for Convergent Real Sequences
    (see Part~(b) of Theorem~\Ref{ThmC60.10}), that the sequence ${\zeta}$ conveges to~$|1/L|$,
    and thus is bounded (see Part~(b) of Theorem~\Ref{ThmC20.10A}). In particular,
    there exists a number $M\,>\,0$ such that $|z_{k}|\,\,{\leq}\,\,M$, that is, $|1/x_{k}|\,\,{\leq}\,\,M$, for each~$k$.
    It follows that $|x_{k}|\,\,{\geq}\,\,m \,=\, 1/M\,>\,0$ for each~$k$.
    
\V

            \subsection{\small{\bf Corollary}}
            \label{CorC60.25}

\V

        The theory of convergence for bounded sequences of real numbers can be reduced to
    the corresponding theory for sequences all of whose values lie in the open unit interval~$(0,1)$.
    More precisely, suppose that ${\xi} \,=\, (x_{1},x_{2},\,{\ldots}\,x_{k},\,{\ldots}\,)$ is a bounded sequence of reals. Let $a$ and $b$ be numbers such that
    $a\,<\,x_{k}\,<\,b$ for each index~$k$. so that $x_{k}{\in}[a,b]$.
    Define a second sequence ${\zeta} \,=\, (z_{1}, z_{2},\,{\ldots}\,z_{k},\,{\ldots}\,)$ by the rule
    ${\displaystyle z_{k} \,=\, \frac{x_{k}-a}{b-a}}$ for each~$k$ for each~$k$. Then $0\,<\,z_{k}\,<\,1$ for each~$k$,
    and the original sequence ${\xi}$ is convergent if, and only if, the new sequence ${\zeta}$ is convergent.

\V

        The simple proof is left as an exercise. %EXERCISE

\V
\V

            \subsection{\small{\bf Remark}}
            \label{RemrkC60.40}

        In the Quotient Rule (Part~(c) of the preceding theorem), it is assumed that for all $k$ one has $b_{k} \,\,{\neq}\,\, 0$.
    This is done so that all the ratios $a_{k}/b_{k}$ are defined; in other words, so that ${\rho}$ is a true sequence.
    In light of  Definition~\Ref{DefC10.40}, however, it is clear that one really need only require that for all sufficiently large $k$ one has $b_{k} \,\,{\neq}\,\, 0$.
    And it is a simple exercise (left to the reader, of course) to show that the hypothesis $B \,\,{\neq}\,\, 0$, already included in the statement of the Quotient Rule, guarantees that $b_{k} \,\,{\neq}\,\, 0$ if $k$ is large enough.
    Thus, the Quotient Rule for Convergent Sequences can be rephrased in the following more general form:

        \h `Assume, in addition, that the limit $B$ is nonzero.
    Then ${\displaystyle \lim_{k \,{\rightarrow}\, {\infty}} \frac{a_{n}}{b_{n}}  \,=\, \frac{A}{B}}$, where the limit is taken in the `Extended Sense' of Definition~\Ref{DefC10.40}'

\V

            \subsection{\small{\bf Examples}}
            \label{ExampC60.50}

\V

        (1) In Corollary~\Ref{CorC20.15} it is shown that if $0\,<\,r\,<\,1$ then $\lim_{k \,{\rightarrow}\, {\infty}} r^{k} \,=\, 0$.
    Here is an alternate proof.

        As in the earlier proof, note that the sequence ${\xi} \,=\, (r,r^{2},r^{3},\,{\ldots}\,)$ is strictly decreasing and bounded below by $0$, hence this sequence has a (finite) limit $L\,\,{\geq}\,\,0$.
    Next, notice that $r^{k+1} \,=\, r^{k}{\cdot}r$, so that the sequence ${\tau} \,=\, (r^{2},r^{3},\,{\ldots}\,)$ is obtained by multiplying each term of the sequence ${\xi}$ by $r$.
    By the Constant Factor Rule for Convergent Sequences (i.e., Part~(a) of Theorem~\Ref{ThmC60.10}) it follows that ${\tau}$ converges to $Lr$.
    But clearly ${\tau}$ is a subsequence of ${\xi}$, so by Part~(b) of Theorem~\Ref{ThmC20.10A} it follows that ${\tau}$ has the {\em same} limit as ${\xi}$.
    Thus $L \,=\, Lr$, hence $L(1-r) \,=\, 0$. Since $r \,\,{\neq}\,\, 1$, it follows that $L \,=\, 0$.

\V

        (2) Consider the sequence ${\xi}$ whose $k$-th term $x_{k}$ is given by
        \begin{displaymath}
        x_{k} \,=\, \frac{3k^{2}+k+5}{6k^{2}+1} \mbox{ for $k$ in ${\NN}$}.
        \end{displaymath}
    We determined the limit of this sequence in Part~(3) of Example~\Ref{ExampC10.20} by first guessing that this limit should be~$1/2$.
    Analyzing such a limit propblem becomes much easier using the limit laws obtained in this section.

        Indeed, factor out the highest power of $k$ in both the numerator and denominator of the fraction defining $x_{k}$, and simplify, to get:
        \begin{displaymath}
        x_{k} \,=\, \frac{k^{2}\,(3 + 1/k + 5/k^{2})}{k^{2}\,(6 + 1/k^{2})}
     \,=\, 
        \frac{3 + 1/k + 5/k^{2}}{6 + 1/k^{2}}.
        \end{displaymath}
    It is clear that
        \begin{displaymath}
        \lim_{k \,{\rightarrow}\, {\infty}} \left(3 + \frac{1}{5} + \frac{5}{k^{2}}\right) \,=\, 3,
    \mbox{ and }
        \lim_{k \,{\rightarrow}\, {\infty}} \left(6 + \frac{1}{k^{2}}\right)
     \,=\, 6.
        \end{displaymath}
    It follows from the Quotient Rule for Limits that the original sequence ${\xi}$ is convergent,
    and that $\lim_{k \,{\rightarrow}\, {\infty}} x_{k} \,=\, 3/6$, which agrees with the result obtained in the earlier treatment of this sequence.

\VV

    There are a few `algebraic rules' which apply to sequences which have {\em infinite} limits.

\V

            \subsection{\small{\bf Theorem}}
            \label{ThmC60.55}

\V

        Suppose that ${\alpha} \,=\, (a_{1},a_{2},\,{\ldots}\,)$ and ${\beta} \,=\, (b_{1},b_{2},\,{\ldots}\,)$ are sequences of real numbers.

\V

        (a) If $\lim_{k \,{\rightarrow}\, {\infty}} a_{k} \,=\, +{\infty}$ and ${\beta}$ is bounded below,
    then $\lim_{k \,{\rightarrow}\, {\infty}} (a_{k} + b_{k}) \,=\, +{\infty}$.

        Similarly, if $\lim_{k \,{\rightarrow}\, {\infty}} a_{k} \,=\, -{\infty}$ and ${\beta}$ is bounded above, then $\lim_{k \,{\rightarrow}\, {\infty}} (a_{k} + b_{k}) \,=\, -{\infty}$.

\V

        (b) If $\lim_{k \,{\rightarrow}\, {\infty}} a_{k} \,=\, +{\infty}$ and there exists $m\,>\,0$ such that $b_{k}\,\,{\geq}\,\,m$ for all~$k$,
    then $\lim_{k \,{\rightarrow}\, {\infty}} (a_{k}\,b_{k}) \,=\, +{\infty}$.

        Similarly, if $\lim_{k \,{\rightarrow}\, {\infty}} a_{k} \,=\, +{\infty}$
    and there exists $m\,<\,0$ such that $b_{k}\,\,{\leq}\,\,m$ for all~$k$, then $\lim_{k \,{\rightarrow}\, {\infty}} (a_{k}\,b_{k}) \,=\, -{\infty}$.

\V

        (c) If $\lim_{k \,{\rightarrow}\, {\infty}} a_{k} \,=\, -{\infty}$ and there exists $m\,>\,0$ such that $b_{k}\,\,{\geq}\,\,m$ for all~$k$,
    then $\lim_{k \,{\rightarrow}\, {\infty}} (a_{k}\,b_{k}) \,=\, -{\infty}$.

        Similarly, if $\lim_{k \,{\rightarrow}\, {\infty}} a_{k} \,=\, -{\infty}$
    and there exists $m\,<\,0$ such that $b_{k}\,\,{\leq}\,\,m$ for all~$k$, then $\lim_{k \,{\rightarrow}\, {\infty}} (a_{k}\,b_{k}) \,=\, +{\infty}$.

\V

        (d) If there exists $m\,>\,0$ such that $a_{k}\,\,{\geq}\,\,m$ for all $k$,
    and if $b_{k}\,>\,0$ for all $k$ and $\lim_{k \,{\rightarrow}\, {\infty}} b_{k} \,=\, 0$,
    then $\lim_{k \,{\rightarrow}\, {\infty}} a_{k}/b_{k} \,=\, +{\infty}$.

        Similarly, if there exists $m\,>\,0$ such that $a_{k}\,\,{\geq}\,\,m$ for all $k$,
    and if $b_{k}\,<\,0$ for all $k$ and $\lim_{k \,{\rightarrow}\, {\infty}} b_{k} \,=\, 0$,
    then $\lim_{k \,{\rightarrow}\, {\infty}} a_{k}/b_{k} \,=\, -{\infty}$.

\V

        (e) If there exists $m\,<\,0$ such that $a_{k}\,\,{\leq}\,\,m$ for all $k$,
    and if $b_{k}\,>\,0$ for all $k$ and $\lim_{k \,{\rightarrow}\, {\infty}} b_{k} \,=\, 0$,
    then $\lim_{k \,{\rightarrow}\, {\infty}} a_{k}/b_{k} \,=\, -{\infty}$.

        Similarly, if there exists $m\,<\,0$ such that $a_{k}\,\,{\leq}\,\,m$ for all $k$,
    and if $b_{k}\,<\,0$ for all $k$ and $\lim_{k \,{\rightarrow}\, {\infty}} b_{k} \,=\, 0$,
    then $\lim_{k \,{\rightarrow}\, {\infty}} a_{k}/b_{k} \,=\, +{\infty}$.

\V

        The simple proofs are left as exercises. %EXERCISES

\VV

        The portions of the preceding theorems which involve $\lim_{k \,{\rightarrow}\, {\infty}} a_{k}/b_{k}$ omit two important cases:

        \h \underline{The `$0/0$' Case} In this case one has $\lim_{k \,{\rightarrow}\, {\infty}} a_{k} \,=\, \lim_{k \,{\rightarrow}\, {\infty}} b_{k} \,=\, 0$.

        \h \underline{The `${\infty}/{\infty}$' Case} In this case one has $\lim_{k \,{\rightarrow}\, {\infty}} a_{k} \,=\, \lim_{k \,{\rightarrow}\, {\infty}} b_{k} \,=\,  \,{\pm}\, {\infty}$.

\noindent The next result sometimes can used to handle these cases. It works by replacing the quotients
    $a_{k}/b_{k}$ by new quotients $c_{k}/d_{k}$ which may be able to avoid the $0/0$ and ${\infty}/{\infty}$ difficulties.

\V

            \subsection{\small{\bf Theorem} (Stoltz-Cesaro Theorem)}
            \label{ThmC60.55A}\IndBD{sequences}{Stoltz-Cesaro theorem}

\V

       Let ${\alpha} \,=\, (a_{1},a_{2},\,{\ldots}\,)$ and ${\beta} \,=\, (b_{1},b_{2},\,{\ldots}\,)$ be sequences of real numbers.
    Assume that ${\alpha}$ and ${\beta}$ satisfy the following conditions:

\VA

        \h (i)\, ${\beta}$ is strictly increasing, and

        \h (ii) ${\displaystyle \lim_{k \,{\rightarrow}\, {\infty}} \frac{a_{k+1} - a_{k}}{b_{k+1} - b_{k}} \,=\, L}$ for some extended real number~$L$.

\VA

\noindent Then:

\V

        (a) (`$0/0$ Case')\, Suppose that both the sequences ${\alpha}$ and ${\beta}$ converge to~$0$.
    Then ${\displaystyle \lim_{k \,{\rightarrow}\, {\infty}} \frac{a_{k}}{b_{k}} \,=\, L}$.


\V

        (b) (`${\infty}/{\infty}$' Case)\, Suppose that $b_{k}\,>\,0$ for all $k$ and that the sequence ${\beta}$ has limit~$+{\infty}$.
    Then ${\displaystyle \lim_{k \,{\rightarrow}\, {\infty}} \frac{a_{k}}{b_{k}} \,=\, L}$.


        \underline{Notes to Parts (a) and~(b)} (i) Because the sequence ${\beta}$ is assumed to be strictly increasing,
    in the `$0/0$' case the sequence ${\beta}$ must approach~$0$ from below; in particular, $b_{k}\,<\,0$ so the ratio $a_{k}/b_{k}$ is defined.

        (ii) The hypothesis in the `${\infty}/{\infty}$' case, that $b_{k}\,>\,0$ for every index~$k$, is included to simplifiy the discussion.
    Indeed, the hypothesis that $\lim_{k \,{\rightarrow}\, {\infty}} b_{k} \,=\, +{\infty}$ already guarantees that $b_{k}\,>\,0$,
    so that the fraction $a_{k}/b_{k}$ makes sense, for all but a finite number of values of~$k$.
    If one understands the limit here in the `extended' sense of Definition~\Ref{DefC10.40},
    then the requirement that $b_{k}\,>\,0$ for {\em all} $k$ can be dropped and the result remains true.

\V

        (c) Similar results hold for both cases if, instead,
    one assumes that the sequence ${\beta}$ is strictly decreasing,
        and in the `${\infty}/{\infty}$' case one assumes that $\lim_{k \,{\rightarrow}\, {\infty}} b_{k} \,=\, -{\infty}$.

\V

        {\bf Proof} \underline{Preliminary Discussion for the Proofs of Parts~(a) and~$(b)$}

        Let $y$ be a real number such that $y\,<\,L$, and let $y'$ be a real number such that $y\,<\,y'\,<\,L$.
    Hypothesis~(ii) allows one to use Statement~A of Theorem~\Ref{ThmC40.80} to say that there exists a natural number $N$ such that if $k\,\,{\geq}\,\,N$,
    then ${\displaystyle y'\,<\,\frac{a_{k+1}-a_{k}}{b_{k+1}-b_{k}}}$. Hypothesis~(i) implies that $b_{k} - b_{k-1}\,>\,0$ for each~$k$,
     so it follows from the usual order properties of ${\RR}$ that if $k\,\,{\geq}\,\,N$,

    then $y'\,(b_{k+1}-b_{k})\,<\,(a_{k+1}-a_{k})$. If $n$ is any natural number and $k\,\,{\geq}\,\,N$ then one gets the following string of inequalities:
        \begin{displaymath}
        y'\,(b_{k+1}-b_{k})\,<\,(a_{k+1}-a_{k}), \,
        y'\,(b_{k+2}-b_{k+1})\,<\,(a_{k+2}-a_{k+1}), \,
        \,{\ldots}\,
        y'\,(b_{k+n}-b_{k+n-1})\,<\,(a_{k+n}-a_{k+n-1}), \,
        \end{displaymath}
    Add both sides of these inequalities, using the fact that in this `collapsing sum' most of the terms cancel, to get
        \begin{displaymath}
        y'\,(b_{k+n} - b_{k})\,<\,(a_{k+n} - a_{k})
        \end{displaymath}
    Note that $b_{k}-b_{k+n} \,=\, (b_{k+1}-b_{k}) + (b_{k+2} - b_{k+1}) + \,{\ldots}\,+ (b_{k+n} - b_{k+n-1})$,
    a sum of positive numbers, so $b_{k+n}-b_{k}\,>\,0$. It follows that
        \begin{displaymath}
        y' \,<\,\frac{a_{k+n}-a_{k}}{b_{k+n}-b_{k}} \mbox{ if $k\,\,{\geq}\,\,N$ and $n{\in}{\NN}$} \h ({\ast})
        \end{displaymath}

        Hypothesis (ii) likewise allows one to use Statement~B of Theorem~\Ref{ThmC40.80} to prove that
    if $z$ and $z'$ are real numbers such that $L\,<\,z'\,<\,z$, there exists $N$ such that if $k\,\,{\geq}\,\,N$, then
        \begin{displaymath}
        z' \,>\,\frac{a_{k+n}-a_{k}}{b_{k+n}-b_{k}} \mbox{ if $k\,\,{\geq}\,\,N$ and $n{\in}{\NN}$} \h ({\ast}{\ast})
        \end{displaymath}

\V

        \underline{Proof of (a)}\, In this `$0/0$' case it is given that the sequences ${\alpha}$ and ${\beta}$ converge to~$0$.
    Note that, because ${\beta}$ is strictly increasing, it follows that $b_{k}\,<\,0$ for each index~$k$.
    Since every subsequence of a convergent sequence also converges to the same limit, it then follows that for each index $k$ one has
    $\lim_{n \,{\rightarrow}\, {\infty}} a_{k+n} \,=\, 0$ and $\lim_{n \,{\rightarrow}\, {\infty}} b_{k+n} \,=\, 0$.
    In particular, if $k\,\,{\geq}\,\,N$ then by letting $n$ approach~${\infty}$ in Inequality~$({\ast})$, and using the Quotient Rule for Limits (which applies since $b_{k}\,>\,0$),
    and recalling that $y'$ was chosen so that $y\,<\,y'$, one gets
        \begin{displaymath}
        y\,<\,y'\,\,{\leq}\,\,\frac{(-a_{k})}{(-b_{k})}
         \,=\, \frac{a_{k}}{b_{k}} \mbox{ if $k\,\,{\geq}\,\,N$}.
        \end{displaymath}
    That is, the sequence $(a_{1}/b_{1}, a_{2}/b_{2},\,{\ldots}\,a_{k/b_{k},\,{\ldots}\,})$ satisfies Statement~A of Theorem~\Ref{ThmC40.80}.
    In a similar way, one can use Inequality~$({\ast}{\ast})$ to show that this sequence also satisfies Statement~B of the same theorem.
    It now follows that ${\displaystyle \lim_{k \,{\rightarrow}\, {\infty}} \frac{a_{k}}{b_{k}} \,=\, L}$, as claimed.

\V%

        \underline{Proof of (b)}\, Apply Statement~A of Theorem~\Ref{ThmC40.80} once again to show that
    there exists a natural number $N$ such that if $k\,\,{\geq}\,\,N$, Inequality~$({\ast})$ holds for all~$n$ in~${\NN}$.
    Multiply both sides of the fraction which appears in Inequality~$({\ast})$ by the positive quantity $b_{k+n}-b_{k}$ and do some simple algebra to get
        \begin{displaymath}
        y'\,(b_{k+n}) - y'\,b_{k} + a_{k}\,<\, a_{k+n}
        \end{displaymath}
    This inequality is valid for every index $k\,\,{\geq}\,\,N$, but it is technically easier from here on to focus on the case $k \,=\, N$.
    One then gets
        \begin{displaymath}
        y'\,(b_{N+n}) - y'\,b_{N} + a_{N}\,<\, a_{N+n} \mbox{ for every $n$ in ${\NN}$}
        \end{displaymath}
    Divide by the positive quantity $b_{N+n}$ to get
        \begin{displaymath}
        y' - y'\,\frac{b_{N}}{b_{N+n}} + \frac{a_{N}}{b_{N+n}}\,<\,\frac{a_{N+n}}{b_{N+n}} \mbox{ for every $n$ in ${\NN}$}.
        \end{displaymath}
    Since $y'$ and thus $N$ are held fixed -- $N$ is determined by the choice of $z$ and $z'$ -- it is clear,
    from the hypothesis that $\lim_{k \,{\rightarrow}\, {\infty}} b_{k} \,=\, +{\infty}$, that
        \begin{displaymath}
        \lim_{n \,{\rightarrow}\, {\infty}} \left(-y'\,\frac{b_{N}}{b_{N+n}} +
         \frac{a_{N}}{b_{N+n}}\right) \,=\, 0.
        \end{displaymath}
    In particular, since $y\,<\,y'$, there exists an index $N'$ such that if $n\,\,{\geq}\,\,N'$ then
        \begin{displaymath}
        y\,<\,\left(-y'\,\frac{b_{N}}{b_{N+n}} +
         \frac{a_{N}}{b_{N+n}}\right)\,<\,\frac{a_{n+n}}{b_{N+n}} \mbox{ if $n\,\,{\geq}\,\,N'$}
        \end{displaymath}
    This implies that
        \begin{displaymath}
        y\,<\,\frac{a_{k}}{b_{k}} \mbox{ if $k\,\,{\geq}\,\,N+N'$}
        \end{displaymath}
    It follows that the sequence $(a_{1}/b_{1}, a_{2}/b_{2},\,{\ldots}\,a_{k}/b_{k})$
    satisfies Statement~A of Theorem~\Ref{ThmC40.80}. A similar proof shows that this sequence also satisfies Statement~B,
    and thus this sequence has limit~$L$, as claimed.


\V

        \underline{Proof of (c)} It is a simple exercise to reduce this to the results obtained in Parts~(a) and~(b). %% EXERCISE

\VV

            \subsection{\small{\bf Examples}}
            \label{RemrkC60.55B}

\V

\hspace*{\parindent}(1) Let ${\alpha} \,=\, (a_{1}, a_{2},\,{\ldots}\,a_{k},\,{\ldots}\,)$
    and ${\beta} \,=\, (b_{1}, b_{2},\,{\ldots}\,b_{k},\,{\ldots}\,)$ be given by the formulas
        \begin{displaymath}
        a_{k} \,=\, 3\,k + \frac{5}{k^{2}} \mbox{ and }
        b_{k} \,=\, 7\,k + 2 - \frac{4}{k^{3}}.
        \end{displaymath}
    Suppose one wishes to determine the convergence of the sequence ${\xi} \,=\, (x_{1}, x_{2},\,{\ldots}\,x_{k},\,{\ldots}\,)$, where $x_{k} \,=\, a_{k}/b_{k}$.
    It is easy to see that $\lim_{k \,{\rightarrow}\, {\infty}} a_{k} \,=\, +{\infty}$ and $\lim_{k \,{\rightarrow}\, {\infty}} b_{k} \,=\, +{\infty}$,
    so that the `Standard Quotient Rules' described in Theorem~\Ref{ThmC60.30} and in Theorem~\Ref{ThmC60.55} do not apply.
    However, one sees easily that ${\beta}$ is strictly increasing. Futhermore, one computes that
        \begin{displaymath}
        a_{k}-a_{k+1} \,=\, 3 + \frac{5}{k^{2}} - \frac{5}{(k+1)^{2}}
	\mbox{ and }
        b_{k} - b_{k+1} \,=\, 7 - \frac{4}{k^{3}} + \frac{4}{(k+1)^{3}}.
        \end{displaymath}
    Note that these are both convergent sequences, with limits $3$ and~$7$, respectively.
    It then follows directly from the standard Quotient Rule for Convergent Sequences, Part~(c) of Theorem~\Ref{ThmC60.30}, that
        \begin{displaymath}
        \lim_{k \,{\rightarrow}\, {\infty}} \frac{a_{k}-a_{k+1}}{b_{k}-b_{k+1}}
     \,=\, \frac{3}{7}
        \end{displaymath}
    Thus, the hypotheses of the Stoltz-Cesaro theorem are satisfied, so one can conclude that $\lim_{k \,{\rightarrow}\, {\infty}} x_{k} \,=\, 3/7$.

\V

        (2) Suppose that $x_{k} \,=\, a_{k}/b_{k}$, where $a_{k} \,=\, 5\,k^{2} + 2\,k - 9$ and $b_{k} \,=\, 4 - k^{2}$.
    Once again, $\lim_{k \,{\rightarrow}\, {\infty}} a_{k} \,=\, +{\infty}$ and $\lim_{k \,{\rightarrow}\, {\infty}} b_{k} \,=\, -{\infty}$,
    so the standard Quotient Rule does not apply. However it is also clear that $b_{k+1}\,<\,b_{k}$ for each~$k$,
    so it makes sense to try the Stoltz-Cesaro theorem. One computes:
        \begin{displaymath}
        a_{k} - a_{k+1} \,=\, 10\,k + 7 \mbox{ and } 
        b_{k} - b_{k+1} \,=\, -2\,k-1.
        \end{displaymath}
    To use the Stoltz-Cesaro theorem, we next consider the limit of the sequence ${\zeta} \,=\, (z_{1}, z_{2},\,{\ldots}\,z_{k},\,{\ldots}\,)$,
    where ${\displaystyle z_{k} \,=\, \frac{c_{k}}{d_{k}}}$, with $c_{k} \,=\, a_{k}-a_{k+1} \,=\, 10\,k + 7$ and $d_{k} \,=\, b_{k}-b_{k+1} \,=\, -2\,k-1$.
    Once again, one cannot use the standard Quotient Rule to compute $\lim_{k \,{\rightarrow}\, {\infty}} (c_{k}/d_{k})$
    because $\lim_{k \,{\rightarrow}\, {\infty}} c_{k} \,=\, +{\infty}$ and $\lim_{k \,{\rightarrow}\, {\infty}} d_{k} \,=\, -{\infty}$.
    However, if one tries the Stoltz-Cesaro Theorem one more time, one gets an answer.
    Indeed, $c_{k}-c_{k+1} \,=\, (10\,k + 7) - (10\,k + 17) \,=\, -7$; likewise, $d_{k}-d_{k+1} \,=\, (-2\,k -1) - (-2\,k - 3) \,=\, 1$.
    It is clear that ${\displaystyle \lim_{k \,{\rightarrow}\, {\infty}} \frac{c_{k}-c_{k+1}}{d_{k}- d_{k+1}} \,=\, -7/1 \,=\, -7}$.
    Apply the Stoltz-Cesaro theorem to get ${\displaystyle \lim_{k \,{\rightarrow}\, {\infty}} \frac{c_{k}}{d_{k}} \,=\, -7}$;
    that is, ${\displaystyle \lim_{k \,{\rightarrow}\, {\infty}} \frac{a_{k}-a_{k+1}}{b_{k}-b_{k+1}} \,=\, -7}$.
    This last fact then implies that ${\displaystyle \lim_{k \,{\rightarrow}\, {\infty}} \frac{a_{k}}{b_{k}} \,=\, -7}$.


\V


            \subsection{\small{\bf Remarks}}
            \label{RemrkC60.55C}

\V

\hspace*{\parindent}(1) The reason for the name `$0/0$' case in Part~(a) of the Stoltz-Cesaro theorem is fairly obvious:
    the `$0$' in the numerator corresponds to the hypothesis $\lim_{k \,{\rightarrow}\, {\infty}} a_{k} \,=\, 0$,
    while the `$0$' in the demominator corresponds to $\lim_{k \,{\rightarrow}\, {\infty}} b_{k} \,=\, 0$.

        In Part~(b), however, why the name $`{\infty}/{\infty}$' is not so clear,
    since there is no hypothesis of the form $\lim_{k \,{\leftarrow}\, {\infty}} a_{k} \,=\,  \,{\pm}\, {\infty}$.
    Actually, such a hypothesis would be redundant `most of the time'. Indeed, it is an easy exercise to show that
    the given hypotheses in Part~(b) imply that $\lim_{k \,{\rightarrow}\, {\infty}} a_{k} \,=\,+{\infty}$ if $L\,>\,0$,
    while they imply that $\lim_{k \,{\rightarrow}\, {\infty}} a_{k} \,=\, -{\infty}$ if $L\,<\,0$.

\V

        (2) The Stoltz-Cesaro theorem is a specialized tool, to be used only when it is needed; mainly for limit problems of the form $0/0$ or~${\infty}/{\infty}$.
    Using Stoltz-Cesaro on a limit problem which is {\em not} of these forms is likely to give no answer, or worse, the wrong anwser.
    For instance, suppose that for each $k$ one has $a_{k} \,=\, 1$ and $b_{k} \,=\, 1/k$.
    One computes that ${\displaystyle \lim_{k \,{\rightarrow}\, {\infty}}(a_{k+1}-a_{k})/(b_{k+1}-b_{k}) \,=\, 0}$, since $a_{k+1}-a_{k} \,=\, 0$ for each~$k$.
    However, one computes directly that ${\displaystyle \lim_{k \,{\rightarrow}\, {\infty}} a_{k}/b_{k} \,=\, +{\infty}}$.
    Of course, the hypotheses of the Stoltz-Cesaro theorem are not satisfied in this example.

        \underline{Moral of the Story} Don't try to use a theorem if its hypotheses are not satisfied.

\V

        (3) The preceding discussion may bring back memories of elementary calculus; specifically, of {\em L'H\^{o}pital's Rule}.
    (If you don't recall what that refers to, don't worry; it is covered in Chapter~\Ref{ChaptE}.)
    Indeed, the Stoltz-Cesaro theorem is often called {\bf L'H\^{o}pital's Rule for sequences of real numbers}.
    \IndBD{L'H\^{o}pital's Rule}{for sequences of real numbers}.

\V

        The following simple consequences of the Stoltz-Cesaro theorem are important in their own right.
    The first was stated explicitly, and proved, by Cauchy in his {\em Cours d'analyse} of~$1821$.
    In light of the simple relation between them, it seems likely that Cauchy was also aware of the truth of the second.


\VV


            \subsection{\small{\bf Corollary}}
            \label{CorC40.85}

\V

\hspace*{\parindent}(a) Let ${\alpha} \,=\, (a_{1}, a_{2},\,{\ldots}\,a_{k},\,{\ldots}\,)$
    be a real sequence such that $\lim_{k \,{\rightarrow}\, {\infty}} (a_{k+1}-a_{k}) \,=\, L$ for some extended real number~$L$.
    Then ${\displaystyle \lim_{k \,{\rightarrow}\, {\infty}} \frac{a_{k}}{k} \,=\, L}$.

\V

        (b) Let ${\xi} \,=\, (x_{1}, x_{2},\,{\ldots}\,x_{k},\,{\ldots}\,)$ be a real sequence such that
    $\lim_{k \,{\rightarrow}\, {\infty}} x_{k} \,=\, L$ for some extended real number~$L$.
    Let ${\mu} \,=\, (m_{1}, m_{2},\,{\ldots}\,m_{k},\,{\ldots}\,)$ be the related sequence given by
        \begin{displaymath}
        m_{1} \,=\, x_{1},\, \,m_{2} \,=\, \frac{x_{1}+m_{2}}{2},\,{\ldots}\,m_{k} \,=\, \frac{m_{1} + x_{2}+\,{\ldots}\,+x_{k}}{k},\,{\ldots}\,;
        \end{displaymath}
    that is, $m_{k}$ is the arithmetic mean (i.e., average) of the first $k$ terms of the original sequence~${\xi}$.
    Then $\lim_{k \,{\rightarrow}\, {\infty}} m_{k} \,=\, L$.

\V

        {\bf Proof} (a) This corresponds to the special case of the Stoltz-Cesaro theorem in which one sets $b_{k} \,=\, k$ for each index~$k$.

\V

        (b) Define a sequence ${\alpha} \,=\, (a_{1}, a_{2},\,{\ldots}\,a_{k},\,{\ldots}\,)$
    by the rule $a_{k} \,=\, x_{1} + x_{2} + \,{\ldots}\,+ x_{k}$ for each~$k$. Note that $a_{k}/k \,=\, m_{k}$, and $a_{k+1}-a_{k} \,=\, x_{k+1}$. 
    The hypothesis that ${\xi}$ has limit~$L$ implies that $\lim_{k \,{\rightarrow}\, {\infty}} x_{k+1} \,=\, L$;
    that is, $\lim_{k \,{\rightarrow}\, {\infty}} (a_{k+1}-a_{k}) \,=\, L$. It follows from Part~(a) of this Corollary,
    and the fact that $a_{k}/k \,=\, m_{k}$ for each~$k$, that $\lim_{k \,{\rightarrow}\, {\infty}} m_{k} \,=\, L$.

\V


        {\bf Remarks} (1) The numbers $m_{k}$ defined above are called the {\bf Cesaro means},
    or, less commonly the {\bf Cesaro averages},\IndBD{sequences}{Cesaro means of a sequence}
    associated with the sequence~${\zeta}$, in honor of the Italian mathematician Ernesto Cesaro (c. $1888$).

\V

        (2) The converses of the statements proved in the preceding corollary are not true. 
    For example, it is an easy exercise to construct a sequence
    ${\xi} \,=\, (x_{1}, x_{2},\,{\ldots}\,x_{k},\,{\ldots}\,)$ which does not have a limit but whose corresponding sequence of Cesaro means does.
%% EXERCISE

\VV

%-----------------------
%-------------------------------
%\StartSkip{

                \section{{\bf The Bolzano-Weierstrass Theorem}}
                \label{SectC30}\IndB{ZZ Sections}{\Ref{SectC30} The Bolzano-Weierstrass Theorem}


        One of the most commonly used techniques in analysis the process of seeking a subsequence, of a given real sequence ${\xi}$,
    which has a limit even though ${\xi}$ itself need not have a limit. The next result gives the theoretical underpinnings for that method.
    It is one of the most important theorems in {\ThisText}.

\V

            \subsection{\small{\bf Theorem} (The Bolzano-Weierstrass Theorem for Sequences of Real Numbers - Standard Form)}\IndBD{sequences}{Bolzano-Weierstrass theorem for real sequences, standard form}
            \label{ThmC30.10}

\V

        Every bounded sequence of real numbers has a convergent subsequence.

\V

        \underline{Standard Proof} Since, by hypothesis, the sequence ${\xi}$ is bounded,
    there exist real numbers $a$ and $b$, with $a\,<\,b$, such that $a\,\,{\leq}\,\,x_{k}\,\,{\leq}\,\,b$ for every index~$k$.


    Now construct a bisection sequence $[a_{1},b_{1}]$, $[a_{2},b_{2}]$,\,{\ldots}\, (see Definition~\Ref{DefB30.05A}) as follows:

        \h (i) \underline{Initial Step} Let $[a_{1}, b_{1}]$ be any closed interval such that $x_{k}{\in}[a_{1}, b_{1}]$ for infinitely many values of the index~$k$;
    for instance, $a_{1} \,=\, a$ and $b_{1} \,=\, b$ will do.

        \h (ii) \underline{Inductive Step} Suppose that for some $m$ in ${\NN}$ the interval $[a_{m},b_{m}]$
    has been constructed so that $x_{k}{\in}[a_{m},b_{m}]$ for infinitely many indices~$k$.
    It is clear that at least one of the two halves of the interval $[a_{m},b_{m}]$ has the analogous property;
    namely, that there are infinitely many indices $k$ such that $x_{k}$ lies in that half of $[a_{m},b_{m}]$.
    If the left-half interval has that property, define $[a_{m+1},b_{m+1}]$ to be that left-half interval.
    If the left-half interval of $[a_{m},b_{m}]$ does {\em not} have that property,
    then the right-half interval does; in that case, define $[a_{m+1},b_{m+1}]$ to be that right-half interval.
    Continuing this way, one gets a bisection sequence $[a_{1},b_{1}]$, $[a_{2},b_{2}]$,\,{\ldots}\,$[a_{m},b_{m}]$,\,{\ldots}\,
    with the property that for each $m$ there are infinitely many indices $k$ such that $x_{k}{\in}[a_{m},b_{m}]$.

        For each $m$ let $A_{m}$ denote the set of indices $k$ such that $x_{k}{\in}[a_{m},b_{m}]$. It is clear that
    ${\cal A} \,=\, (A_{1},A_{2},\,{\ldots}\,A_{m},\,{\ldots}\,)$ is a subsequence structure of infinite order; see Definition~\Ref{DefA40.45}.
    Let $B \,=\, \{k_{1}\,<\,k_{2}\,<\,\,{\ldots}\,\,<\,k_{m}\,<\,\,{\ldots}\,\}$ be a cross section of ${\cal A}$; see Definition~\Ref{DefA40.60}.
    Then the subsequence ${\zeta} \,=\, (x_{k_{1}},x_{k_{2}},\,{\ldots}\,x_{k_{m}},\,{\ldots}\,)$
    of ${\xi}$ associated with $B$ has the property that $k_{m}{\in}A_{m}$ and thus $x_{k_{k}}{\in}[a_{m},b_{m}]$.
    The Bisection Principle implies that there is exactly one real number~$c$ which lies in each interval~$[a_{m},b_{m}]$.
    By Remark~\Ref{RemrkC20.10BB}~(1), together with the Squeeze Property for sequences, it follows that the subsequence ${\zeta}$ converges to~$c$.

\VV

        There is a straight-forward extension of the Bolzano-Weierstrass Theorem which drops the requirement that the sequence ${\xi}$ be bounded.

\V

            \subsection{\small{\bf Theorem} (Extended Bolzano-Weierstrass Theorem for Real Sequences -- Extended Form)}\IndBD{sequences}{Bolzano-Weierstrass theorem for real sequences, extended form}
            \label{ThmC40.60}


\V

        Suppose that ${\xi} \,=\, (x_{1},x_{2},\,{\ldots}\,)$ is a real sequence.
    Then there is a subsequence of ${\xi}$ which has a limit (in the sense of Definition~\Ref{DefC40.10} above).
    More precisely:

\V


        \h (i)\,\, If ${\xi}$ is unbounded above, then ${\xi}$ has a strictly increasing subsequence whose limit is~$+{\infty}$.

       \h (ii)\, If ${\xi}$ is unbounded below, then ${\xi}$ has a strictly decreasing subsequence whose limit is~$-{\infty}$.

        \h (iii) If ${\xi}$ is bounded above and below (i.e., bounded), then ${\xi}$ has a convergent subsequence.

\V

        \underline{Proof}


\V

        (i) Define an infinite-order subsequence structure ${\cal A} \,=\, (A_{1},A_{2},\,{\ldots}\,A_{m},\,{\ldots}\,)$ as follows:

\VA

        \h $A_{1}$ is the set of all indices $k$ such that $x_{k}\,>\,1$. Let $n_{1}$ be the smallest element of the set~$A_{1}$.
    Since, by hypothesis, the sequence ${\xi}$ is unbounded above, it is clear that $A_{1}$ is an infinite subset of~${\NN}$.

        \h $A_{2}$ is the set of all indices $k$ such that $x_{k}\,>\,\max\,\{x_{n_{1}},2\}$.
    Let $n_{2}$ be the smallest element of~$A_{2}$. It is clear that $A_{2}$ is an infinite subset of~$A_{1}$, and that $n_{2}\,>\,n_{1}$.

        \h Continuing this way, suppose that $A_{j}$ and $n_{j}$ have are already defined for all $j \,=\, 1,\,{\ldots}\,m$.
    Then $A_{m+1}$ is the set of all indices $k$ such that $x_{k}\,>\,\max\,\{x_{m}, m+1\}$,
    and $n_{m+1}$ is the smallest element of~$A_{m+1}$. It is clear that $n_{1}\,<\,n_{2}\,<\,\,{\ldots}\,\,<\,n_{m}\,<\,$.

\VA

        Let $B \,=\, \{n_{1}\,<\,n_{2}\,<\,\,{\ldots}\,\,<\,n_{m}\,<\,\,{\ldots}\,\}$;
    note that $B$ is the minimal cross section of subsequence structure ${\cal A}$.
    It is clear that the subsequence ${\xi}_{B} \,=\, (x_{n_{1}}, x_{n_{2}},\,{\ldots}\,x_{n_{m}},\,{\ldots}\,)$ of ${\xi}$
    associated with the set $B$ is strictly increasing and has limit~$+{\infty}$, as required.



        (ii) The proof in this case is similar to that in~(i); the details are  left as an exercise. %% EXERCISE?

\V

        (iii) This case is simply a restatement of the standard Bolzano-Weierstrass Theorem (see Theorem~\Ref{ThmC30.10}),
    which has already been proved.

\V


            \subsection{\small{\bf Definition}}
            \label{DefC30.15A}\IndBD{sequences}{subsequential limits of a real sequence}

\V

        Let ${\xi} \,=\, (x_{1}, x_{2},\,{\ldots}\,x_{k},\,{\ldots}\,)$ be a sequence of real numbers.
    An extended real number which can be expressed as the limit of a subsequence of ${\xi}$ is called a {\bf subsequential limit of~$\Bfm{{\xi}}$}.
    The set of all subsequential limits of a given sequence ${\xi}$ is denoted by~${\cal L}[{\xi}]$.

\V

        Note that the extended Bolzano-Weierstrass Theorem can now be phrased to say that for every real sequence ${\xi}$ the set ${\cal L}[{\xi}]$ is nonempty.

\VV


        There is another version of the Bolzano-Weierstrass Theorem which is closer to Weierstrass' original formulation.
    It helps to first introduce some terminology.

\V

            \subsection{\small{\bf Definition}}\IndBD{sets}{accumulation point of a set of reals}
            \label{DefC40.665}

\V

        Let $X$ be a set of real numbers with infinitely many elements. A real  number $c$ is said to be an {\bf accumulation point of $\Bfm{X}$},
    and the set $X$ is said to {\bf accumulate at~$\Bfm{c}$}, provided that for every ${\varepsilon}\,>\,0$
    there are infinitely many elements $x$ in $X$ such that $|x-c|\,<\,{\varepsilon}$.

\V

        {\bf Examples} (1) Let $X \,=\, \{1, 1/2, 1/3,\,{\ldots}\,1/n,\,{\ldots}\,\}$
    be the set of reciprocals of natural numbers. It is easy to see that this set accumulates at exactly one real number, namely at $c \,=\, 0$.

\V

        (2) Let $X'$ be the set from the preceding example, together with the number~$0$; thus, $X' \,=\, \{0,1,1/2,\,{\ldots}\,1/n\,{\ldots}\,\}$.
    Once again, this set accumulates at exactly one real number, namely~$0$.

\V

        (3) Every rational number is an accumulation point of the set of all irrational numbers.

\V

        (4) Let $X \,=\, {\NN}$. Then this set accumulates at no number. (Note that the definition does not allow $+{\infty}$ or $-{\infty}$ as accumulations points of a set.)

\V

        {\bf Remark} Many texts use the terminology `limit point of $X$'\IndBD{accumulation point of a set of reals}{limit point of a set}
    or `cluster point of $X$'\IndBD{accumulation point of a set of reals}{cluster point of a set} instead of `accumulation point'.


\V

        (2) Examples (1) and (2) above illustrate the fact an accumulation point of a set is allowed, but not required, to be an element of that set.

\V

        (3) The preceding remark illustrates a subtle ambiguity of language. The phrase `point of $X$' suggests
    -- for obvious reasons -- that one is speaking about an element of the set~$X$.
    Adding the modifier `accumulation' to that phrase suggests that one is speaking about an element of $X$ which has a special property.
    The same type of ambiguity arises when, for example, states that a number $M$ is the supremum of a set~$X$:
    the phrasing suggests that this statement entails that $M$ must be an element of $X$, when in fact it need not.
    Since it is unlikely that mathematicians will ever completely avoid such `abuses of language',
    the only solution is to be aware of the problem and to read definitions carefully.

\VV

            \subsection{\small{\bf Theorem} (Bolzano-Weierstrass Theorem for Sets of Real Numbers)}\IndBD{sets}{Bolzano-Weierstrass theorem for sets of reals}
            \label{ThmC40.70}

\V

        Let $X$ be a bounded infinite subset of~${\RR}$. Then $X$ has at least one accumulation point.

\V

        {\bf Proof} Let $S$ be the set of real numbers $y$ such that $y\,\,{\leq}\,\,x$ for all but a finite number of elements of the set~$X$.
    Since, by hypothesis, the set $X$ is bounded, hence bounded below, it is clear that
    $S \,\,{\neq}\,\, {\emptyset}$, since it obviously contains every lower bound of~$X$.
    Also it is clear that $S$ is a convex set which is unbounded below. Finally, the fact that $X$ is bounded above and an infinite subset of ${\RR}$
    implies that $S \,\,{\neq}\,\, {\RR}$, since every upper bound of $X$ fails to satisfy the defining property of~$S$.
    In summary, the set $S$ satisfies the hypothses of the Bolzano's Right-Endpoint Principle.
    It follows that there exists a real number $B$ such that either $S \,=\, (-{\infty},B)$ or $S \,=\, (-{\infty},B]$.
    It is easy to see that $B$ is an accumulation point of the original set~$X$.

\V

        {\bf Remarks} (1) There is an obvious extension of the preceding theorem to unbounded infinite subsets of~${\RR}$
    if one allows $ \,{\pm}\, {\infty}$ as possible accumulation points, but it hardly seems worth the effort.

\V

        (2) It would have been faster to prove this result as a simple corollary of the sequential version of the Bolzano-Weierstrass Theorem,
    but it seemed appropriate to use Bolzano's Principle on a theorem that includes his name.
    One can easily modify this argument to prove the `sequential' version of the Bolzano-Weierstrass theorem using Bolzano's Principle.

%EXERCISE

\V

        (3) The statement of the Bolzano-Weierstrass Theorem for bounded infinite subsets makes sense in any ordered field.
    It is easy to show that in that context it is equivalent to all the other versions of `Completeness' studied in Chapter~\Ref{ChaptB},
    and thus could have been used as the `Completeness Axiom'. One advantage in doing that would be
    that its statement probably requires the least preparation of all the candidates suggested for that axiom in Chapter~\Ref{ChaptB}.
    The biggest disadvantage to using it is that it is even less `obviously true', i.e., `axiomatic', than, say, the Least-Upper-Bound Principle.

\VV

        Parts (i) and (ii) of Theorem~\Ref{ThmC40.60}, the Extended Bolzano-Weierstrass Theorem for Sequences, suggest a natural question:
    Does every {\em bounded} sequence of real numbers have at least one {\em strictly} monotonic subsequence?
    A little thought makes it clear that the answer is `No'; for example, a constant real sequence has no strictly monotonic subsequence.
    However, since such a sequence is monotonic, one is led to a modified question: does every bounded sequence have a {\em monotonic} subsequence?
    The answer to that question is provided by the next result.

\V

            \subsection{\small{\bf Theorem} (The Monotonic-Subsequences Theorem)}
            \label{ThmC30.15}\IndBD{sequences}{monotonic-subsequences theorem}

\V

        Every sequence of real numbers has a monotonic subsequence.

        More precisely, if sequence does not have a constant subsequence, then it has a strictly monotonic subsequence.

\V

        {\bf Proof} Parts (i) and (ii) of the preceding theorem verify the truth of the statement in the case of unbounded sequences.
    Indeed, the monotonic subsequences in that case can be chosen to be {\em strictly} monotonic.

        Thus, let us now assume that ${\xi} \,=\, (x_{1}, x_{2},\,{\ldots}\,x_{k},\,{\ldots}\,)$ is a bounded real sequence.
    If ${\xi}$ has a constant subsequence, that subsequence is automatically monotonic, so the result follows.
    Thus, assume for the rest of this proof that ${\xi}$ does not have a constant subsequence.
    Equivalently, assume that there is no real number $c$ such that $x_{k} \,=\, c$ for infinitely many indices~$k$.

\V

        \underline{Special Case} Assume that ${\xi}$ is convergent, with limit $L$ in~${\RR}$.
    Let $B$ be the set of indices $k$ such that $x_{k}\,>\,L$, and let $C$ be the set of indices $k$ such that $x_{k}\,<\,L$.
    By the assumption that ${\xi}$ does not have a constant subsequence, at least one of the sets $B$ or $C$ must be an infinite subset of~${\NN}$.
    To be definite, assume that $B \,=\,
    \{k_{1}\,<\,k_{2}\,<\,\,{\ldots}\,\,<\,k_{m}\,<\,\,{\ldots}\,\}$ is infinite.
    Let ${\zeta} \,=\, \{z_{1}, z_{2},\,{\ldots}\,z_{m},\,{\ldots}\,\}$ be given by the formula
    $z_{m} \,=\, 1/(x_{k_{m}}-L)$, so each term $z_{m}$ is positive. The hypothesis that
    $L \,=\, \lim_{k \,{\rightarrow}\, {\infty}} x_{k} \,=\, L$ implies that $\lim_{m \,{\rightarrow}\, {\infty}} z_{m} \,=\, +{\infty}$.
    In particular, Part~(i) of the preceding theorem implies that ${\zeta}$ has a subsequence
    ${\tau} \,=\, (z_{k_{m_{1}}}, z_{k_{m_{2}}},\,{\ldots}\,z_{k_{m_{n}}}\,{\ldots}\,)$
    which is strictly increasing. It follows easily, from the usual order properties of ${\RR}$,
    that the corrresponding subsequence $(x_{k_{m_{1}}}, x_{k_{m_{2}}},\,{\ldots}\,x_{k_{m_{n}}},\,{\ldots}\,)$
    of ${\xi}$ is strictly decreasing; in particular, it is strictly monotonic, as required.

        Suppose, instead, that the set $C \,=\, \{j_{1}\,<\,j_{2}\,<\,\,{\ldots}\,\,<\,j_{n}\,<\,\,{\ldots}\,\}$
    is infinite. Let ${\sigma} \,=\, (s_{1}, s_{2},\,{\ldots}\,s_{k},\,{\ldots}\,)$, be the sequence given by
    $s_{k} \,=\, -x_{k}$ for each index~$k$; that is, ${\sigma} \,=\, -{\xi}$. Then ${\sigma}$ converges to $L' \,=\, -L$,
    and $C$ is the set of all indices for which $s_{k}\,>\,L'$. It follows from what was just proved, but now applied to the sequence ${\sigma}$ and limit~$L'$,
    that ${\sigma}$ has a strictly decreasing subsequence. Since $x_{k} \,=\, -s_{k}$,
    it follows that ${\xi}$ has a subsequence which strictly increasing, hence strictly monotonic, as required.

\V

        \underline{General Case} If ${\xi}$ is bounded then, by the Bolzano-Weierstrass Theorem, it has a subsequence ${\eta}$ which is convergent.
    By applying the results of the `Special Case' above to ${\eta}$, one sees that ${\eta}$ has a subsequence ${\sigma}$ which is strictly monotonic.
    But ${\sigma}$ is then a {\em sub}subsequence of ${\xi}$, hence a subsequence of ${\xi}$, which is strictly monotonic, as required.

\V

        {\bf Remark} There are proofs of the Monotonic-Subsequences Theorem which are shorter and, in a sense, more elegant.
    An advantage of the proof given here is that it starts by proving a special case in which the claimed result is intuitively obvious,
    namely the case ${\xi}$ is unbounded above. It then reduces all the remaining cases to this one,
    using the Bolzano-Weierstrass Theorem as the main tool. The technique of first proving the desired result in a simple special case,
    then reducing the other cases to this special case, is a powerful tool in mathematics.


\VV

        The next results provide useful tools for showing whether given sequence has a limit.


            \subsection{\small{\bf Theorem}}
            \label{ThmC30.20}

\V

        Let ${\xi} \,=\, (x_{1},x_{2},\,{\ldots}\,)$ be a sequence of real numbers, and let $L$ be an extended real number.


\V

        (a) A necessary and sufficient condition for ${\xi}$ to \underline{not} have $L$ as a limit
    is that there exist an extended real number $L' \,\,{\neq}\,\, L$ and a subsequence ${\tau} \,=\, (t_{1}, t_{2},\,{\ldots}\,t_{n}, \,{\ldots}\,)$
    of ${\xi}$ such that $\lim_{n \,{\rightarrow}\, {\infty}} t_{n} \,=\, L'$.

\V


        (b) A necessary and sufficient condition for ${\xi}$ to have a limit is that all subsequences of ${\xi}$ which has the same limit.

\V

        \underline{Proof}

\V

        (a)  \underline{The Condition is Necessary} Suppose that ${\xi}$ does not have limit~$L$.
    Then either Statement~A or Statement~B of Theorem~\Ref{ThmC40.80} must fail to hold.
    If Statement~A fails to hold, then there exists a real number $y\,<\,L$ such that $x_{k}\,\,{\leq}\,\,y$ for infinitely many indices~$k$.
    Let $A$ be the set of such indices, so that $A$ is an infinite subset of~${\NN}$,
    and let ${\zeta} \,=\, (z_{1},z_{2},\,{\ldots}\,z_{k},\,{\ldots}\,)$ be the corresponding subsequence ${\xi}_{A}$ of~${\xi}$.
    By the Extended Bolzano-Weierstrass theorem, the sequence ${\zeta}$ has a subsequence
    ${\tau} \,=\, (t_{1}, t_{2},\,{\ldots}\,t_{n},\,{\ldots}\,)$
which has a limit; call this limit~$L'$.
    Since $z_{m}\,\,{\leq}\,\,y$ for all~$m$, it follows that $t_{n}\,\,{\leq}\,\,y$ for all~$n$, and thus $L'\,\,{\leq}\,\,y\,<\,L$.
    Since ${\tau}$ is also a subsequence of the original sequence~${\xi}$, it follows
    from Part~(c) of Theorem~\Ref{ThmC20.10A} that ${\xi}$ does not have $L$ as limit. A similar argument works if, instead, Statement~B fails to hold.

        \underline{The Condition is Sufficient} Indeed, suppose that there is a subsequence ${\tau}$ of ${\xi}$ which has limit $L' \,\,{\neq}\,\, L$.
    Then, by Part~(a) of Theorem~\Ref{ThmC20.10A}, the subsequence ${\tau}$ cannot have limit~$L$,
    hence by Part~(c) of the same theorem the original sequence ${\xi}$ also cannot have limit~$L$.

\V

        (b) \underline{The Condition is Necessary} This is simply the statement of Part~(c) of Theorem~\Ref{ThmC20.10A}.

        \underline{The Condition is Sufficient} Suppose that every subsequence of ${\xi}$ which has a limit has the same limit; call it~$L$.
    Then, by the `Necessary' portion of Part~(a), it is {\em not} the case that ${\xi}$ does {\em not} converge to $L$.
    In other words, ${\xi}$ {\em does} converge to $L$.

\V

        {\bf Remark} The converse of Part~(c) of Theorem~\Ref{ThmC20.10A} can be expressed as follows:

\VA

         \h If every subsequence of ${\xi}$ has limit~$L$, then ${\xi}$ has limit~$L$.

\VA

\noindent This coverse is, in fact, correct, but for trivial reasons: if {\em every} subsequence of ${\xi}$ has limit~$L$,
    then~${\xi}$, being itself a subsequence of~${\xi}$, has limit~$L$.

        In contrast, the hypotheses of Part~(b) of Theorem~\Ref{ThmC30.20} involve only a special type of subsequences of~${\xi}$.
    The proof of Part~(b), far from being trivial, involves the results of Part~(a), whose proof in turn uses the Extended Bolzano-Weierstrass theorem.


\VV

                \section{{\bf The Cauchy Criterion for Convergent Real Sequences}}
                \label{SectC70}\IndB{ZZ Sections}{\Ref{SectC70} The Cauchy Criterion for Convergent Real Sequences}

        {\bf Introduction}

\V

        The definition of `limit of a real sequence' seems to require that one have a candidate $L$ in mind for the purported limit.
    Indeed, most of the results and examples presented so far involve, in perhaps a hidden way, a plausible candidate for~$L$.
    For example, in the proof of the Monotonic-sequences Principle, the candidate is the supremum of the term-set (when the sequence is monotonic up).
    Likewise, in the proof of the (standard) Bolzano-Weierstrass Theorem the candidate is the unique point $c$ determined by the bisection sequence.
    (The main exceptions are those limits which are computed using algebra to reduce
    a complicated limit problem to a finite number of simpler problems whose answers are already known.)

        The present section is devoted to a criterion for convergence which does not require a `candidate' in advance.
    Note, however, that it is a criterion for {\em convergence}, not for `existence of a limit', since it applies only to bounded sequences.

\V

        \underline{Preliminary Observation}\,Anyone who deals with the convergence of sequences of real numbers is familiar with the following fact:

\VA

        \h If a sequence ${\xi} \,=\, (x_{1}, x_{2}\,{\ldots}\,)$ of real numbers is convergent,
    then the distance between consecutive terms approaches $0$; that is, 
        \begin{equation}
        \label{EqnC.10}
        \lim_{k \,{\rightarrow}\, {\infty}} (x_{k+1}-x_{k}) \,=\, 0.
        \end{equation}

\VA

    Actually, the following -- apparently stronger -- result holds.

\V

            \subsection{\small{\bf Theorem}}
            \label{ThmC70.10}

\V

        Suppose that ${\xi} \,=\, (x_{1},x_{2},\,{\ldots}\,x_{k}\,{\ldots}\,)$ is a sequence of real numbers.
    Let $m$ be a natural number, and consider the sequence ${\delta}_{({\xi}; m)} \,=\, (d_{1},d_{2},\,{\ldots}\,d_{k},\,{\ldots}\,)$ defined by the rule
        \begin{displaymath}
        d_{k} \,=\, x_{m+k}-x_{k} \mbox{ for each $k$ in ${\NN}$}.
        \end{displaymath}
    That is, ${\delta}_{({\xi};m)} \,=\, (x_{m+1} - x_{m},\, x_{m+2} - x_{m},\, x_{m+3}-x_{m},\,\,{\ldots}\,)$.
    
        \underline{Conclusion}
    A necessary condition for the sequence ${\xi}$ to be convergent is that for each $m$ the associated sequence ${\delta}_{({\xi};m)}$ converges to $0$;
    that is, for each $m$ in ${\NN}$ one has
        \begin{equation}
        \label{CondC.20}
        \lim_{k \,{\rightarrow}\, {\infty}} (x_{m+k} - x_{k}) \,=\, 0
        \end{equation}

\V

        {\bf Proof} Suppose that ${\xi}$ is convergent, and let $L \,=\, \lim_{k \,{\rightarrow}\, {\infty}} x_{k}$.
    Let ${\zeta} \,=\, (z_{1},z_{2},\,{\ldots}\,z_{k},\,{\ldots}\,)$ be the sequence given by the rule $z_{k} \,=\, x_{m+k}$ for each $k$ in ${\NN}$.
    Note that ${\zeta}$ is a subsequence of ${\xi}$; thus, by Part~(c) of Theorem~\Ref{ThmC20.10A}, the sequence ${\zeta}$ also converges to $L$; that is, $\lim_{k \,{\rightarrow}\, {\infty}} x_{m+k} \,=\, L$.
    Note also that $d_{k} \,=\, z_{k}-x_{k}$ for each $k$ in ${\NN}$, so by Theorem~\Ref{ThmC60.30}
    it follows that the sequence ${\delta}_{({\xi};m)}$ is also convergent, and
        \begin{displaymath}
        \lim_{k \,{\rightarrow}\, {\infty}} d_{k} \,=\, \left(\lim_{k \,{\rightarrow}\, {\infty}} x_{m+k}\right) - \left(\lim_{k \,{\rightarrow}\, {\infty}} x_{k}\right) \,=\, L-L \,=\, 0,
        \end{displaymath}
    as required.

\V

            \subsection{\small{\bf Remarks}}
            \label{RemrkC70.12}

\V

\hspace*{\parindent}(1) The phrase `apparently stronger' is added before the word `result' above.
    It is true that Theorem~\Ref{ThmC70.10} does have a stronger conclusion than Equation~\Ref{EqnC.10} from the same hypothesis;
    indeed, that equation is the special case $m \,=\, 1$ of Condition~\Ref{CondC.20}.
    However, it is a simple exercise to show that if Equation~\Ref{EqnC.10} holds, then so does Condition~\Ref{CondC.20} for {\em every}~$m$.

%% Make it an EXERCISE

\V

        (2) It is useful to reformulate Condition~\Ref{CondC.20} above as follows, so the definition of `convergence' is invoked directly:


        \begin{equation}
        \label{CondC.30}
        \mbox{For each ${\varepsilon}\,>\,0$ and $m$ in ${\NN}$,  there is $N$ in ${\NN}$ such that if $k\,\,{\geq}\,\,N$, then $|x_{k+m}-x_{k}|\,<\,{\varepsilon}$}
        \end{equation}

\V

        The preceding theorem provides a {\em necessary} condition for a sequence to be convergent.
    The next example, when combined with Remark~(1) above, shows that, unfortunately, this condition is not {\em sufficient} for convergence.
    (`Unfortunately': many students lose points on calculus exams because they believe that the condition {\em is} sufficient to prove convergence.)

\V

            \subsection{\small{\bf Example}}
            \label{ExampC70.15}

\V

        Let $x_{n} \,=\, \sqrt{n}$. (The existence of square roots is proved in Theorem~\Ref{ThmC20.70}.)
    It is clear that this sequence is unbounded, and thus not convergent. Nevertheless, one has
        \begin{displaymath}
        x_{k+1}-x_{k} \,=\, \sqrt{k+1} - \sqrt{k}
     \,=\, 
        \frac{(\sqrt{k+1} - \sqrt{k})\,(\sqrt{k+1} + \sqrt{k})}{\sqrt{k+1} + \sqrt{k}}
     \,=\, 
        \frac{(\sqrt{k+1})^{2} - (\sqrt{k})^{2}}{\sqrt{k+1} + \sqrt{k}}
     \,=\, 
        \end{displaymath}
        \begin{displaymath}
        \frac{(k+1) - k}{\sqrt{k+1}  \sqrt{k}}
     \,=\, 
        \frac{1}{\sqrt{k+1} + \sqrt{k}}
        \end{displaymath}
    It is then clear that $\lim_{k \,{\rightarrow}\, {\infty}} (x_{k+1} - x_{k}) \,=\, 0$.


%% EXERCISE Construct an example of a {\em bounded} sequence which satisfies CondC.20 but fails to be convergent; use `Bisections. Also do the 1 + 1/2 + ... + 1/n example.

\VV


        There is another condition, quite similar to the alternate formulation given in Condition~\Ref{CondC.30} above,
    which turns out to be not just necessary, but also sufficient, for a sequence of real numbers to be convergent.
    It is traditionally called the `Cauchy Criterion for the convergence of a sequence of reals',
    although others stated versions of it earlier; for example, Bolzano published it in~$1817$. %% Euler was another?

\V

            \subsection{\small{\bf Definition} (The Cauchy Criterion and Cauchy Sequences in ${\RR}$)}
            \label{DefC70.40}\IndBD{sequences}{Cauchy criterion for convergence in ${\RR}$}
\IndBD{sequences}{Cauchy sequences in ${\RR}$}

        A sequence ${\xi} \,=\, (x_{1},x_{2},\,{\ldots}\,)$ of real numbers is said to satisfy the {\bf Cauchy criterion} provided
        \begin{equation}
        \label{CondC.100}
        \mbox{For every ${\varepsilon}\,>\,0$, there exists a number $N$ in ${\NN}$ such that if $k\,\,{\geq}\,\,N$,
    then $|x_{k+m}-x_{k}|\,<\,{\varepsilon}$ for every $m$ in~${\NN}$}
        \end{equation}
    If this occurs then one says that ${\xi}$ is a {\bf Cauchy sequence (of real numbers)}.

\VV

        {\bf Remarks} (1) Compare Condition~\Ref{CondC.100} given here with Condition~\Ref{CondC.30} above.
    The only difference is the location of the phrase `for every $m$ in~${\NN}$'.
    Nevertheless, this apparent `small' change of phrasing has a major impact on the meaning of the conditions.
    Indeed, Condition~\Ref{CondC.30} says that if both ${\varepsilon}$ and $m$ are chosen in advance,
    then one can chose a value of $N$, which depends on both ${\varepsilon}$ and $m$, for which the given inequality holds.
    In contrast, Condition~\Ref{CondC.100} requires that $N$ can be chosen in terms of ${\varepsilon}$, but its choice should not depend on~$m$.

\V

        (2) The Cauchy criterion is often phrased in the following alternate, but equivalent, form:
        \begin{equation}
        \label{CondC.100A}
        \mbox{For every ${\varepsilon}\,>\,0$ there exists a natural number $N$ such that if $j,l\,\,{\geq}\,\,N$ then $|x_{j}-x_{l}|\,<\,{\varepsilon}$}
        \end{equation}
    The equivalence of this form with that given in the definition is easy to verify. %% EXERCISE?

\VV

        The importance of the concept of `Cauchy sequence' in analysis is explained by the following result.

\V

            \subsection{\small{\bf Theorem} (The Cauchy Convergence Theorem for Real Sequences)}\IndA{Cauchy tests for convergence}\IndBD{Cauchy tests for convergence}{Cauchy's convergence theorem for sequences of reals}
            \label{ThmC70.20}

        Suppose that ${\xi} \,=\, (x_{1},x_{2},\,{\ldots}\,)$ is a sequence of real numbers.
    Then a necesary and sufficient condition for ${\xi}$ to be convergent is that ${\xi}$ be a Cauchy sequence.

\V

        \underline{Proof}:

\V

        (a) (`Necessity') Suppose that ${\xi}$ converges to some number $L$.
    Let ${\varepsilon}\,>\,0$ be given, and choose $N$ in ${\NN}$ large enough that if $j$ satisfies $j\,\,{\geq}\,\,N$ then $|x_{j}-L|\,<\,{\varepsilon}/2$.
    In particular, suppose that $k\,\,{\geq}\,\,N$ and $m{\in}{\NN}$, so that $m+k\,\,{\geq}\,\,N$ as well. Then one has
        \begin{displaymath}
        |x_{m+k}-L|\,<\,\frac{{\varepsilon}}{2} \mbox{ and } |x_{k}-L|\,<\,\frac{{\varepsilon}}{2}.
        \end{displaymath}
    Combine this with the Triangle Inequality to get
        \begin{displaymath}
        |x_{m+k}-x_{k}|\,\,{\leq}\,\,|x_{m+k}-L| + |L-x_{k}|\,<\,\frac{{\varepsilon}}{2} + \frac{{\varepsilon}}{2} \,=\, {\varepsilon};
        \end{displaymath}
    That is, ${\xi}$ satisfies the Cauchy criterion, as required.

\V

        (b) (`Sufficiency') First note that ${\xi}$ satisfies the Cauchy criterion, it is a bounded sequence.
    More precisely, let ${\varepsilon}_{0} \,=\, 1$. (In reality, one can choose 
${\varepsilon}_{0}$ to be any positive number;
    it is mere convention to choose it to be the best-known positive number, namely~$1$.)
    Choose $N$ in ${\NN}$ large enough that if $k\,\,{\geq}\,\,N$ then $|x_{m+k} - x_{k}|\,<\,1$ for all $m$ in~${\NN}$.
    In particular, it follows that if $n\,\,{\geq}\,\,N$ then $|x_{n} - x_{N}|\,<\,1$.
    Indeed, if $n \,=\, N$ the inequality is trivially true, while if $n\,>\,N$ it follows from the definition of $N$, using $k \,=\, N$ and $m \,=\, n-N$.
    Now use the Triangle Inequality to get
        \begin{displaymath}
        |x_{n}| \,=\, |(x_{n} - x_{N}) + x_{N}|\,\,{\leq}\,\,|x_{n}-x_{n}| + |x_{N}|\,<\,1+|x_{N}| \mbox{ for all $n\,\,{\geq}\,\,N$}
        \end{displaymath}
    It follows from Part~(c) of Theorem~\Ref{ThmB30.15}, `A sequence which is eventually bounded is bounded',
    that the given sequence is bounded, as claimed.

        It follows from the Bolzano-Weierstrass Theorem that the given sequence ${\xi}$ has a convergent subsequence.
    Let $A \,=\, \{k_{1}\,<\,k_{2}\,<\,\,{\ldots}\,\,<\,k_{n}\,<\,\,{\ldots}\,\}$
    be an infinite subset of ${\NN}$ such that the subsequence ${\zeta} \,=\,(z_{1},z_{2},\,{\ldots}\,z_{n},\,{\ldots}\,) \,=\, {\xi}_{A}$
    is convergent, and let $L$ the corresponding limit. In light of Part~(c) of Theorem~\Ref{ThmC20.10A},
    it is clear that $L$ is the only reasonable candidate to be the limit of the full sequence~${\xi}$.
    To see that the original sequence ${\xi}$ does, in fact, converge to~$L$, first note that, by the Triangle Inequality, for all $k$ and $m$ one has
        \begin{displaymath}
        |L-x_{k}|\,\,{\leq}\,\,|L-x_{m+k}| + |x_{m+k} - x_{k}| \h ({\ast})
        \end{displaymath}
    Now suppose that ${\varepsilon}\,>\,0$ is given. Let $N$ in ${\NN}$ be large enough that if $k\,\,{\geq}\,\,N$, then $|x_{m+k} - x_{k}|\,<\,{\varepsilon}/2$.
    If $k\,\,{\geq}\,\,N$ choose $m$ so that $m+k$ is of the form $k_{n}$ for some index~$n$,
    so that $x_{m+k}$ is a term of the subsequence~${\zeta}$, and so that $m$ is large enough to guarantee that $|L-x_{m+k}|\,<\,{\varepsilon}/2$.
    (The existence of such $m$ follows from the convergence of the subsequence ${\zeta}$ to~$L$.)
    For such a choice of $m$, Inequality~$({\ast})$ implies that for every $k\,\,{\geq}\,\,N$ one has
        \begin{displaymath}
        |L-x_{k}|\,<\,\frac{{\varepsilon}}{2} + \frac{{\varepsilon}}{2} \,=\, {\varepsilon}
        \end{displaymath}
    The desired result follows.

% EXERCISE Give alternate proof using Theorem~\Ref{ThmC30.20}

\V

        {\bf Remark} The theoretical importance of the Cauchy Criterion is that it applies to all sequences,
    and its hypotheses can be checked knowing only the terms of the sequence, without needing a candidate for the limit.

\VV


        The next result illustrates a type of situation in which the Cauchy Criterion can be used.


\V

            \subsection{\small{\bf Theorem}}
            \label{ThmC70.50}

\V

        Suppose that ${\xi} \,=\, (x_{1},x_{2},\,{\ldots}\,)$ is a sequence of real numbers with the property that there exists a real number ${\lambda}$,
    with $0\,\,{\leq}\,\,{\lambda}\,<\,1$, such that

        \begin{equation}
        \label{IneqC.110}
        |x_{k+2}-x_{k+1}|\,\,{\leq}\,\,{\lambda}|x_{k+1}-x_{k}| \mbox{ for all sufficiently large $k$ in ${\NN}$}.
        \end{equation}
    Then the sequence ${\xi}$ is a Cauchy sequence (and thus is convergent).


\V

        \underline{Proof}: \underline{Case 1}: Assume that Inequality~\Ref{IneqC.110} holds for {\em all} $k$ in ${\NN}$.

        First note that the result is obvious if ${\lambda} \,=\, 0$ or if $x_{2} \,=\, x_{1}$.
    Indeed, if ${\lambda} \,=\, 0$ then one clearly has $|x_{k+2}-x_{k+1}| \,=\, 0{\cdot}|x_{k+1}-x_{k}|$ for all $k$ in ${\NN}$,
    so that $x_{2} \,=\, x_{3} \,=\, x_{m} \,=\, \,{\ldots}\,$ for all $m\,\,{\geq}\,\,2$.
    Likewise, if $x_{2} \,=\, x_{1}$ then it is equally clear that ${\xi}$ is a constant sequence.

        Thus, assume that $x_{2} \,\,{\neq}\,\, x_{1}$ and that ${\lambda}$ satisfies $0\,<\,{\lambda}\,<\,1$.
    It is easy to see, by repeatedly using the given hypothesis on ${\xi}$, that
        \begin{displaymath}
        |x_{k+1}-x_{k}|\,\,{\leq}\,\,{\lambda}^{k-1}|x_{2}-x_{1}| \mbox{ for each $k$ in ${\NN}$}
        \end{displaymath}
    Indeed, let $A$ be the set of all natural numbers $k$ for which this inequality is valid.
    It is clear that $1{\in}A$, since for $k \,=\, 1$ the condition to be verified becomes
        $|x_{2}-x_{1}|\,\,{\leq}\,\,{\lambda}^{0}|x_{2}-x_{1}|$; and since ${\lambda}^{0} \,=\, 1$ if ${\lambda} \,\,{\neq}\,\, 0$,
    this in turn reduces to $|x_{2}-x_{1}|\,\,{\leq}\,\,1{\cdot}|x_{2}-x_{1}|$, which is certainly true.
    Next, suppose that $k{\in}A$. Then
        \begin{displaymath}
        |x_{k+2}-x_{k+1}|\,\,{\leq}\,\,{\cdot}|x_{k+1}-x_{k}|\,\,{\leq}\,\,{\lambda}{\cdot}\left({\lambda}^{k-1}|x_{2}-x_{1}|\right)
     \,=\, {\lambda}^{k}|x_{2}-x_{1}|.
        \end{displaymath}
    Thus $(k+1)$ is also in $A$.
    Now the Principle of Mathematical Induction implies that $A \,=\, {\NN}$, and the desired inequality follows.

        To show that the sequence ${\xi}$ is Cauchy, note that if $k$ and $m$ are in ${\NN}$, then use the preceding inequality, together with the classic `Add-and-Subtract Trick' (see Theorem~\Ref{ThmB10.100}) to get
        \begin{displaymath}
        |x_{k+m}-x_{k}| \,=\, |(x_{k+m}-x_{k+m-1}) + (x_{k+m-1}-x_{k+m-2}) + \,{\ldots}\, + (x_{k+1}-x_{k})|\,\,{\leq}\,\,
        \end{displaymath}
        \begin{displaymath}
        |x_{k+m}-x_{k+m-1}| + \,{\ldots}\, + |x_{k+1}-x_{k}|\,\,{\leq}\,\,
         \left({\lambda}^{k+m-2} + {\lambda}^{k+m-3} + \,{\ldots}\, + {\lambda}^{k-1}\right)|x_{2}-x_{1}|.
        \end{displaymath}
    By the results of Proposition~\Ref{ThmB25.80} one can write
        \begin{displaymath}
        {\lambda}^{k+m-2} + {\lambda}^{k+m-3} + \,{\ldots}\, + {\lambda}^{k-1} \,=\, {\lambda}^{k-1}({\lambda}^{m-1} + {\lambda}^{m-1}+\,{\ldots}\,+1) \,=\, {\lambda}^{k-1}\left(\frac{{\lambda}^{m}}{1-{\lambda}}\right) \,=\, 
    \frac{{\lambda}^{k+m-1}}{1-{\lambda}}.
        \end{displaymath}
    Thus
        \begin{displaymath}
        |x_{k+m}-x_{k}|\,\,{\leq}\,\,\left(\frac{{\lambda}^{k+m-1}}{1-{\lambda}}\right)|x_{2}-x_{1}|\,\,{\leq}\,\,\left(\frac{{\lambda}^{k}}{1-{\lambda}}\right)|x_{2}-x_{1}|
        \end{displaymath}
    It follows from Corollary~\Ref{CorC20.15} that for every ${\varepsilon}\,>\,0$ there exists $B$ such that if $k\,\,{\geq}\,\,B$ then ${\lambda}^{k}\,<\,{\varepsilon}\left(\frac{1-{\lambda}}{|x_{2}-x_{1}|}\right)$.
    The fact that ${\xi}$ is a Cauchy sequence now follows.

\V

        \underline{Case 2}: Now consider the general case.
    That is, assume that there exists a natural number $m$ such that Inequality~\Ref{IneqC.110} holds for all $k\,\,{\geq}\,\,m$.
    Let ${\tau} \,=\, (t_{1},t_{2},\,{\ldots}\,)$ be the sequence $(x_{m},x_{m+1},\,{\ldots}\,)$;
    that is $t_{j} \,=\, x_{m+j-1}$ for each $j$ in ${\NN}$. It is clear that the sequence ${\tau}$ satisfies the hypotheses for Case~1, and thus ${\tau}$ is a Cauchy sequence.

\VV

            \subsection{\small{\bf Example}}
            \label{ExampC70.55}

\V

        Recall Heron's `Divide-and-Average' method for computing square roots; see Theorem~\Ref{ThmC20.70}:
    If $C$ and $x_{1}$ are positive real numbers, then Heron's method produces an infinite sequence ${\xi} \,=\, (x_{1}, x_{2},\,{\ldots}\,x_{k},\,{\ldots}\,)$
such that for each index $k$ one has
    ${\displaystyle x_{k+1} \,=\, \frac{1}{2}\,\left(x_{k} + \frac{C}{x_{k}}\right)}$. We have already seen that $x_{k}\,>\,0$ for each~$k$.
    One can use the preceding theorem to give an alternate proof that the sequence ${\xi}$ is convergent and its limit is~$\sqrt{C}$.

        More precisely, note that
        \begin{displaymath}
        x_{k+2} - x_{k+1} \,=\, \frac{1}{2}\,\left(x_{k+1} + \frac{C}{x_{k+1}}\right)
    -
        \frac{1}{2}\,\left(x_{k} + \frac{C}{x_{k}}\right)
     \,=\, 
        \end{displaymath}
        \begin{displaymath}
        \frac{1}{2}\,\left(x_{k+1} - x_{k} + C\,\left(\frac{1}{x_{k+1}} - \frac{1}{x_{k}}\right)\right)
     \,=\, 
        \frac{1}{2}\,(x_{k+1}-x_{k})\,\left(1-\frac{C}{x_{k+1}\,x_{k}}\right)
        \end{displaymath}
    From the definition of $x_{k+1}$ in terms of $x_{k}$ one computes that $2\,x_{k+1}\,x_{k} \,=\, x_{k}^{2} + C\,>\,C$,
    so that $2\,>\,C/(x_{k+1}\,x_{k})\,>\,0$. From this one gets $|1-C/(x_{k+1}\,x_{k})|\,<\,1$ and thus
        \begin{displaymath}
        \left|x_{k+2} - x_{k+1}\right|\,<\,\frac{1}{2}\,\left|x_{k+1}-x_{k}\right|
        \end{displaymath}
    Thus, the hypotheses of the previous theorem are satisfied, with ${\lambda} \,=\, 1/2$, so the sequence ${\xi}$ is convergent.
    Let $L$ be its limit. Since each $x_{k}\,>\,0$, it follows that $L\,\,{\geq}\,\,0$.
    If it were the case that $L \,=\, 0$, it would follow that $x_{k} + A/x_{k}$ would diverge to~$+{\infty}$, which is impossible since ${\xi}$ is convergent.
    Thus $L\,>\,0$, so since $\lim_{k \,{\rightarrow}\, {\infty}} x_{k} \,=\, \lim_{k \,{\rightarrow}\, {\infty}} x_{k+1} \,=\, L$,
    it follows that ${\displaystyle L \,=\, \frac{1}{2}\,\left(L + \frac{C}{L}\right)}$,
    hence $2\,L^{2} \,=\, L^{2} + C$; that is, $L^{2} \,=\, C$. In other words, $L$ is the positive square root of~$C$, as expected.



\VV

%%%
\begin{quotation}
{\footnotesize \underline{\Note}\IndB{\notes}{on what Cauchy and Bolzano knew about sequences} (on what Cauchy and Bolzano knew about sequences)

\V

        Because of the length of this {\Note}, the reader should recall that nothing in any
    {\Note} is needed for the understanding of the main body or the appendices of {\ThisText}. 
    In particular, this {\Note} is of a speculative nature; for example, it attempts to read the minds of Cauchy and Bolzano, both of whom are long dead.
    Thus, even more than is always the case, feel free to ignore it.

        Most of the remarks below on what Cauchy and Bolzano may have been thinking are based on the following sources:

        \h (1) Cauchy's {\em Cours d'analyse} of $1821$.

        \h (2) Bolzano's {\em Rein analytischer Beweis des Lehrsatzes\,{\ldots}\,} of $1817$.

\V

        {\bf On What Cauchy Knew}\,With few exceptions, such as Part~(a) of Corollary~\Ref{CorC40.85} in {\ThisText},
    Cauchy treats the basic theory of limits of real sequences as if it was all `well known' to his readers;
    so well known, in fact, that he does not provide explicit statements, much less proofs,
    of many of the important results such as the Sum and Product Rules for convergent sequences,
    as well as what we now call the (extended) `Bolzano-Weierstrass Theorem for Real Sequences'.
    As for the theorem that a real sequence is convergent if, and only if, it is what we call a `Cauchy sequence',
    Cauchy does at least clearly define that concept and state the theorem; but he gives no clue about how to prove it.

        How could Cauchy think of such facts as being obvious, when modern treatments of analysis,
    such as {\ThisText}, work hard to state them explicitly and to prove them rigorously?

        In the case of the Product Rule for convergent sequences, for instance, it can be argued that Cauchy is correct:
    it {\em is} `obvious' that the rule is valid. Indeed, a common statement of that rule
    is that if $a_{k}$ is close to $A$ and $b_{k}$ is close to~$B$, then the product $a_{k}\,b_{k}$ is close to the the product $A\,B$.
    At the present time, as in Cauchy's time, the truth of that statement is accepted without thinking.
    For instance, if one uses a calculator to compute $\sqrt{2}$ times $\sqrt{3}$,
    one knows that the calculator does not use the exact values of $\sqrt{2}$ and $\sqrt{3}$; it must round off after a some number of decimal places.
    Nevertheless, one accepts the calculator's answer as being very close to the true value~$\sqrt{6}$.

        What about the Bolzano-Weierstrass Theorem? In Chapter~VI of his {\cal Cours d'analyse},
    Cauchy considers sequences of the form ${\xi} \,=\, (x_{1}, x_{2},\,{\ldots}\,x_{k},\,{\ldots}\,)$,
    where $x_{k} \,=\, \sqrt[k]{|A_{k}|}$ for some quantities~$A_{k}$. He then considers explicitly
    the set of all subsequential limits of such a sequence (see Definition~\Ref{DefC30.15A}),
    and he makes important applications of them. (We consider these applications later in {\ThisText}.)
    Cauchy's applications make no sense unless every sequence of reals has a subsequence with a limit. But how could he think this fact is `obvious'?
    What follows is a speculative attempt to read Cauchy's mind.

        First of all, every mathematician of the time would have believed the truth of the extended Bolzano-Weierstrass Theorem for {\em unbounded} sequences,
    since all it says is that an unbounded sequence has a subsequence which diverges to one of the infinities, a fact that really {\em is} obvious.
    Thus, the issue reduces to the case of bounded sequences; that is, to the standard formulation of the Bolzano-Weierstrass theorem.
    One approach is to give the standard proof found earlier in {\ThisText}, using a Bisection-Sequence argument.
    Such an argument appears to be well within Cauchy's abilities, since he uses similar arguments elsewhere in his {\em Cours d'analyse}.
    However, what follows is an elementary proof which does make the (standard) Bolzano-Weierstrass theorem seem `obviously true'.
    Indeed, it requires only a basic understanding of how the standard decimal representation of real numbers works, together with the following:

    \underline{Fact}\, Since there are only ten decimal digits, in any infinite list of decimal digits at least one such digit must appear infinitely often.

\V

        {\bf An Elementary Proof of the Standard Bolzano-Weierstrass Theorem} Recall that we are given a bounded sequence
    ${\xi} \,=\, (x_{1}, x_{2},\,{\ldots}\,x_{k}, \,{\ldots}\,)$ of real numbers.
    There is no loss in generality in assuming that for each index $k$ one has $0\,<\,x_{k}\,<\,1$; see Corollary~\Ref{CorC60.25}.

        Choose a decimal representation for each of the numbers $x_{k}$, and fix it for the discussion.
    Define a real number $c$ whose decimal representations $c \,=\, 0.d_{1}\,d_{2}\,\,{\ldots}\,d_{m}\,\,{\ldots}\,$,
    where the digits $d_{1}$, $d_{2}$,\,{\ldots}\,$d_{m}$\,{\ldots}\, are obtained as follows:

        \h $d_{1}$ is the smallest decimal digit which appears as the first digit of infinitely many of the terms of the sequence~${\xi}$.
    (That there exists such $d_{1}$ follows from the `Fact' above.)
    The infinitely many terms of the sequence ${\xi}$ with first digit equal to $d_{1}$ then form a subsequence of~${\xi}$.

        \h $d_{2}$ is the smallest decimal digit which appears as the {\em second}
    decimal digit of infinitely many of the terms of the subsequence formed in the preceding step.
    This leads to a subsubsequence of ${\xi}$ whose terms each have first digit $d_{1}$ and second digit~$d_{2}$.

        Continuing this way by induction, suppose that decimal digits $d_{1}$, $d_{2}$,\,{\ldots}\,$d_{m}$
    have already been constructed, together with a subsequence of ${\xi}$ whose terms
    each have first $m$ digits equal to $d_{1}$, $d_{2}$,\,{\ldots}\,$d_{m}$, respectively.
    Let $d_{m+1}$ be the smallest decimal digit which appears as the $m+1$-st digit of infinitely many terms of the preceding subsequence.
    Clearly the resulting number  $c \,=\, 0.d_{1}\,d_{2}\,{\ldots}\,d_{m}\,\,{\ldots}\,$
    is a subsequential limit of~${\xi}$.

\V

        {\bf Remark} The restriction to sequences with values in the interval $(0,1)$ is made to simplify the discussion;
    Cauchy would have had no problem dealing with the general decimal representation instead.
    Also, the proof given here is purposely formulated in a somewhat informal manner, much as a mathematician might have done in Cauchy's era,
    using minimal notation. It can easily be made to look more modern by introducing, say, a subsequence structure ${\cal A} \,=\, (A_{1},A_{2},\,{\ldots}\,A_{m},\,{\ldots}\,)$,
    where $A_{1}$ is the infinite set of indices $k$ for which the first digit of $x_{k}$ is~$d_{1}$,
    $A_{2}$ is the infinite set of $k$ in $A_{1}$ for which the second digit of $x_{k}$ is~$d_{2}$, and so on.

\VV

        If one grants that Cauchy had Bolzano-Weierstrass Theorem available as a tool -- and that this theorem is `obviously' true --
    then the fact that Cauchy sequences must be convergent does become an `obvious' fact; see the proof of Theorem~\Ref{ThmC30.10} above.


        Perhaps the key issue is not that Cauchy could prove the sufficiency of his criterion, but that he thought of stating it in the first place.


\VV

        {\bf On What Bolzano Knew}\, It may appear silly to ask whether Bolzano knew the Bolzano-Weierstrass theorem,
    in either the set-theoretic or the sequential form, since his name appears in the title of the theorem. Nevertheless, the question bears asking.

        The reason is that the connection of Weierstrass' name to the theorem was common by the $1880$s,
    but it appears that Bolzano's name was attached to the result only about~$1899$.
    The justification given by some historians for adding Bolzano's name to that of Weierstrass for this result is that
    Weierstrass' result is a simple consequence of a theorem from Bolzano's $1817$, the one called the Right-Endpoint Principle in {\ThisText}.
    Indeed, the proof given above for the set-theoretic version of this theorem shows that this is true.
    However, there is a big difference between having the tools to prove a given result and actually knowing to even state it in the first place.
    So the question becomes: did Bolzano know to even state the result?

        It turns out that Bolzano did publish the statement of what we now call the Bolzano-Weierstrass theorem (for sets, at least) in~$1830$. Unfortunately, 
    this paper was not discovered until a century later. Bolzano refers to a proof of it in one of his other papers,
    but apparently no copy of that paper has been discovered. In any event, it does seem clear that Bolzano knew the result in question,
    so it is fair to attach his name to it. It is also clear that Weierstrass knew and proved the result,
    independently of Bolzano, and proved it without using Bolzano's Right-Endpoint Principle.
    Thus attaching both names to the result makes good sense. Whether Bolzano knew the result as early as his $1817$ seems to be still an open question.

\V

        The question of what Bolzano knew about what we call the Cauchy Criterion for convergence of a sequence is much clearer:
    He not only stated it, in essentially the same way that Cauchy did, but (unlike Cauchy) Bolzano actually tried to prove it.   

        Unfortunately, Bolzano's `proof' seems totally confused, and the universal opinion is that it proves nothing.

    However, there may be a core of truth in Bolzano's argument. Indeed, the fact that the Cauchy Criterion is sufficient to prove convergence is,
    indeed, `obviously true', at least in a sense that could convince calculus students:

\VA


       \h \underline{Plausibility Argument} Let ${\xi} \,=\, (x_{1}, x_{2},\,{\ldots}\,x_{k}, \,{\ldots}\,)$ be a Cauchy sequence of real numbers. 
    Since ${\xi}$ is bounded, without loss of generality assume that $x_{k}{\in}(0,1)$ for every index~$k$.
    (This is not a serious restriction on~${\xi}$; see Corollary~\Ref{CorC60.25}.
    It what follows, `digit' refers to `decimal digit to the right of the decimal point'.

        Let $m$ be a natural number, and let $N$ in ${\NN}$ be large enough that for every $n$ in ${\NN}$ one has $|x_{N+n}-x_{N}|\,<\,1/10^{m+1}$.
    Since the first $m$ digits of $1/10^{m+1}$ are~$0$, it is clear that the first $m$ digits of $x_{N}$ and $x_{N+n}$ are the same.
    Otherwise stated, all terms of the sequence from the $N$-th term on have the same first $m$ digits; i.e., the first $m$ digits of the sought limit are known.
    Since $m$ is arbitrary, one can determine {\em all} the digits $d_{1}$, $d_{2}$,\,{\ldots}\,$d_{m}$,\,{\ldots}\, of the purported limit.
    That is, the number $c \,=\, 0.d_{1}\,d_{2}\,\,{\ldots}\,d_{m}\,\,{\ldots}\,$
    is the only reasonable candidate to be the desired limit, and it is easy to see that this number works.

\VA

        Before reading further, contemplate whether you find the preceding argument to be valid.

\VV

        The argument {\em almost} works; unfortunately, there are exceptional cases in which it breaks down.
    For example, consider the numbers $x_{1} \,=\, 0.1\,0\,0\,{\ldots}\,0\,{\ldots}\,$
    and $x_{2} \,=\, 0.0\,9\,9\,\,{\ldots}\,9\,{\ldots}\,$. Since $x_{1} \,=\, x_{2}$,
    it follows that $|x_{1}-x_{2}|\,<\,1/10^{m+1}$ for {\em every}~$m$; yet clearly these decimal representations have no digits in common.

        In a sense, Bolzano's `proof' seems to be similar to the plausibility argument just presented above.
    Indeed, he `approximates', by $x_{n}$ for some large $n$, the `true value', of the limit being sought,
    and more-or-less states that the ability to do this to an arbitrarily small error somehow provides the desired limit.
    It could be he had the `decimal digits' plausibility above argument in mind, but realized that it had to be modified.

        Actually, there is a simple fix to the plausibility argument which is still based on the idea of determining the decimal digits of the desired limit.
    For example, Bolzano might have carried out the argument used in the `Elementary Proof of the Standard Bolzano-Weierstrass Theorem' above, but
     only in the context of the Cauchy Criterion, to get the same $c$. (It really is obvious that
    a Cauchy sequence which has a convergent subsequence is convergent.)In any event,
    Bolzano does go on to make the observation that one might be tempted to think that the limit he obtains must be irrational.
    (He does not make clear {\em why} one might be so tempted.) He counters with the further observation that the sequence
    $0.1$, $0.1\,1$,$0.1\,1\,1$,\,{\ldots}\, does converge, and the limit, $1/9$, is rational.

\VV

        {\bf General Comments} Most historians claim that, regardless of any arguments, given or not given,
    neither Cauchy not Bolzano could have proved that the Cauchy Criterion implies convergence,
    because neither understood the structure of the real numbers well enough -- especially `completeness'. How compelling is this argument?

        It is clear that both men understood that the real numbers form what is called (in Chapter~\Ref{ChaptB}) an `ordered field'.

        As for `completeness', Cauchy explicitly stated that every bounded monotonic sequence of reals has a limit, which is equivalent to `Completeness'.
    Indeed, he uses this, together with what we now call the Archimedes Property, which of course he also believed to hold,
    to prove what we call the Bisection Principle. That is, the facts about real numbers which Cauchy accepted as true,
    and thus which could be included in a list of axioms of the reals, included all the axioms given in Chapter~\Ref{ChaptB} of {\ThisText}.

        The story for Bolzano's understanding of `Completeness' is more complicated.
    Indeed, Bolzano explicitly stated what in {\ThisText} is called the `Right-Endpoint Principle', which is also equivalent to `Completeness'.
    Had he simply stated it, we could say that Bolzano also knew that the real numbers satisfy `Completeness', and thus had enough `axioms' to do analysis.
    However, it appears that he did not view this Principle as being obvious enough to be `axiomatic', so he tried to reduce it to something simpler,
    namely the fact that the Cauchy Criterion implies converegence; and since this fact in turn is not so `axiomatic',
    he tried to prove it in terms of somethimg simpler yet. As was mentioned above, this attempt was basically unsuccessful.

        In other words, both mathematicians understood the usual axioms of the real number systems,
    which is generally all that modern texts in analysis ask their readers to accept.

        What Cauchy and Bolzano did {\em not} do is to prove that their axiom systems for the real numbers
    could be represented by some `model' constructed from a simpler system such as the rational numbers.
    However, it can be argued that they did not seek such a `model' for their axioms because such a model had existed for a long time,
    and was already familiar to everyone who had learned the basics of arithmetic:
    namely, the real numbers, described analytically in terms of the standard decimal representation, and not geometrically as points on a line.
    The idea of a sequence converging to a number being the same as being able to determine
    the decimal digits of that number in a step-by-step manner would have been familiar:
    that is how one computes, say,  $\sqrt{2}$ using Bisection or Heron methods.
    The question for both mathematicians was not whether the real numbers exist; they may have thought of that question as a nonissue.
    Instead, the goal was defining concepts such as continuity and infinite sums more precisely,
    and proving theorems more rigorously, using the accepted properties of the real numbers -- accepted, that is, as of $1817$ and~$1821$.

        Lest one accuse Cauchy and Bolzano as being naive in this matter, note that even those modern texts which include a construction of the reals from,
    say, the rationals, often carry out that development in an `optional section': one can skip it yet still learn analysis.
    And even when such texts stress the importance of that construction, they often leave all the hard work to the reader as exercises.
}%EndFootNoteSize
\end{quotation}
%##


%----------------------
\StartSkip{

                \section{{\bf Limits and Suprema/Infima; ${\limsup}$} and ${\liminf}$}
                \label{SectC50}\IndB{ZZ Sections}{\Ref{SectC50} Limits and Suprema/Infima; ${\limsup}$ and ${\liminf}$}



\V
\V

        There is a simple construction which associates a pair of monotonic sequences, one `up', the other `down', with any given bounded sequence of real numbers.

\V

            \subsection{\small{\bf Construction}}
            \label{ConstC50.30}

        Let ${\xi} \,=\, (x_{1},x_{2},\,{\ldots}\,x_{n},\,{\ldots}\,)$ be a bounded sequence of real numbers.
    Thus, there exist real numbers $m$ and $M$ such that $m\,\,{\leq}\,\,x_{k}\,\,{\leq}\,\,M$ for all $k$ in ${\NN}$.
    From this one can easily construct examples of sequences ${\alpha} \,=\, (a_{1},a_{2},\,{\ldots}\,)$ and ${\beta} \,=\, (b_{1},b_{2},\,{\ldots}\,)$ with the following properties:

        \h (i) ${\alpha}$ is monotonic up, and $a_{k}\,\,{\leq}\,\,x_{k}$ for all $k$;

        \h (ii) ${\beta}$ is monotonic down, and $b_{k}\,\,{\geq}\,\,x_{k}$ for all $k$.

\noindent For instance, simply let $a_{k} \,=\, m$ and $b_{k} \,=\, M$ for all $k$ in ${\NN}$.

        One can do much better; indeed, one can construct a unique `best possible' pair of sequences associated with ${\xi}$ that satisfy (i) and (ii).

        First note that if a sequence ${\alpha}$ exists which satisfies (i), then clearly $a_{1}$ must satisfy $a_{1}\,\,{\leq}\,\,{\inf}\,\{x_{1},x_{2},\,{\ldots}\,\}$.
    For if $a_{1}$ did not satisfy this inequality, then one would have $a_{1}\,>\,{\inf}\,\{x_{1},x_{2},\,{\ldots}\,\}$.
    By the defining properties of the infimum of a set, this in turn would imply that there must exist an index $k$ such that $a_{1}\,>\,x_{k}$.
    From the requirement that the sequence ${\alpha}$ should be monotonic up one then sees that $a_{k}\,\,{\geq}\,\,a_{1}$ and thus $a_{k}\,>\,x_{k}$, contradicting~(i).
    This argument also shows that if such ${\alpha}$ exists, then one must have
        \begin{displaymath}
        a_{k}\,\,{\leq}\,\,{\inf}\,\{x_{k},x_{k+1},\,{\ldots}\,\} \mbox{ for each $k$ in ${\NN}$}.
        \end{displaymath}
   A similar argument shows that if a sequence ${\beta}$ which satisfies~(ii) exists, then ${\beta}$ must satisfy
        \begin{displaymath}
        b_{k}\,\,{\geq}\,\,{\sup}\,\{x_{k},x_{k+1},\,{\ldots}\,\} \mbox{ for each $k$ in ${\NN}$}.
        \end{displaymath}

        The preceding discussion motivates the following.


\V

            \subsection{\small{\bf Definition}}
            \label{DefC50.40}

        (a) Let ${\xi} \,=\, (x_{1},\,{\ldots}\,x_{k},\,{\ldots}\,)$ be a sequence of real numbers which is bounded above.
    The {\bf upper envelope associated with ${\xi}$}, denoted ${\xi}^{+}$, is the sequence whose $k$-th term $M_{k}({\xi})$ is given by the rule
        \begin{displaymath}
        M_{k}({\xi}) \,=\, {\sup}\,\{x_{k},x_{k+1},\,{\ldots}\,\} \mbox{ for all $k$ in ${\NN}$}.
        \end{displaymath}
    (The fact that the suprema $M_{k}({\xi})$ exist, and are numbers, follows from the Supremum Principle.)

\V


         (b) Similarly, if the sequence ${\xi}$ is bounded below then the {\bf lower envelope associated with ${\xi}$},
    denoted ${\xi}^{-}$, has $k$-th term $m_{k}({\xi})$ given by
        \begin{displaymath}
        m_{k}({\xi}) \,=\, {\inf}\,\{x_{k},x_{k+1},\,{\ldots}\,\} \mbox{ for all $k$ in ${\NN}$}.
        \end{displaymath}
    (The fact that these infima exist, and are real numbers, follows from the Infimum Principle.)


    \underline{Note} If the context makes clear which sequence ${\xi}$ is under consideration, one may abbreviate the notations $M_{k}({\xi})$ and $m_{k}({\xi})$ to $M_{k}$ and $m_{k}$, respectively.

\V

            \subsection{\small{\bf Remark}}
            \label{RemrkC50.50}

        There is an obvious extension of the concept of `upper envelope' to the case in which ${\xi}$ is unbounded above.
    Indeed, in this case it is clear that ${\sup}\,\{x_{k},x_{k+1},\,{\ldots}\,\} \,=\, +{\infty}$ for {\em all} indices $k$, so the natural definition would be ${\xi}^{+} \,=\, (+{\infty},+{\infty},\,{\ldots}\,)$.
    Likewise, there is an obvious way to extend the notion of `lower envelope' to allow sequences which are unbounded below.
    However, it does not appear to be worth the effort to introduce these extensions, so we don't.

\V
\V

        The next result shows how questions about convergence of arbitrary sequences can be reduced to the theory for monotonic sequences.
    To simplify the statement of the theorem, the hypothesis of `boundedness' is included.
    This is a reasonable restriction, however, since unbounded sequences cannot be convergent.

\V

            \subsection{\small{\bf Theorem} (The Upper/Lower-Envelopes Theorem)}
            \label{ThmC50.60}

        Let ${\xi} \,=\, (x_{1},x_{2},\,{\ldots}\,)$ be a bounded sequence of real numbers,
    and let ${\xi}^{+} \,=\, (M_{1},M_{2},\,{\ldots}\,)$ and ${\xi}^{-} \,=\, (m_{1},m_{2},\,{\ldots}\,)$
    be the corresponding upper and lower envelopes associated with ${\xi}$, as described in Definition~\Ref{DefC50.40}. Then:

        (a) The upper envelope ${\xi}^{+} \,=\, (M_{1},M_{2},\,{\ldots}\,)$ is monotonic down,
    and the lower envelope ${\xi}^{-} \,=\, (m_{1},m_{2},\,{\ldots}\,)$ is monotonic up.
    Furthermore, one has
        \begin{displaymath}
        m_{k}\,\,{\leq}\,\,x_{k}\,\,{\leq}\,\,M_{k} \mbox{ for each $k$ in ${\NN}$}.
        \end{displaymath}

\V

        (b) The monotonic sequences ${\xi}^{+}$ and ${\xi}^{-}$ are bounded, and thus are convergent.

\V

        (c) The original sequence ${\xi}$ is convergent if, and only if, $\lim_{k \,{\rightarrow}\, {\infty}} (M_{k}-m_{k}) \,=\, 0$.
    When this occurs, one has $\lim_{k \,{\rightarrow}\, {\infty}} x_{k} \,=\, \lim_{k \,{\rightarrow}\, {\infty}} m_{k} \,=\, \lim_{k \,{\rightarrow}\, {\infty}} M_{k}$.

\V

         \underline{Proof}:

\V

        (a) For convenience let $A_{k}$ denote the set $\{x_{k},x_{k+1},\,{\ldots}\,\}$,
    so that $m_{k} \,=\, {\inf}\,A_{k}$, and $M_{k} \,=\, {\sup}\,A_{k}$.
    It is clear that $A_{k+1} \,{\subseteq}\, A_{k}$ for each $k$ in ${\NN}$, so that by Part~(b) of Theorem~\Ref{ThmB30.150} one has, for each index $k$,
        \begin{displaymath}
        m_{k+1} \,=\, {\inf}\,A_{k+1}\,\,{\geq}\,\,{\inf}\,A_{k} \,=\, m_{k},
    \mbox{ and } 
        M_{k+1} \,=\, {\sup}\,A_{k+1}\,\,{\leq}\,\,{\sup}\,A_{k} \,=\, M_{k}.
        \end{displaymath}
    That is, the claimed monotonicity holds.

\V

        (b) The monotonicity properties of the sequences ${\xi}^{+}$ and ${\xi}^{-}$ imply that $m_{k}\,\,{\geq}\,\,m_{1}$ and $M_{k}\,\,{\leq}\,\,M_{1}$ for all indices $k$.
    Combining this with the hypothesis $m_{k}\,\,{\leq}\,\,x_{k}\,\,{\leq}\,\,M_{k}$ (and, of course, using Transitivity of Order in ${\RR}$), one then obtains
        \begin{displaymath}
        m_{1}\,\,{\leq}\,\,m_{k}\,\,{\leq}\,\,x_{k}\,\,{\leq}\,\,M_{k}\,\,{\leq}\,\,M_{1} \mbox{ for all indices $k$}.
        \end{displaymath}
    In particular, both of the sequences ${\xi}^{+}$ and ${\xi}^{-}$ are bounded below by $m_{1}$ and bounded above by $M_{1}$.
    The fact that these sequences are convergent then follows from the Monotonic-Sequence Principle (Part~(b) of Theorem~\Ref{ThmC20.10B})

\V

        (c) Let $A \,=\, \lim_{k \,{\rightarrow}\, {\infty}} m_{k}$ and $B \,=\, \lim_{k \,{\rightarrow}\, {\infty}} M_{k}$ be the (real) limits whose existence is proved in Part~(b).

    \h (i)\, Assume that $\lim_{k \,{\rightarrow}\, {\infty}} (M_{k}-m_{k}) \,=\, 0$, so that $A \,=\, B$. Since, by Part~(a), one also has $m_{k}\,\,{\leq}\,\,x_{k}\,\,{\leq}\,\,M_{k}$, 
    the Squeeze Property for Sequences (Part~(c) of Theorem~\Ref{ThmC20.10B}) can be used to conclude that $\lim_{k \,{\rightarrow}\, {\infty}} x_{k}$ exists and equals the common limit of ${\xi}^{-}$ and ${\xi}^{+}$.

        Conversely, suppose that the sequence ${\xi}$ is convergent, and let $L \,=\, \lim_{k \,{\rightarrow}\, {\infty}} x_{k}$.
    Let $y$ and $z$ be real numbers such that $y\,<\,L\,<\,z$.
    Then there exists a number $B$ such that if $k\,\,{\geq}\,\,B$ then $y\,<\,x_{k}\,<\,z$.
    From this it is clear that for $k\,\,{\geq}\,\,B$ the number $y$ is a lower bound for the set $A_{k}$ and $z$ is an upper bound for $A_{k}$.
    Thus, by the basic properties of `supremum' and `infimum', if $k\,\,{\geq}\,\,B$ then 
    $y\,\,{\leq}\,\,m_{k}\,\,{\leq}\,\,M_{k}\,\,{\leq}\,\,z$.
    Thus, by Theorem~\Ref{ThmC20.100} it follows that the sequences ${\xi}^{-}$ and ${\xi}^{+}$ converge to $L$.

\V
\V

        To illustrate the preceding result, let us use it to give a second proof of Part~(b) of Theorem~\Ref{ThmC20.10A}.
    That is, suppose ${\xi} \,=\, (x_{1},x_{2},\,{\ldots}\,)$ is a convergent sequence of real numbers, with $\lim_{k \,{\rightarrow}\, {\infty}} x_{k} \,=\, L$,
    and that ${\zeta} \,=\, (z_{1},z_{2},\,{\ldots}\,)$ is a subsequence of ${\xi}$.
     We want to show that ${\zeta}$ also converges to $L$ by using the preceding theorem.

\V

        \underline{Proof Using Theorem~\Ref{ThmC50.60}}: First, recall from Theorem~\Ref{ThmA30.60} that to each infinite subset $A$ of ${\NN}$ there is a (unique) strictly increasing bijection ${\Psi}_{A}:{\NN} \,{\rightarrow}\, A$ of ${\NN}$ with $A$, given as follows:
        \begin{displaymath}
        {\Psi}_{A}(1) \,=\, \min\,A; {\Psi}_{A}(j+1) \,=\, \min\,A{\setminus}
    \{ {\Psi}_{A}(1),\,{\ldots}\,{\Psi}_{A}(j)\} \mbox{ for each $j$ in ${\NN}$}.
        \end{displaymath}
    Also recall (from the same theorem) that the subsequence ${\zeta}$ of the given sequence ${\xi}$ can be expressed in the form ${\zeta} \,=\, {\xi}{\circ}{\Psi}_{A}$ for at least one infinite subset $A$ of ${\NN}$.
    (In terms of the notation used in Proof~(A) above, $A \,=\, \{k_{1},k_{2},\,{\ldots}\,\}$, and ${\Psi}_{A}(j) \,=\, k_{j}$ for each $j$.)
    Since $k_{j} \,=\, {\Psi}_{A}(j)\,\,{\geq}\,\,j$ for each $j$ in ${\NN}$, and ${\Psi}_{A}$ is strictly increasing,
    it follows that the set $\{z_{j},z_{j+1},\,{\ldots}\,\}$ is a subset of the set $\{x_{j},x_{j+1},\,{\ldots}\,\}$.
Now apply Theorem~\Ref{ThmB30.150} to conclude that
        \begin{displaymath}
        {\inf}\,\{x_{j},x_{j+1},\,{\ldots}\,\} \,\,{\geq}\,\, 
        {\inf}\,\{z_{j},z_{j+1},\,{\ldots}\,\} \mbox{ and }
        {\sup}\,\{z_{j},z_{j+1},\,{\ldots}\,\} \,\,{\leq}\,\, 
        {\sup}\,\{x_{j},x_{j+1},\,{\ldots}\,\}
        \end{displaymath}
    Using the notation of Definition~\Ref{DefC50.40}, one can then say
        \begin{equation} 
        \label{IneqC.50}
        m_{j}({\xi})\,\,{\leq}\,\,m_{j}({\zeta})
    \,\,{\leq}\,\,
        M_{j}({\zeta})\,\,{\leq}\,\,M_{j}({\xi}) \mbox{ for each index $j$}.
        \end{equation}
    By the Upper/Lower-Envelopes Theorem (Theorem~\Ref{ThmC50.60}), combined with the hypothesis that $\lim_{k \,{\rightarrow}\, {\infty}} x_{k} \,=\, L$, one knows that the sequences ${\xi}^{+} \,=\, (M_{1}({\xi}),M_{2}({\xi}),\,{\ldots}\,)$ and ${\xi}^{-} \,=\, (m_{1}({\xi}),m_{2}({\xi}),\,{\ldots}\,)$ both converge to~$L$.
    Now apply Inequality~\Ref{IneqC.50} and the Squeeze Property for Sequences (Part~(c) of Theorem~\Ref{ThmC20.10B}) to conclude that the sequences ${\zeta}^{+} \,=\, (M_{1}({\zeta}),M_{2}({\zeta}),\,{\ldots}\,)$ and ${\zeta}^{-} \,=\, (m_{1}({\zeta}),m_{2}({\zeta}),\,{\ldots}\,)$ also both converge to~$L$.
    Finally, apply the Upper/Lower-Envelopes Theorem again to conclude that the subsequence ${\zeta}$ converges to~$L$, as claimed.


\V
\V

        We have seen repeatedly that it is important to analyse the convergence properties of subsequences of a given sequence.
    The next definition provides terminology to aid in that analysis.

            \subsection{\small{\bf Definition}}
            \label{DefC50.80}

        Let ${\xi} \,=\, (x_{1},x_{2},\,{\ldots}\,)$ be a real sequence. An extended real number $L$
    is said to be a {\bf subsequential limit of ${\xi}$} if there exists a subsequence $(x_{k_{1}},x_{k_{2}},\,{\ldots}\,)$ of ${\xi}$
    such that $L \,=\, \lim_{j \,{\rightarrow}\,{\infty}} x_{k_{j}} \,=\, L$.

        The set of all subsequential limits of ${\xi}$ is denoted by ${\cal L}[{\xi}]$.

\V

            \subsection{\small{\bf Remarks}}
            \label{RemrkC50.90}


        (1) It follows from Theorem~\Ref{ThmC40.60}, (i.e., the Extended Bolzano-Weierstrass Theorem), that the set ${\cal L}[{\xi}]$ is nonempty.

\V

        (2) It follows from Part~(g) of Theorem~\Ref{ThmC40.30} that ${\xi}$ has a limit if, and only if the set ${\cal L}[{\xi}]$ has precisely one element.
    Also, when this condition occurs, that element equals $\lim_{k \,{\rightarrow}\, {\infty}} x_{k}$.

\V

        (3) The notation ${\cal L}_{{\xi}}$ for the set of all subsequential limits of the sequence ${\xi}$ is not standard;
    indeed, there appears to be no standard notation for this set. In his {\em Cours d'analyse} Cauchy suggested a notation using `double parentheses'; namely,
    $\lim\,((x_{1}, x_{2},\,{\ldots}\,x_{n},\,{\ldots}\,))$. It appears that this notation was not widely accepted.

\V

        There is a simple characterization of the set of subsequential limits of a sequence.

\V


            \subsection{\small{\bf Theorem}}
            \label{ThmC50.100}

        Let ${\xi} \,=\, (x_{1},x_{2},\,{\ldots}\,)$ be a sequence of real numbers.

\V

        (a) Let $L$ be a real number. Then $L$ is an element of ${\cal L}[{\xi}]$ if, and only if, for every pair of numbers $y$ and $z$ such that $y\,<\,L\,<\,z$, there are infinitely many indices $k$ such that $y\,<\,x_{k}\,<\,z$.

        \underline{Alternate Phrasing}: $L$ is an element of ${\cal L}[{\xi}]$ if, and only if, for every ${\varepsilon}\,>\,0$ there are infinitely many indices $k$ such that $|L-x_{k}|\,<\,{\varepsilon}$.

\V

        (b) The quantity $+{\infty}$ is an element of ${\cal L}[{\xi}]$ if, and only if, the sequence ${\xi}$ is unbounded above.


\V
        (c) Likewise, the quantity $-{\infty} is in {\cal L}[{\xi}]$ if, and only if, ${\xi}$ is unbounded below.


\V

        \underline{Proof} (a) Suppose that $L{\in}{\cal L}[{\xi}]$, and let $(x_{k_{1}},x_{k_{2}},\,{\ldots}\,)$ be a subsequence of ${\xi}$ which converges to $L$.
    Let $y$ and $z$ be numbers such that $y\,<\,L\,<\,z$. Then by Part~(a) of Theorem~\Ref{ThmC20.10A} there is a number $B$ such that if $j\,\,{\geq}\,\,B$ then $y\,<\,x_{k_{j}}\,<\,z$.
    Since the indices $k_{1}$, $k_{2}$,\,{\ldots}\,form a strictly increasing sequence of natural numbers, it follows that there are infinitely many different indices $k_{j}$ with $j\,\,{\geq}\,\,B$,
    Thus there are infinitely many indices $k$ such that $y\,<\,x_{k}\,<\,z$; for instance, the $k$'s of the form $k_{j}$ with $j\,\,{\geq}\,\,B$.

        Conversely, suppose that for each $y$ and $z$ in ${\RR}$ such that $y\,<\,L\,<\,z$ there are infinitely many indices $k$ such that $y\,<\,x_{k}\,<\,z$.
    Choose an infinite strictly increasing sequence of indices $k_{1}\,<\,k_{2}\,<\,\,{\ldots}\,$ as follows:

        (i) $x_{k_{1}}$ satisfies $L-1\,<\,x_{k_{1}}\,<\,L+1$.

        (ii) Suppose that indices $k_{1}\,<\,k_{2}\,<\,\,{\ldots}\,k_{m}$ have been chosen. By hypothesis, there are infinitely many indices $k$ such that
        \begin{displaymath}
        L-\frac{1}{m+1}\,<\,x_{k}\,<\,L+\frac{1}{m+1}. \h ({\ast})
        \end{displaymath}
    From among these, choose $k_{m+1}$ so that $k_{m+1}\,>\,k_{m}$.
    Then it is clear that the subsequence $(x_{k_{1}},x_{k_{2}},\,{\ldots}\,)$ has the property that
        \begin{displaymath}
        |L-x_{k_{j_{m}}}|\,<\,\frac{1}{m} \mbox{ for each $m$ in ${\NN}$}.
        \end{displaymath}
    Since $\lim_{m \,{\rightarrow}\, {\infty}} 1/m \,=\, 0$, the Squeeze Property implies that this subsequence converges to $L$.

        The proof that the alternate phrasing also works is left to the reader;
    see the proof of Part~(a) of Theorem~\Ref{ThmC20.10A}.

\V

         (b) Suppose that $+{\infty}{\in}{\cal L}[{\xi}]$.
 Then (by the definition of  the set ${\cal L}[{\xi}]$) there is a subsequence of ${\xi}$ which has $+{\infty}$ as limit.
    Clearly that subsequence is unbounded above, and thus ${\xi}$ itself is unbounded above.

        Conversely, suppose that ${\xi}$ is unbounded above.
    Then, by Case~(ii) of Theorem~\Ref{ThmC40.60}, there exists a subsequence of ${\xi}$ which has $+{\infty}$ as its limit; thus, $+{\infty}{\in}{\cal L}[{\xi}]$.

\V

        (c) Apply the conclusions of Part~(b) to the sequence $(-x_{1},-x_{2},\,{\ldots}\,)$.
    The details are left to the reader.

\V
\V

        The preceding result allows one to easily show that the limit properties of a sequence of real numbers do not depend on the order in which one writes down the terms of the sequence.

\V
\V

            \subsection{\small{\bf Theorem}}
            \label{ThmC50.110}

        Let ${\xi} \,=\, (x_{1},x_{2},\,{\ldots}\,)$ be a sequence of real numbers,
    and suppose that ${\sigma} \,=\, (s_{1},s_{2},\,{\ldots}\,)$ is a sequence obtained by permuting the terms of the sequence ${\xi}$.
    That is, suppose that ${\sigma}$ can be expressed in the form ${\sigma} \,=\, {\xi}{\circ}F$, where $F:{\NN} \,{\rightarrow}\, {\NN}$ is a bijection of ${\NN}$ onto ${\NN}$.
    Then ${\cal L}[{\sigma}] \,=\, {\cal L}[{\xi}]$.

\V

        \underline{Proof} First, suppose that $L$ is a real number in the set ${\cal L}[{\sigma}]$,
    and let $y$ and $z$ be numbers such that $y\,<\,L\,<\,z$. By Theorem~\Ref{ThmC50.100} there are infinitely many indices $j$ such that $y\,<\,s_{j}\,<\,z$.
    Let $A$ be the set of such indices, and let $B \,=\, F[A]$. Since $F$ is one-to-one, the set $B$ is also an infinite subset of ${\NN}$.
    Suppose that $k{\in}B$, so that $k \,=\, F(j)$ for a unique $j$ in $A$.
    Then, since $s_{j} \,=\, x_{F(j)} \,=\, x_{k}$, it follows that $x_{k}$ also satisfies $y\,<\,x_{k}\,<\,z$ for all $k$ in the infinite set~$B$.
    Thus, by Theorem~\Ref{ThmC50.100} again, one sees that $L{\in}{\cal L}[{\xi}]$ as well.
    A similar argument shows that if one of the infinities is in ${\cal L}[{\sigma}]$ then it is also in ${\cal L}[{\xi}]$.
    Combining these results leads to the conclusion that ${\cal L}[{\sigma}] \,{\subseteq}\, {\cal L}[{\xi}]$.

        By reversing the roles of ${\sigma}$ and ${\xi}$ in the preceding argument,
    which one can do because $F$ is invertible and thus one can write ${\xi} \,=\, {\sigma}{\circ}F^{-1}$,
    one sees that ${\cal L}[{\xi}] \,{\subseteq}\, {\cal L}[{\sigma}]$ as well.
    Thus, ${\cal L}[{\sigma}] \,=\, {\cal L}[{\xi}]$, as claimed.

\V
            \subsection{\small{\bf Corollary}}
            \label{CorC50.120}

        Suppose that ${\xi}$ and ${\sigma}$ are real sequences which differ only by a permutation of their indices;
    that is, there exists a bijection $F:{\NN} \,{\rightarrow}\, {\NN}$ of ${\NN}$ onto itself such that ${\sigma} \,=\, {\xi}{\circ}F$.
    Then ${\xi}$ has a limit if, and only if, ${\sigma}$ has a limit; and when this occurs, their limits are equal.

\V

        \underline{Proof} Combine the results of the preceding theorem with the result stated in Remark~\Ref{RemrkC50.90}~(2) above.

\V
\V


            \subsection{\small{\bf Examples}}
            \label{ExampC50.130}

        \hspace*{\parindent}(1) Let ${\xi} \,=\, (x_{1},x_{2},\,{\ldots}\,)$ be the sequence given by $x_{k} \,=\, (-1)^{k-1}$ for each $k$ in ${\NN}$;
    that is, ${\xi} \,=\, (1,-1,1,-1,\,{\ldots}\,)$.
    This sequence is bounded (note that $|x_{k}| \,=\, 1$ for all $k$), so neither $+{\infty}$ nor $-{\infty}$ is in ${\cal L}[{\xi}]$.
    Likewise, if $L$ is a real number such that $L \,\,{\neq}\,\, 1$ and $L \,\,{\neq}\,\, -1$,
    then there exists ${\varepsilon}\,>\,0$ such that $|L-x_{k}|\,\,{\geq}\,\,{\varepsilon}$ for all indices $k$.
    Indeed, let ${\varepsilon} \,=\, \min\,\{|L-1|,|L+1|\}$.

        In contrast, it is clear that $1$ and $-1$ are both in ${\cal L}[{\xi}]$.
    For instance, no matter which ${\varepsilon}\,>\,0$ is chosen, there are infinitely many indices $k$ for which $|1-x_{k}|\,<\,{\varepsilon}$;
    indeed, one has $1-x_{k} \,=\, 0$ whenever $k$ is odd.
    Likewise, $|-1-x_{k}| \,=\, 0\,<\,{\varepsilon}$ if $k$ is even.

        Thus one has ${\cal L}[{\xi}] \,=\, \{-1,1\}$.

\V

        (2) Recall that the set ${\QQ}$ of all rational numbers is countable (see Corollary~\Ref{CorA20.80}).
    Thus, there exists a bijection ${\alpha}:{\NN} \,{\rightarrow}\, {\QQ}$ which maps ${\NN}$ one-to-one onto ${\QQ}$.
    As such, the map ${\alpha}$ is an infinite sequence $(a_{1},a_{2},\,{\ldots}\,)$ of rational numbers in which each rational number appears exactly once.

        \underline{Claim}: ${\cal L}[{\alpha}] \,=\, {\RR}\,{\cup}\,\{-{\infty},+{\infty}\}$.

        \underline{Proof of Claim} First, note that the set ${\QQ}$ is unbounded above and below, so the same is true for the sequence ${\alpha}$.
    Thus $-{\infty}$ and $+{\infty}$ are elements of the set ${\cal L}[{\alpha}]$.

        Next, notice that if $L$ is a real number and if $y$ and $z$ are numbers such that $y\,<\,L\,<\,z$,
    then there are infinitely many rational numbers in the open interval $(y,z)$.
    Since each rational number corresponds to exactly one index $k$, there exist infinitely many indices $k$ such that $y\,<\,a_{k}\,<\,z$.

        The claim now follows by applying Part~(a) of Theorem~\Ref{ThmC50.100}.

\V
\V

            \subsection{\small{\bf Theorem}}
            \label{ThmC50.140}

        Let ${\xi} \,=\, (x_{1},x_{2},\,{\ldots}\,)$ be a sequence of real numbers, and let ${\cal L}[{\xi}]$ be the corresponding set of subsequential limits of ${\xi}$.
    Then the set ${\cal L}[{\xi}]$ has both a maximum element and a minimum element.
    That is, there are quantities $L_{1}$ and $L_{2}$ such that

       \h (i) $L_{1}$ and $L_{2}$ are elements of ${\cal L}[{\xi}]$, and

       \h (ii) $L_{1}\,\,{\geq}\,\,L\,\,{\geq}\,\,L_{2}$ for all $L$ in ${\cal L}[{\xi}]$.

\noindent (Of course, we allow the possibility that $L_{1}$ or $L_{2}$ could be an infinity.)


\V

        \underline{Proof}: Let us first show that ${\cal L}[{\xi}]$ has a maximum element.
    There are several cases to consider.

\V

        \underline{Case 1} Suppose that ${\cal L}[{\xi}]$ is a singleton set $\{L\}$.
    (By Part~(g) of Theorem~\Ref{ThmC40.30} this corresponds to the situation in which $\lim_{k \,{\rightarrow}\, {\infty}} x_{k} \,=\, L$.)
    In this case one sees that the choice $L_{1} \,=\, L$ is the desired maximum element.
    (Of course it is also the desired minimum element, but we are not yet ready to discuss the minimum element in general.)

\V

        \underline{Case 2} Suppose that $+{\infty}{\in}{\cal L}[{\xi}]$.
    Then clearly $L_{1} \,=\, +{\infty}$ is the desired maximum.

\V

        \underline{Case 3} Suppose that ${\cal L}[{\xi}]$ is not a singleton set, and assume also that $+{\infty}$ is not in ${\cal L}[{\xi}]$.
    Then, by Part~(a) of Theorem~\Ref{ThmC50.100}, ${\xi}$ is bounded above.
    Let $M$ in ${\RR}$ be an upper bound for ${\xi}$, so that $M\,\,{\geq}\,\,x_{k}$ for all $k$.
    Let ${\zeta} \,=\, (z_{1},z_{2},\,{\ldots}\,)$ be any subsequence of ${\xi}$ which has a limit.
    Since ${\zeta}$ is a subsequence of ${\xi}$, it follows that $M\,\,{\geq}\,\,z_{j}$ for all $j$ in ${\NN}$, and thus $M\,\,{\geq}\,\,\lim_{j \,{\rightarrow}\, {\infty}} z_{j}$.
    Now let $X \,=\, {\cal L}[{\xi}]{\setminus}\{-{\infty}\}$.
    Then $X$ is a nonempty set of real numbers which is bounded above by the number $M$.
    Let $L_{1} \,=\, {\sup}\,X$, so that $L_{1}$ is a real number and $M\,\,{\geq}\,\,L_{1}\,\,{\geq}\,\,L$ for all $L$ in $X$.
    Of course $L_{1}\,>\,-{\infty}$, so certainly $L_{1}\,\,{\geq}\,\,L$ for all $L$ in ${\cal L}[{\xi}]$; that is, $L_{1}$ satisfies Condition~(ii) above.

    Next, note that because of Theorem~\Ref{ThmC50.10} one knows that there exists a monotonic-up sequence of numbers $b_{1}$, $b_{2}$,\,{\ldots}\, in $X$
    (hence in ${\cal L}[{\xi}]$) such that $L_{1} \,=\, \lim_{j \,{\rightarrow}\, {\infty}} b_{j}$.
    Since $b_{j}{\in}{\cal L}[{\xi}]$, it follows that there exists a sequence of subsequences ${\tau}_{1} \,=\, (t_{11},t_{12},\,{\ldots}\,)$, ${\tau}_{2} \,=\, (t_{21},t_{22},\,{\ldots}\,)$, \,{\ldots}\, of ${\xi}$ such that $b_{j} \,=\, \lim_{m \,{\rightarrow}\, {\infty}} t_{jm}$ for each $j \,=\, 1,2,\,{\ldots}\,$.

        Now let ${\varepsilon}\,>\,0$ be given. By the Alternate Phrasing of Part~(a) of Theorem~\Ref{ThmC50.100} there exist infinitely many $k$ such that $|L_{1}-b_{k}|\,<\,{\varepsilon}/2$.
    Let $p$ be one such index, and consider the corresponding subsequence ${\tau}_{p} \,=\, (t_{p1},t_{p2},\,{\ldots}\,)$, so that $b_{p} \,=\, \lim_{j \,{\rightarrow}\, {\infty}} t_{pj}$.
    By the definition of convergent sequence, there exist infinitely many indices $j$ such that $|b_{p}-t_{pj}|\,<\,{\varepsilon}/2$.
    By the Triangle Inequality one then has
        \begin{displaymath}
        |L_{1}-t_{pj}| \,=\, |L_{1}-b_{p}| + |b_{p}-t_{pj}|\,<\,\frac{{\varepsilon}}{2} + \frac{{\varepsilon}}{2} \,=\, {\varepsilon}
        \end{displaymath}
    for infinitely many indices $j$.
    Since each number $t_{pj}$ is of the form $x_{k_{j}}$ for some index $k_{j}$, and since the indices $k_{j}$, $j \,=\, 1,2,\,{\ldots}\,$ form a strictly increasing sequence, it follows that $|L_{1}-x_{k}|\,\,{\leq}\,\,{\varepsilon}$ for infinitely many indices $j$.
    Thus, by Theorem~\Ref{ThmC50.100} again, it follows that $L_{1}{\in}{\cal L}[{\xi}]$, as claimed. 

        
        The proof that ${\cal L}[{\xi}]$ has a minimum element can be carried out in a similar manner.
    Or, better yet, one can simply apply to the sequence $-{\xi}$ the result just obtained about the maximium element; the details are left to the reader.
        
\V
\V


            \subsection{\small{\bf Definition}}
            \label{DefC50.150}


        Let ${\xi} \,=\, (x_{1},x_{2},\,{\ldots}\,)$ be a sequence of real numbers, and (as usual) let ${\cal L}[{\xi}]$ denote the corresponding set of subsequential limits of ${\xi}$.
    The maximum element of ${\cal L}[{\xi}]$ is called the {\bf limit superior of ${\xi}$}.
    It is denoted by an expression such as ${\displaystyle \limsup_{k \,{\rightarrow}\, {\infty}} x_{k}}$, or, on occasion,by $\limsup\,{\xi}$.
    (The symbol $\limsup$ is pronounced `lim~soup'.)
    This quantity is also called the {\bf upper limit of the sequence ${\xi}$}. As such, it is sometimes denoted by an expression such as $\overline{\mbox{lim}}_{k \,{\rightarrow}\, {\infty}} x_{k}$;
    but some authors combine the phrase `upper limit' with the `$\limsup$' notation.

        Likewise, the minimum element of the set ${\cal L}[{\xi}]$ is called the {\bf limit inferior}, or the {\bf lower limit}, of the sequence ${\xi}$,
    and it is denoted by expressions such as ${\displaystyle \liminf_{k \,{\rightarrow}\, {\infty}} x_{k}}$ or $\underline{\mbox{lim}}_{k \,{\rightarrow}\, {\infty}} x_{k}$.

\V

            \subsection{\small{\bf Examples}}
            \label{ExamC50.160}

\V

\hspace*{\parindent}(1) Let ${\xi} \,=\, (x_{1}, x_{2},\,{\ldots}\,)$ be the sequence given by $x_{k} \,=\, (-1)^{k-1}$ for each $k$ in ${\NN}$.
    From the results obtained in Example~\Ref{ExampC50.130}~(1) one sees that $\limsup\,{\xi} \,=\, +1$, $\limsup\,{\xi} \,=\, -1$.

\V

        (2) Let ${\alpha} \,=\, (a_{1}, a_{2},\,{\ldots}\,)$ be a sequence of real numbers.
    Let $X$ be the set of all real numbers $R\,\,{\geq}\,\,0$ such that the sequence ${\rho} \,=\, (a_{1}R, a_{2}R^{2}, a_{3}R^{3}, \,{\ldots}\,)$
    is a bounded sequence.
    It is clear that the set $X$ is nonempty; for example, $R \,=\, 0$ is obviously in $X$.
    Then one can show that ${\sup}\,X \,=\, 1/\limsup {\xi}$, where ${\xi}\,=\, (|a_{1}|, \sqrt[2]{|a_{2}|}, \sqrt[3]{|a_{3}|}, \,{\ldots}\,\sqrt[k]{|a_{k}|} \,{\ldots}\,)$.

        \underline{Note} It appears that the concept -- although not the terminology -- of `limit superior' is due Cauchy in his proof of a version of the preceding result.



\V


            \subsection{\small{\bf Theorem}}
            \label{ThmC50.170}

        Let ${\xi} \,=\, (x_{1},x_{2},\,{\ldots}\,)$ be a sequence of real numbers.

\V

        (a) A necessary and sufficient condition for $\limsup {\xi} \,=\, +{\infty}$ is that the sequence ${\xi}$ be unbounded above.
    
    Likewise, a necessary and sufficient condition for $\liminf {\xi} \,=\, -{\infty}$ is that the sequence ${\xi}$ be unbounded below.


\V

        (b) Suppose that ${\xi}$ is bounded above, and let $L$ be a real number.
    Then a necessary and sufficient condition for $L$ to equal $\limsup\,_{k \,{\rightarrow}\, {\infty}} x_{k}$ is that for every number ${\varepsilon}\,>\,0$ one has

        \h (i) $x_{k}\,>\,L+{\varepsilon}$ for only finitely many indices $k$; and

        \h (ii) $x_{k}\,>\,L-{\varepsilon}$ for infinitely many indices $k$.

        \V

        (c) Suppose that ${\xi}$ is bounded below, and let $L$ be a real number.
    Then a necessary and sufficient condition for $L$ to equal $\liminf\,_{k \,{\rightarrow}\, {\infty}} x_{k}$ is that for every number ${\varepsilon}\,>\,0$ one has

        \h (i) $x_{k}\,<\,L-{\varepsilon}$ for only finitely many indices $k$; and

        \h (ii) $x_{k}\,<\,L+{\varepsilon}$ for infinitely many indices $k$.

\V

\V

        (d) Suppose that ${\xi}$ is bounded above, and let ${\xi}^{+} \,=\, (M_{1}({\xi}),M_{2}({\xi}),\,{\ldots}\,)$ denote the upper envelope associated with ${\xi}$ (see Definition~\Ref{DefC50.40}). Then
        \begin{equation}
        \label{EqnC.55a}
        \limsup_{k \,{\rightarrow}\, {\infty}} x_{k} \,=\, \lim_{k \,{\rightarrow}\, {\infty}} M_{k}({\xi}). 
        \end{equation}
    Likewise, suppose that ${\xi}$ is bounded below, and let ${\xi}^{-} \,=\, (m_{1}({\xi}),m_{2}({\xi}),\,{\ldots}\,)$ denote the lower envelope associated with ${\xi}$. Then
        \begin{equation}
        \label{EqnC.55b}
        \liminf_{k \,{\rightarrow}\, {\infty}} x_{k} \,=\, \lim_{k \,{\rightarrow}\, {\infty}} m_{k}({\xi}). 
        \end{equation}


        The simple proof is left as an exercise.

\V

            \subsection{\small{\bf Corollary}}
            \label{CorC50.180}

        Let ${\xi} \,=\, (x_{1},x_{2},\,{\ldots}\,)$ be a sequence of real numbers.
    A necessary and sufficient condition for the sequence ${\xi}$ to have a limit $L$ is that
        \begin{displaymath}
        {\limsup}_{k \,{\rightarrow}\, {\infty}}\,x_{k} \,=\, {\liminf}_{k \,{\rightarrow}\, {\infty}}\, x_{k} \,=\, L.
        \end{displaymath}


        \underline{Proof} This follows from Parts~(a) and~(d) of the preceding theorem when combined with the results of Theorem~\Ref{ThmC50.60}.


\V

        NOTE: Some mathematics texts use Equation~\Ref{EqnC.55a} as the {\em definition} of the limit superior of a real sequence which is bounded above.
    Likewise, they use Equation~\Ref{EqnC.55b} as the {\em definition} of the limit inferior of a real sequence which is bounded below.
    Normally, however, such texts would write these equations in the form
        \begin{displaymath}
        \limsup_{k \,{\rightarrow}\, {\infty}} x_{k} \,=\, \lim_{k \,{\rightarrow}\, {\infty}} \left({\sup}\,\{x_{k},x_{k+1},\,{\ldots}\,\}\right)
        \end{displaymath}
    and
        \begin{displaymath}
        \liminf_{k \,{\rightarrow}\, {\infty}} x_{k} \,=\, \lim_{k \,{\rightarrow}\, {\infty}} \left({\inf}\,\{x_{k},x_{k+1},\,{\ldots}\,\}\right). 
        \end{displaymath}
    That is, they probably would not introduce the auxiliary notion of `enveloping sequence'.

        In contrast, some texts define the concepts of ${\limsup\, {\xi}}$ and $\liminf\, {\xi}$ in terms of the conditions stated in Parts~(b) and~(c) of Theorem~\Ref{ThmC50.170}.

\V
\V

\begin{quotation}
{\footnotesize \underline{Remark on the `sup' and `inf' Terminology}

        Many students get confused when trying to sort out the differences between the word `supremum' and the phrase `limit superior'.
    One obvious source of this confusion is that mathematicians have elected to use the same abbreviation, namely `sup', for both `supremum' and `superior'.
    A similar confusion holds between `infimum' and `limit inferior'.
    Thus, it may be useful to briefly consider the linguistic backgrounds of these words and phrases.

       It has already been stated that the words `supremum' and `infimum' are of Latin origin;
    indeed, one often uses the Latin version of their plurals, (`suprema' and  `infima' respectively).
    In any event, these words are \underline{nouns} which mean (roughly) `highest one' and `lowest one', respectively.

        In contrast, the phrases `limit superior' and `limit inferior' are word-for-word translations into English of the Latin phrases `limes superior' and `limes inferior'.
    Unfortunately, word-for-word translations between languages with different structures often produce awkward phrasings.
    Indeed,  in Latin the words `superior' and `inferior' are \underline{adjectives}; and as such they are placed after the noun they modify, `limes', because that's proper Latin grammar.
    English, in contrast, is a language in which an attributive adjective is normally placed {\em before} the noun it modifies;
    thus  better translations would have been `superior limit' and `inferior limit'.
    Of course, these last phrases have virtually the same meanings in English as the corresponding `upper limit' and `lower limit' mentioned in Definition~\Ref{DefC50.150}.

        The situation gets even less clear if a textbook introduces the notations $\limsup$ and $\liminf$,
    but fails to tell its readers what phrases they correspond to.
    Lacking any guidance, the reader of such a textbook is then likely to conclude -- incorrectly -- that these symbols should be pronounced `limit supremum' and `limit infimum'.
}%EndFootNoteSize
\end{quotation}

}%EndSkip
%------------------------
\V
\V

%---------------------------------------

\V
\V


                \section{{\bf Closed Subsets of ${\RR}$}}
                \label{SectC80}

           \subsection{\small{\bf Definition}}
            \label{DefC80.10}

        A subset $X$ of ${\RR}$ is said to be {\bf closed under sequential convergence} provided that the following condition is true:

\VA

        \h If ${\xi} \,=\, (x_{1},x_{2},\,{\ldots}\,)$ is a sequence of elements of $X$ which converges to a number $L$, then the limit $L$ is also in $X$.

\VA

\noindent It is convenient in this definition to allow $X$ to be the empty set.

\noindent NOTE: The statement `A subset $X$ of ${\RR}$ is closed under sequential convergence' is normally abbreviated
    to simply `$X$ is a closed subset' or, even more simply, `$X$ is closed' if no confusion is likely.

\V

            \subsection{\small{\bf Examples}}
            \label{ExampC80.20}

\V

\hspace*{\parindent}(1) Every finite subset of ${\RR}$, including the empty set, is closed.
    Indeed, since there is no sequence ${\xi}$ which lies in the empty set, so the fact that ${\emptyset}$ is closed is trivially true.
    And if $X$ is a nonempty subset of ${\RR}$, then any sequence in $X$ which is convergent must eventually be constant, so its limit must be in~$X$.

\V

        (2) Suppose that $a$ and $b$ are real numbers such that $a\,<\,b$.

        Part~(e) of Theorem~\Ref{ThmC20.10B} says, in effect,
    that $[a,b]$ is a closed subset of ${\RR}$.
    Otherwise stated, a closed interval in ${\RR}$ is a closed subset of~${\RR}$.

        In contrast, the interval $(a,b]$ is {\em not} a closed subset of~${\RR}$.
    For example, let $x_{k} \,=\, a + (b-a)/k$ for each index~$k$. Then it is clear that $x_{k}{\in}(a,b]$ for each~$k$,
    and that $\lim_{k \,{\rightarrow}\, {\infty}} x_{k} \,=\, a$, but $a \not \in (a,b]$.

        In a similar manner one can prove that neither $[a,b)$ nor $(a,b)$ is a closed subset of~${\RR}$.

\V

        (3) In light of the preceding example, one might conjecture that an interval in ${\RR}$ is a closed subset of ${\RR}$ in the sense of Definition~\Ref{DefC80.10} above,
    if, and only if, it is a `closed interval' in the sense of Definition~\Ref{DefB20.130}.
   This conjecture is not true. Indeed, the interval ${\RR}$ is obviously a closed subset of ${\RR}$, but is not a closed interval.
    The same holds for intervals of the form $[a,+{\infty})$ and $(-{\infty},b]$, where $a$ and $b$ are real numbers.

\V

        (4) The set ${\QQ}$ of all rational numbers is not closed in~${\RR}$, since every irrational number is the limit of a sequence of rational numbers.
    Likewise, the set ${\RR}\,{\setminus}\,{\QQ}$ of all irrational  numbers is not closed in~${\RR}$.

\V

        (5) The Cantor Ternary Set is a closed subset of ${\RR}$.
    Indeed, suppose that ${\xi} \,=\, (x_{1},x_{2},\,{\ldots}\,x_{k},\,{\ldots}\,)$ is a Cauchy sequence of points in $C$.
    Define a number $b$ in $C$ as follows:

        \h (i)\, Let $B_{1}$ be large enough that if $i,j\,\,{\geq}\,\,B_{1}$ then $|x_{i}-x_{j}|\,<\,1/3$.
    It follows that all the numbers $x_{j}$ with $j\,\,{\geq}\,\,B_{1}$ have the same first $1$-free ternary digit; see Part~(f) of Theorem~\Ref{ThmB30.65B}.
    (Note that if a pair of $1$-free ternary digits are not equal, then they differ by~$2$.)

        \h (ii) In a similar mannner one shows that for each $k$ in ${\RR}$ there is $B_{k}$ such that the first~$k$ $1$-free ternary digits are the same for all $x_{j}$ such that $j\,\,{\geq}\,\,B_{k}$.

        From this one gets an infinite sequence of $1$-free ternary digits $d_{1}$, $d_{2}$,\,{\ldots}\,$d_{k}$,\,{\ldots}\,.
    Let $L \,=\,  \stackrel{3}{.}d_{1}d_{2}\,{\ldots}\,d_{k}\,{\ldots}\,$.
    Then it is easy to see that $L \,=\,$ $\lim_{k \,{\rightarrow}\, {\infty}} x_{k}$, and clearly $L$ is an element of $C$.
    It follows that $C$ is a closed subset of ${\RR}$.

        \underline{Remark} It is an easy exercise to show that the other `Cantor sets' described in Chapter~\Ref{ChaptA} are also closed subsets of~${\RR}$.
%% EXERCISE



\VV

            \subsection{\small{\bf Remarks}}
            \label{RemrkC80.30}

        \hspace*{\parindent}(1) The ambiguity in the use of `closed' described above in the context of intervals can be easily avoided.
    For example, many authors would carefully write something like `Let $I$ be an interval of the form $[a,b]$, where $a$ and $b$ are real numbers'.
    Others would write `Let $I$ be a closed bounded interval in~${\RR}$'. In the latter case the meaning of `closed' is still, technically speaking, ambiguous;
    however, under either interpretation the nature of the resulting set is not: it must be
    an interval which is a closed subset of ${\RR}$ which is bounded in~${\RR}$, which means it must be of the form $[a,b]$ with $a$, $b$ in~${\RR}$.


\V

        (2) The use of the word `closed' in connection with limits is similar to the use of the same word in some other contexts;
    for instance:

       \h `The set ${\NN}$ of all natural numbers is closed under the operation of addition, but it is not closed under the operation of subtraction.'

       \h `The set ${\QQ}$ of all rational numbers is closed under the operations of addition, subtraction, multiplication, and division by nonzero numbers.'

\noindent That is, by simply adding natural numbers together, one cannot `escape' outside the realm of natural numbers;
    but subtracting natural numbers can lead us outside the set ${\NN}$ and into the larger set ${\ZZ}$ of all integers.
    In contrast, in ${\QQ}$ it is impossible to `escape' ${\QQ}$ by doing the standard algebraic operations to numbers in ${\QQ}$.

        Thus ${\QQ}$ is `closed' in the algebraic sense, but {\em not} `closed' in the limit sense.


\VV

%-----------------
\StartSkip{

        There are some simple ways to form new closed subsets of ${\RR}$ from old ones.

\V

            \subsection{\small{\bf Theorem}}
            \label{ThmC80.40}

        \hspace*{\parindent}(a) The union of finitely many closed subsets of ${\RR}$ is also a closed subset of ${\RR}$.
    More precisely, if $X_{1}$, $X_{2}$,\,{\ldots}\,$X_{m}$ are closed subsets of ${\RR}$, then $X_{1}\,{\cup}\,X_{2}\,{\cup}\,\,{\ldots}\,\,{\cup}\,X_{m}$ is also closed.

\V

        (b) The intersection of an arbitrary family of closed subsets of ${\RR}$ is also closed.
    More precisely, if ${\cal F}$ is a nonempty family of closed subsets of ${\RR}$, then ${\bigcap}\,{\cal F}$ is also closed.

\V

        \underline{Proof} (a) Let $X \,=\, X_{1}\,{\cup}\,X_{2}\,{\cup}\,\,{\ldots}\,X_{m}$, and suppose that ${\xi} \,=\, (x_{1},x_{2},\,{\ldots}\,)$ is a sequence in $X$ which converges to some number $L$.
    Clearly at least one of the sets $X_{1}$,\,{\ldots}\,$X_{m}$ must contain $x_{k}$ for infinitely many values of the index $k$.
    Let $X_{r}$ be such a set, and let $A_{0}$ be the set of all indices $k$ such that $x_{k}{\in}X_{r}$.
    Then $A_{0}$ determines a subsequence ${\zeta} \,=\, (z_{1},z_{2},\,{\ldots}\,)$ of ${\xi}$ such that $z_{j}{\in}X_{r}$ for each $j$.
    Since ${\zeta}$ is a subsequence of the convergent sequence ${\xi}$, it follows that ${\zeta}$ also converges, and to the same limit, namely $L$.
    Since, by hypothesis, $X_{r}$ is closed, it follows that $L{\in}X_{r}$, and thus $L$ is in the union $X$.
    Thus, $X$ is closed.

\V

        (b) Let $Y \,=\, {\bigcap} {\cal F}$, and suppose that ${\xi} \,=\, (x_{1},x_{2},\,{\ldots}\,)$ is a sequence in $Y$ which converges to some number $L$.
    Then, by definition of `intersection', one has $x_{k}$ in $X$ for every set $X$ in the family ${\cal F}$.
    Thus, ${\xi}$ is also a convergent sequence in the set $X$.
    Since, by hypothesis, $X$ is closed, it follows that $L$, the limit of ${\xi}$, is also an element of $X$.
    Since this is true for every set $X$ in the family ${\cal F}$, it follows that $L$ is also in the intersection of this family;
    that is, $L{\in}Y$.
    It follows that $Y$ is closed in ${\RR}$.

\V

        \underline{Remark} It is left as an exercise for the reader to find an example of countably infinite family of closed subsets of ${\RR}$ whose union is {\em not} a closed subset of ${\RR}$.
    A careful reading of the proof of Part~(a) should suggest how to construct such an example.

\V
\V

        One simple consequence of Part~(b) oof the preceding theorem is that one can associate to every subset $W$ of ${\RR}$ a unique `smallest' closed subset of ${\RR}$ which contains $W$ as a subset.
    More precisely:

            \subsection{\small{\bf Theorem}}
            \label{ThmC80.50}

        Let $W$ be a subset of ${\RR}$. Then there is a unique closed subset of ${\RR}$, denoted by $\overline{W}$, such that:

        (i) $W \,{\subseteq}\, \overline{W}$ and

        (ii) If $X$ is any closed subset of ${\RR}$ such that $W \,{\subseteq}\, X$ then $\overline{W} \,{\subseteq}\, X$.

\noindent In other words, $\overline{W}$ is the {\em smallest} closed subset of ${\RR}$ containing $W$.

\V

        \underline{Proof} Let ${\cal F}$ be the family of {\em all} closed subsets $X$ of ${\RR}$ such that $W \,{\subseteq}\, X$.
    (Note that the family ${\cal F}$ is nonempty; for example, the set ${\RR}$ is in the family ${\cal F}$.)
    Let $\overline{W} \,=\, {\bigcap}\, {\cal F}$. It is clear, by the definition of `intersection', that $W \,{\subseteq}\, {\overline{W}}$;
     and it follows from Theorem~\Ref{ThmC80.40} that $\overline{W}$ is closed in ${\RR}$.
    Thus, this set $\overline{W}$ satisfies Condition~(i) is satisfied.

        Likewise, if $X$ is any closed subset of ${\RR}$ such that $W \,{\subseteq}\, X$, then (by definition of the family ${\cal F}$), $X{\in}{\cal F}$, and thus ${\bigcap}\,{\cal F} \,{\subseteq}\, X$; that is, Condition~(ii) is satisfied.

\V

            \subsection{\small{\bf Definition}}
            \label{DefC80.60}

        Let $W$ be a subset of ${\RR}$.
    The set $\overline{W}$ described in the preceding theorem is called the {\bf closure of~$W$}.


\V

            \subsection{\small{\bf Examples}}
            \label{ExampC80.65}

        \hspace*{\parindent}(1) Suppose that $I$ is an open interval in ${\RR}$, so that $I \,=\, (a,b)$ where $-{\infty}\,\,{\leq}\,\,a\,<\,b\,\,{\leq}\,\,+{\infty}$.
    The possibilities for the closure $\overline{I}$ of $I$ depend on the nature of $a$ and $b$:

        \h (a) If $a$ and $b$ are both finite, then $\overline{I} \,=\, [a,b] \,=\, I\,{\cup}\,\{a,b\}$.

        \h (b) If $a$ is finite and $b \,=\, +{\infty}$, then $\overline{I} \,=\, [a,+{\infty}) \,=\, I\,{\cup}\,\{a\}$.

        \h (c) If $a \,=\, -{\infty}$ and $b$ is finite, then $\overline{I} \,=\, (-{\infty},b] \,=\, I\,{\cup}\,\{b\}$.

        \h (d) If $a$ and $b$ are both infinite, so that $I \,=\, {\RR}$, then $\overline{I} \,=\, (-{\infty},+{\infty}) \,=\, I$.

\noindent The verification of these statements is left to the reader as an exercise.

\V

        (2) It is easy to show that the closure of the set of rational numbers is ${\RR}$.
    Likewise, the closure of the set of irrational numbers is also ${\RR}$.

\V

        (3) Let $W \,=\, \{1, 1/2, 1/3, \,{\ldots}\,\}$. Then it is easy to see that $\overline{W} \,=\, W\,{\cup}\,\{0\}$.




            \subsection{\small{\bf Theorem}}
            \label{ThmC80.70}

        Let $W$ be a subset of ${\RR}$.

\V

        (a) A subset $W$ of ${\RR}$ is closed if, and only if, $W \,=\, \overline{W}$.
\V

        (b) If $W \,\,{\neq}\,\, {\emptyset}$, then $\overline{W}$ consists every real number $L$ which can be expressed as the limit of a convergent sequence whose entries are all in $W$.

\V

        \underline{Proof} (a) If $W \,=\, \overline{W}$, then $W$ is closed, since it equals the closed set $\overline{W}$.
    Conversely, if $W$ is closed then $W$ is a closed set containing $W$, hence by~(ii) of Theorem~\Ref{ThmC80.50}, one has $\overline{W} \,{\subseteq}\, W$.
    However, by Part~(i) of the same theorem one also has $W \,{\subseteq}\, \overline{W}$. Thus $W \,=\, \overline{W}$.

\V

        (b) Let $Y$ be the set of all real numbers $L$ which can be expressed as the limit of a convergent sequence whose entries are all in $W$.


        It is clear that if $X$ is a closed subset such that $W  \,{\subseteq}\,  X$, then every element of $Y$ is an element of $X$.
    Indeed, if ${\xi}$ is a sequence in $W$ which is convergent, then it is also a sequence in $X$ which is convergent, and thus (by definition of $X$ being closed) its limit is an element of $X$.
    Since (by definition) every element of $Y$ arises this way, it follows that $Y$ is a subset of $X$.

    It is also clear that $W$ is a subset of $Y$.
    Indeed, suppose that $w$ is an element of $W$, and let ${\sigma} \,=\, (s_{1},s_{2},\,{\ldots}\,)$ be the constant sequence such that $s_{k} \,=\, w$ for all $k$.
    Clearly $w \,=\, \lim_{k \,{\rightarrow}\, {\infty}} s_{k}$, so that $w{\in}Y$.
 
    Finally, suppose that ${\xi} \,=\, (x_{1},x_{2},\,{\ldots}\,)$ is a sequence in $Y$ which converges to some real number $L$.
    By Part~(a) of Theorem~\Ref{ThmC50.100} it follows that there is a subsequence ${\zeta} \,=\, (z_{1}, z_{2},\,{\ldots}\,)$ of ${\xi}$ such that $|z_{j}-L|\,<\,1/(2j)$ for each index~$j$.
    However, since $z_{j}$ is in $Y$, there exists an element $w_{j}$ in $W$ such that $|z_{j}-w_{j}|\,<\,1/(2j)$.
    Thus by the Triangle Inequality one has
        \begin{displaymath}
        \left|w_{j}-L\right|\,\,{\leq}\,\,|w_{j}-z_{j}| + |z_{j}-L|\,<\,\frac{1}{2j} + \frac{1}{2j} \,=\, \frac{1}{j}.
        \end{displaymath}
    It follows that $\lim_{j \,{\rightarrow}\, +{\infty}} w_{j} \,=\, L$, so that $L$ is the limit of a sequence of points in $W$.
    That is, $L$ is an element of the set $Y$, and thus $Y$ is a closed set.

        Combining the preceding results, one sees that $Y$ is a closed subset of ${\RR}$ which has $W$ as a subset,
    and which itself is a subset of every closed subset of ${\RR}$ containing $W$.
    Now apply Theorem~\Ref{ThmC80.50} to conclude that $Y \,=\, \overline{W}$, as claimed.
    


\V

        To understand what it means to belong to the closure $\overline{X}$ of a set $X$,
    it helps to know what it means to {\em not} belong to that closure.

\V

            \subsection{\small{\bf Theorem}}
            \label{ThmC80.75}

        Suppose that $X$ is a subset of ${\RR}$ whose closure $\overline{X}$ is {\em not} ${\RR}$.
    A necessary and sufficient condition for a number $w$ to be an element of ${\RR}{\setminus}\overline{X}$ is that there exist ${\varepsilon}\,>\,0$ such that if $|w-x|\,<\,{\varepsilon}$ then $x \not \in X$.
 
        Alternate Phrasing of the Condition: There exists an open interval $(y,z)$, with $y\,<\,w\,<\,z$, such that $(y,z)\,{\cap}\,X \,=\, {\emptyset}$.

\V

        \underline{Proof} Suppose that no such ${\varepsilon}\,>\,0$ exists. 
    Then for every $k$ in ${\NN}$ there exists $x_{k}$ in $X$ such that $|y-x_{k}|\,<\,1/k$.
    It is clear that the sequence $(x_{1},x_{2},\,{\ldots}\,)$ converges to $y$, so by the preceding theorem, $y{\in}\overline{X}$.

        Conversely, if such ${\varepsilon}\,>\,0$ exists, then for every sequence ${\xi} \,=\, (x_{1},x_{2},\,{\ldots}\,)$ of points in $X$ one has $|y-x_{k}|\,\,{\geq}\,\,{\varepsilon}$ for all $x_{k}$.
    Thus every sequence in $X$ fails to converge to $y$; thus, by the preceding theorem, $y$ is not in $\overline{X}$.

        The proof of the alternate phrasing is left to the reader.

\V

        The next result is simply a rephrasing of the preceding theorem using the terminology of Definition~\Ref{DefB30.185}.

\V

            \subsection{\small{\bf Corollary}}
            \label{CorC80.77}

        A subset $W$ of ${\RR}$ is a closed subset if, and only if, its complement $U \,=\, {\RR}{\setminus}W$ is an open subset of ${\RR}$.

\V
\V

            \subsection{\small{\bf Theorem}}
            \label{ThmC80.80}

        Let $X$ be a subset of ${\RR}$.
    Then $X$ is a closed subset of ${\RR}$ if, and only if, every bounded set of the form $[-M,M]\,{\cap}\,X$, with $M\,>\,0$,
    is a closed subset of ${\RR}$.

\V

        \underline{Proof}

        (The `Only if' part) Suppose that $X$ is a closed subset of ${\RR}$. Let $M$ be a positive number, and let ${\xi} \,=\, (x_{1},x_{2},\,{\ldots}\,)$ be a convergent sequence, with limit $L$,
    such that $x_{k}{\in}[-M,M]\,{\cap}\,X$.
    Then, since $[-M,M]$ is closed (see Example~\Ref{ExampC80.20}~(2)), one has that $L{\in}[-M,M]$;
    and since (by hypothesis) $X$ is closed, one also has that $L{\in}X$.
    Thus, $L{\in}[-M,M]\,{\cap}\,X$.
    It follows that $[-M,M]\,{\cap}\,X$ is closed.

        (The `If' part) Suppose that, for every $M\,>\,0$, the set $[-M,M]\,{\cap}\,X$ is closed.
    Let ${\xi} \,=\, (x_{1},x_{2},\,{\ldots}\,)$ be a sequence of numbers in $X$ which converges to a number $L$.
    Then the sequence ${\xi}$ is bounded; that is, there exists a number $M\,>\,0$ such that $|x_{k}|\,\,{\leq}\,\,M$ for all $k$ in ${\NN}$.
    This implies that $x_{k}{\in}[-M,M]$ for all $k$, and thus $x_{k}{\in}[-M,M]\,{\cap}\,X$.
    The hypothesis that $[-M,M]\,{\cap}\,X$ is closed then implies that $L{\in}[-M,M]\,{\cap}\,X$, and thus $L{\in}X$.
    Thus, $X$ is closed.

\V
\V

        The preceding result suggests that the theory of closed sets in ${\RR}$ can be reduced  to the study of closed sets which are also bounded.

\V

            \subsection{\small{\bf Definition}}
            \label{DefC80.90}

        A subset $X$ of ${\RR}$ is said to be {\bf compact} if $X$ is both closed and bounded.

\V

                    \subsection{\small{\bf Theorem}}
            \label{ThmC80.100}


        Let $X$ be a nonempty subset of ${\RR}$.
    Then the following statements are equivalent:

        (i) $X$ is compact.

\V

        (ii) (Bolzano-Weierstrass Property for Subsets of ${\RR}$) Every sequence in $X$ has a subsequence which converges to some point in $X$.

\V

        \underline{Proof}

\V

        \underline{(i) implies (ii)}: Let ${\xi} \,=\, (x_{1},x_{2},\,{\ldots}\,)$ be a sequence in $X$.
    By hypothesis $X$ is compact; in particular, $X$ must be a bounded set, so that ${\xi}$ is a bounded sequence.
    Thus the Bolzano-Weierstrass Theorem for Bounded Sequences (i.e., Theorem~\Ref{ThmC30.10}) implies that ${\xi}$ has a convergent subsequence.
    Let ${\zeta} \,=\, (z_{1},z_{2},\,{\ldots}\,)$ be such a subsequence, and let $L \,=\, \lim_{j \,{\rightarrow}\, {\infty}} z_{j}$.
    Clearly each $z_{j}{\in}X$, so that ${\zeta}$ is a convergent sequence in $X$.
    Since, by hypothesis, $X$ is compact, and thus closed, it follows that $L{\in}X$. Thus, Statement~(ii) also holds.

\V

        \underline{(ii) implies (i)} First note that if Condition~(ii) holds, then certainly $X$ is bounded,.
    Indeed, if $X$ were {\em not} bounded then there would be a sequence $(t_{1},t_{2},\,{\ldots}\,)$ of points in $X$ such that either $\lim_{k \,{\rightarrow}\, {\infty}} t_{k} \,=\, +{\infty}$ or $\lim_{k \,{\rightarrow}\, {\infty}} t_{k} \,=\, -{\infty}$, which would contraduct Condition~(ii).
    Likewise, assuming that $X$ is not closed would imply the existence of a sequence ${\xi}$ converning to some number $L$ {\em not} in $X$.
    It would follow that every subsequence of ${\xi}$ also converges to $L$, and thus  cannot converge to a number {\em in} $X$, contradicting Condition~(ii).

            \subsection{\small{\bf Remarks}}
            \label{RemrkC80.110}

        \hspace*{\parindent}(1) Many texts use the Bolzano-Weierstrass Property for Subsets of ${\RR}$ as the defining condition for the concept of `compact subset of ${\RR}$'.
    The advantage of doing this is that the Bolzano-Weierstrass definition of `compact' works properly in the more-general context of `metric spaces'.
    In that context the `closed-and-bounded' idea still makes sense; but it is not strong enough to produce the desired theorems.
    Since we do not consider `metric spaces' in {\ThisText}, however, it seems more natural to approach the concept of `compactness' as in Definition~\Ref{DefC80.90}.

\V

        (2) There is yet another approach to `compactness'  which is used in some texts; it is based on the so-called `Heine-Borel' property.
    This property is quite a bit more complicated than either the `closed-and-bounded' or the `Bolzano-Weierstrass' properties.
    It has the advantage over `Bolzano-Weierstrass' that it works not only in `metric spaces', but also in the much more general context of `topological spaces'.
    Since we do not consider `topological spaces' in {\ThisText}, it seems better to relegate the `Heine-Borel' approach to optional material.

}% EndSkip
%---------------------------------
\V

%------------
\StartSkip{

                \section{{\bf Limits of Real Functions Defined on Intervals}}
                \label{SectC90}\IndB{ZZ Sections}{\Ref{SectC90} Limits of Real Functions on Intervals}

        Up to now, the limits which we have studied in this chapter all involve real-valued functions defined on the {\em discrete} set ${\NN}$;
    in other words, {\em sequences} of real numbers.

        In this section we carry out a similar treatment for the other type of limit already seen in elementary calculus,
    namely limits of the form $\lim_{x \,{\rightarrow}\, c} g(x)$.
    In this situation the quantity $c$ is usually a (finite) real number, although it can also be one of the infinities.
    
\V

        In order to speed up the treatment of `limits on intervals', it is convenient to begin with the idea of `one-sided limits', and then use that to get the usual definition.

\V


            \subsection{\small{\bf Definition}}\IndBD{limits}{one-sided limits}
            \label{DefC90.10}

        \hspace*{\parindent} Let $I \,=\, (a,b)$ be an open interval in ${\RR}$, with $-{\infty}\,\,{\leq}\,\,a\,<\,b\,\,{\leq}\,\,+{\infty}$.
    Let $f:I \,{\rightarrow}\, {\RR}$ be a real-valued function which is defined at each point of the open interval $I$.
    (Recall that the notation $f:(a,b) \,{\rightarrow}\, {\RR}$ implies that the function is defined at each point of the set $(a,b)$;
    but $f$ may be defined at other points as well.)

        (1) \underline{Limits From Below/Left-hand Limits}
   \IndBD{limits}{left-hand limits}\IndBD{limits}{limits from below}

        \h (a) Let $L$ be a real number.
    One says that {\bf $f(x)$ approaches the limiting value $L$ as $x$ approaches $b$ from below}, or {\bf as $x$ increases to $x$},
    written ${\displaystyle \lim_{x{\nearrow}b} f(x) \,=\, L}$, provided the following condition holds:
        \begin{equation}
        \label{CondC.120}
        \mbox{if ${\varepsilon}\,>\,0$, then there exists $y$ in $(a,b)$
    such that if $y\,<\,x\,<\,b$ then $|L-f(x)|\,<\,{\varepsilon}$}.
        \end{equation}

        \h (b) One says that {\bf $f(x)$ approaches $+{\infty}$ as $x$ approaches $b$ from below}, written ${\displaystyle \lim_{x{\nearrow}b} f(x) \,=\, +{\infty}}$,
    provided the following condition holds:
        \begin{equation}
        \label{CondC.130}
        \mbox{for each $M$ there exists $y$ in $(a,b)$ such that if $x$ satisfies $y\,<\,x\,<\,b$, then $f(x)\,\,{\geq}\,\,M$
}.
        \end{equation}

        \h (c) One says that {\bf $f(x)$ approaches $-{\infty}$ as $x$ approaches $b$ from below}, written ${\displaystyle \lim_{x{\nearrow}b} f(x) \,=\, -{\infty}}$,
    provided the following condition holds:
        \begin{equation}
        \label{CondC.140}
        \mbox{if $M$ in ${\RR}$ there exists $y$ in $(a,b)$ such that if $x$ satisfies $y\,<\,x\,<\,b$, then $f(x)\,\,{\leq}\,\,M$
}.
        \end{equation}

        The statement, `$f(x)$ approaches $L$ as $x$ approaches $b$ from below', is often rephrased as `$L$ is the left-hand limit of $f$ at $b$';
    it is then usually written in the alternate form
        \begin{displaymath}
        L \,=\, \lim_{x \,{\rightarrow}\, b^{-}} f(x)
        \end{displaymath}

\V

        (2) \underline{Limits From Above/Right-hand Limits}       \IndBD{limits}{right-hand limits}\IndBD{limits}{limits from above}

        Let $L$ be an extended real number. One says that {\bf $f(x)$ approaches $L$ as $x$ approaches $a$ from above},
    written $\lim_{x{\searrow}a} f(x) \,=\, L$, provided one has
        \begin{equation}
        \label{EqnC.150}
        \lim_{x{\nearrow}-a} f(-x) \,=\, L.
        \end{equation}
    (In the preceding equation, one applies Part~(1) of this definition to the function $g:(-b,-a) \,{\rightarrow}\, {\RR}$ which is given by the formula $g(x) \,=\, f(-x)$ for all $x$ in $(-b,-a)$.)

        The statement, `$f(x)$ approaches $L$ as $x$ approaches $a$ from above', is often rephrased as `$L$ is the right-hand limit of $f$ at $a$';
    it is then usually written in the alternate form
        \begin{displaymath}
        L \,=\, \lim_{x \,{\rightarrow}\, a^{+}} f(x)
        \end{displaymath}

\V

        \underline{Note} The limits `from below' and `from above' discussed above are combined under the general heading of `one-sided limits'.

\V

        (3) \underline{Two-Sided Limits}\IndBD{limits}{two-sided limits}
    Suppose that $I \,=\, (a,b)$ is an open interval as above, and assume that $c$ is a real number such that 
$a\,<\,c\,<\,b$.
    Suppose that $f$ is a real-valued function which is defined for all $x$ in the interval $I$, except possibly at $c$.
    Let $L$ be an extended real number.
    One says that {\bf $f(x)$ approaches $L$ as $x$ approaches $c$}, written ${\displaystyle \lim_{x \,{\rightarrow}\, c} f(x) \,=\, L}$,
    provided the one-sided limits ${\displaystyle \lim_{x{\nearrow}c}} f(x)$ and ${\displaystyle \lim_{x{\searrow}c} f(x)}$ both exist and both equal $L$.

        It is customary to refer to the `two-sided' limit $L$ defined here more simply as `the limit of $f$ at $c$,
    saving the terminology `two-sided' when one needs to emphasize that values of $x$ on both sides of $c$ are being considered.

\V

            \subsection{\small{\bf Remarks}}
            \label{RemrkC90.12}

\V

\hspace*{\parindent}(1) In {\ThisText} we mainly use the notations ${\displaystyle \lim_{x{\nearrow}b}}$ and
    ${\displaystyle \lim_{x{\searrow}a}}$ instead of the corresponding ${\displaystyle \lim_{x \,{\rightarrow}\, b^{-}}}$ and
    ${\displaystyle \lim_{x \,{\rightarrow}\, a^{+}}}$ notations, despite the fact that the latter notations are more common.
    The main reason for preferring `${x\nearrow}b$' over $x \,{\rightarrow}\, b^{-}$, for example, is this:
    the minus sign in the notation $b^{-}$ makes one think of `decreasing', whereas the notation `${\nearrow}$'
    better reflects the actual nature of the limit process in question; namely, that $x$ is {\em increasing} towards~$b$.

\V

        (2) It is customary to avoid the use of both the $x{\nearrow}b$ and $x \,{\rightarrow}\, b^{-}$ notations when $b \,=\, +{\infty}$,
    and to use instead $x \,{\rightarrow}\, +{\infty}$. This causes no confusion since it is clear that the horizontal arrow in the expression
    $x \,{\rightarrow}\, +{\infty}$ cannot refer to a `two-sided' limit -- there are no values of $x$ `to the right of~$+{\infty}$!
    A similar comment applies to the expression $x \,{\rightarrow}\, -{\infty}$.

\V

        (3) There is yet another space-saving notation in common use for the result of taking one-sided limits.
    Namely, one uses the abbreviations $f(a+)$ and $f(b-)$ for the one-sided limits
    $\lim_{x \,{\rightarrow}\, a^{+}}$ and $\lim_{x \,{\rightarrow}\, b^{-}}$, respectively;
    that is, for $\lim_{x{\searrow}a} f(x)$ and $\lim_{x{\nearrow}b} f(x)$.
    (There appears to be no corresponding notation $f({\searrow}a)$ and $f({\nearrow}b)$ in use.)
    Note that this $a^{+}/b^{-}$ notation does {\em not} require the function $f$ to actually be defined at $x \,=\, a$ or $x \,=\, b$, and we do allow that freedom here.
    However, in {\ThisText} we restrict the use of this notation as follows:

        \h (i)\, The limiting values $a$ and $b$ of $x$ must be finite. This avoids the odd-looking possibilities $f(-{\infty}+)$ and $f(+{\infty}-)$.

        \h (ii) The quantities $f(a+)$ and $f(b-)$ must also be finite. This avoids the temptation to go from writing, say, `$f(a+) \,=\, +{\infty}$' to then `defining the value $f(a) \,=\, +{\infty}$'.
    (Recall that in {\ThisText} we don't allow a function to assume either of the infinities as actual `values'.)

\VV

        The most important of the various limits described above is the the two-sided case in which $L$ is a (finite) number.
    It is useful to describe that situation is a more familiar way.

\V

            \subsection{\small{\bf Theorem} (The ${\varepsilon}{\delta}$ Formulation)}
            \label{ThmC90.15}

        Suppose that $c$ is an element of an open interval $I$ in ${\RR}$.
    Let $f$ be a real-valued function which is defined for all $x$ in $I$ except possibly at $x \,=\, c$, and let $L$ be a real number.
    Then the following statements are equivalent:

        (i) ${\displaystyle \lim_{x \,{\rightarrow}\, c} f(x) \,=\, L}$.

\V

        (ii) For every ${\varepsilon}\,>\,0$ there exists ${\delta}\,>\,0$ such that if $x{\in}I$ satisfies $0\,<\,|x-c|\,<\,{\delta}$ then $|f(x)-L|\,<\,{\varepsilon}$.

\V


        \underline{Proof} Suppose first that Statement~(i) holds. That is, suppose that the following conditions hold:
        \begin{displaymath}
        ({\alpha}) \h \lim_{x{\nearrow}c} f(x) \,=\, L
    \mbox{ and }
        ({\beta})  \h \lim_{x{\searrow}c} f(x) \,=\, L.
        \end{displaymath}
    Let ${\varepsilon}\,>\,0$ be given. From the meaning of Condition~${\alpha}$, there exists $y$ in ${\RR}$, with $a\,<\,y\,<\,c$, such that if $x$ in ${\RR}$ satisfies $y\,<\,x\,<\,c$ then $|f(x)-L|\,<\,{\varepsilon}$.
    Likewise, by Condition~${\beta}$ there must exist $z$ in ${\RR}$, with $c\,<\,z\,<\,b$, such that if $x$ in ${\RR}$ satisfies $c\,<\,x\,<\,z$ then $|f(x)-L|\,<\,{\varepsilon}$.
    Now let ${\delta} \,=\, {\min\,}\{c-y,z-c\}$.
    Clearly ${\delta}\,>\,0$. Also, ${\delta}\,\,{\leq}\,\,c-y$ and ${\delta}\,\,{\leq}\,\,z-c$; thus $y\,\,{\leq}\,\,c-{\delta}$ and $c+{\delta}\,\,{\leq}\,\,z$.
    Now suppose that $|x-c|\,<\,{\delta}$ and $x \,\,{\neq}\,\, c$.
    then either $c-{\delta}\,<\,x\,<\,c$ or $c\,<\,x\,<\,c+{\delta}$.
    In the former case one gets $y\,\,{\leq}\,\,c-{\delta}\,<\,x\,<\,c$, while in the latter one gets $c\,<\,x\,<\,c+{\delta}\,\,{\leq}\,\,z$.
    In each case it follows (from the construction of the numbers $y$ and $z$) that $|f(x)-L|\,<\,{\varepsilon}$.
    Thus, Statement~(i) implies Statement~(ii).

        Conversely, suppose that Statement (ii) holds. Let ${\varepsilon}\,>\,0$ be given, and let ${\delta}\,>\,0$ be a number such that $a\,<\,c-{\delta}$ and $c+{\delta}\,<\,b$.
    Set $y \,=\, c-{\delta}$ and $z \,=\, c+{\delta}$.
    If $x$ satisfies $y\,<\,x\,<\,c$ then $|x-c|\,<\,c-y \,=\,{\delta}$ and thus $|f(x)-L|\,<\,{\varepsilon}$.
    Thus, Condition~${\alpha}$ is true.
    Similarly, if $c\,<\,x\,<\,z$ then $|x-c|\,<\,{\delta}$ hence $|f(x)-L|\,<\,{\varepsilon}$.
    Thus Condition~${\beta}$ is also true, hence Statement~(i) holds.

\V

        \underline{Remark} Many texts -- probably a majority of them -- use Statement~(ii) as their definition of `${\displaystyle \lim_{x \,{\rightarrow}\, c} f(x) \,=\, L}$' in the case $c$ and $L$ are both finite.
    The main advantage of Definition~\Ref{DefC90.10} is that it covers all the various possibilities more efficiently.

\V




            \subsection{\small{\bf Examples}}
            \label{ExampC90.20}

\hspace*{\parindent}(1) Suppose that $I \,=\, (a,b)$ is an open interval in ${\RR}$, and assume that $c$ is a real number such that $a\,<\,c\,<\,b$.

\V

        \h (i) Let $A$ be a number. If $f(x) \,=\, A$ for all $x$ in $I$ such that $x \,\,{\neq}\,\, c$,
    then ${\displaystyle \lim_{x \,{\rightarrow}\, c} f(x) \,=\, A}$.

        \h (ii) If $g(x) \,=\, x$ for all $x$ in $I$ such that $x \,\,{\neq}\,\, c$, then ${\displaystyle \lim_{x \,{\rightarrow}\, c} g(x) \,=\, c}$.

\V

        To see why (i) holds, note that $|f(x)-A| \,=\, 0$ for all $x$ in $I$ such that $x \,\,{\neq}\,\, c$.
    Let ${\delta}\,>\,0$ be such that $a\,<\,c-{\delta}$ and $c+{\delta}\,<\,b$.
    let ${\varepsilon}\,>\,0$ be given.
    Then if $|x-c|\,<\,{\delta}$ and $x \,\,{\neq}\,\, c$ one has $|f(x)-A| \,=\, 0\,<\,{\varepsilon}$.
    Thus, by Theorem~\Ref{ThmC90.15} one has ${\displaystyle \lim_{x \,{\rightarrow}\, c} f(x) \,=\, A}$.

        As for (ii), note that $|g(x)-c| \,=\, |x-c|$. Let ${\varepsilon}\,>\,0$ be given.
    Choose ${\delta}\,>\,0$ small enough that $a\,<\,c-{\delta}$, $c+{\delta}\,<\,b$, and ${\delta}\,<\,{\varepsilon}$.
    Then if $|x-c|\,<\,{\delta}$ and $x \,\,{\neq}\,\, c$, one has $|g(x)-c| \,=\, |x-c|\,<\,{\varepsilon}$.
    Thus, by Theorem~\Ref{ThmC90.15} again, one has ${\displaystyle \lim_{x \,{\rightarrow}\, c} g(x) \,=\, c}$.

        In both parts of this example, the reason for requiring ${\delta}$ to be small enough that $a\,<\,c-{\delta}$ and $c+{\delta}\,<\,b$ is so that the condition $|x-c|\,<\,{\delta}$, $x \,\,{\neq}\,\, c$ forces $x$ to be in the domain of $f$.
    

\V
\V


            \subsection{\small{\bf Remark}}
            \label{RemrkC90.30}


        In Part~(1) of Definition~\Ref{DefC90.10}, the left endpoint $a$ of the interval $I$ plays no essential role.
    More precisely, if $a'$ is another quantity such that the given function $f$ is defined on the open interval $(a',b)$,
    then the truth of the statement ${\displaystyle \lim_{x{\nearrow}b} f(x) \,=\, L}$ does not depend on whether one thinks of $f$ as being defined on $(a,b)$ or on $(a',b)$.
    One needs to know the values of $f$ in some open interval, no matter how small, which has $b$ as its `right endpoint'.
    A similar comment holds for statements of the form ${\displaystyle \lim_{x{\searrow}b} f(x)}$ and ${\displaystyle \lim_{x \,{\rightarrow}\,c} f(x) \,=\, L}$.
    Because of this, it is customary to be vague about the precise domain on which $f$ is defined, as long as it is clear that it includes an appropriate open interval.

\V
\V

        The next result shows that the theory for limits of functions on an interval can be reduced to the corresponding theory for limits of sequences.

\V

            \subsection{\small{\bf Theorem}}
            \label{ThmC90.40}

        Throughout this theorem $(a,b)$ denotes an open interval in ${\RR}$,  $L$ is an extended real number,
    and, in Parts~(a) and~(b), $f$ is a real-valued function defined on the interval $(a,b)$.

\V

        (a) The statement `${\displaystyle \lim_{x{\nearrow}b} f(x) \,=\, L}$' is true if, and only if, the following statement is true:

        `${\displaystyle \lim_{k \,{\rightarrow}\, {\infty}} f(x_{k}) \,=\, L}$ for every sequence ${\xi} \,=\, (x_{1},x_{2},\,{\ldots}\,)$ of numbers such that $a\,<\,x_{k}\,<\,b$ for all $k$ and ${\displaystyle \lim_{k \,{\rightarrow}\, {\infty}} x_{k} \,=\, b}$'.

\V

        (b) Likewise, the statement `${\displaystyle \lim_{x{\searrow}a} f(x) \,=\, L}$' is true if, and only if, the following statement is true:

        `${\displaystyle \lim_{k \,{\rightarrow}\, {\infty}} f(x_{k}) \,=\, L}$ for every sequence ${\xi} \,=\, (x_{1},x_{2},\,{\ldots}\,)$ of numbers such that $a\,<\,x_{k}\,<\,b$ for all $k$ and ${\displaystyle \lim_{k \,{\rightarrow}\, {\infty}} x_{k} \,=\, a}$'.

\V

        (c) Suppose instead that $f$ is a real-valued function defined at all points of $(a,b)$ except possibly at a certain number $c$ such that $a\,<\,c\,<\,b$.
    Then the statement `${\displaystyle \lim_{x \,{\rightarrow}\, b} f(x) \,=\, L}$' is true if, and only if, the following statement is true:

        `${\displaystyle \lim_{k \,{\rightarrow}\, {\infty}} f(x_{k}) \,=\, L}$ for every sequence ${\xi} \,=\, (x_{1},x_{2},\,{\ldots}\,)$ of numbers such that $a\,<\,x_{k}\,<\,b$ and $x_{k} \,\,{\neq}\,\, c$ for all $k$ and ${\displaystyle \lim_{k \,{\rightarrow}\, {\infty}} x_{k} \,=\, c}$'.

\V

        \underline{Proof}:

\V

        (a) To show the `Only if' half, suppose that the statement ${\displaystyle \lim_{x{\nearrow}b} f(x) \,=\, L}$ holds, and let ${\xi} \,=\, (x_{1},x_{2},\,{\ldots}\,)$
    be a real sequence such that $a\,<\,x_{k}\,<\,b$ for all $k$, and ${\displaystyle \lim_{k \,{\rightarrow}\, {\infty}} x_{k} \,=\, b}$.

        \underline{Case 1} Assume that $L$ is finite.
    Let ${\varepsilon}\,>\,0$ be given.
    Then, by Part~(1a) of Definition~\Ref{DefC90.10}, there exists a number $y$ such that if $y\,<\,x\,<\,b$ then $|L-f(x)|\,<\,{\varepsilon}$.
    For such $y$, the hypothesis that $b \,=\, \lim_{k \,{\rightarrow}\, {\infty}} x_{k}$ implies that there exists $M$ such that if $k\,\,{\geq}\,\,M$ then $y\,<\,x_{k}\,<\,b$.
    Combining these facts then implies that if $k\,\,{\geq}\,\,M$ then one has
        \begin{displaymath}
        |L-f(x_{k})|\,<\,{\varepsilon} \mbox{ for all $k\,\,{\geq}\,\,M$}.
        \end{displaymath}
    Thus, ${\displaystyle \lim_{k \,{\rightarrow}\, {\infty}} f(x_{k}) \,=\, L}$, as required.

        \underline{Case 2} Suppose that $L \,=\, +{\infty}$.
    Let $B$ in ${\RR}$ be given.
    Then by Part~(1b) of Definition~\Ref{DefC90.10} it follows that there exists $y$, satisfying $a\,<\,y\,<\,b$,
    such that if $y\,<\,x\,<\,b$ then $f(x)\,\,{\geq}\,\,B$.
    As in Case~(1) above, there exists $M$ such that if $k\,\,{\geq}\,\,M$ then $y\,<\,x_{k}\,<\,b$.
    Thus for $k\,\,{\geq}\,\,M$ one has $f(x_{k})\,\,{\geq}\,\,B$.
    It follows that ${\displaystyle \lim_{k \,{\rightarrow}\, {\infty}} f(x_{k}) \,=\, +{\infty}}$.

        \underline{Case 3} Suppose that $L \,=\, -{\infty}$.
    This case can be handled much as in Case~2.
    Or, perhaps better, one can simply apply Case~(2) to the function $g$ given by $g(x) \,=\, -f(x)$ for all $x$ in $(a,b)$.
    In either event, the details are left to the reader.

        To show the `If' part, assume that the statement `${\displaystyle \lim_{x{\nearrow}b} f(x) \,=\, L}$' is {\em not} true.
    Once again there are three cases to consider.

        \underline{Case 1} Suppose that $L$ is finite.
    Then there exists ${\varepsilon}_{0}\,>\,0$ such that for every $y$ which satisfies $a\,<\,y\,<\,b$ there exists $x$, with $y\,<\,x\,<\,b$, such that $|L-f(x)|\,\,{\geq}\,\,{\varepsilon}_{0}$.
    In particular, let $(y_{1},y_{2},\,{\ldots}\,)$ be a sequence of numbers such that $a\,<\,y_{1}\,<\,y_{2}\,<\,\,{\ldots}\,\,<\,b$ and such that $b \,=\, \lim_{k \,{\rightarrow}\, {\infty}} y_{k}$.
    (It is clear that many such sequences exist.) Let $x_{k}$ be a number such that $y_{k}\,<\,x_{k}\,<\,b$ and $|L-f(x_{k})|\,\,{\geq}\,\,{\varepsilon}_{0}$.
    It is clear, by the Squeeze Property for Sequences, that $b \,=\, \lim_{k \,{\rightarrow}\, {\infty}} x_{k}$.
    But since $|L-f(x_{k})|\,\,{\geq}\,\,{\varepsilon}_{0}\,>\,0$ for all indices $k$, it follows that the sequence $(f(x_{1}),f(x_{2}),\,{\ldots}\,)$ does not converge to $L$.

        \underline{Case 2} Suppose that $L \,=\, +{\infty}$. Then there exists $B$ such that for every $k$ in ${\NN}$ there exists $x_{k}$ such that $a\,<\,x_{k}\,<\,b$ but $f(x_{k})\,<\,B$.
    Clearly the corresponding sequence $(f(x_{1}),f(x_{2}),\,{\ldots}\,)$ does not converge to $+{\infty}$.

        \underline{Case 3} Suppose that $L \,=\, -{\infty}$. This case is similar to Case~2, and is left to the reader.

\V

        (b) Let $g(x) \,=\, f(-x)$ for $x$ in the open interval $(-b,-a)$.
    Then the result to be proved concerning $f$ on $(a,b)$ reduces to applying the results of Part~(1) to $g$.
    The details are left to the reader.

\V

        (c) \underline{The `Only if' Half} Suppose that the statement `${\displaystyle \lim_{x \,{\rightarrow}\, c} f(x) \,=\, L}$' is true.
    Then, by Part~(3) of Definition~\Ref{DefC90.10}, one must have the following facts:

        \h (i)\, ${\displaystyle \lim_{x{\nearrow}c} f(x) \,=\, L}$, and 

        \h (ii) ${\displaystyle \lim_{x{\searrow}c} f(x) \,=\, L}$.

\noindent Now let ${\xi} \,=\, (x_{1},x_{2},\,{\ldots}\,)$ be a sequence of real numbers such that for each index $k$ one has $a\,<\,x_{k}\,<\,b$ and $x_{k} \,\,{\neq}\,\, c$,
    and such that ${\displaystyle \lim_{k \,{\rightarrow}\, {\infty}} x_{k} \,=\, c}$.
    Let $A$ be the set of all the indices $k$ such that $x_{k}\,<\,c$ and let $B$ be the set of indices $k$ such that $x_{k}\,>\,c$.
    It is clear that ${\NN} \,=\, A\,{\cup}\,B$.
    Once again, there are three cases to consider.

        \underline{Case 1} Suppose that $B$ is a finite subset of ${\NN}$.
    Then $A$ must be an infinite set,
    hence $a\,<\,x_{k}\,<\,c$ for all sufficiently large $k$. Now Fact~(i), when combined with the results of Part~(a) of the current theorem,
    implies that ${\displaystyle \lim_{k \,{\rightarrow}\, {\infty}} f(x_{k}) \,=\, L}$.

        \underline{Case 2} Now suppose that $A$ is a finite subset of ${\NN}$, so $B$ must be an infinite set.
    Then Fact~(ii), when combined with the results of Part~(b) of the current theorem,
    implies that ${\displaystyle \lim_{k \,{\rightarrow}\, {\infty}} f(x_{k}) \,=\, L}$.

        \underline{Case 3} Suppose that $A$ and $B$ are both infinite.
    Then these subsets of ${\NN}$ determine, infinite subsequences ${\zeta} \,=\, (z_{1},z_{2},\,{\ldots}\,)$ and ${\tau} \,=\, (t_{1},t_{2},\,{\ldots}\,)$
    of ${\xi}$ such that $z_{j}{\in}(a,c)$ and $t_{j}{\in}(c,b)$ for all indices $j$.
    Also, Part~(e) of Theorem~\Ref{ThmC40.30} implies that the sequences ${\zeta}$ and ${\tau}$ both have the same limit, namely $L$.
    Now apply the results of Part~(a) of the current theorem to the sequence ${\zeta}$, and the results of Part~(b) to ${\tau}$,
    to conclude that ${\displaystyle \lim_{j \,{\rightarrow}\, {\infty}} f(z_{j}) \,=\, L}$ and
    ${\displaystyle \lim_{j \,{\rightarrow}\, {\infty}} f(t_{j}) \,=\, L}$.
    It then follows from Part~(h) of Theorem~\Ref{ThmC40.30}, with $m$ in that result equal to~$2$,
    that ${\displaystyle \lim_{k \,{\rightarrow}\, {\infty}} f(x_{k}) \,=\, L}$.

        The desired result now follows.

        \underline{The `If' Half} Now assume that the given statement, namely that

`${\displaystyle \lim_{k \,{\rightarrow}\, {\infty}} f(x_{k}) \,=\, L}$ for every sequence ${\xi} \,=\, (x_{1},x_{2},\,{\ldots}\,)$ of numbers such that $a\,<\,x_{k}\,<\,b$ and $x_{k} \,\,{\neq}\,\, c$ for all $k$ and ${\displaystyle \lim_{k \,{\rightarrow}\, {\infty}} x_{k} \,=\, c}$'

is true. In particular, it would follow that if ${\xi} \,=\, (x_{1},x_{2},\,{\ldots}\,)$ satisfies $a\,<\,x_{k}\,<\,c$ for all $k$ and $\lim_{k \,{\rightarrow}\, {\infty}} x_{k} \,=\, c$, then $\lim_{k \,{\rightarrow}\, {\infty}} f(x_{k}) \,=\, L$.
    Part~(a) of this theorem now implies that ${\displaystyle \lim_{x{\nearrow}c}} f(x) \,=\, L$.
    A similar argument, but using Part~(b), shows that $\lim_{x{\searrow}c} f(x) \,=\, L$.
    Finallly, Part~(3) of Definition~\Ref{DefC90.10} says that ${\displaystyle \lim_{x \,{\rightarrow}\, c} f(x) \,=\, L}$, as claimed.

\V
\V

            \subsection{\small{\bf Corollary}}
            \label{CorC90.45}

\V

        The conclusions of the preceding theorem remain true if, in the statement of the hypotheses,
    the phrase `for every sequence ${\xi}$' is replaced by the phrase `for every strictly monotonic sequence ${\xi}$'.

\V

        The simple proof is left as an exercise. \Q

\V
\V

        The fact that we have already proved numerous results about limits of sequences, combined with the previous theorem,
    makes it much easier to prove analogous results for limits of functions on intervals.
    The next theorem illustrates this statement. In order to simplify the exposition, these results focuses on `two-sided' limits, which is the case that arises most frequently in practice.
    The reader is given the task to formulate and prove the corresponding `one-sided' results.

\V

            \subsection{\small{\bf Theorem}}
            \label{ThmC90.50}

        Let $(a,b)$ be an open interval in ${\RR}$, and let $c$ be a real number such that $a\,<\,c\,<\,b$.
    For convenience let $X_{c} \,=\, (a,b){\setminus}\{c\}$;
    thus, $X_{c} \,=\, (a,c)\,{\cup}\,(c,b)$ is the union of the two disjoint open intervals $(a,c)$ and $(c,b)$.
    Let $f$ be a real-valued function that is defined on the set $X_{c}$.
    Finally, let $L$ be an extended real number.

\V

        (a) If the function $f$ has a limit at $c$, then this limit is unique.
    That is, if the statements ${\displaystyle \lim_{x \,{\rightarrow}\, c} f(x) \,=\, L_{1}}$ and ${\displaystyle \lim_{x \,{\rightarrow}\, c} f(x) \,=\, L_{2}}$ are both true, then $L_{1} \,=\, L_{2}$.

\V

        (b) If there is a constant $A$ such that $f(x) \,=\, A$ for all $x$ in $X_{c}$, then ${\displaystyle \lim_{x \,{\rightarrow}\, c} f(x)} \,=\, A$.

\V

        (c) Suppose that there exist real numbers $m$ and $M$, with $m\,<\,M$, such that $f(x){\in}[m,M]$ for all $x$ in the set $X_{c}$.
    If $f$ has a limit at $c$, then ${\displaystyle \lim_{x \,{\rightarrow}\, c} f(x)}$ is also in the set $[m,M]$.

\V

        (d) (Squeeze Property for Limits on Intervals) Suppose that $f$, $g$ and $h$ are real-valued functions defined on the set $X_{c}$, and that the following conditions hold:

        \h (i) $g(x){\in}\mbox{Seg}\,[f(x),h(x)]$ for all $x$ in $X_{c}$.

        \h (ii) ${\displaystyle \lim_{x \,{\rightarrow}\, c} f(x)
     \,=\, \lim_{x \,{\rightarrow}\, c} h(x) \,=\, L}$.

    Then ${\displaystyle \lim_{x \,{\rightarrow}\, c} g(x) \,=\, L}$.

\V

        (e) Suppose that the quantity $L$ is a real number (i.e., $L$ is finite).
    Then the following statements are equivalent:

        \h (i)\,\, ${\displaystyle \lim_{x \,{\rightarrow}\, c} f(x) \,=\, L}$;

        \h (ii)\, ${\displaystyle \lim_{x \,{\rightarrow}\, c} |L-f(x)| \,=\, 0}$;

        \h (iii) ${\displaystyle \lim_{x \,{\rightarrow}\, c} (L-f(x)) \,=\, 0}$.

\V

        (f) Suppose that ${\displaystyle \lim_{x \,{\rightarrow}\, c} f(x) \,=\, 0}$.
    If $g$ is real-valued function which is bounded on the set $X_{c}$, then ${\displaystyle \lim_{x \,{\rightarrow}\, c} f(x){\cdot}g(x) \,=\, 0}$.

\V

        The proof of each part of this theorem reduces, in much the same way, to corresponding known facts about limits of real sequences.
    Thus we prove only one portion of the theorem here, Part~(d), just to illustrate the technique;
    the proof of the remaining parts is then left as an exercise for the reader.

\V

        \underline{Proof of Part (d)} Let ${\xi} \,=\, (x_{1},x_{2},\,{\ldots}\,)$ be a sequence of real numbers such that $x_{k}{\in}X_{c}$ for each index $k$, and assume that $c \,=\, \lim_{k \,{\rightarrow}\, {\infty}} x_{k}$.
    From the hypothesis that $g(x){\in}\mbox{Seg}\,[f(x),h(x)]$ for all $x$ in $X_{c}$,
    it follows that $g(x_{k}){\in}\mbox{Seg}\,[f(x_{k}),h(x_{k})]$ for every index $k$.
    In addition, from the hypothesis that ${\displaystyle \lim_{x \,{\rightarrow}\, c} f(x) \,=\, \lim_{x \,{\rightarrow}\, c} h(x) \,=\, L}$, 
    combined with the `only if' portion of Part~(3) of Theorem~\Ref{ThmC90.40}, it follows that
        \begin{displaymath}
        \lim_{k \,{\rightarrow}\, {\infty}} f(x_{k}) \,=\, \lim_{k \,{\rightarrow}\, {\infty}} h(x_{k}) \,=\, L
        \end{displaymath}
    It now follows from the various `Squeeze Properties' of sequences that $\lim_{k \,{\rightarrow}\, {\infty}} g(x_{k}) \,=\, L$ as well.
    Since this is true for {\em all} such sequences ${\xi}$, it then follows from the `if' portion of Part~(3) of Theorem~\Ref{ThmC90.40} that ${\displaystyle \lim_{x \,{\rightarrow}\, c} g(x) \,=\, L}$, as claimed.

\V

        The next results provides analogs, for limits of functions on intervals, of the calculational results for sequences discussed in Section~\Ref{SectC60}.
    As in the preceding theorem, we phrase the results for two-sided limits, and leave to the reader to formulate and prove the corresponding one-sided results.

            \subsection{\small{\bf Theorem}}
            \label{ThmC90.60}
        
        Suppose that $(a,b)$ is an open interval in ${\RR}$ and that $c$ is a real number such that $a\,<\,c\,<\,b$.
    Let $X_{c} \,=\, (a,b){\setminus}\{c\} \,=\, (a,c)\,{\cup}\,(c,b)$, and let $f:X_{c} \,=\, {\RR}$ be a function whose domain includes $X_{c}$.

\V

        (a) (Constant Factor Rule) Suppose that ${\displaystyle \lim_{x \,{\rightarrow}\, c} f(x) \,=\, L}$ for some real number $L$.
    Then for every real number $A$ one has ${\displaystyle \lim_{x \,{\rightarrow}\, c} (A{\cdot}f)(x) \,=\, A{\cdot}L}$.

\V

        (b) Suppose that ${\displaystyle \lim_{x \,{\rightarrow}\, c}} f(x) \,=\, L$, where $L$ is an extended real number.
    Then ${\displaystyle \lim_{x \,{\rightarrow}\, c}} |f(x)| \,=\, |L|$.

\V

        (c) Suppose that ${\displaystyle \lim_{x \,{\rightarrow}\, c} f(x) \,=\, L}$,
    where $L$ is an extended real number. Also, assume that $L \,\,{\neq}\,\, 0$.
    Then there exists ${\delta}\,>\,0$ such that if $0\,<\,|c-x|\,\,{\leq}\,\,{\delta}$ then $f(x) \,\,{\neq}\,\, 0$.

        More precisely:

        \underline{Case (i)} If $L\,>\,0$ and $m$ is any number such that $0\,<\,m\,<\,L$, then there exists ${\delta}\,>\,0$ so that $f(x)\,\,{\geq}\,\,m$ for all $x$ such that $0\,<\,|x-c|\,\,{\leq}\,\,{\delta}$.

        \underline{Case (ii)} If $L\,<\,0$ and $m\,>\,0$ is any number such that $L\,<\,-m\,<\,0$, then there exists ${\delta}\,>\,0$ so that $f(x)\,\,{\leq}\,\,-m$ for all $x$ such that $0\,<\,|x-c|\,\,{\leq}\,\,{\delta}$.

\V


        As was the case with the preceding theorem, we prove here only one portion of the stated result -- namely, Case~(i) of Part~(d) -- and leave the rest as an easy, but instructive, exercise for the reader.

\V

        \underline{Proof of Case~(i) of Part~(c)} Suppose that $L\,>\,0$ and that the conclusion in~(c) is {\em not} true for a given $f$.
    Then there would have to exist a number $m$, with $0\,<\,m\,<\,L$, such that the following holds:

        for every $k$ in ${\NN}$ there exists a number $x_{k}$ in the set $X_{c}$ such that $|c-x_{k}|\,<\,1/k$ and $f(x_{k})\,<\,m$.
    Let ${\xi} \,=\, (x_{1},x_{2},\,{\ldots}\,)$ be the corresponding infinite sequence.
    Since, by construction, one has $|c-x_{k}|\,<\,1/k$, it is clear that the sequence ${\xi}$ converges to $c$.
    Thus by Part~(c) of Theorem~\Ref{ThmC90.40} it follows that $\lim_{k \,{\rightarrow}\, {\infty}} f(x_{k}) \,=\, L$.
    However, one also has, by construction, $f(x_{k})\,<\,m$ for each $k$; thus by Part~(e) of Theorem~\Ref{ThmC20.10B} one also has $L\,\,{\leq}\,\,m$.
    This contradicts the hypothesis that $0\,<\,m\,<\,L$.

\V
\V

            \subsection{\small{\bf Theorem}}
            \label{ThmC90.70}

        Suppose that $(a,b)$ is an open interval in ${\RR}$ and that $c$ is a real number such that $a\,<\,c\,<\,b$.
    Let $X_{c} \,=\, (a,b){\setminus}\{c\} \,=\, (a,c)\,{\cup}\,(c,b)$, and let $f:X_{c} \,=\, {\RR}$ and $g:X_{c} \,{\rightarrow}\, {\RR}$ be functions whose domains include the set $X_{c}$.
    Assume that ${\displaystyle \lim_{x \,{\rightarrow}\, c} f(x) \,=\, A}$ and ${\displaystyle \lim_{x \,{\rightarrow}\, c} g(x) \,=\, B}$,
    where $A$ and $B$ are real numbers.
    Then:

\V
        (a) (Sum Rule) ${\displaystyle \lim_{x \,{\rightarrow}\, c} (f(x)+g(x)) \,=\, A+B}$.

\V

        (b) (Difference Rule) ${\displaystyle \lim_{x \,{\rightarrow}\, c} (f(x)-g(x)) \,=\, A-B}$.

\V

        (c) (Product Rule) ${\displaystyle \lim_{x \,{\rightarrow}\, c} (f(x){\cdot}g(x)) \,=\, A{\cdot}B}$.

\V

        (d)(Quotient Rule) Suppose that, in addition, $g(x) \,\,{\neq}\,\, 0$ when $x{\in}X_{c}$, and that $B \,\,{\neq}\,\, 0$.
    Then ${\displaystyle \lim_{x \,{\rightarrow}\, c} (f(x)/g(x)) \,=\, A/B}$.

    Note: The requirement that $g(x) \,\,{\neq}\,\, 0$ when $x{\in}X_{c}$ is included just to simplify the statement of the result.
    In light of Part~(c) of Theorem~\Ref{ThmC90.60}, however, it is clear that this condition can be dropped because of the hypothesis that $B \,\,{\neq}\,\, 0$.
    (Recall that whether the limit as $x \,{\rightarrow}\, c$ exists or not depends only on the behavior of the function when $x$ is arbitrarily close to $c$.)

\V

        Once again, each part of this theorem can be proved by using the corresponding result for sequences.
    Thus, let us carry out the process in just one part -- namely Part~(c) -- and leave the rest of the proof to the reader.


\V

        \underline{Proof of Part (c)} Let ${\xi} \,=\, (x_{1},x_{2},\,{\ldots}\,)$ be a sequence of real numbers in $X_{c}$ such that $\lim_{k \,{\rightarrow}\, {\infty}} x_{k} \,=\, c$.
    Then the hypotheses that ${\displaystyle \lim_{x \,{\rightarrow}\, c} f(x) \,=\, A}$ and ${\displaystyle \lim_{x \,{\rightarrow}\, c} g(x) \,=\, B}$ imply, 
    by Theorem~\Ref{ThmC90.40}, that
        \begin{displaymath}
        \lim_{k \,{\rightarrow}\, {\infty}} f(x_{k}) \,=\, A \mbox{ and }
        \lim_{k \,{\rightarrow}\, {\infty}} g(x_{k}) \,=\, B.
        \end{displaymath}
    It now follows from Part~(b) of Theorem~\Ref{ThmC60.30} (i.e., from the Product Rule for Convergent Sequences) that
        \begin{displaymath}
        \lim_{k \,{\rightarrow}\, {\infty}} (f(x_{k}){\cdot}g(x_{k})) \,=\, A{\cdot}B.
        \end{displaymath}
    Since this is true for every such sequence, it follows from Theorem~\Ref{ThmC90.40} again that
        \begin{displaymath}
        \lim_{x \,{\rightarrow}\, c} (f(x){\cdot}g(x)) \,=\, A{\cdot}B,
        \end{displaymath}
    as claimed.

\V

            \subsection{\small{\bf Remark}}
            \label{RemrkC90.80}

        The preceding result provides `Sum', `Difference', `Product' and `Quotient' rules for limits.
    It is natural to ask about a correponding `Composition' rule for limits.
    The obvious statement would be something like this:

        `If ${\displaystyle \lim_{x \,{\rightarrow}\, c}} f(x) \,=\, b$ and ${\displaystyle \lim_{y \,{\rightarrow}\, b} g(y) \,=\, L}$, then
    ${\displaystyle \lim_{x \,{\rightarrow}\, c} g(f(x)) \,=\, L}$.'

\noindent Intuitively, this equation seems quite plausible:  As $y$ gets close to $b$, $g(y)$ gets close to $L$,
    and as $x$ gets close to $a$, the quantity $y \,=\, f(x)$ gets close to $b$;
    thus combining these facts seems to imply the `law' stated above.

    Unfortunately, this `law' is \underline{not} true in general.

\V

        \underline{Example}: Suppose that $f(x) \,=\, 1$ for all $x$ in ${\RR}$.
    Likewise, suppose that $g:{\RR} \,{\rightarrow}\, {\RR}$ is given by the rule that $g(y) \,=\, 4$ if $y \,\,{\neq}\,\, 1$, but $g(1) \,=\, 0$.
    It is clear that
        \begin{displaymath}
        \lim_{x \,{\rightarrow}\, 0} f(x) \,=\, 1 \mbox{ and } 
        \lim_{y \,{\rightarrow}\, 1} g(x) \,=\, 4
        \end{displaymath}
    However, $g(f(x)) \,=\, g(1) \,=\, 0$ for all $x$, so that ${\displaystyle \lim_{x \,{\rightarrow}\, 0} g(f(x))} \,=\, 0$, not $4$ as the `law' would have predicted.

\V
\V

        There are also analogs for `limits on intervals' of the Monotonic-Sequences Principle (i.e., Part~(b) of Theorem~\Ref{ThmC20.10B}).
    Of necessity, these analogs involve one-sided limits.

\V

            \subsection{\small{\bf Theorem (Limits of Monotonic Functions)}}
            \label{ThmC90.90}

        Let $(a,b)$ be an open interval in ${\RR}$, so that $-{\infty}\,\,{\leq}\,\,a\,<\,b\,\,{\leq}\,\,+{\infty}$.

\V

       (a) Suppose that a function $f:(a,b) \,{\rightarrow}\, {\RR}$ is monotonic up on $(a,b)$. Then the one-sided limits $\lim_{x{\nearrow}b} f(x)$ and $\lim_{x{\searrow}a} f(x)$ both exist.
    More precisely,
        \begin{displaymath}
        \lim_{x{\nearrow}b} f(x) \,=\, {\sup}\,\{f(x): a\,<\,x\,<\,b\}
    \mbox{ and }
        \lim_{x{\searrow}a} f(x) \,=\, {\inf}\,\{f(x): a\,<\,x\,<\,b\}
        \end{displaymath}
    In particular, 
        \begin{displaymath}
        \lim_{x{\searrow}a} f(x)\,\,{\leq}\,\,f(u)\,\,{\leq}\,\,\lim_{x{\nearrow}b} f(x) \mbox{ for all $u$ in $(a,b)$}.
        \end{displaymath}

\V

        (b) Suppose, instead, that the function $f:(a,b) \,{\rightarrow}\, {\RR}$ is monotonic down on $(a,b)$.
    Then the one-sided limits $\lim_{x{\nearrow}b} f(x)$ and $\lim_{x{\searrow}a} f(x)$ both exist.
    More precisely,
        \begin{displaymath}
        \lim_{x{\nearrow}b} f(x) \,=\, {\inf}\,\{f(x): a\,<\,x\,<\,b\}
    \mbox{ and }
        \lim_{x{\searrow}a} f(x) \,=\, {\sup}\,\{f(x): a\,<\,x\,<\,b\}
        \end{displaymath}
    In particular, 
        \begin{displaymath}
        \lim_{x{\searrow}a} f(x)\,\,{\geq}\,\,f(u)\,\,{\geq}\,\,\lim_{x{\nearrow}b} f(x) \mbox{ for all $u$ in $(a,b)$}.
        \end{displaymath}

\V

        \underline{Proof}

\V

\hspace*{\parindent}(a) Let $M \,=\, {\sup}\,\{f(x): a\,<\,x\,<\,b\}$.

        \underline{Case 1} Suppose $M$ is finite. Let ${\varepsilon}\,>\,0$ be given. Then, by basic properties of `supremum',
    there exists $c$ in $(a,b)$ such that $0\,\,{\leq}\,\,M-f(c)\,<\,{\varepsilon}$.
    By the hypothesis that $f$ is monotonic up, and the fact that $M\,\,{\geq}\,\,f(x)$, 
    it follows that if $c\,<\,x\,<\,b$ then $0\,\,{\leq}\,\,M-f(x)\,\,{\leq}\,\,M-f(c)\,<\,{\varepsilon}$.
    Thus $\lim_{x{\nearrow}b} f(x) \,=\, M$.

        \underline{Case 2} Suppose that $M \,=\, +{\infty}$. Then if $B$ is any number, there exists a number $c$ in $(a,b)$ such that $f(c)\,\,{\geq}\,\,B$.
    The monotonicity hypothesis then implies that if $c\,<\,x\,<\,b$ then $f(x) \,\,{\geq}\,\,f(c)\,\,{\geq}\,\,B$.
    Thus $\lim_{x{\nearrow}b} f(x) \,=\, +{\infty}$.

        The proof that the one-sided limit $\lim_{x{\searrow}a} f(x)$ also exists, and is finite if, and only if, $f$ is bounded below, is similar, and is left as an exercise for the reader.

\V

        (b) If $f$ is monotonic down on $(a,b)$, then apply the results of Part~(a) to the function $g \,=\, -f$.

\V

        \underline{Note} Monotonic functions are studied in more detail in several later chapters.

}%\EndSkip
%------------

%------------
\StartSkip{

                \section{{\bf Some Extensions of the Preceding Theory}}
                \label{SectC100}\IndB{ZZ Sections}{\Ref{SectC100} Some Extensions of the Preceding Theory}


        The preceding theory of limits of sequences of real numbers is perfectly adequate for many purposes in analysis.
    There are numerous extensions of that theory, however, in which one is able to assign a `limit value' to some sequences which do {\em not} have limits (finite or infinite) in the usual sense.
    The need for such extensions usually comes out of the advanced portions of analysis, and thus are not easy to motivate at this stage of {\ThisText}.
    Thus, we shall discuss only a couple of important examples, and then only briefly.

\V
\V

        {\bf I -- Limits of Averages}

\V
\V


            \subsection{\small{\bf Theorem}}
            \label{ThmC100.20}

\V

        Suppose that ${\xi} \,=\, (x_{1},x_{2},\,{\ldots}\,x_{k},\,{\ldots}\,)$ is an infinite sequence of real numbers for which $\lim_{k \,{\rightarrow}\, {\infty}} x_{k}$ exists,
    in the sense of Definition~\Ref{DefC40.10}.
    Let $L \,=\, \lim_{k \,{\rightarrow}\, {\infty}} x_{k}$.

\V


        (a) Define a second sequence ${\zeta} \,=\, (z_{1},z_{2},\,{\ldots}\,z_{k},\,{\ldots}\,)$ from ${\xi}$ by the rule
        \begin{displaymath}
        z_{k} \,=\, \frac{x_{1}+x_{2}+\,{\ldots}\,+x_{k}}{k} \mbox{ for each $k$ in ${\NN}$}.
        \end{displaymath}
    Then
        \begin{displaymath}
        \lim_{k \,{\rightarrow}\, {\infty}} z_{k} \,=\, L.
        \end{displaymath}

        Note:  One calls ${\zeta}$ the {\bf sequence of averages of increasing order associated with ${\xi}$}.


\V

        (b) Let $m$ be a fixed positive integer, and define a third sequence ${\tau} \,=\, (t_{1},t_{2},\,{\ldots}\,t_{k},\,{\ldots}\,)$ by the rule
        \begin{displaymath}
        t_{k} \,=\, \frac{x_{k}+x_{k+1}+\,{\ldots}\,+x_{k+m-1}}{m} \mbox{ for each $k$ in ${\NN}$}.
        \end{displaymath}
    Then
        \begin{displaymath}
        \lim_{k \,{\rightarrow}\, {\infty}} t_{k} \,=\, L.
        \end{displaymath}

        Note: In statistical theory one calls ${\tau}$ the {\bf sequence of moving averages of fixed order $m$ associated with ${\xi}$}.

\V

        {\bf Proof}

\V

        (a) \underline{Case 1} Suppose that the sequence ${\xi}$ is convergent; that is, $L$ is a real number.
    First note that if $m$ and $l$ are any positive integers, then one easily computes that
        \begin{displaymath}
        z_{m+l}-L \,=\, \frac{x_{1}+ \,{\ldots}\,+x_{m} + x_{m+1} + \,{\ldots}\,x_{m+l}}{m+l} - L \,=\, 
        \end{displaymath}
        \begin{displaymath}
    \frac{(x_{1}+\,{\ldots}\,+x_{m})-L}{m+l} + \frac{(x_{m+1}-L) + (x_{m+2}-L) + \,{\ldots}\,+ (x_{m+l}-L)}{m+l}.
        \end{displaymath}
    By using the Extended Triangle Inequality one then obtains
        \begin{displaymath}
        |z_{m+l}-L|\,\,{\leq}\,\,\frac{|(x_{1}+\,{\ldots}\,+x_{m})-mL|}{m+l} + 
    \frac{|x_{m+1}-L| + \,{\ldots}\, + |x_{m+l}-L|}{m+l}.
        \end{displaymath}
    Let ${\varepsilon}\,>\,0$ be given. By the hypothesis that the sequence ${\xi}$ converges to $L$ there exists $m$ in ${\NN}$ such that if $j{\in}{\NN}$ then
        \begin{displaymath}
        |x_{m+j}-L|\,<\,\frac{{\varepsilon}}{2}.
        \end{displaymath}
   Choose such $m$ and keep it fixed for the rest of the argument. For such $m$ the quantity $|(x_{1}+\,{\ldots}\,+x_{m})-mL|$ is then also fixed,
    so by choosing $r$ in ${\NN}$ large enough then one can be sure that if $l$ in ${\NN}$ satisfies $l\,\,{\geq}\,\,r$, then
        \begin{displaymath}
        \frac{|(x_{1}+\,{\ldots}\,+x_{m})-mL|}{m+l}\,<\,\frac{{\varepsilon}}{2}.
        \end{displaymath}
    For such $l$ one also has $1/(m+l)\,<\,1/l$, so
        \begin{displaymath}
        \frac{|x_{m+1}-L| + \,{\ldots}\, + |x_{m+l}-L|}{m+l}\,\,{\leq}\,\,
    \frac{|x_{m+1}-L| + \,{\ldots}\, + |x_{m+l}-L|}{l}\,<\,\frac{{\varepsilon}/2+{\varepsilon}/2+\,{\ldots}\,+{\varepsilon}/2}{l} \,=\, \frac{l({\varepsilon}/2)}{l} \,=\, \frac{{\varepsilon}}{2}.
        \end{displaymath}
    Combining these results, one sees that if $k\,\,{\geq}\,\,m+r$, then
        \begin{displaymath}
        |z_{k}-L|\,\,{\leq}\,\,\frac{{\varepsilon}}{2} + \frac{{\varepsilon}}{2} \,=\, {\varepsilon}.
        \end{displaymath}
    It follows from this argument that the sequence ${\zeta}$ converges to $L$, as claimed.

        \underline{Case 2} Suppose that $L \,=\, +{\infty}$ or $L \,=\, -{\infty}$. Then a similar analysis shows that $\lim_{k \,{\rightarrow}\, {\infty}} z_{k} \,=\, L$.
    The details are left to the reader as an exercise.

\V

        (b) The simple proof is left to the reader as an exercise.

\V
\V

        The preceding results can be rephrased as follows:

        \h (a) A \underline{necessary} condition for the original sequence ${\xi}$ to satisfy $\lim_{k \,{\rightarrow}\, {\infty}} x_{k} \,=\, L$
    is that the sequence ${\zeta}$ satisfy $\lim_{k \,{\rightarrow}\, {\infty}} z_{k} \,=\, L$.

        \h (b) A \underline{necessary} condition for the original sequence ${\xi}$ to satisfy $\lim_{k \,{\rightarrow}\, {\infty}} x_{k} \,=\, L$
    is that, for each $m$, the sequence ${\tau}$ satisfy $\lim_{k \,{\rightarrow}\, {\infty}} t_{k} \,=\, L$.

        The word `necessary' is underlined to emphasize the fact that neither of the conditions $\lim_{k \,{\rightarrow}\, {\infty}} z_{k} \,=\, L$ nor $\lim_{k \,{\rightarrow}\, {\infty}} t_{k} \,=\, L$ is sufficient to enure that $\lim_{k \,{\rightarrow}\, {\infty}} x_{k} \,=\, L$ holds.

\V
\V


            \subsection{\small{\bf Examples}}
            \label{ExampC100.30}

\V

\hspace*{\parindent}
        (1) Let ${\xi} \,=\, (x_{1},x_{2},\,{\ldots}\,)$ be the sequence defined by the rule $x_{k} \,=\, (-1)^{k-1}$ for each $k$ in ${\NN}$.\
    Written as an ordered list it takes the form
        \begin{displaymath}
        {\xi} \,=\, (1,-1,1,-1,\,{\ldots}\,)
        \end{displaymath}
    It is clear that this sequence fails to have a limit, since there are subsequences which converge to different limits.
    However, one easily computes that
        \begin{displaymath}
        x_{1} + \,{\ldots}\, + x_{k} \,=\, \left\{
        \begin{array}{ll}
        1 & \mbox{if $k$ is odd} \\
        0 & \mbox{if $k$ is even}
        \end{array}
                \right.
        \end{displaymath}
    It follows that
        \begin{displaymath}
        z_{k} \,=\, \left\{
        \begin{array}{ll}
        {\displaystyle \frac{1}{k}} & \mbox{if $k$ is odd} \\
        &                                \\
        0 & \mbox{if $k$ is even}
        \end{array}
                        \right.
        \end{displaymath}
    That is,
        \begin{displaymath}
        {\zeta} \,=\, \left(1,0,\frac{1}{3},0,\frac{1}{5},0,\,{\ldots}\,\right),
        \end{displaymath}
    which clearly converges to $0$.

\V

        (2) Let ${\xi}$ be the same sequence discussed in the preceding example, and let $m$ be a fixed element of ${\NN}$.
    It is easy to see that for each $k$ in ${\NN}$ the $k$-th term $t_{k}$ of the corresponding sequence ${\tau}$ of moving averages of order $m$ is given by
        \begin{displaymath}
        t_{k} \,=\, \frac{x_{k}+x_{k+1}+\,{\ldots}\,+x_{k+m-1}}{m} \,=\, 
        \left\{
        \begin{array}{ll}
        {\displaystyle \frac{1}{m}} & \mbox{for all $k$ if $m$ is odd} \\
                                    &                      \\
        0                           & \mbox{for all $k$ if $m$ is even}
        \end{array}
                        \right.
        \end{displaymath}
    Thus the sequence ${\tau}$ is always a constant sequence, and thus convergent, no matter which $m$ is chosen as order.
    However, the limit of ${\tau}$ certainly does depend on $m$.

\V

        The preceding examples illustrate an intriguing possibility: Given an infinite sequence ${\xi} \,=\, (x_{1},x_{2},\,{\ldots}\,)$ of real numbers, 
    there may be a reasonable way to assign a `limiting value' to ${\xi}$, even if $\lim_{k \,{\rightarrow}\, {\infty}} x_{k}$ does not exist in the sense of
    Definition~\Ref{DefC40.10}.
    It turns out that such `convergence schemes' are of considerable use in advanced portions of analysis, especially those involving Fourier Series.
    In particular, the scheme described in Example~(1) above is associated with the name of `Cesaro'.

\V
\V

            \subsection{\small{\bf Definition}}
            \label{DefC100.40}

        Let ${\xi} \,=\, (x_{1},x_{2},\,{\ldots}\,)$ be an infinite sequence of real numbers.
    Let ${\zeta} \,=\, (z_{1},z_{2},\,{\ldots}\,)$ be the corresponding sequence of average values, so that
        \begin{displaymath}
        z_{k} \,=\, \frac{x_{1} + \,{\ldots}\, + x_{k}}{k} \mbox{ for each $k$ in ${\NN}$}.
        \end{displaymath}
    Let $L$ be an extended real number.
    One says that the original sequence ${\xi}$ {\bf has limit $L$ in the sense of Cesaro}, which one writes in symbols as
        \begin{displaymath}
    \lim_{k \,{\rightarrow}\, {\infty}} x_{k} \,=\, L \h (C),
        \end{displaymath}
    provided the sequence ${\zeta}$ satisfies $\lim_{k \,{\rightarrow}\, {\infty}} z_{k} \,=\, L$;
    in the last equation, the limit stands for the usual limit as described in Definition~\Ref{DefC40.10}.

\V

        {\bf Remark} It could happen that the second sequence ${\zeta}$ does not have a limit, but that $\lim_{k \,{\rightarrow}\, {\infty}} z_{k} \,=\, L \h (C)$ holds.
    In this case one says that the original sequence ${\xi}$ {\bf has limit $L$ in the second-order Cesaro sense},
   and one writes $\lim_{k \,{\rightarrow}\, {\infty}} x_{k} \,=\, L \h (C_{2})$.
    Limits of higher order `in the Cesaro sense' can be defined analogously.
    With this extension, the original Cesaro limits are then indicated by $C_{1}$.

\V
\V

        {\bf II -- Double Sequences}

\V
\V


            \subsection{\small{\bf Definition}}
            \label{DefC100.70}

        Let $X$ be a nonempty set.

\V

        (1) A {\bf double sequence with values in $X$} is a function ${\Xi}:{\NN}{\times}{\NN} \,{\rightarrow}\, {\RR}$.
    It is often convenient to picture a double sequence as an infinite matrix, i.e., as an infinite rectangular array:
        \begin{displaymath}
        {\Xi} \,=\, \left[
        \begin{array}{llclc}
        x_{11} & x_{12} & \,{\ldots}\,& x_{1j} & \,{\ldots}\, \\
        x_{21} & x_{22} & \,{\ldots}\,& x_{2j} & \,{\ldots}\, \\
      {\vdots} &        &             &        &              \\
        x_{i1} & x_{i2} & \,{\ldots}\,& x_{ij} & \,{\ldots}\, \\
      {\vdots} &        &             &        &
        \end{array}
                                \right]
        \end{displaymath}
    In this representation the entry $x_{jk}$ located at the intersection of the $j$-th row and $k$-th column is the value ${\Xi}(j,k)$.
    It is convenient to refer to $(j,k)$ as a {\bf double index}.

\V

        (2) More generally, if $k$ is any natural number then a {\bf $k$-fold sequence with values in $X$} is a function ${\Xi}:{\NN}^{k} \,{\rightarrow}\, X$.
    If the specific value of $k$ is not given, one also refers to ${\Xi}$ as a {\bf multiple sequence}.

        \underline{Special Cases} $1$-fold sequences are also called {\bf simple sequences}; $2$-fold sequences are also called {\bf double sequences}, and so on.

    One normally denotes the values of a $k$-fold sequence using $k$ subscripts:
    ${\Xi}(j_{1},j_{2},\,{\ldots}\,j_{k}) \,=\, x_{j_{1}j_{2}\,{\ldots}\,j_{k}}$, for example.
    In analogy with the case $k \,=\, 2$, it is sometimes useful to refer to a $k$-fold sequence ${\Xi}$ as an {\bf infinite $k$-dimensional array}, and to write ${\Xi} \,=\, (x_{j_{1},j_{2},\,{\ldots}\,j_{k}})$ as a shorthand.
    When $k \,=\, 3$ one can picture such a multiple sequence as a three-dimensional version of the matrix above,
    but it normally not particularly useful to do so; needless to say, when $k\,\,{\geq}\,\,4$ the temptation to `draw' such a picture disappears.

\V
\V


            \subsection{\small{\bf Examples}}
            \label{ExampC100.80}

\hspace*{\parindent}
        (1) Suppose that, for each $i$ in ${\NN}$, ${\xi}_{i}$ denotes a simple sequence with values in a nonempty set $X$;
    thus, ${\xi}_{i}:{\NN} \,{\rightarrow}\, X$ is an $X$-valued function defined on ${\NN}$.
    Let us write
        \begin{displaymath}
        {\xi}_{i} \,=\, (u_{i1},u_{i2},\,{\ldots}\,) \mbox{ for each $i$ in ${\NN}$}.
        \end{displaymath}
    Then this data determine a double sequence ${\Xi}:{\NN}{\times}{\NN} \,{\rightarrow}\, {\RR}$ by the rule
        \begin{displaymath}
        {\Xi}(i,j) \,=\, u_{ij} \mbox{ for each $i,j$ in ${\NN}$}.
        \end{displaymath}
    In terms of the matrix viewpoint, the simple sequence ${\xi}_{i}$ forms the $i$-th row of the matrix which represents ${\Xi}$.

        It is clear that every double sequence with values in $X$ can be formed this way.

\V

        (2) Suppose that ${\xi} \,=\, (x_{1},x_{2},\,{\ldots}\,)$ denotes a simple sequence of real numbers.
    Then the statement of the Cauchy Criterion for ${\xi}$ involves a certain double sequence ${\Xi} \,=\, (u_{ij})$ defined by the rule
        \begin{displaymath}
        u_{ij} \,=\, x_{i}-x_{j} \mbox{ for each $i,j$ in ${\NN}$}.
        \end{displaymath}
    It makes sense to call this the {\bf double sequence of differences associated with the simple sequence ${\xi}$}.

\V
\V

        We are going to define below a notion of `limit of a double sequence of real numbers'.
    A natural way to motivate that definition is with a natural requirement. Namely:

\V

        \h `{\em Any reasonable definition of what it means for a double sequence ${\Xi} \,=\, (u_{ij})$ to converge a value $L$ should imply,
    as a special case, that if ${\xi} \,=\, (x_{1},x_{2},\,{\ldots}\,)$ is any simple sequence of reals,
    then ${\xi}$ is a Cauchy sequence if, and only if, the double sequence of differences associated with ${\xi}$ (see Example~(2) above) converges to $0$.}'

\V

        The following definition satisfies this requirement perfectly. It was published (in 1897) by the German mathematician Alfred Pringsheim;
    incidently, he was also the father-in-law of the great German novelist Thomas Mann.

\V

            \subsection{\small{\bf Definition} (The Pringsheim Limit of a Double Sequence)}
            \label{DefC100.100}

\V


        Let ${\Xi}:{\NN}{\times}{\NN} \,{\rightarrow}\, {\RR}$ be a double sequence of real numbers.
    That is, ${\Xi} \,=\, (u_{ij})$, where $u_{ij}{\in}{\RR}$ for each $i,j{\in}{\NN}$.

\V

        (1) Let $L$ be a real number. One says that ${\Xi}$ {\bf converges to $L$ in the sense of Pringsheim} if for every ${\varepsilon}\,>\,0$ there exists a number $B$ such that
        \begin{displaymath}
        |L- u_{ij}|\,<\,{\varepsilon} \mbox{ for all $i,j{\in}{\NN}$ such that $i\,\,{\geq}\,\,B$ and $j\,\,{\geq}\,\,B$}.
        \end{displaymath}
    In this case one writes $\lim_{i,j \,{\rightarrow}\, {\infty}}^{P} x_{ij} \,=\, L$.
    One sometimes refers to the Pringsheim limit $\lim_{i,j \,{\rightarrow}\, {\infty}}^{P} u_{ij}$ as the {\bf double limit of ${\Xi}$}.

\V

        (2) One says that $\lim_{i,j \,{\rightarrow}\, {\infty}} u_{ij} \,=\, +{\infty}$ if for every $M$ in ${\RR}$ there exists $B$ such that if $i\,\,{\geq}\,\,B$ and $j\,\,{\geq}\,\,B$ then $u_{ij}\,\,{\geq}\,\,M$.

\V

        (3) One says that $\lim_{i,j \,{\rightarrow}\, {\infty}} u_{ij} \,=\, -{\infty}$ if $\lim_{i,j \,{\rightarrow}\, {\infty}} (-u_{ij}) \,=\, +{\infty}$, in the sense of Part~(2).

\V
\V

        {\bf Remark}

        The reader should be able to generalize the preceding definitions to the case of multiple sequences of arbitrary order.


\V
\V

            \subsection{\small{\bf Examples}}
            \label{ExampC100.110}

\V

\hspace*{\parindent}
        (1) Define ${\Xi}:{\NN}{\times}{\NN} \,{\rightarrow}\, {\RR}$ by the rule
        \begin{displaymath}
        {\Xi}(i,j) \,=\, u_{ij} \,=\,  \left\{
        \begin{array}{cl}
        i & \mbox{if $j \,=\, 1$} \\
          &                       \\
        j & \mbox{if $i \,=\, 1$} \\
          &                       \\
        {\displaystyle 1+\frac{1}{i+j}} & \mbox{if $i\,\,{\geq}\,\,2$ and $j\,\,{\geq}\,\,2$} \\
          &
        \end{array}
                                    \right.
        \end{displaymath}
    In terms of the `matrix' interpretation, one has
        \begin{displaymath}
        {\Xi} \,=\, \left[
        \begin{array}{llclc}
        1 & 2 & \,{\ldots}\,& j & \,{\ldots}\, \\
        2 & 5/4 & \,{\ldots}\,& 1+1/(2+j) & \,{\ldots}\, \\
      {\vdots} &        &             &        &              \\
        i & 1+1/(i+2) & \,{\ldots}\,& 1+1/(i+j) & \,{\ldots}\, \\
      {\vdots} &        &             &        &
        \end{array}
                                \right]
        \end{displaymath}


        \underline{Claim} The double sequence ${\Xi}$ converges to $1$ in the Pringsheim sense.

        \underline{Proof of Claim} Let ${\varepsilon}\,>\,0$ be given. Note that if $i\,\,{\geq}\,\,2$ and $j\,\,{\geq}\,\,2$ then $u_{ij} \,=\, 1+1/(i+j)$, so that
        \begin{displaymath}
        |1-u_{ij}| \,=\, \left| 1-\left(1+\frac{1}{i+j}\right)\right| \,=\, -\frac{1}{i+j}.
        \end{displaymath}
    Now let $B$ be any number such that $B\,>\,\max\{2,1/{\varepsilon}\}$. Then for all $i,j$ in ${\NN}$ such that $i,j\,\,{\geq}\,\,B$ one has $i\,\,{\geq}\,\,2$ and $j\,>\,2$ and $i+j\,\,{\geq}\,\,B$. Thus
        \begin{displaymath}
        \left|1- u_{ij}\right| \,=\, \frac{1}{i+j}\,\,{\leq}\,\,\frac{1}{B} \,<\,{\varepsilon}.
        \end{displaymath}
    Thus, $\lim_{i,j \,{\rightarrow}\, {\infty}}^{P} u_{ij} \,=\, 1$.

\V

        (2) Let ${\Xi} \,=\, (u_{ij})$ be the double sequence given by the rule
        \begin{displaymath}
        {\Xi}(i,j) \,=\, u_{ij} \,=\, \frac{1}{i} \mbox{ for each $i$ and $j$ in ${\NN}$}.
        \end{displaymath}
    Thus, the matrix associated with ${\Xi}$ is
        \begin{displaymath}
        \left[
        \begin{array}{llclc}
        1 & 1 & \,{\ldots}\,& 1 & \,{\ldots}\, \\
        1/2 & 1/2 & \,{\ldots}\,& 1/2 & \,{\ldots}\, \\
      {\vdots} &        &             &        &              \\
        1/i & 1/i & \,{\ldots}\,& 1/i & \,{\ldots}\, \\
      {\vdots} &        &             &        &
        \end{array}
                                \right]
        \end{displaymath}
    It is easy to see, by an analysis similar to that carried out in Example~(1), that $\lim_{i,j \,{\rightarrow}\, {\infty}}^{P} u_{ij} \,=\, 0$;
    in particular, the double sequence ${\Xi}$ is convergent in the sense of Pringsheim.
    Notice also that the simple sequences formed by the rows of this matrix are convergent in the sense of Definition~\Ref{DefC10.10},
    but none of these simple sequences has $0$ as limit; indeed, the sequence formed by the $i$-th row converges to $1/i$.
    In contrast, the simple sequences formed by the {\em columns} of the matrix all converge to $0$.

\V

        (3) Let ${\Xi} \,=\, (u_{ij})$ be the double sequence given by the rule $u_{ij} \,=\, i/(i+j)$ for each $i,j$ in ${\NN}$.
    It is easy to see that the simple sequences formed from the rows of the infinite matrix associated with ${\Xi}$ all converge to $0$, while the simple sequences formed from the columns all converge to $1$.
    In symbols:
        \begin{displaymath}
        \lim_{j \,{\rightarrow}\, {\infty}} \frac{i}{i+j} \,=\, 0 \mbox{ for all $i$ in ${\NN}$},
        \end{displaymath}
    while
        \begin{displaymath}
        \lim_{i \,{\rightarrow}\, {\infty}} \frac{i}{i+j} \,=\, 1 \mbox{ for all $j$ in ${\NN}$}.
        \end{displaymath}
    It is a simple exercise to check that the double sequence ${\Xi}$ does {\em not} have a limit in the sense of Pringsheim.

\V

        (4) Let ${\Xi} \,=\, (u_{ij})$ be the double sequence given by the rule
        \begin{displaymath}
        u_{ij} \,=\, \frac{(-1)^{j-1}}{i} + \frac{(-1)^{i-1}}{j} \mbox{ for all $i,j$ in ${\NN}$}.
        \end{displaymath}
    It is a simple exercise to show that ${\Xi}$ does converge in the sense of Pringsheim;
    indeed, that $\lim_{i,j \,{\rightarrow}\, {\infty}}^{P} u_{ij} \,=\, 0$.
    In contrast, none of the rows of the matrix $(u_{ij})$ forms a convergent simple sequence; likewise, none of the columns of this matrix converge.

\V
\V

            \subsection{\small{\bf Remarks}}
            \label{RemrkC100.120}

\V

\hspace*{\parindent}
        (1) In Example~(1) above the double sequence ${\Xi}$ converges to~$1$, yet ${\Xi}$ is not bounded.
    For example, the entry in the $j$-th column of the first row is $j$, which can become arbitrarily large.
    This contrasts strongly with the properties of convergent {\em simple} sequences, which of course must be bounded.

\V

        (2) The preceding examples also illustrate the fact that if a double sequence ${\Xi}$ converges to a number $L$ in the sense of Pringsheim,
    it may happen that some simple `subsequences' of ${\Xi}$ may fail to converge in the sense of Definition~\Ref{DefC10.10},
    or they may converge, but to a value different from $L$.
    This also contrasts strongly with the behavior of subsequences of simple sequences.

\V
\V


        The definition of `limit of a double sequence' given above is based largely on Part~(2) of Example~\Ref{ExampC100.80}.
    An obvious alternative would be to base the concept instead on Part~(1) of the same example;
    namely, in terms of repeatedly computing simple limits.
    Unfortunately, the examples given in~\Ref{ExampC100.110} show that this procedure has its difficulties.
    Nevertheless, one can frequently reduce the calculation of the Pringsheim limit of a double sequence to the repeated calculation of simple limits.
    Since we have already developed an extensive theory for simple limits, this often simplifies the calculation of double limits. 
    The next result makes this more precise.

\V
\V

            \subsection{\small{\bf Theorem}}
            \label{ThmC100.130}

\V

\hspace*{\parindent}
        (a) Suppose that ${\Xi}$ is a double sequence of real numbers which is formed, as in Part~(1) of Example~\Ref{ExampC100.80} above,
    as a sequence of sequences.
    That is, for each $i$ in ${\NN}$ one is given a simple sequence ${\xi}_{i}:{\NN} \,{\rightarrow}\, {\RR}$,
    and ${\Xi}$ is the double sequence whose matrix has ${\alpha}_{i}$ as $i$-th row for each $i$ in ${\NN}$. Let $u_{ij}$ denote the $(i,j)$-entry of ${\Xi}$, so that the $j$-th entry of ${\alpha}_{i}$ is $u_{ij}$.

        If the Pringsheim limit $\lim_{i,j \,{\rightarrow}\, {\infty}}^{P} u_{ij}$ exists and equals $L$, as described in Definition~\Ref{DefC100.40} above,
    and if, for each $i$ in ${\NN}$, the sequences ${\alpha}_{i}$ has finite limit $L_{i}$, then
        \begin{displaymath}
        L \,=\, \lim_{i \,{\rightarrow}\, {\infty}} L_{i};
        \end{displaymath}
    the limit on the right side of the last equation is the usual limit for simple sequences as described in Definition~\Ref{DefC40.10}.

        \underline{Alternate Formulation} The preceding equation is often written
        \begin{equation}
        \label{EqnC.160A}
        \lim_{i,j \,{\rightarrow}\, {\infty}}^{P} u_{ij} \,=\, \lim_{i \,=\, 1}^{{\infty}} \lim_{j \,=\, 1}^{{\infty}} u_{ij}.
        \end{equation}

\V

        (b) A similar result holds if one thinks of the double sequence ${\Xi}$ as being formed by its columns.
    Then the rule becomes:
        \begin{equation}
        \label{EqnC.160B}
        \lim_{i,j \,{\rightarrow}\, {\infty}}^{P} u_{ij} \,=\, \lim_{j \,=\, 1}^{{\infty}} \lim_{i \,=\, 1}^{{\infty}} u_{ij}.
        \end{equation}


\V

        {\bf Proof}

\V

        (a) \underline{Case 1} Suppose that $L$ is finite. Let ${\varepsilon}\,>\,0$ be given, and let $M$ be a number large enough that if $i,j\,\,{\geq}\,\,M$ then $|L-u_{ij}|\,<\,{\varepsilon}$.
    Fix the index $i$ for the moment, and let $j \,{\rightarrow}\, {\infty}$.
    Then by the usual rules for convergence of simple sequences one gets
        \begin{displaymath}
        {\varepsilon}\,\,{\geq}\,\,\lim_{j \,{\rightarrow}\, {\infty}} |L-u_{ij}| \,=\, |L-\lim_{j \,{\rightarrow}\, {\infty}} u_{ij}| \,=\, |L-L_{i}|
        \end{displaymath}
    That is, for every ${\varepsilon}\,>\,0$ there exists $B$ so that if $i\,\,{\geq}\,\,B$ then $|L-L_{i}|\,\,{\leq}\,\,{\varepsilon}$.
    It follows that $\lim_{i \,{\rightarrow}\, {\infty}} L_{i} \,=\, L$, as claimed.

        \underline{Case 2} Suppose that $L \,=\, +{\infty}$ or $L \,=\, -{\infty}$.
    These situations are handled in much the same way; the details are left to the reader.

\V

        (b) Apply the results of Part~(a) to the transpose of the matrix $(u_{ij})$;
    that is, to the matrix $(w_{ij})$ given by the rule $w_{ij} \,=\, u_{ji}$ for each $i,j$ in ${\NN}$.

\V
\V

            \subsection{\small{\bf Definition}}
            \label{DefC100.140}

        Each of the expressions appearing on the right sides of Equations~\Ref{EqnC.160A} and~\Ref{EqnC.160B} is called an {\bf iterated (simple) limit}.
}%\EndSkip
%------------

\newpage

% Exercises_M140AB_C.TeX   Exercises for Chapter C

%
% Revised: 07/06/2016 Encoding: Western ASCII
%

%% NOTE: Copy the 52 lines below to each chapter, and change the Chapter letters.

%\thispagestyle{myheadings}


%\markboth{Exercises for Chapter~\ref{ChaptC} -}{Exercises for Chapter~\ref{ChaptC} -}

\newcommand{\ExCa}{{\bf \ref{ChaptC} - \,1} }
\newcommand{\ExCb}{{\bf \ref{ChaptC} - \,2} }
\newcommand{\ExCc}{{\bf \ref{ChaptC} - \,3} }
\newcommand{\ExCd}{{\bf \ref{ChaptC} - \,4} }
\newcommand{\ExCe}{{\bf \ref{ChaptC} - \,5} }
\newcommand{\ExCf}{{\bf \ref{ChaptC} - \,6} }
\newcommand{\ExCg}{{\bf \ref{ChaptC} - \,7} }
\newcommand{\ExCh}{{\bf \ref{ChaptC} - \,8} }
\newcommand{\ExCi}{{\bf \ref{ChaptC} - \,9} }
\newcommand{\ExCj}{{\bf \ref{ChaptC} -  10} }
\newcommand{\ExCk}{{\bf \ref{ChaptC} -  11} }
\newcommand{\ExCl}{{\bf \ref{ChaptC} -  12} }
\newcommand{\ExCm}{{\bf \ref{ChaptC} -  13} }
\newcommand{\ExCn}{{\bf \ref{ChaptC} -  14} }
\newcommand{\ExCo}{{\bf \ref{ChaptC} -  15} }
\newcommand{\ExCp}{{\bf \ref{ChaptC} -  16} }
\newcommand{\ExCq}{{\bf \ref{ChaptC} -  17} }
\newcommand{\ExCr}{{\bf \ref{ChaptC} -  18} }
\newcommand{\ExCs}{{\bf \ref{ChaptC} -  19} }
\newcommand{\ExCt}{{\bf \ref{ChaptC} -  20} }
\newcommand{\ExCu}{{\bf \ref{ChaptC} -  21} }
\newcommand{\ExCv}{{\bf \ref{ChaptC} -  22} }
\newcommand{\ExCw}{{\bf \ref{ChaptC} -  23} }
\newcommand{\ExCx}{{\bf \ref{ChaptC} -  24} }
\newcommand{\ExCy}{{\bf \ref{ChaptC} -  25} }
\newcommand{\ExCz}{{\bf \ref{ChaptC} -  26} }


\newcommand{\ExCaa}{{\bf \ref{ChaptC} - 27} }
\newcommand{\ExCab}{{\bf \ref{ChaptC} - 28} }
\newcommand{\ExCac}{{\bf \ref{ChaptC} - 29} }
\newcommand{\ExCad}{{\bf \ref{ChaptC} - 30} }
\newcommand{\ExCae}{{\bf \ref{ChaptC} - 31} }
\newcommand{\ExCaf}{{\bf \ref{ChaptC} - 32} }
\newcommand{\ExCag}{{\bf \ref{ChaptC} - 33} }
\newcommand{\ExCah}{{\bf \ref{ChaptC} - 34} }
\newcommand{\ExCai}{{\bf \ref{ChaptC} - 35} }
\newcommand{\ExCaj}{{\bf \ref{ChaptC} - 36} }
\newcommand{\ExCak}{{\bf \ref{ChaptC} - 37} }
\newcommand{\ExCal}{{\bf \ref{ChaptC} - 38} }
\newcommand{\ExCam}{{\bf \ref{ChaptC} - 39} }
\newcommand{\ExCan}{{\bf \ref{ChaptC} - 40} }
\newcommand{\ExCao}{{\bf \ref{ChaptC} - 41} }
\newcommand{\ExCap}{{\bf \ref{ChaptC} - 42} }
\newcommand{\ExCaq}{{\bf \ref{ChaptC} - 43} }
\newcommand{\ExCar}{{\bf \ref{ChaptC} - 44} }
\newcommand{\ExCas}{{\bf \ref{ChaptC} - 45} }
\newcommand{\ExCat}{{\bf \ref{ChaptC} - 46} }
\newcommand{\ExCau}{{\bf \ref{ChaptC} - 47} }
\newcommand{\ExCav}{{\bf \ref{ChaptC} - 48} }
\newcommand{\ExCaw}{{\bf \ref{ChaptC} - 49} }
\newcommand{\ExCax}{{\bf \ref{ChaptC} - 50} }
\newcommand{\ExCay}{{\bf \ref{ChaptC} - 51} }
\newcommand{\ExCaz}{{\bf \ref{ChaptC} - 52} }



                       \section{EXERCISES FOR CHAPTER~\ref{ChaptC}}
                        \label{SectCEX}

\V
\V
\V
\V

\noindent  \ExCa In each part of this exercise, determine -- directly from the definition of `limit' -- whether the given sequence is convergent or not.


\V

        (a) ${\xi} \,=\,(x_{1},x_{2},\,{\ldots}\,x_{k},\,{\ldots}\,)$, where ${\displaystyle x_{k} \,=\, \frac{3k+100}{k^{2} + 3}}$ for each $k$ in ${\NN}$.

\V

        (b) ${\tau} \,=\, (t_{1},t_{2},\,{\ldots}\,)$ where $t_{k} \,=\, {\displaystyle \frac{3k+17}{8k-17}}$ for each $k$ in ${\NN}$.

\V

        (c) ${\alpha} \,=\, (a_{1},a_{2},\,{\ldots}\,a_{k},\,{\ldots}\,)$, where ${\displaystyle a_{k} \,=\, \frac{3k^{2} + (-1)^{k}k}{k^{2}}}$.

\V

        (d) ${\beta} \,=\, (b_{1},b_{2},\,{\ldots}\,b_{k},\,{\ldots}\,)$, where $b_{k} \,=\, \sqrt{k+1}-\sqrt{k}$ for each $k$ in ${\NN}$.
    (You may assume that every positive real number has a unique positive square root.)

\V
\V                                                                        

 \noindent \ExCb (a) Prove that if $c$ is a real number such that $c \,>\, -1$, then $(1+c)^{k} \,\,{\geq}\,\, 1+kc$ for all natural numbers $k$.

        \V

        (b) Use the conclusion of Part~(a) to show that if $|r|\,<\,1$ then the sequence ${\xi} \,=\, (1, r, r^{2},\,{\ldots}\,)$, whose $k$-th term is $r^{k-1}$, coverges to~$0$.

\V
\V

\noindent \ExCc (a) \underline{Prove or Disprove}: Let ${\tau} \,=\, (t_{1}, t_{2}, \,{\ldots}\,)$ be a convergent sequence of real numbers,
    and suppose that ${\displaystyle \lim_{k {\rightarrow} {\infty}} t_{k} \,\,{\geq}\,\, a}$ for some number~$a$.
    Then there exists a number $N$ such that $k \,\,{\geq}\,\, N$ implies $t_{k} \,\,{\geq}\,\, a$.

\V

        (b) \underline{Prove or Disprove}: Let ${\xi} \,=\, (x_{1}, x_{2}, \,{\ldots}\,)$ be a convergent sequence of real numbers,
    and suppose that ${\displaystyle \lim_{k {\rightarrow} {\infty}} x_{k} \,>\, a}$ for some number~$a$.
    Then there exists a number $N$ such that $k \,\,{\geq}\,\, N$ implies $x_{k} \,>\,a$.

%% From (I) in HW #3 in W04 Math 140A. Notation changed.

\V
\V

\noindent \ExCd In both parts of this problem, ${\alpha} \,=\, (a_{1},a_{2},\,{\ldots}\,)$ and ${\beta} \,=\, (b_{1},b_{2},\,{\ldots}\,)$ are sequences of real numbers such that $a_{k} \,\,{\leq}\,\, b_{k}$ for all $k$ in ${\NN}$,
    and ${\displaystyle \lim_{k {\rightarrow} {\infty}} |b_{k}-a_{k}|} \,=\, 0$.

\V

        (a) \underline{Prove or Disprove}: Suppose that there exists a number $c$ such that $a_{k} \,\,{\leq}\,\, c \,\,{\leq}\,\, b_{k}$ for all $k{\in}{\NN}$.
    Then each of the sequences ${\alpha}$ and ${\beta}$ is convergent.

\V

        (b) \underline{Prove or Disprove}: Suppose that for each $k{\in}{\NN}$ there exists a number $c$ such that
    $a_{k} \,\,{\leq}\,\, c \,\,{\leq}\,\, b_{k}$.
    Then each of the sequences ${\alpha}$ and ${\beta}$ is convergent.

%% From (III) in HW #3 in W04 Math 140A. Notation changed.

\V
\V

\noindent \ExCe In each part of this problem, show that Statement~(i) implies Statement~(ii).
    Do this using only the axioms for an ordered field -- that is, without using `Completeness' of ${\RR}$.


\V

        (a) (i) The Bisection Principle; \, (ii) The Monotonic Sequences Principle

\V

        (b) (i) The Monotonic Sequences Principle; \, (ii) The Bisection Principle.

\V

        (c) (i) The Monotonic-Up Sequences Principle; \, (ii) The Supremum Principle.

\V
\V

\noindent \ExCf Prove Parts (a) and (b) of Theorem~\Ref{ThmC40.30} %% Thm C.5.4 on p.181


\V
\V

\noindent \ExCg Prove Parts (c) and (d) of Theorem~\Ref{ThmC40.30} %% Thm C.5.4 on p.181


\V
\V

\noindent \ExCh Prove Parts (e) and (f) of Theorem~\Ref{ThmC40.30} %% Thm C.5.4 on p.181

\V
\V

\noindent \ExCi Prove Parts (g) and (h) of Theorem~\Ref{ThmC40.30} %% Thm C.5.4 on p.181

\V
\V

\noindent \ExCj Let ${\xi} \,=\, (x_{1},x_{2},\,{\ldots}\,)$ be a sequence of {\em positive} real numbers,
    and let ${\rho} \,=\, (r_{1},r_{2},\,{\ldots}\,)$ be the corresponding sequence of ratios: $r_{k} \,=\, x_{k+1}/x_{k}$ for each $k$ in ${\NN}$.

\V

        \underline{Prove or Disprove} If the sequence ${\xi}$ is convergent, then so is the sequence ${\rho}$.

\V
\V

\noindent \ExCk {\bf Definition} Let $X$ be a nonempty subset of ${\RR}$. A point $c$ in ${\RR}$ is said to be an {\bf accumulation point of $X$} provided that there exists a sequence ${\xi} \,=\, (x_{1}, x_{2},\,{\ldots}\,x_{k},\,{\ldots}\,)$ of points in $X$ such that

        \h (i)\, for all $k$ in ${\NN}$ one has $x_{k} \,\,{\neq}\,\, c$;
        \h (ii) $\lim_{k \,{\rightarrow}\, {\infty}} x_{k} \,=\, c$.

\V

        (a) Prove that if a subset $X$ of ${\RR}$ has an accumulation point then $X$ is an infinite set.

\V

        (b) Give an example of a subset $X$ of ${\RR}$ which has exactly five accumulation points, three of which are elements of $X$, two which are not elements of $X$.

\V
\V

\noindent \ExCl (a) Prove the {\bf Bolzano-Weierstrass Theorem for Sets in ${\RR}$}:
    If $X$ is a bounded infinite subset of ${\RR}$, then $X$ has an accumulation point.
    (See Exercise~\ExCk for the definition of `accumulation point'.) 

\V

        (b) \underline{Prove or Disprove} If $X$ is an uncountable subset of ${\RR}$, then $X$ has an accumulation point.

\V
\V

\noindent \ExCm Give an alternate proof of Theorem~\Ref{ThmC30.30} along the lines of the proof of Theorem~\Ref{ThmC20.90}, the `Odd/Even Convergence Theorem'. %% i.e., Thm C.2.15 p.172

\V
\V

\noindent \ExCn Suppose that ${\xi}$ is a bounded infinite sequence of real numbers, and that ${\cal F} \,=\, (A_{1},A_{2},\,{\ldots}\,A_{n},\,{\ldots}\,)$ is a countably infinite family of infinite subsets $A_{n}$ of ${\NN}$ such that ${\NN} \,=\, {\bigcup}_{n=1}^{{\infty}} A_{n}$.

        \underline{Prove or Disprove}: If all of the subsequences of ${\xi}$ corresponding to the subsets $A_{n}$ in the family ${\cal F}$ converge to the same real number $L$, then the original sequence ${\xi}$ also converges to $L$.

        \underline{Note} See Remark~\Ref{RemrkC30.40}~(2) for the origin of this problem.


\V
\V

\noindent \ExCo Prove Parts (a), (b) and (c) of Theorem~\Ref{ThmC40.30}. %% C.5.4

\V
\V

\noindent \ExCp Prove Parts (d) and (e) of Theorem~\Ref{ThmC40.30}. %% C.5.4

\V
\V

\noindent \ExCq Prove Parts (f) and (g) of Theorem~\Ref{ThmC40.30}. %% C.5.4

\V
\V

\noindent \ExCr Prove Part (h) of Theorem~\Ref{ThmC40.30}. %% C.5.4

\V
\V

\noindent \ExCs Define a sequence ${\xi} \,=\, (x_{1},x_{2},\,{\ldots}\,x_{k},\,{\ldots}\,)$ by the rule
        \begin{displaymath}
        x_{1} \,=\, 1, \, x_{k+1} \,=\, \frac{1+x_{k}}{2+x_{k}} \mbox{ for each $k$ in ${\NN}$}
        \end{displaymath}

\V

        (a) Show that the sequence ${\xi}$ is monotonic down and bounded below by~$0$.

\V

        (b) Determine the limit of the sequence ${\xi}$.

%% See Example 2.11, p.39, of Fitzpatrick

\V
\V

\noindent \ExCt Suppose that ${\alpha} \,=\, (a_{1},a_{2},\,{\ldots}\,)$ and ${\beta} \,=\, (b_{1},b_{2},\,{\ldots}\,)$ are sequences such that

        \h (i)\, $b_{k} \,=\, {\displaystyle \frac{3+a_{k}}{5+a_{k}}}$ for all $k$ in ${\NN}$.
        \h (ii) $\lim_{k \,{\rightarrow}\, {\infty}} b_{k} \,=\, 10$.

\noindent Determine whether the sequence ${\alpha}$ is convergent; if it is, determine its limit.

\V
\V


\noindent \ExCu In the `Remark' immediately after the proof of Theorem~\Ref{ThmC30.10} %% Thm C.4.1, "Bolzano-Weierstrass p. 178
    it is indicated that the Bolzano-Weierstrass Theorem is equivalent to the following result:

        \h  Every bounded sequence has a monotonic convergent subsequence.

        \underline{Problem} Prove this last result using the Infinimum/Supremum Principles.

 %% NOTE: Automatically starts on a new page
