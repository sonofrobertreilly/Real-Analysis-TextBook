% Exercises_M140AB_G.TeX   Exercises for Chapter 

%
% Revised:05/06/11
%

%% NOTE: Copy the 52 lines below to each chapter, and change the Chapter letters.

%\thispagestyle{myheadings}


%\markboth{Exercises for Chapter~\ref{ChaptG} -}{Exercises for Chapter~\ref{ChaptG} -}

\newcommand{\ExGa}{{\bf \ref{ChaptG} - \,1} }
\newcommand{\ExGb}{{\bf \ref{ChaptG} - \,2} }
\newcommand{\ExGc}{{\bf \ref{ChaptG} - \,3} }
\newcommand{\ExGd}{{\bf \ref{ChaptG} - \,4} }
\newcommand{\ExGe}{{\bf \ref{ChaptG} - \,5} }
\newcommand{\ExGf}{{\bf \ref{ChaptG} - \,6} }
\newcommand{\ExGg}{{\bf \ref{ChaptG} - \,7} }
\newcommand{\ExGh}{{\bf \ref{ChaptG} - \,8} }
\newcommand{\ExGi}{{\bf \ref{ChaptG} - \,9} }
\newcommand{\ExGj}{{\bf \ref{ChaptG} -  10} }
\newcommand{\ExGk}{{\bf \ref{ChaptG} -  11} }
\newcommand{\ExGl}{{\bf \ref{ChaptG} -  12} }
\newcommand{\ExGm}{{\bf \ref{ChaptG} -  13} }
\newcommand{\ExGn}{{\bf \ref{ChaptG} -  14} }
\newcommand{\ExGo}{{\bf \ref{ChaptG} -  15} }
\newcommand{\ExGp}{{\bf \ref{ChaptG} -  16} }
\newcommand{\ExGq}{{\bf \ref{ChaptG} -  17} }
\newcommand{\ExGr}{{\bf \ref{ChaptG} -  18} }
\newcommand{\ExGs}{{\bf \ref{ChaptG} -  19} }
\newcommand{\ExGt}{{\bf \ref{ChaptG} -  20} }
\newcommand{\ExGu}{{\bf \ref{ChaptG} -  21} }
\newcommand{\ExGv}{{\bf \ref{ChaptG} -  22} }
\newcommand{\ExGw}{{\bf \ref{ChaptG} -  23} }
\newcommand{\ExGx}{{\bf \ref{ChaptG} -  24} }
\newcommand{\ExGy}{{\bf \ref{ChaptG} -  25} }
\newcommand{\ExGz}{{\bf \ref{ChaptG} -  26} }


\newcommand{\ExGaa}{{\bf \ref{ChaptG} - 27} }
\newcommand{\ExGab}{{\bf \ref{ChaptG} - 28} }
\newcommand{\ExGac}{{\bf \ref{ChaptG} - 29} }
\newcommand{\ExGad}{{\bf \ref{ChaptG} - 30} }
\newcommand{\ExGae}{{\bf \ref{ChaptG} - 31} }
\newcommand{\ExGaf}{{\bf \ref{ChaptG} - 32} }
\newcommand{\ExGag}{{\bf \ref{ChaptG} - 33} }
\newcommand{\ExGah}{{\bf \ref{ChaptG} - 34} }
\newcommand{\ExGai}{{\bf \ref{ChaptG} - 35} }
\newcommand{\ExGaj}{{\bf \ref{ChaptG} - 36} }
\newcommand{\ExGak}{{\bf \ref{ChaptG} - 37} }
\newcommand{\ExGal}{{\bf \ref{ChaptG} - 38} }
\newcommand{\ExGam}{{\bf \ref{ChaptG} - 39} }
\newcommand{\ExGan}{{\bf \ref{ChaptG} - 40} }
\newcommand{\ExGao}{{\bf \ref{ChaptG} - 41} }
\newcommand{\ExGap}{{\bf \ref{ChaptG} - 42} }
\newcommand{\ExGaq}{{\bf \ref{ChaptG} - 43} }
\newcommand{\ExGar}{{\bf \ref{ChaptG} - 44} }
\newcommand{\ExGas}{{\bf \ref{ChaptG} - 45} }
\newcommand{\ExGat}{{\bf \ref{ChaptG} - 46} }
\newcommand{\ExGau}{{\bf \ref{ChaptG} - 47} }
\newcommand{\ExGav}{{\bf \ref{ChaptG} - 48} }
\newcommand{\ExGaw}{{\bf \ref{ChaptG} - 49} }
\newcommand{\ExGax}{{\bf \ref{ChaptG} - 50} }
\newcommand{\ExGay}{{\bf \ref{ChaptG} - 51} }
\newcommand{\ExGaz}{{\bf \ref{ChaptG} - 52} }



                       \section{EXERCISES FOR CHAPTER~\ref{ChaptG}}
                        \label{SectGEX}

\V
\V
\V
\V

\noindent \ExGa Prove Part (a) of Corollary G.1.6 % CorG20.60
    by using the definition of unordered sums given in the Notes.
    (The proof of this result given in the Notes is based on the `epsilon' approach to unordered sums that is outlined in Statement (ii) in Theorem G.1.4. 
    % ThmG20.50
     Thus, this exercise asks you to prove the same result, but based directly on the `$f = f^{+} - f^{-}$' approach to ordered sums.)

\V
\V

\noindent \ExGb Prove Corollary G.1.8. % CorG20.67

\V
\V

\noindent \ExGc It is mentioned in Remark G.1.5 % RemrkG20.55
    on Page~362 that many authors use the `${\varepsilon}$'
    characterization of `unordered sum' as their defnition of this concept instead of our `$f \,=\, f^{+}-f^{-}$' approach.
    One minor difficulty in doing so is that one must then show that the value one assigns to $\sum_{X} f$ is unique, an issue which the `$f \,=\, f^{+}-f^{-}$' approach avoids.

        \underline{Problem} Using the `${\varepsilon}$' approach to unordered sums, and without using the results of Theorem~G.1.4, % ThmG20.50
    prove that the value $C$ which that approach assigns to a convergent unordered sum $\sum_{X} f$ is unique.

\V
\V

\noindent \ExGd Suppose that $f$ and $g$ are real-valued functions defined on an infinite set $X$,
    and suppose that $|f(x)|\,\,{\leq}\,\,|g(x)|$ for all $x$ in $X$.
    Prove that if the unordered sum $\sum_{X} g$ is defined then so is $\sum_{X} f$.

        \underline{Note} This result is called the {\bf Comparison Test for Unordered Sums}

\V
\V

\noindent \ExGe Suppose that $f$ and $g$ are real-valued functions defined on ${\NN}$; 
    assume that the unordered sums $\sum_{{\NN}} f$ and $\sum_{{\NN}} g$ are convergent, with values $A$ and $B$, respectively.
    Define $P:{\NN}{\times}{\NN} \,{\rightarrow}\, {\RR}$ by the rule $P(i,j) \,=\, f(i){\cdot}g(j)$ for each $(i,j)$ in ${\NN}{\times}{\NN}$.

        (a) Show that the unordered sum $\sum_{{\NN}{\times}{\NN}} P$ is also convergent,
    and that ${\displaystyle \sum_{{\NN}{\times}{\NN}}} P \,=\, A{\cdot}B$.

\V

        (b) Define a third function $h:{\NN} \,{\rightarrow}\, {\RR}$ by the rule
        \begin{displaymath}
        h(k) \,=\, f(1){\cdot}g(k) + f(2){\cdot}g(k-1) + \,{\ldots}\,+ f(k){\cdot}g(1) \mbox{ for each $k$ in ${\NN}$}.
        \end{displaymath}
    (Thus, $h(1) \,=\, f(1)g(1)$, $h(2) \,=\, f(1)g(2) + f(2)g(1)$, $h(3) \,=\, f(1)g(3) + f(2)g(2) + f(3)g(1)$, and so on.)

        Show that the unordered sum $\sum_{{\NN}} h$ is convergent and that $\sum_{{\NN}} h \,=\, A{\cdot}B$.

\V
\V

\noindent \ExGf In this exercise we generalize the `Middle-Thirds' characterization of the Cantor Ternary Set.

\V

        Let $r$ be a number such that $0\,<\,r\,<\,1$. If $I$ is any closed interval of length $L$,
    the `middle $r$-portion of $I$' is the {\em open} subinterval $J$ of $I$ which is symmetric about the midpoint of $I$ and whose length is $rL$.
    This concept leads to the following construction:

       \h \underline{Step 1} Remove the middle $r$-portion of the unit interval $[0,1]$.
    What remains is the disjoint union of two closed subintervals of $[0,1]$, each of length $(1-r)/2$.

        \h \underline{Step 2} Remove the middle $r$-portion of each of the subintervals obtained in Step~1,
    obtaining $4$ disjoint closed subintervals of $[0,1]$, all of equal length.

        \h \underline{General Step} Continue this process of removing middle $r$-portions.

\noindent Let $C_{r}$ be the set of points of $[0,1]$ which do {\em not} get removed during this process.

\V

        \underline{Problem}

\V

        (a) Show that the set $C_{r}$ is closed in ${\RR}$, and that this set has no nonempty open subsets.

\V

        (b) Show that $C_{r}$ is an uncountable set.

\V

        (c) Show that $C_{r}$ has measure~$0$.


\V
\V

\noindent \ExGf In this exercise we modify the construction described in the preceding exercise.

\V

       \h \underline{Step 1} Remove the middle $1/2$-portion of the unit interval $[0,1]$.
    What remains is the disjoint union of two closed subintervals of $[0,1]$, each of length $1/4$.

        \h \underline{Step 2} Remove the middle $1/2^{2}$-portion of each of the subintervals obtained in Step~1,
    obtaining $4$ disjoint closed subintervals of $[0,1]$, all of equal length.

        \h \underline{General Step} Continue this process of removing middle $1/2^{k}$-portion, increasing $k$ by~$1$ at each stage.

\noindent Let $\hat{C}$ be the set of points of $[0,1]$ which do {\em not} get removed during this process.


\V

        \underline{Problem}

\V

        (a) Show that the set $\hat{C}$ is closed in ${\RR}$, and that this set has no nonempty open subsets.

\V

        (b) Show that $\hat{C}$ is an uncountable set.

\V

        (c) Show that the set $\hat{C}$ has positive measure.

\V

        \underline{Remark} Because $\hat{C}$ has these properties, it is sometimes called a {\bf fat Cantor set}\index{fat Cantor set}.

\V
\V

\noindent \ExGh \underline{Prove or Disprove} If $\sum_{k=1}^{{\infty}} x_{k}$ is a convergent infinite series and ${\alpha} \,=\, (a_{1},a_{2},\,{\ldots}\,a_{k},\,{\ldots}\,)$ is a sequence of positive numbers which converges to~$0$,
    then the series $\sum_{k=1}^{{\infty}} a_{k}x_{k}$ is also convergent.

\V
\V

\noindent \ExGi Suppose that ${\alpha} \,=\, (a_{1},a_{2},\,{\ldots}\,)$ is a sequence as in the statement of the Alternating-Series Test. % ThmG30.120
    That is, $a_{k}\,>\,0$ for each index $k$, the sequence ${\alpha}$ is monotonic down, and $\lim_{k \,{\rightarrow}\, {\infty}} a_{k} \,=\, 0$.
    For each $k$ in ${\NN}$ let $s_{k}$ denote the $k$-th partial sum of the alternating series $\sum_{k=1}^{{\infty}} (-1)^{k-1}a_{k}$,
    and let $L$ be the sum of that series.


\V

        (a) Prove that if ${\varepsilon}\,>\,0$ and if $a_{k+1}\,\,{\leq}\,\,{\varepsilon}$ then $|L-s_{k}|\,\,{\leq}\,\,{\varepsilon}$.

\V

        (b) Suppose that, in addition, the sequence ${\alpha}$ has the property that $a_{k}-a_{k+1}\,\,{\geq}\,\,a_{k+1}-a_{k+2}$ for each $k$ in ${\NN}$;
    that is, the {\em differences} of consecutive terms form a monotonic-down sequence.
    Prove that if ${\varepsilon}\,>\,0$ and $a_{k}\,\,{\leq}\,\,2{\varepsilon}$ then $|L-s_{k}|\,\,{\leq}\,\,{\varepsilon}$.

\V

        (c) Use the results of Parts (a) and~(b) to determine how many terms of the Alternating Harmonic Series one should use to estimate the value of ${\ln}\,2$ with error less than $1/1000$ in magnitude. Compare the results.

\V
\V

\noindent \ExGj Let ${\xi} \,=\, (x_{1}, x_{2},\,{\ldots}\,x_{k},\,{\ldots}\,)$ be given by the rule
        \begin{displaymath}
        x_{k} \,=\, \left\{
        \begin{array}{rl}
        {\displaystyle \frac{1.0000001}{k}} & \mbox{if $k$ is odd}    \\
                                            &                         \\
        -{\displaystyle \frac{0.9999999}{k}} & \mbox{if $k$ is even}
        \end{array}
                           \right.
        \end{displaymath}

\V

        (a) Show that the series $\sum_{k=1}^{{\infty}} x_{k}$ is divergent.

\V

        (b) Explain why this does not contradict the Alternating Series Test. (Or does it ???)
    

\V
\V

\noindent \ExGk Suppose that ${\xi} \,=\, (x_{1},x_{2},\,{\ldots}\,)$ is a monotonically decreasing sequence of nonnegative numbers.
    Prove that the series $\sum_{j=1}^{{\infty}} x_{j}$ is convergent if, and only if, the series $\sum_{j=0}^{{\infty}} 2^{j}x_{2^{j}}$ is convergent.

\V
\V

\noindent \ExGl \underline{Prove or Disprove} If the series $\sum_{k=1}^{{\infty}} x_{k}$ is divergent, then so is the series $\sum_{k=1}^{{\infty}} kx_{k}$. % Apostol p. 233 Also 205B W05 HW 6

\V
\V

\noindent \ExGm (a) Show that if $n \,\,{\geq}\,\, 2$ then ${\displaystyle \frac{1}{2} + \frac{1}{3} +  \,{\ldots}\,
    + \frac{1}{n}  \,<\, {\ln}\,(n)  \,<\,  1 + \frac{1}{2} + \frac{1}{3} +  \,{\ldots}\, + \frac{1}{n-1}}$.

\V

        (b) Show that the sequence $(a_{n})_{n \,\,{\geq}\,\, 1}$, defined by the rule ${\displaystyle a_{n} \,=\, 1 + \frac{1}{2} + \frac{1}{3} +  \,{\ldots}\, + \frac{1}{n} - {\ln}}\,n$,
    is bounded and strictly decreasing.

        \underline{Remark}: It follows from the preceding that the sequence ${\alpha} \,=\, (a_{1},a_{2},\,{\ldots}\,)$ given here is convergent.
    The limit of this sequence is an important number called {\em Euler's Constant}; its standard symbol is~${\gamma}$.
    The value of ${\gamma}$, to $10$ decimal places, is $0.5772156649$.

\V

        (c) For each $k{\in}{\NN}$ let $b_{k} \,=\, 1/2+1/4+1/6+ \,{\ldots}\, 1/(2k)$ denote the sum of the reciprocals of the first $k$ {\em even} positive integers;
    likewise, let $c_{k} \,=\, 1+1/3+1/5+ \,{\ldots}\, +1/(2k-1)$ denote the sum of the reciprocals of the first $k$ {\em odd} positive integers;

        \underline{Problem}: Use results of Parts~(a) and~(b) to show that
        \begin{displaymath}
        \lim_{k {\rightarrow} {\infty}} \left(b_{k} - \frac{{\ln}\,(k)}{2}\right) \,=\,  \frac{{\gamma}}{2}, \mbox{ and }
    \lim_{k {\rightarrow} {\infty}} \left(c_{k} - \frac{{\ln}\,(k)}{2}\right) \,=\, \frac{{\gamma}}{2} + {\ln}\,(2)
        \end{displaymath}

\V

        (d) Use the results of Part~(c) to give an alternate proof that the Alternating Harmonic Series converges to ${\ln}\,(2)$. % 205C S05 HW 1

\V
\V

\noindent \ExGn Suppose that $\sum_{k=1}^{{\infty}} x_{k}$ is an infinite series such that $\sum_{k=1}^{{\infty}} x_{k}^{+}$
    and $\sum_{k=1}^{{\infty}} x_{k}^{-}$ both diverge to~$+{\infty}$, and such that $\lim_{k \,{\rightarrow}\, {\infty}} x_{k} \,=\, 0$.
    Prove that some rearrangement of $\sum_{k=1}^{{\infty}} x_{k}$ is conditionally convergent.

\V�
\V

\noindent \ExGo Prove that if $f:(a,b) \,{\rightarrow}\, {\RR}$ is real-analytic on an open interval $(a,b)$, and if $f(x) \,=\, 0$ for all $x$ in some subinterval of $(a,b)$, then $f(x) \,=\, 0$ for all $x$ in $(a,b)$.

\V
\V

\noindent \ExGp Prove that the exponential function satisfies the usual `Law of Exponents', namely $e^{x+y} \,=\, e^{x}{\cdot}e^{y}$, by using the product of the Maclaurin series for for $e^{x}$ and~$e^{y}$.

\V
\V

\noindent \ExGq Suppose that ${\alpha} \,=\, (a_{0},a_{1},\,{\ldots}\,)$ is a sequence such that $a_{0} \,=\, 1$.
    Assume that $r$ is a positive number such that $\sum_{k=1}^{{\infty}} |a_{k}|r^{k}\,<\,1$.
    Define a second sequence ${\beta} \,=\, (b_{0},b_{1},b_{2},\,{\ldots}\,)$ recursively by the rule
        \begin{displaymath}
        b_{0} \,=\, 1; \, b_{k} \,=\, - \left(a_{k} + a_{k-1}b_{1} + \,{\ldots}\,+ a_{1}b_{k-1}\right) \mbox{ if $k\,\,{\geq}\,\,1$}
        \end{displaymath}
        \underline{Problem} Show that the radius of convergence of the power series $\sum_{j=0}^{{\infty}} b_{j}x^{j}$ is at least as large as~$r$.
 
 \V
 \V
 
\noindent \ExGr (a) Determine what the Root Test tells us about the convergence or divergence of the following series:
        \begin{displaymath}
        \frac{1}{2} + 0 + 0 + \frac{1}{2^{2}} + 0 + 0 + 0 + 0 + \frac{1}{2^{3}}
    + 0 + 0 + 0 + 0 + 0 + 0 + \frac{1}{2^{4}} + \,{\ldots}\,;
        \end{displaymath}
    the general pattern is that between the consecutive {\em non}zero terms $1/2^{k}$ and $1/2^{k+1}$ there are $2k$ zeroes.

\V

        (b) Determine what the Ratio Test tells us about the convergence or divergence of the following series of positive terms:
        \begin{displaymath}
        \frac{1}{2^{2}} + \frac{1}{2} + \frac{1}{2^{4}} + \frac{1}{2^{3}} + \frac{1}{2^{6}} + \frac{1}{2^{5}} + \,{\ldots}\,;
        \end{displaymath}
    the general pattern is that if $k \,=\, 2m-1$, then the $k$-th term is $1/2^{2m}$, while if $k \,=\, 2m$, then the $k$-th term is $1/2^{2m-1}$.

\V

        (c) Determine the sums of the series discussed in Parts~(a) and~(b).

\V
\V

\noindent \ExGs {\bf Dirichlet's Test for Uniform Convergence} Consider an infinite series $\sum_{j=1}^{{\infty}} f_{j}$ of real-valued functions $f_{1}$, $f_{2}$, \,{\ldots}\, defined on a nonempty set $X \,{\subseteq}\, {\RR}$,
    and for each $k$ let $s_{k}:X \,{\rightarrow}\, {\RR}$ denote the corresponding $k$-th partial sum;
    that is, $s_{k} \,=\, f_{1} + f_{2} + \,{\ldots}\, + f_{k}$.
    Suppose that the sequence $(s_{1}, s_{2},\,{\ldots}\,)$ is uniformly bounded on $X$;
    that is, there is a positive real number $M$ such that for all $x$ in $X$ and all $k$ in ${\NN}$ one has $|s_{k}(x)|\,\,{\leq}\,\,M$.
    Let ${\gamma} \,=\, (g_{1},g_{2},\,{\ldots}\,)$ be a sequence of real-valued functions on $X$ such that $g_{k+1}(x)\,\,{\leq}\,\,g_{k}(x)$ for all $x$ in $X$.
    Assume further that the sequence ${\gamma}$ converges uniformly on $X$ to the zero function.
    Then the series $\sum_{j=1}^{{\infty}} f_{j}g_{j}$ converges uniformly on~$X$.

\V
\V

\noindent \ExGt {\bf Abel's Test for Uniform Convergence} Consider an infinite series $\sum_{j=1}^{{\infty}} f_{j}$
    of real-valued functions $f_{1}$, $f_{2}$, \,{\ldots}\, defined on a nonempty set $X \,{\subseteq}\, {\RR}$;
    suppose that this series  converges uniformly on $X$.
        Let ${\gamma} \,=\, (g_{1},g_{2},\,{\ldots}\,)$ be a sequence of real-valued functions on $X$ such that $g_{k+1}(x)\,\,{\leq}\,\,g_{k}(x)$ for all $x$ in $X$.
    Assume further that the sequence ${\gamma}$ is uniformly bounded on $X$; that is, there is a positive real number $M$ such that for all $x$ in $X$ and all $k$ in ${\NN}$ one has $|g_{k}(x)|\,\,{\leq}\,\,M$.

    Then the series $\sum_{j=1}^{{\infty}} f_{j}g_{j}$ converges uniformly on~$X$. % Apostol P. 24 Ex 9.13

\V
\V

\noindent \ExGu Prove that if $\sum_{k=1}^{{\infty}} a_{k}$ is an absolutely convergent series then the series $\sum_{k=1}^{{\infty}} a_{k}{\sin}\,(kx)$ is uniformly convergent on ${\RR}$. % Apostol p. 249 # 9.22

\V
\V


\noindent \ExGv Prove that the series ${\displaystyle \sum_{k=1}^{{\infty}} \frac{(-1)^{k-1}}{\sqrt{k}} {\sin}\,\left(1+\frac{x}{k}\right)}$ converges uniformly on every closed bounded interval of ${\RR}$. % Apostol p. 249 # 9.20

\V
\V

\noindent \ExGw Let ${\alpha} \,=\, (a_{1},a_{2},\,{\ldots}\,)$ be a monotonic down sequence of positive numbers.
    Prove that if the series $\sum_{k=1}^{{\infty}} a_{k}{\sin}\,(kx)$ converges uniformly on ${\RR}$, then $\lim_{k \,{\rightarrow}\, {\infty}} ka_{k} \,=\, 0$.
    \underline{The converse is also true, but is definitely harder.} % Theorem of Chaundry and Jolliffe. See Zygmund's "Trig Series" book, p.108.

% Apostol p.249 # 9.23


\V
\V

\noindent \ExGx Recall that the Cantor set can be constructed by removing certain open intervals -- the `middle thirds' -- from the unit interval $[0,1]$.
    Let $X$ be the set of those open intervals, and define a function $f:X \,{\rightarrow}\, {\RR}$ by the rule that if $I \,=\, (a,b)$ is in $X$ then $f(I) \,=\, b-a$.

        \underline{Problem} Compute the unordered sum $\sum_{X} f$.


