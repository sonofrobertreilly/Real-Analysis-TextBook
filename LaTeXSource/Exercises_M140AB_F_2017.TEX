% Exercises_M140AB_F.TeX   Exercises for Chapter 
%% NOTE: Some of these are actually for material in the orginal Chapter C
%%       which will eventually be moved to Chapter F


%
% Revised: 06/26/09
%

%% NOTE: Copy the 52 lines below to each chapter, and change the Chapter letters.

%\thispagestyle{myheadings}


%\markboth{Exercises for Chapter~\ref{ChaptF} -}{Exercises for Chapter~\ref{ChaptF} -}

\newcommand{\ExFa}{{\bf \ref{ChaptF} - \,1} }
\newcommand{\ExFb}{{\bf \ref{ChaptF} - \,2} }
\newcommand{\ExFc}{{\bf \ref{ChaptF} - \,3} }
\newcommand{\ExFd}{{\bf \ref{ChaptF} - \,4} }
\newcommand{\ExFe}{{\bf \ref{ChaptF} - \,5} }
\newcommand{\ExFf}{{\bf \ref{ChaptF} - \,6} }
\newcommand{\ExFg}{{\bf \ref{ChaptF} - \,7} }
\newcommand{\ExFh}{{\bf \ref{ChaptF} - \,8} }
\newcommand{\ExFi}{{\bf \ref{ChaptF} - \,9} }
\newcommand{\ExFj}{{\bf \ref{ChaptF} -  10} }
\newcommand{\ExFk}{{\bf \ref{ChaptF} -  11} }
\newcommand{\ExFl}{{\bf \ref{ChaptF} -  12} }
\newcommand{\ExFm}{{\bf \ref{ChaptF} -  13} }
\newcommand{\ExFn}{{\bf \ref{ChaptF} -  14} }
\newcommand{\ExFo}{{\bf \ref{ChaptF} -  15} }
\newcommand{\ExFp}{{\bf \ref{ChaptF} -  16} }
\newcommand{\ExFq}{{\bf \ref{ChaptF} -  17} }
\newcommand{\ExFr}{{\bf \ref{ChaptF} -  18} }
\newcommand{\ExFs}{{\bf \ref{ChaptF} -  19} }
\newcommand{\ExFt}{{\bf \ref{ChaptF} -  20} }
\newcommand{\ExFu}{{\bf \ref{ChaptF} -  21} }
\newcommand{\ExFv}{{\bf \ref{ChaptF} -  22} }
\newcommand{\ExFw}{{\bf \ref{ChaptF} -  23} }
\newcommand{\ExFx}{{\bf \ref{ChaptF} -  24} }
\newcommand{\ExFy}{{\bf \ref{ChaptF} -  25} }
\newcommand{\ExFz}{{\bf \ref{ChaptF} -  26} }


\newcommand{\ExFaa}{{\bf \ref{ChaptF} - 27} }
\newcommand{\ExFab}{{\bf \ref{ChaptF} - 28} }
\newcommand{\ExFac}{{\bf \ref{ChaptF} - 29} }
\newcommand{\ExFad}{{\bf \ref{ChaptF} - 30} }
\newcommand{\ExFae}{{\bf \ref{ChaptF} - 31} }
\newcommand{\ExFaf}{{\bf \ref{ChaptF} - 32} }
\newcommand{\ExFag}{{\bf \ref{ChaptF} - 33} }
\newcommand{\ExFah}{{\bf \ref{ChaptF} - 34} }
\newcommand{\ExFai}{{\bf \ref{ChaptF} - 35} }
\newcommand{\ExFaj}{{\bf \ref{ChaptF} - 36} }
\newcommand{\ExFak}{{\bf \ref{ChaptF} - 37} }
\newcommand{\ExFal}{{\bf \ref{ChaptF} - 38} }
\newcommand{\ExFam}{{\bf \ref{ChaptF} - 39} }
\newcommand{\ExFan}{{\bf \ref{ChaptF} - 40} }
\newcommand{\ExFao}{{\bf \ref{ChaptF} - 41} }
\newcommand{\ExFap}{{\bf \ref{ChaptF} - 42} }
\newcommand{\ExFaq}{{\bf \ref{ChaptF} - 43} }
\newcommand{\ExFar}{{\bf \ref{ChaptF} - 44} }
\newcommand{\ExFas}{{\bf \ref{ChaptF} - 45} }
\newcommand{\ExFat}{{\bf \ref{ChaptF} - 46} }
\newcommand{\ExFau}{{\bf \ref{ChaptF} - 47} }
\newcommand{\ExFav}{{\bf \ref{ChaptF} - 48} }
\newcommand{\ExFaw}{{\bf \ref{ChaptF} - 49} }
\newcommand{\ExFax}{{\bf \ref{ChaptF} - 50} }
\newcommand{\ExFay}{{\bf \ref{ChaptF} - 51} }
\newcommand{\ExFaz}{{\bf \ref{ChaptF} - 52} }



                       \section{EXERCISES FOR CHAPTER~\ref{ChaptF}}
                        \label{SectFEX}

\V
\V
\V
\V

\noindent \ExFa In each part of this exercise, either construct an example of a sequence ${\xi}$ for which ${\cal L}[{\xi}] $ is the indicated set, or show that no such sequence exists.

\V

        (a) ${\cal L}[{\xi}] \,=\, \{-1, 3, {\pi}\}$ \h (b) ${\cal L}[{\xi}] \,=\, {\NN}$

\V
\V


\noindent  \ExFb  Note: In doing Part (b) of this exercise you may use the result of Part~(a), whether you proved it or not.
    Likewise, in doing Part~(c) you may assume the results of Parts~(a) and~(b).

\V

        (a) Prove Part~(b) of Theorem~C.6.15. (Hint: Show that Condition~(i) is a necessary and sufficient condition for the number $L$ to satisfy the inequality $L\,\,{\geq}\,\,\limsup\,{\xi}$,
    while Condition~(ii) is a necessary and sufficient condition for $L$ to satisfy the inequality $L\,\,{\leq}\,\,\limsup\,{\xi}$.)

\V

        (b) Prove Part~(c) of Theorem~C.6.15. %% ThmC50.170

\V

        (c) Prove Part~(d) of Theorem~C.6.15.

\V
\V

\noindent \ExFc Let ${\xi} \,=\, (x_{1},x_{2},\,{\ldots}\,)$ be a sequence of real numbers.
    Show that
        \begin{displaymath}
        \liminf_{k \,{\rightarrow}\, {\infty}} x_{k} \,=\, -\limsup_{k \,{\rightarrow}\, {\infty}} -x_{k} \mbox{ and }
        \limsup_{k \,{\rightarrow}\, {\infty}} x_{k} \,=\, -\liminf_{k \,{\rightarrow}\, {\infty}} -x_{k}.
        \end{displaymath}
    (As usual, in the case of the quantities $+{\infty}$ and $-{\infty}$ one has $-(+{\infty}) \,=\, -{\infty}$ and $-(-{\infty}) \,=\, +{\infty}$.)

\V
\V

\noindent \ExFd Suppose that ${\alpha} \,=\, (a_{1},a_{2},\,{\ldots}\,)$ and ${\beta} \,=\, (b_{1},b_{2},\,{\ldots}\,)$ are bounded sequences of real numbers.

\V

        (a) Show that
        \begin{displaymath}
        \limsup_{n \,{\rightarrow}\, {\infty}} (a_{n} + b_{n})\,\,{\leq}\,\,\left(\limsup_{n \,{\rightarrow}\, {\infty}} a_{n}\right) +
    \left(\limsup_{n \,{\rightarrow}\, {\infty}} b_{n}\right)
        \end{displaymath}

%% Apostol Ex 8.2 (modified); assigned in HW #1 in 205C 2004/5

\V

        (b) Find an example of bounded ${\alpha}$ and ${\beta}$ for which the inequality in Part~(a) is strict.

\V
\V

\noindent \ExFe Suppose that ${\xi} \,=\, (x_{1},x_{2},\,{\ldots}\,)$ is a sequence of positive numbers.

\V

        (a) Prove that $\limsup_{k \,{\rightarrow}\, {\infty}} \sqrt[k]{x_{k}}\,\,{\leq}\,\,
        \limsup_{k \,{\rightarrow}\, {\infty}} \frac{x_{k+1}}{x_{k}}$.
    (Hint: Look at the `Hint' for Part~(a) of Exercise~\ExFb.)

\V

        (b) Prove that $\liminf_{k \,{\rightarrow}\, {\infty}} \frac{x_{k+1}}{x_{k}} 
\,\,{\leq}\,\,
        \liminf_{k \,{\rightarrow}\, {\infty}} \sqrt[k]{x_{k}}$.

\V
\V

\StartSkip{
\noindent \ExFf (a) Construct an example of a countably infinite family of closed bounded subsets of ${\RR}$ whose union is bounded but not a closed subset.

\V

    \h    (b) Construct an example of a countably infinite family of closed bounded subsets of ${\RR}$ whose union is closed but not a bounded subset.

\V
\V
}%EndSkip

\noindent \ExFf Suppose that $X_{1}$, $X_{2}$,\,{\ldots}\,$X_{k}$,\,{\ldots}\, is a nested sequence of nonempty compact subsets of ${\RR}$.
    (`Nested': For each $k$ one has $X_{k+1} \,{\subseteq}\, X_{k}$.) Prove that the intersection of the family $\{X_{k}: k{\in}{\NN}\}$ is nonempty.
        Note: This result is often called the {\bf Cantor Intersection Theorem}.

\V
\V

\noindent \ExFg \underline{Prove or Disprove} If $X$ is a nonempty subset of ${\RR}$ such that every continuous function $f:X \,{\rightarrow}\, {\RR}$ is bounded on $X$, then $X$ is compact.

\V
\V

\noindent \ExFh Let $f:X \,{\rightarrow}\, {\RR}$ be a continuous function whose domain $X$ is a nonempty compact subset of~${\RR}$.
    Let $Y$ be the image $f[X]$ of $X$ under~$f$.

\V

        (a) Prove that $Y$ is also a compact subset of $Y$.

\V

        (b) Suppose that, in addition, $f$ maps $X$ bijectively onto $Y$. Prove that the inverse map $f^{-1}:Y \,{\rightarrow}\, X$ is continuous on $Y$.

\V

        (c) Determine whether conclusion of Part~(b) remains true if the hypothesis `$X$ is compact' is omitted.

\V
\V

\noindent \ExFi Let ${\xi} \,=\, (x_{1},x_{2},\,{\ldots}\,)$ be a sequence of real numbers, and let $M \,=\, {\sup}\,\{x_{1},x_{2},\,{\ldots}\,\}$.

\V

        \underline{Prove or Disprove}: The quantity $M$ is an element of ${\cal L}[{\xi}]$ \underline{if} for each index $j$ one has $x_{j} \,\,{\neq}\,\, M$.

\V

        \underline{Prove or Disprove}: The quantity $M$ is an element of ${\cal L}[{\xi}]$ \underline{only if} for each index $j$ one has $x_{j} \,\,{\neq}\,\, M$.


\V
\V


\noindent \ExFj {\bf Definition} Let ${\cal F}$ be a nonempty family of real-valued functions defined on a nonempty subset $X$ of ${\RR}$.
    One says that the family ${\cal F}$ is {\bf uniformly bounded on $X$} provided that there exists a number $M$ such that $|f(x)|\,\,{\leq}\,\,M$ for all points $x$ in $X$ and for all functions $f$ in the family~${\cal F}$.

\V

        (a) Give an example of a nonempty family ${\cal F}$ of functions $f:{\RR} \,{\rightarrow}\, {\RR}$ such that each function $f$ in ${\cal F}$ is bounded on ${\RR}$ but ${\cal F}$ is not uniformly bounded on ${\RR}$.

\V

        (b) \underline{Prove or Disprove} If a sequence of functions $f_{k}:X \,{\rightarrow}\, {\RR}$ converges uniformly on $X$ to a function $f:X \,{\rightarrow}\, {\RR}$,
    and if for each $k$ the function $f_{k}$ is bounded on $X$, then the family ${\cal F} \,=\, \{f_{1},f_{2},\,{\ldots}\,\}$ is uniformly bounded on $X$.

%% Apostol Ex 9.1 p.247

\V
\V

\noindent \ExFk Suppose that ${\varphi} \,=\, (f_{1},f_{2},\,{\ldots}\,)$ and ${\gamma} \,=\, (g_{1},g_{2},\,{\ldots}\,)$ are sequences of real-valued functions defined on a nonempty set $X$ in ${\RR}$.
    Assume that ${\varphi}$ converges uniformly on $X$ to $f$ and ${\gamma}$ converges uniformly on $X$ to $g$.

\V

        (a) Prove that the sequence ${\sigma} \,=\, ((f_{1}+g_{1}), (f_{2}+g_{2}),\,{\ldots}\,)$ converges uniformly on $X$ to $f+g$.

\V

        (b) Give an example of such sequences ${\varphi}$ and ${\gamma}$ for which the corresponding sequence ${\mu} \,=\, (f_{1}{\cdot}g_{1}, f_{2}{\cdot}g_{2},\,{\ldots}\,)$ of products fails to be uniformly convergent on $X$.

\V

        (c) Prove that if one assumes, in addition to the uniform convergence on $X$ of ${\varphi}$ and ${\gamma}$,
    that each $f_{k}$ and each $g_{k}$ is bounded on $X$, then the sequence ${\mu}$ does converge uniformly to $f{\cdot}g$ on $X$.

%% Apostol 9-3. Done in 205C Spring 2005 HW 3 (III)

\V
\V

\noindent \ExFl Let $[a,b]$ be a fixed compact interval in ${\RR}$, with $a \,<\, b$;
    and for each $n$ in ${\NN}$ let $f_{n}:[a,b] {\rightarrow} {\RR}$ be a continuous function.

\V

        (a) \underline{Prove or Disprove}: A {\em necessary} condition for the sequence ${\varphi}\,=\, (f_{1}, f_{2}, \,{\ldots}\,)$ to converge uniformly on $[a,b]$
    is that for every Cauchy sequence ${\xi}\,=\, (x_{1}, x_{2}, \,{\ldots}\,) $ in $[a,b]$ the corresponding sequence of values
    $(f_{1}(x_{1}), f_{2}(x_{2}), \,{\ldots}\,f_{k}(x_{k}),\,{\ldots}\,)$ is Cauchy in ${\RR}$.

\V

        (b) \underline{Prove or Disprove}: A {\em sufficient} condition for the sequence ${\varphi}\,=\, (f_{1}, f_{2}, \,{\ldots}\,)$ to converge uniformly on $[a,b]$
    is that for every Cauchy sequence ${\xi}\,=\, (x_{1}, x_{2}, \,{\ldots}\,) $ in $[a,b]$ the corresponding sequence of values
    $(f_{1}(x_{1}), f_{2}(x_{2}), \,{\ldots}\,f_{k}(x_{k}),\,{\ldots}\,)$ is Cauchy in ${\RR}$.

%% M205C S05 HW #3 (IV)

\V
\V

\noindent \ExFm \underline{Prove or Disprove}: If $f:{\RR} \,{\rightarrow}\, {\RR}$ is a bounded monotonic continuous function, then $f$ is uniformly continuous on ${\RR}$.

\V
\V

\noindent \ExFn Suppose that $f:X \,{\rightarrow}\, {\RR}$ and $g:Y \,{\rightarrow}\, {\RR}$ are functions defined on nonempty subsets $X$ and $Y$, 
    respectively, and assume that $f[X] \,{\subseteq}\, Y$.

        \underline{Prove or Disprove}: If $f$ is uniformly continuous on $X$ and $g$ is uniformly continuous on $Y$,
    then their composition $h \,=\, g{\circ}f$ is uniformly continuous on $X$.

\V
\V

\noindent \ExFo Let $X$ be a nonempty subset of ${\RR}$.

\V

        (a) \underline{Prove or Disprove}: If $f:X \,{\rightarrow}\, {\RR}$ is uniformly continuous on $X$ then $|f|:X \,{\rightarrow}\, {\RR}$ is uniformly continuous on $X$.

\V

        (b) \underline{Prove or Disprove}: If $|f|:X \,{\rightarrow}\, {\RR}$ is uniformly continuous on $X$ then $f:X \,{\rightarrow}\, {\RR}$ is uniformly continuous on $X$.

\V
\V

\noindent \ExFp Let $f: \,{\rightarrow}\, {\RR}$ be a real-valued function whose domain is a nonempty subset $X$ of ${\RR}$.
    Prove that the following statements are equivalent:

        \h (i)\, The function $f$ is continuous on $X$.

        \h (ii) For every closed subset $Y$ of ${\RR}$ the set $f^{-1}[Y]$ is of the form $X\,{\cap}\,Z$ for some closed subset $Z$ of ${\RR}$.

\V
\V

\noindent \ExFq Prove Theorem F.4.5 %% ThmF30.50


\V
\V

\noindent \ExFr Prove Dini's Theorem (see Theorem~F.5.1) %% ThmF35.20
    using the Cantor Intersection Theorem (see Exercise~\ExFf) and the results of Exercise~\ExFp above.
    (Hint: First reduce to the case in which the functions $f_{n}$ are decreasing pointwise on $X$ to the zero function. Then for each ${\varepsilon}\,>\,0$ consider the sets $Y_{n}({\varepsilon})$ of the form $Y_{n}({\varepsilon}) \,=\, \{x{\in}X: f_{n}(x)\,\,{\geq}\,\,{\varepsilon}\}$.)


\V
\V

\noindent \ExFs {\bf Definition} (1) A function $f:{\RR} \,{\rightarrow}\, {\RR}$ is said to be an {\bf open function} provided $f[U]$ is open in ${\RR}$ for each open set $U \,{\subseteq}\, {\RR}$.

\V

        (2) A function $f:{\RR} \,{\rightarrow}\, {\RR}$ is said to be a {\bf closed function} provided $f[X]$ is closed in ${\RR}$ for each closed set $X \,{\subseteq}\, {\RR}$.

\V

        \underline{Problem} (a) Give an example of a continuous function $f:{\RR} \,{\rightarrow}\, {\RR}$ which is {\em not} an open function.

\V

        (b) Give an example of a continuous function $f:{\RR} \,{\rightarrow}\, {\RR}$ which is {\em not} a closed function.

\V
\V

\noindent \ExFt (a) \underline{Prove or Disprove} If $f_{1}$ and $f_{2}$ are hybrid functions defined on a closed bounded interval $[a,b]$, then their sum, $f_{1} + f_{2}$, is hybrid on $[a,b]$.

\V

        (b) \underline{Prove or Disprove} If $f_{1}$ and $f_{2}$ are hybrid functions defined on a closed bounded interval $[a,b]$, then their product, $f_{1} {\cdot} f_{2}$, is hybrid on $[a,b]$.

\V
\V

\noindent \ExFu Suppose that $f:[a,b] \,{\rightarrow}\, {\RR}$ is of bounded variation on $[a,b]$.

\V

        (a) \underline{Prove or Disprove} The function $|f|$ is of bounded variation on $[a,b]$.

\V

        (b) Prove that if there is a constant $c\,>\,0$ such that $f(x)\,\,{\geq}\,\,c$ for all $x$ in $[a,b]$,
    then $1/f$ is of bounded variation on $[a,b]$.

\V
\V

\noindent \ExFv Suppose that ${\varphi} \,=\, (f_{1},f_{2},\,{\ldots}\,)$ is a sequence of functions which converges pointwise on $I$ to a function $f$.

\V

        (a) \underline{Prove or Disprove} If each function $f_{k}$ is monotonic up on $I$ then so is $f$.

\V

        (b) \underline{Prove or Disprove} If each function $f_{k}$ is of bounded variation on $I$ then so is $f$.

\V
\V

\noindent \ExFw Suppose that $g$, $f_{1}$, $f_{2}$,\,{\ldots}\,$f_{k}$,\,{\ldots}\, are all monotonic up on an open interval $(a,b)$.
    Suppose further that $\lim_{k \,{\rightarrow}\, {\infty}} f_{k}(x) \,=\, g(x)$ for every {\em rational} number in $(a,b)$.
    Prove that if $c$ is an irrational number in $(a,b)$ such that $g$ is continuous at $c$, then $\lim_{k \,{\rightarrow}\, {\infty}} f_{k}(c) \,=\, g(c)$.

\V
\V

\noindent \ExFx Let $f:[a,b] \,{\rightarrow}\, {\RR}$ be a function of bounded variation on the closed bounded interval $[a,b]$.
    Assume also that $f$ also has the `Intermediate-Value Property' on $[a,b]$; that is, for each $x_{1}$ and $x_{2}$ in $[a,b]$ with $x_{1}\,<\,x_{2}$,
    if $y$ is between $f(x_{1})$ and $f(x_{2})$ then there exists $c$ in $(x_{1},x_{2})$ such that $f(c) \,=\, y$.

        \underline{Problem} Show that $f$ is continuous on $[a,b]$.

\V
\V

\noindent \ExFy Prove the claim made in Remark~(1) of~F.6.23 on Page~351
