%
% Revised: 10/23//2017 %% Encoding Western ASCII
%


                  \chapter{The Riemann Integral}
                  \label{ChaptH}

       \underline{Quotes for Chapter~\Ref{ChaptH}}:

\V

\begin{quotation}
{\footnotesize
        (1) Leibniz's `Whole difference equal to sum of the partial differences.....'
\V

        (2) `It is frequently claimed that Lebesgue integration is as easy to teach as Riemann integration.
    This is probably true, but I have yet to be convinced that it is as easy to learn.'

        T. Korner, `A Companion to Analysis'

}%EndFootNoteSize
\end{quotation}


% Related comment: Even mathematicians whose native language is `Lebesgue integral' need to be able to speak `Riemann integral' to engineers, numerical analysts, etc. Likewise, knowing the LUB Property does not relieve one of knowing `Decimals'.

\VV

        \section{{\bf The Riemann Integral}}
        \label{SectH20}\IndB{ZZ Sections}{\Ref{SectH20} The Riemann Integral}

\V

        {\bf Introduction} One of the most important concepts in elementary calculus is that of the definite integral
    ${\displaystyle \int_{a}^{b} f}$ of a bounded function $f$ over an interval~$[a,b]$.
    Cauchy gave an analytical treatment of this quantity in [CAUCHY~1823] as the basis for his proof of Theorem~\Ref{ThmE45.125B}, the Antiderivative Theorem;
    indeed, he stressed the need for such a theorem in the opening of that book. In contrast with Lebesgue's proof of the Antiderivative Theorem given in 
    Chapter~\Ref{ChaptE}, however, Cauchy's approach does not involve approximating $f$ by functions known to possess antiderivatives on $[a,b]$.

        Later developments in analysis made it important to extend Cauchy's concept of the definite integral
    to one which can apply to discontinuous functions which may not have antiderivatives.
    To be useful, such extensions needed to maintain the major properties enjoyed by the concept in Cauchy's context of continuous functions.
    It turns out that for such an extension the expression ${\displaystyle \int_{a}^{b} f}$
    does not make sense for every bounded function $f$ on an interval $[a,b]$. Thus before computing the value of such an integral,
    one must first determine whether the expression even makes sense under the given extension;
    that is, whether $f$ is `integrable' over $[a,b]$ under this extension. The first major extension
    of Cauchy's formulation of the definite integral is due to Riemann; see [RIEMANN~1854].
    Several other extensions of the definite integral concept appeared over the following decades,
    and it has become customary to refer to Riemann's extension as the `Riemann integral'.


        In {\ThisText} we follow an approach to the Riemann integral, due to Darboux, which simplifies Riemann's original treatment; see [DARBOUX~1875].
    This approach can be based on the following question:

\VA

        \h `What features should any reasonable concept of ${\displaystyle \int_{a}^{b} f}$ include?'

\VA

        Certainly one would want any such concept to include the basic rules of ordinary integral calculus. These include:

\V
        \underline{Rule 1} Suppose $f$ is integrable on $[a,b]$. If $m$ and $M$ satisfy $m\,\,{\leq}\,\,f(x)\,\,{\leq}\,\,M$ for all $x$ in $[a,b]$,
    then  ${\displaystyle m\,(b-a)\,\,{\leq}\,\,\int_{a}^{b} f\,\,{\leq}\,\,M\,(b-a)}$.
    In particular, these should hold if $m \,=\, {\inf}\,\{f(x): x{\in}[a,b]\}$ and $M \,=\, {\sup}\,\{f(x): x{\in}[a,b]\}$.
    Of course since $b-a\,>\,0$, the left-hand inequality holds automatically if $m \,=\, -{\infty}$;
    likewise, the right-hand inequality holds automatically if $M \,=\, +{\infty}$ since $b-a\,>\,0$.
    Hence this rule is of significance primarily when $m$ and $M$ can be chosen to be finite;
    that is, when $f$ is bounded on $[a,b]$. We assume such boundedness throughout this discussion; unbounded functions are considered later.

\VA

        \underline{Rule 2} Suppose again that (bounded) $f$ is integrable on $[a,b]$.
    If $a\,<\,c\,<\,b$, then $f$ should be integrable over each of the subintervals $[a,c]$ and $[c,b]$.
    Furthermore, one should have ${\displaystyle \int_{a}^{b} f \,=\, \int_{a}^{c} f + \int_{c}^{b} f}$.

\V

        \subsection{\small{{\bf Remark}}}
        \label{RemrkH20.15}

\V

        Repeated use of these two rules forms the basis for Darboux's approach to the definite integral.
    For example, repeated use of the first portion of Rule~2 would imply that if $f$ is integrable on $[a,b]$,
    and if ${\cal P} \,=\, \{a \,=\, x_{0}\,<\,x_{1}\,<\,\,{\ldots}\,\,<\,x_{n-1}\,<\,x_{n} \,=\, b\}$ is any partition of $[a,b]$,
    then $f$ is integrable on each subinterval $[x_{j-1},x_{j}]$ of this partition, $1\,\,{\leq}\,\,j\,\,{\leq}\,\,n$.
    Furthermore, Rule~1 would then imply that if for each $j \,=\, 1,2,\,{\ldots}\,n$
    one sets $m_{j} \,=\, {\inf}\,\{f(x):x{\in}[x_{j-1},x_{j}]\}$ and $M_{j} \,=\, {\sup}\,\{f(x): x{\in}[x_{j-1},x_{j}]\}$, then one has
        \begin{displaymath}
        m_{j}\,(x_{j}-x_{j-1})\,\,{\leq}\,\,\int_{x_{j-1}}^{x_{j}} f\,\,{\leq}\,\,M_{j}\,(x_{j}-x_{j-1}) \mbox{ for each $j=1,2,\,{\ldots}\,n$}.
        \end{displaymath}
    Summing over the preceding inequalities and using the second portion of Rule~2 then implies that
        \begin{displaymath}
        \sum_{j=1}^{n} m_{j}\,(x_{j}-x_{j-1})
    \,\,{\leq}\,\,
        \int_{a}^{b} f
    \,\,{\leq}\,\,
        \sum_{j=1}^{n} M_{j}\,(x_{j}-x_{j-1}). %\\\\\
        \end{displaymath}

        The simple proof is left as an exercise. % EXERCISE

\VV

\begin{quotation}
{\footnotesize \underline{\Note}\IndB{\notes}{on Fourier series and the Riemann integral} (on Fourier series and the Riemann integral)
STILL TO BE WRITTEN
}%EndFootNoteSize
\end{quotation}
%##

\VV

        The next definition assigns names to the quantities which appear in the preceding Remark.

\V

        \subsection{\small{{\bf Definition}}}
        \label{DefH20.20}

\V

        Let $f:[a,b] \,{\rightarrow}\, {\RR}$ be a function which is bounded on a closed interval~$[a,b]$;
    continuity of $f$ is not assumed here, nor is it assumed that $[a,b]$ is the full domain of $f$.
 
        For a partition ${\cal P} \,=\, \{a \,=\, x_{0}\,<\,x_{1}\,<\,\,{\ldots}\,\,<\,x_{n-1}\,<\,x_{n} \,=\, b\}$ of $[a,b]$,
    let $m_{j} \,=\, {\inf}\,\{f(x): x{\in}[x_{j-1}, x_{j}]\}$ and let $M_{j} \,=\, {\sup}\,\{f(x): x{\in}[x_{j-1}, x_{j}]\}$.
    For convenience write ${\Delta}x_{j} \,=\, x_{j}-x_{j-1}$ for each index~$j$.
    The number $L(f;{\cal P}) \,=\, \sum_{j=1}^{n} m_{j}\,{\Delta}x_{j}$ is called the
    {\bf lower Darboux sum}\IndB{Darboux, Gaston (1842-1917)}{lower, upper Darboux sums}
    associated with the function $f$ and the partition ${\cal P}$; similarly, the number
    $U(f;{\cal P}) \,=\, \sum_{j=1}^{n} M_{j}\,{\Delta}x_{j}$ is the corresponding {\bf upper Darboux sum}.
    The quantity ${\Delta}(f;{\cal P}) \,=\, U(f;{\cal P})-L(f;{\cal P})$ is the corresponding
    {\bf Darboux difference}\IndB{Darboux, Gaston (1842-1917)}{Darboux difference} associated with the function $f$ and the partition~${\cal P}$.

\V

        \underline{Remarks}\,(1) The notations $m_{j}$, $M_{j}$ and ${\Delta}x_{j}$ are ambiguous, in that they assume that the context makes clear
    which function $f$ and partition ${\cal P}$ is under consideration. Normally this ambiguity causes no problems.

\V

        (2)\, If $f$ is continuous on $[a,b]$, then $m_{j}$ and $M_{j}$ are the minimum and maximum values of $f$ on the subinterval $[x_{j-1},x_{j}]$,
    so the notation $L(f;{\cal P})$ and $U(f;{\cal P})$ agrees with that used in Section~\Ref{SectE45}.

\VV

        The conclusions of Remark~\Ref{RemrkH20.15} above can now be written to say that
    any reasonable definition of ${\displaystyle \int_{a}^{b} f}$ should satisfy the inequalities
        \begin{displaymath}
        L(f;{\cal P})\,\,{\leq}\,\,\int_{a}^{b} f\,\,{\leq}\,\,U(f;{\cal P})
        \end{displaymath}
    for every partition ${\cal P}$ of $[a,b]$. In particular, $\int_{a}^{b} f$ should be an upper bound for the set of all numbers of the form
    $L(f;{\cal P})$ and a lower bound for the set of numbers of the form $U(f;{\cal P})$.
    Thus one ought to have
        \begin{displaymath}
        {\sup}_{{\Pi}[a,b]}\,\{L(f;{\cal P})\}\,\,{\leq}\,\,\int_{a}^{b} f\,\,{\leq}\,\,{\inf}_{{\Pi}[a,b]}\,\{U(f;{\cal P})\},
        \end{displaymath}
    where the sup and inf are both taken over the set ${\Pi}([a,b])$ of all partitions ${\cal P}$ of~$[a,b]$.
    The next result outline properties of these quantities.

\VV

        \subsection{\small{{\bf Lemma}}}
        \label{LemmaH20.33}

\V

        Suppose that $f:[a,b] \,{\rightarrow}\, {\RR}$ is bounded on the interval~$[a,b]$.

\V

        (a) If ${\cal P}$ is any partition of $[a,b]$, then $L(f;{\cal P})\,\,{\leq}\,\,U(f;{\cal P})$.
    Furthermore one has ${\Delta}(f;{\cal P})\,\,{\geq}\,\,0$, with equality if, and only if, $f$ is constant on~$[a,b]$.

\V

        (b) Suppose that ${\cal P}$ and ${\cal Q}$ are partitions of $[a,b]$ such that ${\cal Q}$ is a refinement of ${\cal P}$; that is, ${\cal P} \,{\subseteq}\, {\cal Q}$. Then
        \begin{displaymath}
        L(f;{\cal P})\,\,{\leq}\,\,L(f;{\cal Q})\,\,{\leq}\,\,U(f;{\cal Q})\,\,{\leq}\,\,U(f;{\cal P}).
        \end{displaymath}

\V

        (c) If ${\cal Q}$ and ${\cal R}$ are partitions of $[a,b]$, then $L(f;{\cal Q})\,\,{\leq}\,\,U(f;{\cal R})$.
    In particular, one has
        \begin{displaymath}
        L(f;{\cal Q})\,\,{\leq}\,\,{\sup}_{{\Pi}[a,b]}\,\{L(f;{\cal P})\}
        \,\,{\leq}\,\,{\inf}_{{\Pi}[a,b]}\,\{U(f;{\cal P})\}
        \,\,{\leq}\,\,U(f;{\cal R})
        \end{displaymath}
    for every pair of partitions ${\cal Q}$ and ${\cal R}$ of~$[a,b]$.


\V

        (d) Let $c$ be a number such that $a\,<\,c\,<\,b$. Then a partition ${\cal P}$ of $[a,b]$ contains the point $c$ if, and only if,
    there are partitions ${\cal Q}$ and ${\cal R}$ of the intervals $[a,c]$ and $[c,b]$, respectively, such that ${\cal P} \,=\, {\cal Q}\,{\cup}\,{\cal R}$.
    Furthermore, when this situation holds one has $L(f;{\cal P}) \,=\, L(f;{\cal Q}) + L(f;{\cal R})$ and
    $U(f;{\cal P}) \,=\, U(f;{\cal Q}) + U(f;{\cal R})$.


\V

        {\bf Proof} (a) This statement follows directly from the observation that if $S$ is any nonempty set of real numbers, then ${\inf}\,S\,\,{\leq}\,\,{\sup}\,S$, with equality if, and only if, $S$ is a singleton set.

\V

        (b) Let ${\cal P} \,=\, \{a \,=\, x_{0}\,<\,x_{1}\,<\,\,{\ldots}\,\,<\,x_{n-1}\,<\,x_{n} \,=\, b\}$ be any partition of~$[a,b]$.
    Suppose first that ${\cal Q}$ is obtained from ${\cal P}$ by adjoining a single new point $q$ to the finite set~${\cal P}$.
    To be precise, suppose that $x_{k-1}\,<\,q\,<\,x_{k}$ for some index~$k$. Then one has
        \begin{displaymath}
        L(f;{\cal P}) \,=\, m_{1}\,{\Delta}x_{1} + \,{\ldots}\, + m_{k-1}\,{\Delta}x_{k-1} + m_{k}\,{\Delta}x_{k} + m_{k+1}\,{\Delta}x_{k+1} + \,{\ldots}\, + m_{n}\,{\Delta}x_{n}.
        \end{displaymath}
    In contrast, the sum expressing $L(f;{\cal Q})$ consists of exactly the same terms,
    except that the single term $m_{k}\,{\Delta}x_{k} \,=\, m_{k}\,(x_{k}-x_{k-1})$ is replaced by two terms $m_{k}'\,(q-x_{k-1}) + m_{k}''\,(x_{k}-q)$,
    where $m_{k}' \,=\, {\inf}\,\{f(x): x_{k-1}\,\,{\leq}\,\,x\,\,{\leq}\,\,q\}$ and $m_{k}'' \,=\, {\inf}\,\{f(x): q\,\,{\leq}\,\,x\,\,{\leq}\,\,x_{k}\}$.
    It is clear from properties of the infimum that $m_{k}\,\,{\leq}\,\,m_{k}'$ and $m_{k}\,\,{\leq}\,\,m_{k}''$, and thus
        \begin{displaymath}
        m_{k}\,{\Delta}x_{k} \,=\, m_{k}\,(x_{k}-x_{k-1})
     \,=\, 
        m_{k}\,((q-x_{k-1}) + (x_{k}-q))\,\,{\leq}\,\,
        m_{k}'\,(q-x_{k-1}) + m_{k}''\,(x_{k}-q).
        \end{displaymath}
    In other words, to obtain $L(f;{\cal Q})$ one replaces the term $m_{k}\,{\Delta}x_{k}$ in the sum for $L(f;{\cal P})$
    by the sum $m_{k}'\,(p-x_{k-1}) + m_{k}''\,(x_{k}-p)$, which is either bigger than or at worst equal to the first term. The equation $L(f;{\cal P})\,\,{\leq}\,\,L(f;{\cal Q})$ follows in this special case.
    To get the general case, in which ${\cal Q}$ is obtained from ${\cal P}$ by adjoining any finite number of new elements,
    simply repeat the preceding argument and use mathematical induction. A similar argument shows that $U(f;{\cal Q})\,\,{\leq}\,\,U(f;{\cal P})$.
    Finally, by Part~(a) the inequality $L(f;{\cal Q})\,\,{\leq}\,\,U(f;{\cal Q})$ is true.

\V


        (c) Let ${\cal Q} \,=\, {\cal P}\,{\cup}\,{\cal R}$, so that ${\cal Q}$ is a refinement of both ${\cal P}$ and ${\cal R}$. Then by Part~(b) one has
        \begin{displaymath}
        L(f;{\cal P})\,\,{\leq}\,\,L(f;{\cal Q})\,\,{\leq}\,\,U(f;{\cal Q})\,\,{\leq}\,\,U(f;{\cal R})
        \end{displaymath}
        The desired results then follow from the transitivity property of order and the definitions of `sup' and `inf'.

\V

        (d) The simple proof is left as an exercise. % EXERCISE

\V

        The preceding results suggest the following.

\V

        \subsection{\small{{\bf Definition}}}
        \label{DefH20.25}


\V

        Let $f$ be a function bounded on an interval $[a,b]$.

\V

        A number $B$ is said to be a {\bf Darboux number for $\Bfm{f}$ on $\Bfm{[a,b]}$}\IndB{Darboux, Gaston (1842-1917)}{Darboux number for $f$ on $[a,b]$}
    provided $L(f;{\cal P})\,\,{\leq}\,\,B\,\,{\leq}\,\,U(f;{\cal P})$ for every partition ${\cal P}$ of~$[a,b]$.

\V

        {\bf Remark}\, The preceding discussion can now be reformulated to say that in any reasonable definition of $\int_{a}^{b} f$,
    this number must be a Darboux number of $f$ on~$[a,b]$. Furthermore, Part~(c) of the preceding result implies that such Darboux numbers do exist; 
    namely ${\sup}_{{\Pi}[a,b]}\,\{L(f;{\cal P})\}$ and ${\inf}_{{\Pi}[a,b]}\,\{U(f;{\cal P})\}$,
    together with any number between these values. Darboux now focusses on the case in which there is only one such number,
    hence only one reasonable choice for $\int_{a}^{b} f$.

\V

        \subsection{\small{{\bf Definition}}}
        \label{DefH20.25A}

\V

        Let $f$ be a function bounded on an interval $[a,b]$.

\V

        One says that $f$ is {\bf integrable on $\Bfm{[a,b]}$ in the sense of Darboux}\IndB{integrals}{integrable in the sense of Darboux}\IndB{Darboux, Gaston (1842-1917)}{integrable in the sense of Darboux}
    provided that there exists precisely one Darboux number $B$ for $f$ on~$[a,b]$.
    If this is the case, then one writes ${\displaystyle B \,=\, \int_{a}^{b} f}$, and one calls the expression
    ${\displaystyle \int_{a}^{b}} f$ the {\bf Riemann integral\IndB{integrals}{Riemann integral} of $\Bfm{f}$ over $\Bfm{[a,b]}$}.
    In this context the function $f$ is called the {\bf integrand} and the numbers $a$ and $b$ are called the {\bf limits of integration}.

\V

        {\bf Remarks} (1) It is common to use the classical `variables' notation for functions and write something like ${\displaystyle \int_{a}^{b} f(x)\,dx}$
    in place of the simpler expression ${\displaystyle \int_{a}^{b} f}$. The role of the quantity $x$ here is the as the {\bf variable of integration}.
    The letter $x$ can be replaced by any letter that is not already in mathematical use here.

\V

        (2) The `Darboux lower sum' and `Darboux upper sum' terminology is standard in analysis;
    in contrast, the `Darboux difference' and `Darboux number' terminology, while convenient, is not.

\V

        (3) In calculus texts the quantity ${\displaystyle \int_{a}^{b} f(x)\,dx}$
    described above is normally introduced following an earlier approach of Riemann; see below.
    Riemann's approach to this quantity is technically more complicated than that of Darboux,
    but it avoids the use of the concepts of suprema and infima, concepts which are considered too difficult for elementary calculus.
    Since the two approaches lead to precisely the same set of integrable functions and the same values for their integrals,
    and Riemann's approach is earlier than that of Darboux, it is customary to assign the name
    `Riemann integral' to this quantity, regardless of which approach is taken, but to distinguish between `Darboux integrable' and `Riemann integrable'
    to indicate which approach one follows in obtaining the Riemann integral.
However, there are exceptions:
    some authors refer to the `Darboux integral' instead.

\VV

        For some purposes one needs a somewhat more general formulation of `Darboux number'.

\V

        \subsection{\small{{\bf Lemma}}}
        \label{LemmaH20.34}

\V

        Let ${\cal R}$ be a particular (fixed) partition of~$[a,b]$. Then a necessary and sufficient condition
    for a number $B$ to be a Darboux number for a function $f$ bounded on $[a,b]$ is that
        \begin{displaymath}
        L(f;{\cal Q})\,\,{\leq}\,\,B\,\,{\leq}\,\,U(f;{\cal Q}) \mbox{ for every refinement ${\cal Q}$ of~${\cal R}$}.
        \end{displaymath}

\V

        {\bf Proof}

\V

        The necessity of the given condition is obvious. To show its sufficiency,
    suppose that $B$ satisfies the given condition, and let ${\cal P}$ be any partition of~$[a,b]$.
    As in the proof of Part~(c) above, let ${\cal Q} \,=\, {\cal P}\,{\cup}\,{\cal R}$, so that ${\cal Q}$ is a refinement of~${\cal R}$.
    Then the hypothesis here says that $L(f;{\cal Q})\,\,{\leq}\,\,B\,\,{\leq}\,\,U(f;{\cal Q})$,
    so that by Part~(c) again one has
        \begin{displaymath}
        L(f;{\cal P})\,\,{\leq}\,\,L(f;{\cal Q})\,\,{\leq}\,\,B\,\,{\leq}\,\,U(f;{\cal Q})\,\,{\leq}\,\,U(f;{\cal R})
        \end{displaymath}
    for every partition~${\cal P}$ of $[a,b]$. The claimed sufficiency now follows.

\VV

        \subsection{\small{{\bf Examples}}}
        \label{ExampH20.30}

\V
        In the following examples, the word `integrable' means `integrable in the sense of Darboux'.

\V

        (1) If $f:I \,{\rightarrow}\, {\RR}$ is continuous on an open interval $I$, then, by Part~(b) of Theorem~\Ref{TheoremE45.130}, for every pair of numbers $a$ and $b$ in $I$ such that $a\,<\,b$,
    the function $f$ is integrable on $[a,b]$. Furthermore, ${\displaystyle \int_{a}^{b} f \,=\, F(b)-F(a)}$, where $F$ is any antiderivative of $f$ on~$I$.

\V

        (2) If $f:[a,b] \,{\rightarrow}\, {\RR}$ is monotonic up on $[a,b]$, then $f$ is integrable on $[a,b]$.
    This is trivially true by Part~(1) if $f(a) \,=\, f(b)$, since in this case $f$ is constant, hence continuous, on~$[a,b]$.

        Now assume that $f(a)\,<\,f(b)$. Suppose, in contradiction, that $f$ is {\em not} integrable on~$[a,b]$. Then there exist Darboux numbers
    $B_{1}$ and $B_{2}$ for $f$ on~$[a,b]$ with $B_{1}\,<\,B_{2}$, such that
        \begin{displaymath}
        L(f;{\cal P})\,\,{\leq}\,\,B_{1}\,<\,B_{2}\,\,{\leq}\,\,U(f;{\cal P}) \mbox{ for every partition ${\cal P} \,=\, \{a \,=\, x_{0}\,<\,x_{1}\,<\,\,{\ldots}\,\,<\,x_{n-1}\,<\,x_{n} \,=\, b\}$ of $[a,b]$}.
        \end{displaymath}
    Using the notation from Part~(1) of Definition~\Ref{DefH20.20}, together with the hypothesis that $f$ is monotonic up, one gets
        \begin{displaymath}
        L(f;{\cal P}) \,=\, \sum_{j=1}^{n} f(x_{j-1})\,{\Delta}x_{j}\,\,{\leq}\,\,B_{1}\,<\,B_{2}\,\,{\leq}\,\,\sum_{j=1}^{n} f(x_{j})\,{\Delta}x_{j} \,=\, U(f;{\cal P}).
        \end{displaymath}
    It follows that
        \begin{displaymath}
        0\,<\,B_{2}-B_{1}\,\,{\leq}\,\,\sum_{j=1}^{n} \left(f(x_{j})-f(x_{j-1})\right)\,{\Delta}x_{j}
    \,\,{\leq}\,\,
    \left(\sum_{j=1}^{n} (f(x_{j})-f(x_{j-1}))\right)\,||{\cal P}||
     \,=\, (f(b)-f(a))\,||{\cal P}||
        \end{displaymath}
    for {\em every} partition ${\cal P}$ of $[a,b]$. Now chose ${\cal P}$ so that $||{\cal P}||\,<\,(B_{2}-B_{1})/(f(b)-f(a))$ to get the contradiction.

        Clearly the conclusion, that $f$ is integrable, remains true if one assumes instead that $f$ is monotonic {\em down} on $[a,b]$.

\V

        (3) Let $f:[0,1] \,{\rightarrow}\, {\RR}$ be the restriction to $[0,1]$ of the Dirichlet function,
    so that for $x$ in $[0,1]$ one has $f(x) \,=\, 0$ if $x$ is irrational while $f(x) \,=\, 1$ if $x$ is rational.
    Then clearly $L(f;{\cal P}) \,=\, 0$ and $U(f;{\cal P}) \,=\, 1$ for every partition ${\cal P}$ of~$[0,1]$.
    In particular, $f$ is {\em not} integrable on~$[0,1]$. Indeed, if $B$ is any of the infinitely many numbers in the closed interval $[0,1]$, then $B$ is a Darboux number for $f$ on~[0,1].

\V

        (4) Let $f:[0,1] \,{\rightarrow}\, {\RR}$ be the restriction to $[0,1]$ of the Thomae function;
    that is, for $x$ in $[0,1]$ one has $f(x) \,=\, 0$ if $x$ is irrational, or if $x \,=\, 0$;
    while if $x\,>\,0$ is rational and of the form $x \,=\, p/q$, with $p, q$ being natural numbers having no common factors bigger than~$1$, then $f(x) \,=\, 1/q$.
    It is clear that if ${\cal P}$ is any partition of $[0,1]$, then $L(f;{\cal P}) \,=\, 0$ (since every subinterval of $[0,1]$ includes irrational points),
    and that $U(f;{\cal P})\,\,{\leq}\,\,1$ (since $1/q\,\,{\leq}\,\,1$ for each $q$ in~${\NN}$). In particular, $0$ is a Darboux number for $f$ on~$[0,1]$.
    To see that $0 $ is the only such Darboux number, suppose that $B$ satisfies $0\,<\,B\,\,{\leq}\,\,1$, and let $n$ be any natural number such that $1/n\,<\,B/3$.
    If $x$ in $[0,1]$ satisfies $f(x)\,\,{\geq}\,\,B$, so that $f(x)\,\,{\geq}\,\,B/3\,>\,1/n$, then $x$ can be written in the form $p/q$,
    as described above, with $1\,\,{\leq}\,\,p\,\,{\leq}\,\,q\,<\,n$.
    Clearly there are at most $n^{2}$ such values of~$x$. Now let ${\cal P}$ be the partition of $[0,1]$ into $n^{3}$ subintervals of equal length $1/n^{3}$.
    At most $2\,n^{2}$ of these subintervals contain $x$ such that $f(x)\,>\,1/n$. (The factor $2$ reflects the fact that such $x$ might be an endpoint of two contiguous subintervals.)
    The sum of the terms in the expression for $U(f;{\cal P})$ from those subintervals then is strictly bounded above by $2\,n^{2}/n^{3} \,=\, 2/n$.
    The sum of the remaining terms appearing in $U(f;{\cal P})$ is clearly bounded above by $1/n$, so that $U(f;{\cal P})\,<\,3/n\,<\,B$, so that $B$ is not a Darboux number for $f$ on~$[0,1]$.

\V

        {\bf Remark}\, The last example shows that $f$ being Riemann integrable on $[a,b]$ implies nothing about $f$ having an antiderivative on~$[a,b]$.
    Indeed, the Intermediate-Value Theorem for Derivatives implies that the Thomae function has an antiderivative on {\em no} subinterval of~$[0,1]$.

\VV

        \subsection{\small{{\bf Corollary}}}
        \label{CorH20.35A}

\V

        Suppose that $f:[a,b] \,{\rightarrow}\, {\RR}$ is bounded on the interval~$[a,b]$ and that ${\cal R}$ is a fixed partition of~$[a,b]$.
    Let $L_{(f;{\cal R})} \,=\, \{L(f; {\cal P}): \mbox{${\cal P}$ is a refinement of ${\cal R}$}\}$,
    and let $U_{(f;{\cal R})} \,=\, \{U(f; {\cal Q}): \mbox{ ${\cal Q}$ is a refinement of ${\cal R}$}\}$. Then:

\V

        (a) The sets $L_{(f;{\cal R})}$ and $U_{(f;{\cal R})}$ are nonempty bounded subsets of ${\RR}$, and one has ${\sup}\,L_{(f;{\cal R})}\,\,{\leq}\,\,{\inf}\,U_{(f;{\cal R})}$.
    Furthermore, a number $B$ is a Darboux number for $f$ on $[a,b]$ if, and only if, ${\sup}\,L_{(f;{\cal R})}\,\,{\leq}\,\,B\,\,{\leq}\,\,{\inf}\,U_{(f;{\cal R})}$.

\V

        (b) The function $f$ is integrable on $[a,b]$ in the sense of Darboux if, and only if, ${\sup}\,L_{(f;{\cal R})} \,=\, {\inf}\,U_{(f;{\cal R})}$.
    If this equality holds, then the common value of these two expressions equals ${\displaystyle \int_{a}^{b} f}$.

\V

        {\bf Proof}\, (a) The fact that $L_{(f;{\cal R})}$ and $U_{(f;{\cal R})}$ are nonempty sets of numbers follows from the fact that
    there exists at least one partition of the given interval~$[a,b]$ which is a refinement of~${\cal R}$; for example, ${\cal R}$ itself.
    It then follows from Part~(b) of the preceding theorem that the set $L_{(f;{\cal R})}$ is bounded below and the set $U_{(f;{\cal R})}$ is bounded above,
    namely by the numbers $L(f;{\cal R})$ and $U(f;{\cal R})$, respectively. Furthermore, it follows from Part~(d) of the preceding theorem that if $x{\in}L_{(f;{\cal R})}$ and $y{\in}U_{(f;{\cal R})}$,
    then $x\,\,{\leq}\,\,y$; in particular, $X$ is bounded and $Y$ is bounded below. It now follows from Part~(d) of Theorem~\Ref{ThmB30.08K} that ${\sup}\,L_{(f;{\cal R})}\,\,{\leq}\,\,{\inf}\,U_{(f;{\cal R})}$, as claimed.
    It also follows that if $B$ is any number such that  ${\sup}\,L_{(f;{\cal R})}\,\,{\leq}\,\,B\,\,{\leq}\,\,{\inf}\,U_{(f;{\cal R})}$, then $B$ has the desired property.

\V

        (b) From the proof of the preceding part of this theorem it is clear that if ${\sup}\,L_{(f;{\cal R})}\,<\,{\inf}\,U_{(f;{\cal R})}$,
    then there are infinitely many values of $B$ such that $L(f;{\cal P})\,\,{\leq}\,\,B\,\,{\leq}\,\,U(f;{\cal P})$
    for every refinement ${\cal P}$ of~${\cal R}$, in which case $f$ is {\em not} integrable on $[a,b]$ in the sense of Darboux.
    In contrast, it likewise follows that if ${\sup}\,L_{(f;{\cal R})} \,=\, {\inf}\,U_{(f;{\cal R})}$, then there is precisely one value of $B$ for which
    $L(f;{\cal P})\,\,{\leq}\,\,B\,\,{\leq}\,\,U(f;{\cal P})$ for every such partition; namely the common value of ${\sup}\,L_{(f;{\cal R})}$ and ${\inf}\,U_{(f;{\cal R})}$. The desired result now follows. \Q

\V

        \subsection{\small{{\bf Remarks}}}
        \label{RemrkH20.35A}

\V

\hspace*{\parindent}(1) It follows from Part~(a) of the preceding corollary that the numbers ${\sup}\,L_{(f;{\cal R})}$
    and ${\inf}\,U_{(f;{\cal R})}$ described there do not depend on the choice of the fixed partition ${\cal R}$ of $[a,b]$.
    In particular, suppose that one chooses ${\cal R}$ to be the `trivial' partition $\{a,b\}$ of $[a,b]$, so that $L_{(f;{\cal R})} \,=\, L_{(f;\{a,b\})}$ and $U_{(f;{\cal R})} \,=\, U_{(f;\{a,b\})}$.
    Then the set of partitions ${\cal P}$ referred to in the corollary is simply set of {\em all} partitions of~$[a,b]$.

\V

     (2) There are several popular notations for the quantities ${\sup}\,L_{(f;{\cal R})}$ and ${\inf}\,U_{(f;{\cal R})}$ considered above, for example:

\VA

        \h (i)\,\, ${\sup}\,L_{(f;{\cal R})} \,=\, \underline{I}(f;[a,b])$; ${\inf}\,U_{(f;{\cal R})} \,=\, \overline{I}(f;[a,b])$.

\VA

        \h (ii)\, ${\sup}\,L_{(f;{\cal R})} \,=\, \underline{\int}_{a}^{b} f$; ${\inf}\,U_{(f;{\cal R})} \,=\, \overline{\int}_{a}^{b}\,f$.

\VA

        Because of the notations in~(ii), many texts refer to ${\sup}\,L_{(f;{\cal R})}$ as the {\bf lower Darboux integral of $f$ on $[a,b]$},
    and to ${\inf}\,L_{(f;{\cal R})}$ as the {\bf upper Darboux integral of $f$ on $[a,b]$}.\IndB{Darboux, Gaston (1842-1917)}{lower, upper Darboux integrals}
    In {\ThisText} we avoid the use of such specialized notations and terminology, but the reader needs to know that it does appear elsewhere.

\VV


        There are several alternate ways to characterize the property of a function being integrable on an interval.
    In the following theorem and its proof, the word `integrable' means `integrable in the sense of Darboux'.

\V

        \subsection{\small{{\bf Theorem}}}
        \label{ThmH20.40}

\V

        Let $f:[a,b] \,{\rightarrow}\, {\RR}$ be a function which is bounded on the interval $[a,b]$. Then the following statements are equivalent:

\V

        \underline{Statement (i)}\,\,\,The function $f$ is integrable in the sense of Darboux on $[a,b]$;
    that is, $f$ has exactly one Darboux number for~$[a,b]$.

\V

        \underline{Statement (ii)}\,\, For every ${\varepsilon}\,>\,0$ there exists a partition
    ${\cal R}$ of $[a,b]$ such that ${\Delta}(f;{\cal R})\,<\,{\varepsilon}$, where, as usual,
    ${\Delta}(f;{\cal R})$ is the (nonnegative) Darboux difference $U(f;{\cal R}) - L(f;{\cal R})$ associated with the partition~${\cal R}$.

\V


        \underline{Statement (iii)} For every ${\varepsilon}\,>\,0$ there exists ${\delta}\,>\,0$ such that if ${\cal R}$ is any partition of $[a,b]$ for which $||{\cal R}||\,<\,{\delta}$, then 
        \begin{displaymath}
        {\Delta}(f;{\cal R})\,<\,{\varepsilon}.
        \end{displaymath}


\V

        \underline{Statement (iv)}\, There exists a number $B$ with the following property: for every ${\varepsilon}\,>\,0$
    there exists ${\delta}\,>\,0$ such that if ${\cal P} \,=\, \{a \,=\, x_{0}\,<\,x_{1}\,<\,\,{\ldots}\,\,<\,x_{n-1}\,<\,x_{n} \,=\, b\}$
    is any partition of $[a,b]$ for which $||{\cal P}||\,<\,{\delta}$, then for any ordered list ${\zeta} \,=\, (z_{1}, z_{2},\,{\ldots}\,z_{n})$
    of numbers with $x_{j-1}\,\,{\leq}\,\,z_{j}\,\,{\leq}\,\,x_{j}$ for each $j$ one has
        \begin{displaymath}
        \left|B - \sum_{j=1}^{n} f(z_{j}){\Delta}x_{j}\right|\,<\,{\varepsilon}.
        \end{displaymath}



\V%\\\\

        {\bf Proof}\, The fact that Statement (i) implies Statement~(ii) is a simple consequence of Part~(b) of Corollary~\Ref{CorH20.35A}.
    Indeed, suppose that $f$ is integrable on $[a,b]$, so that by that corollary one has ${\sup}\,L_{(f;\{a,b\})} \,=\, {\inf}\,U_{(f;\{a,b\})}$.
    Let $B$ be the common value of these quantities. It then follows easily from the Approximation Property for infima and suprema that for every ${\varepsilon}\,>\,0$
    there exist partitions ${\cal P}$ and ${\cal Q}$ of $[a,b]$ such that
        \begin{displaymath}
        L(f;{\cal P})\,>\,B-\frac{{\varepsilon}}{2} \mbox{ and } U(f;{\cal Q})\,<\,B+\frac{{\varepsilon}}{2}.
        \end{displaymath}
    Now let ${\cal R} \,=\, {\cal P} \,{\cup}\, {\cal Q}$. The desired inequality follows from Part~(b) of Lemma~\Ref{LemmaH20.33}.

       As for the converse, namely that Statement~(ii) implies Statement~(i), recall that for every partition ${\cal R}$ of $[a,b]$ one has
                \begin{displaymath}
        L(f;{\cal R})\,\,{\leq}\,\,{\sup}\,L_{(f;\{a,b\})}\,\,{\leq}\,\,{\inf}\,U_{(f;\{a,b\})}\,\,{\leq}\,\,U(f;{\cal R})
        \end{displaymath}
    If Statement~(ii) holds, then for every ${\varepsilon}\,>\,0$ there exists ${\cal R}$ such that
        \begin{displaymath}
        0\,\,{\leq}\,\,U_{(f;\{a,b\})} - L_{(f;\{a,b\})}\,\,{\leq}\,\,U(f;{\cal R})-L(f;{\cal R})\,<\,{\varepsilon},
        \end{displaymath}
    which implies that $U_{(f;\{a,b\})} \,=\, L_{(f;\{a,b\})}$, and thus, by Part~(b) of Corollary~\Ref{CorH20.35A}, $f$ is integrable on~$[a,b]$.

\VA

        To see that Statement (ii) implies Statement (iii), let ${\cal R} \,=\, \{a \,=\, y_{0}\,<\,y_{1}\,<\,\,{\ldots}\,\,<\,y_{N-1}\,<\,y_{N} \,=\, b\}$
    be a partition of $[a,b]$ such that ${\Delta}(f;{\cal R})\,<\,{\varepsilon}/2$; such ${\cal R}$ exists by the hypothesis that Statement~(ii) holds.
    Let ${\delta}_{1}$ be the minimum of the lengths of the $N$ subintervals of $[a,b]$ determined by the partition~${\cal R}$; this is (probably) {\em not} the ${\delta}$ referred to in Statement~(iii).
    To set up the search for the true ${\delta}$, let ${\cal P} \,=\, \{a \,=\, x_{0}\,<\,x_{1}\,<\,\,{\ldots}\,\,<\,x_{n-1}\,<\,x_{n} \,=\, b\}$ be any partition of $[a,b]$ such that $||{\cal P}||\,<\,{\delta}_{1}$, and let ${\cal Q} \,=\, {\cal P}\,{\cup}\,{\cal R}$;
    note that ${\cal Q}$ is a refinement of ${\cal R}$, so that ${\Delta}(f;{\cal Q})\,<\,{\varepsilon}/2$ as well.
    Because $0\,<\,x_{j}-x_{j-1}\,<\,{\delta}_{0}$ for each~$j$, it is clear that for each $j$ either the subinterval $[x_{j-1},x_{j}]$ of the partition ${\cal P}$ is also a subinterval of ${\cal Q}$,
    or there exists exactly one index $k$ such that $x_{j-1}\,<\,y_{k}\,<\,x_{j}$; in that case $[x_{j-1},x_{j]}$ is the union of exactly two subintervals for ${\cal Q}$,
    namely $[x_{j-1}, y_{k}]$ and $[y_{k},x_{j}]$. It follows that one can write ${\Delta}(f;{\cal P}) \,=\, {\Delta}(f;{\cal Q}) + S$,
    where $S$ is the sum of at most $N-1$ terms, each of the form $(M_{j}-m_{j})\,(x_{j}-x_{j-1}) - \left((M_{k}'-m_{k}')\,(y_{k}-x_{j-1}) + (M_{k}''-m_{k}'')\,(x_{j}-y_{k})\right)$.
    In this last expression, as usual $m_{j} \,=\, {\inf}\,\{f(x): x_{j-1}\,\,{\leq}\,\,x\,\,{\leq}\,\,x_{j}\}$, and $M_{j} \,=\, {\sup}\,\{f(x): x_{j-1}\,\,{\leq}\,\,x\,\,{\leq}\,\,x_{j}\}$;
    likewise, $m_{k}', M_{k}'$ and $m_{k}'', M_{k}''$ are the analogous quantities on $[x_{j-1},y_{k}]$ and $[y_{k},x_{j}]$, respectively.
    Let $L \,=\, {\sup}\,\{f(x): a\,\,{\leq}\,\,x\,\,{\leq}\,\,b\} - {\inf}\,\{f(x): a\,\,{\leq}\,\,x\,\,{\leq}\,\,b\}$,
    Then each term forming the sum $S$ above has magnitude no bigger than $3\,L\,||{\cal P}||$. Since $S$ has at most $N-1$ such terms, it follows that certainly $|S|\,\,{\leq}\,\,3\,(N-1)\,L\,||{\cal P}||\,\,{\leq}\,\,3\,N\,(L+1)\,||{\cal P}||$.
    Now let $d_{2} \,=\, {\varepsilon}/(6\,N\,(L+1))$, and let ${\delta} \,=\, \min\{{\delta}_{1}, {\delta}_{2}\}$.
    Then $||{\cal P}||\,<\,{\delta}$ implies that $|S|\,<\,{\varepsilon}/3$, and thus
        \begin{displaymath}
        {\Delta}(f;{\cal P}) \,=\, {\Delta}(f;{\cal Q}) + S\,\,{\leq}\,\,{\Delta}(f;{\cal Q}) + |S|
        \,<\, \frac{{\varepsilon}}{2} + \frac{{\varepsilon}}{2} \,=\, {\varepsilon},
        \end{displaymath}
    as required. (The use of $N$ and $L+1$ the weaker estimate for $|S|$, instead of the more accurate $L$ and $N-1$,
    is to avoid the possible division-by-zero problem in defining ${\delta}_{2}$.)

        The proof that Statement~(iii) implies Statement~(ii) is obvious.

\VA

        Next, suppose that Statement~(iii) holds, and thus, as has been shown, Statements~(i) and~(ii) also hold, so that $f$ is integrable on~$[a,b]$. Let ${\displaystyle B \,=\, \int_{a}^{b} f}$.
    Then for every ${\varepsilon}\,>\,0$ there exists ${\delta}\,>\,0$ such that $0\,\,{\leq}\,\,U(f;{\cal P}) - L(f;{\cal P})\,<\,{\varepsilon}$ for any partition
    ${\cal P} \,=\, \{a \,=\, x_{0}\,<\,x_{1}\,<\,\,{\ldots}\,\,<\,x_{n-1}\,<\,x_{n} \,=\, b\}$ for which $||{\cal P}||\,<\,{\delta}$.
    Note that, by definition of `integral', one has $L(;f{\cal P})\,\,{\leq}\,\,B\,\,{\leq}\,\,U(f;{\cal P})$.
   Furthermore, if ${\zeta} \,=\, (z_{1}, z_{2},\,{\ldots}\,z_{n})$ is as in Statement~(iv), then it is clear that 
        \begin{displaymath}
        L(f;{\cal P})\,\,{\leq}\,\,\sum_{j=1}^{n} f(z_{j}){\Delta}x_{j}\,\,{\leq}\,\,U(f;{\cal P})
        \end{displaymath}
    It follows that
        \begin{displaymath}
        \left|B - \left(\sum_{j=1}^{n} f(z_{j}){\Delta}x_{j}\right)\right|
    \,\,{\leq}\,
        \,U(f;{\cal P})-L(f;{\cal P})\,<\,{\varepsilon},
        \end{displaymath}
    as required.

        As for the converse, suppose that $f$ is {\em not} integrable on~$[a,b]$. Then ${\sup}\,L_{(f;\{a,b\})}\,<\,{\inf}\,U_{(f;\{a,b\})}$,
    so there must exist Darboux numbers $B_{1}$ and $B_{2}$ for $f$ on $[a,b]$ such that ${\sup}\,L_{(f;\{a,b\})}\,<\,B_{1}\,<\,B_{2}\,<\,{\inf}\,U_{(f;\{a,b\})}$.
    Let ${\varepsilon}_{0} \,=\, B_{2}-B_{1}\,>\,0$. It is a simple exercise to show, using the Approximation Property for infima and suprema,
    that if ${\cal P} \,=\, \{a \,=\, x_{0}\,<\,x_{1}\,<\,\,{\ldots}\,\,<\,x_{n-1}\,<\,x_{n} \,=\, b\}$ is any partition of~$[a,b]$,
    then there exist ordered lists ${\zeta} \,=\, (z_{1}, z_{2},\,{\ldots}\,z_{n})$ and ${\tau} \,=\, (t_{1}, t_{2},\,{\ldots}\,t_{n})$,
    satisfying $x_{j-1}\,\,{\leq}\,\,z_{j}, t_{j}\,\,{\leq}\,\,x_{j}$ for each $j \,=\, 1,2,\,{\ldots}\,n$, such that
        \begin{displaymath}
        \sum_{j=1}^{n} f(z_{j})\,{\Delta}x_{j}\,\,{\leq}\,\,B_{1}\,<\,B_{2}\,\,{\leq}\,\,\sum_{j=1}^{n} f(t_{j})\,{\Delta}x_{j}.
        \end{displaymath}
    This implies that ${\displaystyle \left|\left(\sum_{j=1}^{n} f(t_{j})\,{\Delta}x_{j}\right) - \left(\sum_{j=1}^{n} f(z_{j})\,{\Delta}x_{j}\right)\right|\,\,{\geq}\,\,B_{2}-B_{1} \,=\, {\varepsilon}_{0}}$.
    Since this holds for {\em every} partition of~$[a,b]$, it follows that Statement~(iv) cannot hold. %EXERCISES

\V

        {\bf Remarks} (1) The description of integrability given by Statement (iv) in the preceding theorem is Riemann's original formulation of the concept;
    for that reason in {\ThisText} we say that a function which satisfies this statement is {\bf integrable in the sense of Riemann}\IndB{integrals}{integrable in the sense of Riemann};
    or, more briefly, it is {\bf Riemann integrable}.

\V

        (2) It is easy to see that if Statement (iv) holds for $f$ on~$[a,b]$, then the number $B$ mentioned there equals ${\displaystyle \int_{a}^{b} f}$; in particular, $B$ is unique.
    In particular, Darboux's approach yields exactly the same class of integrable functions as does Riemann's, and exactly the same value for the integral.
    The usual custom in analysis texts -- including {\ThisText} -- is to refer to the final product by the name `Riemann integral'\IndB{Riemann, Bernhard (1826-1866)}{Riemann integral}, even when the tools used in its analysis are based on the approach of~Darboux.

\V

        (3) Statements (ii) and (iii) in the preceding theorem are analogs of the Cauchy Criterion for the convergence of sequences of numbers.
    More precisely, they can be used to prove the existence of a certain number $B$ without requiring in advance a particular candidate for this number.

\V

        (4) There are several other approaches to integration which are more general than that of Riemann/Darboux and lead to different classes of `integrable' funnctions.
    These approaches are associated with names such as Lebesgue, Stieltjes and Henstock, and are not studied here. %% BUT MAYBE ELSEWHERE IN THE TEXT?

\VV

%\\\\


        \section{{\bf Basic Properties of the Riemann Integral}}
        \label{SectH25}\IndB{ZZ Sections}{\Ref{SectH25} Basic Properties of the Riemann Integral}

\V

        This section considers some of the basic properties of the Riemann integral. The following lemma forms the connecting link with the preceding section.

\V

        \subsection{\small{{\bf Lemma}}}
        \label{LemmaH25.10}

\V

        Suppose that $f:[a,b] \,{\rightarrow}\, {\RR}$ is a bounded function on $[a,b]$, and that $a\,<\,c\,<\,b$.
    Then a number $B$ is a Darboux number for $f$ on $[a,b]$ if, and only if, there exist Darboux numbers $C$ and $D$ for $f$ on $[a,c]$ and $[c,b]$, 
    repectively, such that $B \,=\, C+D$.

\V

        {\bf Proof}\, Suppose first that $C$ and $D$ are Darboux numbers for $f$ on $[a,c]$ and $[c,b]$, respectively.
    Then by Part~(b) of Corollary~\Ref{CorH20.35A} one has
        \begin{displaymath}
        {\sup}\,L_{(f;{\cal Q})}\,\,{\leq}\,\,C\,\,{\leq}\,\,{\inf}\,U_{(f;{\cal Q})}
        \end{displaymath}
    and
        \begin{displaymath}
        {\sup}\,L_{(f;{\cal R})}\,\,{\leq}\,\,C\,\,{\leq}\,\,{\inf}\,U_{(f;{\cal R})},
        \end{displaymath}
    where ${\cal Q}$ and ${\cal R}$ are partitions of $[a,c]$ and $[c,b]$, respectively.
    By Part~(e) of Lemma~\Ref{LemmaH20.33}, to determine the Darboux numbers for $f$ on $[a,b]$ it suffices to restrict attention to partitions ${\cal P}$ of $[a,b]$ such that $c{\in}{\cal P}$;
    that is, to refinements ${\cal P}$ of the special partition ${\cal P}_{0} \,=\, \{a,c,b\}$. For such ${\cal P}$ one can write ${\cal P} \,=\, {\cal Q}\,{\cup}\,{\cal R}$,
    where ${\cal Q} \,=\, {\cal P}\,{\cap}\,[a,c]$ is a partition of $[a,c]$ and ${\cal R} \,=\, {\cal P}\,{\cap}\,[c,b]$ is a partition of~$[c,b]$.
    By Part~(c) of Lemma~\Ref{LemmaH20.33} one has
        \begin{displaymath}
        L(f;{\cal P}) \,=\, L(f;{\cal Q}) + L(f;{\cal R})\,\,{\leq}\,\,C+D
    \,\,{\leq}\,\,
        U(f;{\cal Q}) + U(f;{\cal R}) \,=\, U(f;{\cal P}).
        \end{displaymath}
    It follows that $B \,=\, C+D$ is a Darboux number for $f$ on $[a,b]$, as claimed.

        As for the converse, let ${\cal Q}$ and ${\cal R}$ be partitions of $[a,c]$ and $[c,b]$, 
    respectively,
    and let ${\cal P} \,=\, {\cal Q}\,{\cup}\,{\cal R}$ be the corresponding partition of $[a,b]$ containing~$c$.
    It is easy to see that if ${\cal P}_{0} \,=\, \{a,c,b\}$ as above, while ${\cal Q}_{0} \,=\, \{a,c\}$ and ${\cal R}_{0} \,=\, \{c,b\}$ are the corresponding trivial partitions of $[a,c]$ and $[c,b]$, 
    respectively, then
        \begin{displaymath}
        {\sup}\,L_{(f;{\cal P}_{0})}
         \,=\, 
        {\sup}\,L_{(f;{\cal Q}_{0})}
    +
        {\sup}\,L_{(f;{\cal R}_{0})}.
        \end{displaymath}
    Likewise, one has
        \begin{displaymath}
        {\inf}\,U_{(f;{\cal P}_{0})}
         \,=\, 
        {\inf}\,U_{(f;{\cal Q}_{0})}
    +
        {\inf}\,U_{(f;{\cal R}_{0})}.
        \end{displaymath}
    In other words, if $B_{1}$ is the lowest Darboux number for $f$ on $[a,b]$, then one has
    $B_{1} \,=\, C_{1} + D_{1}$, where $C_{1}$ and $D_{1}$ are the lowest Darboux numbers for $f$ on $[a,c]$ and $[c,b]$, respectively.
    The analogous equation $B_{2} \,=\, C_{2} + D_{2}$ holds for the corresponding highest Darboux numbers.
    Finally, note that if $B$ is any Darboux number for $f$ on $[a,b]$, then one can write
        \begin{displaymath}
        B \,=\, t\,B_{1} + (1-t)\,B_{2} \mbox{ for some number $t$ in $[0,1]$}.
        \end{displaymath}
    One then gets $B \,=\, C+D$ where
        \begin{displaymath}
        C \,=\, t\,C_{1} + (1-t)\,C_{2}
    \mbox{ and }
        D \,=\, t\,D_{1} + (1-t)\,D_{2}
        \end{displaymath}
    are the desired Darboux numbers of $f$ on $[a,c]$ and $[c,b]$, respectively.

\VV

        \subsection{\small{{\bf Theorem}}}
        \label{ThmH25.20}

\V

\hspace*{\parindent}(a) Suppose that $f:[a,b] \,{\rightarrow}\, {\RR}$ can be expressed in the form $f \,=\, A{\cdot}g$ for some constant~$A$, where $g:[a,b] \,{\rightarrow}\, {\RR}$ is integrable on~$[a,b]$.
    Then $f$ is also integrable on~$[a,b]$, and one has
        \begin{displaymath}
        \int_{a}^{b} f \,=\, A{\cdot}\int_{a}^{b} g.
        \end{displaymath}
\V

        (b) Suppose that $f:[a,b] \,{\rightarrow}\, {\RR}$ can be expressed in the form $f \,=\, f_{1} + f_{2}$,
    where the functions $f_{1}:[a,b] \,{\rightarrow}\, {\RR}$ and $f_{2}:[a,b] \,{\rightarrow}\, {\RR}$ are both integrable on $[a,b]$. Then the function $f$ is also integrable on~$[a,b]$, and one has
        \begin{displaymath}
        \int_{a}^{b} f \,=\, \left(\int_{a}^{b} f_{1}\right) + \left(\int_{a}^{b} f_{2}\right).
        \end{displaymath}
    More generally, suppose that $f$ can be expressed on $[a,b]$ in the form
        \begin{displaymath}
        f \,=\, A_{1}{\cdot}g_{1} + \,{\ldots}\, + A_{n}{\cdot}g_{n},
        \end{displaymath}
    where $A_{1}$, \,{\ldots}\,$A_{n}$ are real constants and $g_{1}$,\,{\ldots}\,$g_{n}$ are integrable on~$[a,b]$.
    Then $f$ is also integrable on $[a,b]$, and one has
        \begin{displaymath}
        \int_{a}^{b} f = A_{1}\,\int_{a}^{b} g_{1} + \,{\ldots}\, + \int_{a}^{b} g_{n}.
        \end{displaymath}

\V

        (c) Let $c$ be any number such that $a\,<\,c\,<\,b$. Then $f:[a,b] \,{\rightarrow}\, {\RR}$ is integrable on $[a,b]$ if, and only if, it is integrable on each of the intervals $[a,c]$ and~$[c,b]$.
    Furthermore, if this happens, then one has

        \begin{displaymath}
        \int_{a}^{b} f \,=\, \left(\int_{a}^{c} f\right) + \left(\int_{c}^{b} f\right).
        \end{displaymath}

\V

        (d) Suppose that $f:[a,b] \,{\rightarrow}\, {\RR}$ satisfies the condition that
    $f(x) \,=\, h(x)$ for all but finitely many values of $x$ in~$[a,b]$,
    where $h:[a,b] \,{\rightarrow}\, {\RR}$ is integrable on~$[a,b]$. Then $f$ is also integrable on $[a,b]$, and one has ${\displaystyle \int_{a}^{b} f \,=\, \int_{a}^{b} h}$.

\V


        {\bf Proof}\, (a) The simple proof breaks into the separate cases $A\,>\,0$, $A\,<\,0$ and $A \,=\, 0$, and is left as an exercise. %EXERCISE

\V

        (b) For convenience, set ${\displaystyle B_{1} \,=\, \int_{a}^{b} f_{1}}$ and ${\displaystyle B_{2} \,=\, \int_{a}^{b} f_{2}}$.
    Then by the definition of `integral', the numbers $B_{1}$ and $B_{2}$ are the only ones which satisfy
        \begin{displaymath}
        L(f_{1};{\cal P})\,\,{\leq}\,\,B_{1}\,\,{\leq}\,\,U(f_{1};{\cal P})
        \mbox{ and }
        L(f_{2};{\cal Q})\,\,{\leq}\,\,B_{2}\,\,{\leq}\,\,U(f_{2};{\cal Q})
        \mbox{ for all partitions ${\cal P}$ and ${\cal Q}$ of $[a,b]$}.
        \end{displaymath}
    By summing these inequalities one gets, for ${\cal R} \,=\, {\cal P}\,{\cup}\,{\cal Q}$,
    that if $B$ is any Darboux number for $f$, then
        \begin{displaymath}
        L(f_{1};{\cal P}) + L(f_{2};{\cal Q})\,\,{\leq}\,\,L(f_{1};{\cal R}) + L(f_{2};{\cal R}) \,=\, L(f_{3};{\cal R})
    \,\,{\leq}\,\,
        \end{displaymath}
        \begin{displaymath}
        B\,\,{\leq}\,\,
        U(f;{\cal R})\,\,{\leq}\,\,U(f_{1};{\cal R}) + U(f_{2};{\cal R})
        \,\,{\leq}\,\,
        U(f_{1}; {\cal P}) + U(f_{2};{\cal Q}).
        \end{displaymath}
    In particular, one has
        \begin{displaymath}
        L(f_{1};{\cal P}) + L(f_{2};{\cal Q})\,\,{\leq}\,\,B \mbox{ for all partitions ${\cal P}$ and ${\cal Q}$ of $[a,b]$}.
        \end{displaymath}
    Keeping ${\cal P}$ fixed for the moment, take the supremum over ${\cal Q}$ and use the fact that
    $B_{2}$ is the supremum of the numbers of the form $L(f_{2};{\cal Q})$ to get
        \begin{displaymath}
        L(f_{1};{\cal P}) + B_{2}\,\,{\leq}\,\,B \mbox{ for every partition ${\cal P}$ of $[a,b]$};
        \end{displaymath}
    that is,
        \begin{displaymath}
        L(f_{1};{\cal P})\,\,{\leq}\,\,B - B_{2} \mbox{ for every partition ${\cal P}$ of $[a,b]$}.
        \end{displaymath}
    In a similar manner one gets
        \begin{displaymath}
        B-B_{2}\,\,{\leq}\,\, U(f_{1};{\cal P})\mbox{ for every partition ${\cal P}$ of $[a,b]$}.
        \end{displaymath}
    These facts, when combined with the fact that $B_{1}$ is the unique Darboux number for $f_{1}$ on $[a,b]$, implies that $B-B_{2} \,=\, B_{1}$.
    In other words, there is precisely one Darboux number for $f \,=\, f_{1}+f_{2}$ on $[a,b]$, namely $B \,=\, B_{1}+B_{2}$. The desired result follows.

        The corresponding statement about the function $f \,=\, A_{1}{\cdot}g_{1} + \,{\ldots}\,A_{n}{\cdot}g_{n}$ follows easily using mathematical induction on~$n$.

\V

        (c) This follows easily from the preceding lemma. % EXERCISE?

\V

        (d) The simple proof is left as an exercise. %EXERCISE \Q

\VV

        The next result also uses the integrability of given functions to prove the integrability of a function $f:[a,b] \,{\rightarrow}\, {\RR}$ formed from them.
    In contrast with the preceding theorem, however, it does not provide the precise value of ${\displaystyle \int_{a}^{b} f}$ from the values of the other integrals.

\VV%\\\\

        \subsection{\small{{\bf Theorem}}}
        \label{ThmH25.30}

\V

\hspace*{\parindent}(a) Suppose that $f:[a,b] \,{\rightarrow}\, {\RR}$ can  be expressed as a product $f \,=\, f_{1}{\cdot}f_{2}$,
    where $f_{1}:[a,b] \,{\rightarrow}\, {\RR}$ and $f_{2};[a,] \,{\rightarrow}\, {\RR}$ are integrable on~$[a,b]$. Then $f$ is also integrable on~$[a,b]$.

        More generally, if $f$ can  be expressed as the product $f \,=\, f_{1}{\cdot}f_{2}{\cdot}\,{\ldots}\,{\cdot}f_{k}$
    of finitely many functions which are integrable on~$[a,b]$, then $f$ is also integrable on~$[a,b]$.

\V

        (b) Suppose that $f:[a,b] \,{\rightarrow}\, {\RR}$ can be expressed as a quotient $f \,=\, f_{1}/f_{2}$,
    where $f_{1}:[a,b] \,{\rightarrow}\, {\RR}$ and $f_{2}:[a,b] \,{\rightarrow}\, {\RR}$ are integrable on $[a,b]$.
    Suppose further that there exists a constant $c\,>\,0$ such that $|f_{2}(x)|\,\,{\geq}\,\,c$ for all $x$ in~$[a,b]$. Then $f$ is also integrable on~$[a,b]$.

\V

        (c) Suppose that $f:[a,b] \,{\rightarrow}\, {\RR}$ can be expressed in the form $f \,=\, |g|$ for some function $g:[a,b] \,{\rightarrow}\, {\RR}$ such that $g$ is integrable on~$[a,b]$.
    Then $f$ is also integrable on~$[a,b]$.

\V


        {\bf Proof}\, We shall give direct proofs of these results later, based on an important theoretical characterization of integrability to be developed in the next section.
    Direct simple proofs are also outlined in the exercises at the end of this chapter. \Q %% EXERCISES

\V

        {\bf Remarks} (1) The usual phrasing of Part~(a) of the preceding theorem is this:

\VA

        \h `The product of finitely many functions that are integrable on $[a,b]$ is also integrable on~$[a,]$.'

\VA

\noindent Similar comments about the usual phrasings of Part~(b) of this theorem and Parts~(a) and~(b) of Theorem~\Ref{ThmH25.20}.

        In real-life mathematics, however, one does not normally start with a pair of functions $f_{1}$ and $f_{2}$ and then form their product $f \,=\, f_{1}{\cdot}f_{2}$ (or their sum or quotient).
    Instead, one starts with a function $f$ of interest and seeks out a way to express it in terms of simpler functions using operations such as addition,
    multiplication, division and so on. The phrasings of such results in {\ThisText} attempt to reflect that more realistic usage.

\V

        (2) The converses of various parts of Theorems~\Ref{ThmH25.20} and~\Ref{ThemH25.30} are not true.
    For example, if $f:[a,b] \,{\rightarrow}\, {\RR}$ is already known to be integrable on $[a,b]$, then knowing that $f \,=\, f_{1}+f_{2}$
    or $f \,=\, f_{1}{\cdot}f_{2}$ or $f \,=\, f_{1}/f_{2}$ on $[a,b]$ does {\em not} imply that $f_{1}$ and $f_{2}$ are also integrable on~$[a,b]$.
    Likewise, knowing that $|f|$ is integrable on $[a,b]$ does not imply that $f$ is integrable on~$[a,b]$.
    It is a simple exercise to find appropriate counterexamples. %% EXERCISES

\VV

        There are simple inequalities associated with the Riemann integral which are important for both practical and theoretical discussions.

\V

        \subsection{\small{{\bf Theorem}}}
        \label{ThmH25.40}

\V

        Suppose that $f$ and $g$ are Riemann integrable on the interval $[a,b]$, and that $f(x)\,\,{\leq}\,\,g(x)$ for all $x$ in~$[a,b]$. Then
        \begin{displaymath}
        \int_{a}^{b} f\,\,{\leq}\,\,\int_{a}^{b} g.
        \end{displaymath}
    Furthermore, if in addition both $f$ and $g$ are continuous on~$[a,b]$, then one gets the case of equality in the preceding inequality if, and only if, $f(x) \,=\, g(x)$ for all $x$ in~$[a,b]$.

\V

        {\bf Proof}\,Define $h:[a,b] \,{\rightarrow}\, {\RR}$ by the rule $h(x) \,=\, g(x)-f(x)$ for all $x$ in~$[a,b]$.
    By Parts~(a) and~(b) of Theorem~\Ref{ThmH25.20} it follows that $h$ is also integrable on $[a,b]$ and that ${\displaystyle \int_{a}^{b} h \,=\, \int_{a}^{b} g - \int_{a}^{b} f}$.
    It is clear from the `inequality' hypothesis on $f$ and $g$  that $h(x)\,\,{\geq}\,\,0$ for every $x$ in~$[a,b]$, which implies that $L(h;{\cal P})\,\,{\geq}\,\,0$ for every partition ${\cal P}$ of~$[a,b]$.
    Since ${\displaystyle \int_{a}^{b}} h$ is the supremum of Darboux sums of the form $L(h'{\cal P})$, it follows that $\int_{a}^{b} h\,\,{\geq}\,\,0$,
    which in turn implies that ${\displaystyle \int_{a}^{b} g\,>\,\int_{a}^{b}f}$, as required. The statement about `equality' when $f$ and $g$ are continuous on~$[a,b]$ is left as an exercise. \Q %%EXERCISE



%----------------B
%\StartSkip{

\V


\V

        \section{More on Antiderivatives and the Riemann Integral}
                \label{SectH40}\IndB{ZZ Sections}{\Ref{SectH40} More on Antiderivatives and the Riemann Integral}

\V

        There are important relations between the Riemann integral and the concept of antiderivatives. This was illustrated in Example~\Ref{ExampH20.30},
    where it was it was pointed out that if $f:[a,b] \,{\rightarrow}\, {\RR}$ is continuous on~$[a,b]$, and $F$ is any antiderivative of $f$ on~$[a,]$,
    then $f$ is Riemann integrable on~$[a,b]$ and ${\displaystyle \int_{a}^{b} f \,=\, F(b)-F(a)}$. In this section we explore these relations more completely.

        The first step is to introduce a slight extension of the integral notation. More precisely, up to now the expression ${\displaystyle \int_{a}^{b} f}$ assumes that $[a,b]$ is an interval in~${\RR}$;
    in particular, it requires that the number at the bottom of the integral sign be strictly smaller than the one at the top.
    However, even in elementary calculus one finds it useful to allow $a \,=\, b$ or $a\,>\,b$ in such expressions.
    The motivation for the standard choice for this extension is the observation that the right side of the preceding equation {\em does} make sense independent of the relation between $a$ and~$b$.
    More precisely, one has
        \begin{displaymath}
        F(a) - F(b) \,=\, -(F(b) - F(a)) \mbox{ and } F(c) - F(c)\,=\,0 \mbox{ for every number $c$ in~$[a,b]$}.
        \end{displaymath}
    Thus it makes sense to extend the meaning of the integral as follows: If $f$ is integrable on an interval $[a,b]$, then
        \begin{displaymath}
        \int_{b}^{a} f \,=\, -\left(\int_{a}^{b} f\right) \mbox{ and } \int{c}^{c} f\,=\,0 \mbox{ for every $c$ in $[a,b]$}.
        \end{displaymath}
    In {\ThisText} we refer to this as the {\bf extended integral notation}IndB{integrals}{extended integral notation}.

\V

        Using this extended notation, one can reformulate Part~(c) of Theorem~\Ref{ThmH25.20} as follows:


        \subsection{\small{{\bf Theorem}}}
        \label{ThmH40.15}

        Suppose that $f:I \,{\rightarrow}\, {\RR}$ is Riemann integrable on a closed bounded interval~$I$.
    Then for each triple of numbers $a$, $b$ and $c$ in the interval $I$, regardless of the relative order of these numbers, one has
        \begin{displaymath}
        \int_{a}^{b} f \,=\, \int_{a}^{c} f + \int_{c}^{b};
    \mbox{ equivalently, }
        \int_{a}^{c} f +  + \int_{b}^{a} f \,=\, 0.
        \end{displaymath}

        The simple proof is left as an exercise. % EXERCISE

\VV

        {\bf Example} Suppose that $f:[a,b] \,{\rightarrow}\, {\RR}$ is continuous, and that $u,v:J \,{\rightarrow}\, [a,b]$ are differentiable functions on an open interval $J$ with values in~$[a,b]$.
    Define $G:J \,{\rightarrow}\, {\RR}$ by the rule
        \begin{displaymath}
        G(t) \,=\, \int_{u(t)}^{v(t)} f \mbox{ for all $t$ in $J$};
        \end{displaymath}
    in light of the extended integral notation, this makes sense no matter the relative order of the numbers $u(t)$ and~$v(t)$.
     It is easy to see that $G$ is differentiable on $J$, and
        \begin{displaymath}
        G'(t) \,=\, f(v(t)){\cdot}v'(t) - f(u(t)){\cdot}u'(t) \mbox{ for all $t$ in $J$}.
        \end{displaymath}
    Indeed, the continuity hypothesis implies that $f$ has an antiderivative $F$ on $[a,b]$, and the Cauchy Fundamental Theorem of Calculus implies that
        \begin{displaymath}
        G(t) \,=\, F(v(t)) - F(u(t)).
        \end{displaymath}
    The desired result follows from the Chain Rule for Derivatives.


\VV

        The proof of Cauchy's Antiderivative Theorem given in the preceding chapter differs from Cauchy's original proof.
    The next result carries out his approach in a slightly more general manner.

\V

        \subsection{\small{{\bf Theorem}}}%\\\\
        \label{ThmH40.20}

\V

        Suppose that $f:[a,b] \,{\rightarrow}\, {\RR}$ is Riemann integrable on an interval~$[a,b]$.
    Let $c$ be any number in $[a,b]$, and define a new function $F:[a,b] \,{\rightarrow}\, {\RR}$ by the rule
        \begin{displaymath}
        F(x) \,=\, \int_{c}^{x} f \mbox{ for each $x$ in $[a,b]$};
        \end{displaymath}
    note that once again the extended integral notation is in use. Then:

\V

        (a) The function $F$ is continuous on $[a,b]$.

\V

        (b) If $x_{0}$ is a point of $[a,b]$ at which $f$ is continuous, then $F$ is differentiable at $x_{0}$. and $F'(x_{0}) \,=\, f(x_{0})$.
    (If $x_{0}$ is one of the endpoints $a$ or $b$, then the appropriate one-sided notions of continuity and differentiability are understood to apply.)


\V

        {\bf Proof}%\\\

\V

        Note that if $x_{1}$ and $x_{2}$ are in $[a,b]$ with $x_{1}\,\,{\leq}\,\,x_{2}$, then one has
        \begin{displaymath}
        F(x_{2}) - F(x_{1}) \,=\, \int_{a}^{x_{2}} f - \int_{a}^{x_{1}} f \,=\, \int_{a}^{x_{2}} + \int_{x_{1}}^{a} f
    \,=\, \int_{x_{1}}^{x_{2}} f;
        \end{displaymath}
    see Remark~\Ref{RemrkH30.35}.
    If $M$ is an upper bound for $|f|$ on $[a,b]$, it then follows that
        \begin{displaymath}
        |F(x_{2}) - F(x_{1})|\,\,{\leq}\,\,M\left({\alpha}(x_{2})-{\alpha}(x_{1})\right).
        \end{displaymath}
    Parts (a) and (b) now follow easily.

     To get a more precise estimate, suppose that $a\,<\,x_{1}\,<\,x_{2}\,<\,b$, and let
    $m(x_{2}) \,=\, {\inf}\,\{f(x): x_{1}\,\,{\leq}\,\,x\,\,{\leq}\,\,x_{2}\}$ 
    and
    $M(x_{2}) \,=\, {\sup}\,\{f(x): x_{1}\,\,{\leq}\,\,x\,\,{\leq}\,\,x_{2}\}$.
    Then one has
        \begin{displaymath}
        m(x_{2})({\alpha}(x_{2}) - {\alpha}(x_{1}))\,\,{\leq}\,\,F(x_{2}) - F(x_{1})\,\,{\leq}\,\,M(x_{2})({\alpha}(x_{2})-{\alpha}(x_{1})).
        \end{displaymath}
    Divide all the terms in this inequality by the positive number $x_{2}-x_{1}$ to get
        \begin{displaymath}
        m(x_{2})\left(\frac{{\alpha}(x_{2}) - {\alpha}(x_{1})}{x_{2}-x_{1}}\right)
    \,\,{\leq}\,\,
        \frac{F(x_{2}) - F(x_{1})}{x_{2} - x_{1}} \,\,{\leq}\,\,
        M(x_{2})\left(\frac{{\alpha}(x_{2}) - {\alpha}(x_{1})}{x_{2} - x_{1}}\right)
        \end{displaymath}
    The hypothesis that $f$ is continuous at $x_{1}$, which implies that $\lim_{x_{2} \,{\rightarrow}\, x_{1}} m(x_{2}) \,=\,
    \lim_{x_{2} \,{\rightarrow}\, x_{1}} M(x_{2}) \,=\, f(x_{1})$, combined with the hypothesis that ${\alpha}$ is differentiable at $x_{1}$,
    implies (by the `Squeeze Theorem') that
        \begin{displaymath}
        \lim_{x_{2}{\searrow}x_{1}} \frac{F(x_{2}) - F(x_{1})}{x_{2}-x_{1}} \,=\, f(x_{1}){\alpha}'(x_{1}).
        \end{displaymath}
    A similar argument shows that the left-hand derivative of $F$ at $x_{1}$ exists and equals $f(x_{1}){\alpha}'(x_{1})$.
    The desired result follows.

 

\V

         \subsection{\small{{\bf Corollary}}}
        \label{CorH40.30}

\V

        Suppose that $f:[a,b] \,{\rightarrow}\, {\RR}$ is Riemann integrable on $[a,b]$.
    Let $F:[a,b] \,{\rightarrow}\, {\RR}$ be defined by $F(x) \,=\, \int_{a}^{x} f(t)\,dt$ for each $x$ in $[a,b]$.
    Then $F$ is continuous on $[a,b]$, and if $f$ is continuous at $x$, then $F'(x)$ exists and one has $F'(x) \,=\, f(x)$.

\V

        {\bf Proof} This follows immediately from the previous theorem in the case ${\alpha}(x) \,=\, x$ for all~$x$.

\V

        \underline{Remark} When $f$ is continuous on $[a,b]$, the preceding corollary reduces essentially to Theorem~\Ref{ThmE45.125B},
    `Cauchy's Antiderivative Theorem'.

\V
\V

        In the preceding one uses the definite integral to construct an antiderivative of a continuous function.
    The next result reverses the relationship between antiderivatives and integrals.

\V

        \subsection{\small{{\bf Theorem}}}
        \label{ThmH40.40}

\V

        Suppose that $f:[a,b] \,{\rightarrow}\, {\RR}$ is Riemann integrable on $[a,b]$, and suppose that there exists continuous $F:[a,b] \,{\rightarrow}\, {\RR}$ such that $F'(x) \,=\, f(x)$ for all $x$ in $[a,b]$.
    Then ${\displaystyle \int_{a}^{b} f(x)\,dx \,=\, F(b) - F(a)}$.

\V

        {\bf Proof} Let ${\cal P} \,=\, \{a \,=\, x_{0}\,<\,x_{1}\,<\,\,{\ldots}\,\,<\,x_{k} \,=\, b\}$ be a partition of $[a,b]$,
    and consider the telescoping sum
        \begin{displaymath}
        F(b) - F(a) \,=\, \sum_{j=1}^{k} \left(F(x_{j}) - F(x_{j-1})\right).
        \end{displaymath}
    By the Mean-Value Theorem, for each $j$ there exists $t_{j}$ in $(x_{j-1},x_{j})$ such that
        \begin{displaymath}
        F(x_{j}) - F(x_{j-1}) \,=\, F'(t_{j}){\Delta}x_{j} \,=\, f(t_{j}){\Delta}x_{j}.
        \end{displaymath}
     It follows that $F(b) - F(a)$ equals the Riemann sum $\sum_{j=1}^{k} f(t_{j}){\Delta}x_{j}$, and thus one has
        \begin{displaymath}
        L({\cal P},f)\,\,{\leq}\,\,F(b) - F(a)\,\,{\leq}\,\,U({\cal P},f).
        \end{displaymath}
    However, by hypothesis $f$ is Riemann integrable on $[a,b]$, so there exists only one number $A$ such that
    $L({\cal P},f)\,\,{\leq}\,\,A\,\,{\leq}\,\,U({\cal P},f)$, namely $A \,=\, \int_{a}^{b} f(x)\,dx$. The desired result now follows.

\V
\V

        {\bf Remark} Some authors refer to Theorem~\Ref{ThmH40.20} (or Corollary~\Ref{CorH40.30}) as the `First Fundamental Theorem of Calculus',
    and to Theorem~\Ref{ThmH40.40} as the `Second Fundamental Theorem of Calculus'.
    Other authors number these results in the opposite way. And yet other authors combine the two as individual parts of a single `Fundamental Theorem'.
    In {\ThisText} we shall refer to each as `The Fundamental Theorem of Calculus',
    and let the context make it clear to the reader which version is being used, or simply give the number of the theorem.

\V

        \subsection{\small{{\bf Corollary}}}
        \label{CorH40.50}

\V

        Suppose that $f:[a,b] \,{\rightarrow}\, {\RR}$ is continuous and ${\alpha}:[a,b] \,{\rightarrow}\, {\RR}$ is monotonic up.
    Assume further that ${\alpha}'$ is defined and Riemann integrable on $[a,b]$.
    Then
        \begin{equation}
        \label{EqnH.30}
        \int_{a}^{b} f\,d{\alpha} \,=\, \int_{a}^{b} f(x){\alpha}'(x)\,dx.
        \end{equation}


\V

        {\bf Proof} Since, by hypothesis, $f$ is continuous on $[a,b]$ and ${\alpha}'{\in}{\cal R}_{[a,b]}$,
    it follows from Theorem~\Ref{ThmH20.120} that $f{\in}{\cal R}_{[a,b]}({\alpha})$,
    and from Theorem~\Ref{ThmH30.20} that $f{\cdot}{\alpha}'{\in}{\cal R}_{[a,b]}$.
    Furthermore, it follows from Theorem~\Ref{ThmH40.20} that the function $F:[a,b] \,{\rightarrow}\, {\RR}$ given by the rule
    ${\displaystyle F(x) \,=\, \int_{a}^{x} f\,d{\alpha}}$ is continuous on $[a,b]$ and differentiable on $(a,b)$, and that $F'(x) \,=\, f(x){\alpha}'(x)$.
    Then Theorem~\Ref{ThmH40.40} allows one to conclude that $F(b) - F(a) \,=\, {\displaystyle \int_{a}^{b} f(x){\alpha}'(x)\,dx}$.
    The desired result now follows by noting that ${\displaystyle F(b)-F(a) \,=\, \int_{a}^{b} f\,d{\alpha}}$.
    
\V

        {\bf Remark} With a little more work one can prove the following more general result.

        \subsection{\small{{\bf Theorem}}}
        \label{ThmH40.60}

\V

                Suppose that $f:[a,b] \,{\rightarrow}\, {\RR}$ is in ${\cal R}_{[a,b]}({\alpha})$ and ${\alpha}:[a,b] \,{\rightarrow}\, {\RR}$ is monotonic up.
    Assume further that ${\alpha}'$ is defined and Riemann integrable on $[a,b]$. Then $f{\alpha}'$ is Riemann integrable on $[a,b]$, and Equation~\Ref{EqnH.30} holds.

\V

        See Page~131 of Rudin's `Principles of Mathematical Analysis' (3rd edition) for a proof.

\V
\V

    
        \subsection{\small{{\bf Remark}}}
        \label{RemrkH40.70}

\V

        The left side of Equation~\Ref{EqnH.30} involves the Riemann-Stieltjes integral ${\displaystyle \int_{a}^{b}f\,d{\alpha}}$,
    in which the integrator ${\alpha}$ is assumed to be monotonic up on $[a,b]$. This restriction on ${\alpha}$ was imposed by the initial motivation of the integral in terms of weighted averages;
    but it also facilitated the technical development of the theory. For instance,
    the Darboux approach makes repeated use of the fact that ${\Delta}{\alpha}_{j}\,\,{\geq}\,\,0$.

        The restriction of ${\alpha}$ to be monotonic up implies that the factor ${\alpha}'$,
    which appears in the Riemann integral ${\displaystyle \int_{a}^{b} f(x){\alpha}'(x)\,dx}$ on the right side of Equation~\Ref{EqnH.30},
    must be nonnegative. However, the integral itself makes perfectly good sense even if ${\alpha}'$ changes sign in the interval $[a,b]$.
    This suggests that it might be useful to extend the concept of the Riemann-Stieltjes integral to allow
    integrators ${\alpha}$ which are not of fixed monotonicity throughout the interval $[a,b]$.
    Such extensions have been developed -- indeed, it appears that even Stieltjes allowed such extentions.
    An excellent source for such an extended treatement of the Riemann-Stieltjes integral can be found in Apostol's `Mathematical Analysis (2nd edition)'.
    Instead of defining integrabilty in terms of Riemann's Condition, which uses the hypothesis that ${\Delta}{\alpha}_{j}\,\,{\geq}\,\,0$,
    Apostol uses the Riemann sum approach; see Theorem~\Ref{ThmH20.100}.
    This approach allows quite general integrators. However, one soon restricts to the case in which ${\alpha}$ is of bounded variation on $[a,b]$,
    since in this case the basic theorem, that every continuous function on $[a,b]$ is in ${\cal R}_{[a,b]}({\alpha})$, remains valid.

\V
\V


                \section{{\bf Miscellaneous Results on the Riemann Integral}}
                \label{SectH50}\IndB{ZZ Sections}{\Ref{SectH50} Miscellaneous Results on the Riemann Integral}

\V

        There are numerous results for the Riemann integral which are worth singling out.

\V
\V

        \subsection{\small{{\bf Theorem} (Integration-by-Parts)}}
        \label{ThmH50.20}

\V

        Suppose that $f$ and $g$ are differentiable on $[a,b]$ and that their derivatives are Riemann integrable on $[a,b]$.
    Then $fg'$ and $f'g$ are both integrable on $[a,b]$, and one has
        \begin{equation}
        \label{EqnH.40}
        \int_{a}^{b} f(x)g'(x)\,dx \,=\, f(b)g(b) - f(a)g(a) - \int_{a}^{b} f'(x)g(x)\,dx.
        \end{equation}

\V

        {\bf Proof} Since $f$ and $g$ are differentiable on $[a,b]$, they are certainly continuous, hence Riemann integrable, on $[a,b]$;
    and since, by hypothesis, $f'$  and $g'$ are Riemann integrable on $[a,b]$, it follows that the products $f'g$ and $fg'$ are also in ${\cal R}_{[a,b]}$, and thus so is their sum $f'g+fg'$.
    However, by the Product Rule for Derivatives, this last function is the derivative on $[a,b]$ of the function $H \,=\, fg$.
    Now Theorem~\Ref{ThmH40.40} implies that
        \begin{displaymath}
        H(b) - H(a) \,=\, \int_{a}^{b} f'(x)g(x)\,dx + \int_{a}^{b} f(x)g'(x)\,dx
        \end{displaymath}
    Transpose the term ${\displaystyle \int_{a}^{b} f'(x)g(x)\,dx}$ in this last equation, and note that $H(b)-H(a) \,=\, f(b)g(b) - f(a)g(a)$, to get the desired Equation~\Ref{EqnH.40}.

%Other topics

\V
\V

        \subsection{\small{{\bf Theorem} (First Mean-Value Theorem for Riemann Integrals)}}
        \label{ThmH50.30}

\V

        (a) Suppose that $f{\in}{\cal R}_{[a,b]}$, and let $m^{{\ast}} \,=\, {\inf}\,\{f(x): x{\in}[a,b]\}$ and $M^{{\ast}} \,=\, {\sup}\,\{f(x): x{\in}[a,b]\}$.
    Suppose that $g:[a,b] \,{\rightarrow}\, {\RR}$ is a nonnegative Riemann integrable function on $[a,b]$.
    Then there exists a number ${\mu}$, with $m^{{\ast}}\,\,{\leq}\,\,{\mu}\,\,{\leq}\,\,M^{{\ast}}$,
    such that
        \begin{displaymath}
        \int_{a}^{b} f(x)g(x)\,dx \,=\, {\mu}\int_{a}^{b} g(x)\,dx
        \end{displaymath}
    If, in addition, $f$ is continuous on $[a,b]$, then ${\mu}$ can be chosen to be of the form $f(c)$ for some $c$ in $[a,b]$.

\V


        (b) If the function $f$ in Part~(a) is continuous on $[a,b]$, and if $g(x) \,=\, 1$ for all $x$ in $[a,b]$, 
    then one has ${\displaystyle \frac{1}{b-a}\int_{a}^{b} f(x)\,dx \,=\, f(c)}$ for some number $c$ in $[a,b]$.

\V

        {\bf Proof} (a) Because $g$ is nonnegative, it is clear that $m^{{\ast}}g(x)\,\,{\leq}\,\,f(x)g(x)\,\,{\leq}\,\,M^{{\ast}}g(x)$ for all $x$ in $[a,b]$.
    Then from Part~(c) of Theorem~\Ref{ThmH30.20} one sees that
        \begin{displaymath}
        m^{{\ast}}\int_{a}^{b} g(x)\,dx\,\,{\leq}\,\,\int_{a}^{b} f(x)g(x)\,dx\,\,{\leq}\,\, M^{{\ast}}\int_{a}^{b} g(x)\,dx.
        \end{displaymath}
    If the integral ${\displaystyle \int_{a}^{b} g(x)\,dx}$, which appears on either end of this string of inequalities, equals~$0$, then clearly one also has ${\displaystyle \int_{a}^{b} f(x)g(x)\,dx \,=\, 0}$, so ${\mu}$ can be any number such that $m^{{\ast}}\,\,{\leq}\,\,{\mu}\,\,{\leq}\,\,M^{{\ast}}$.
    If, instead, that integral is {\em not} zero, then choose ${\mu}$ by the rule
        \begin{displaymath}
        {\mu} \,=\, \frac{{\displaystyle \int_{a}^{b} f(x)g(x)\,dx}}{{\displaystyle \int_{a}^{b} g(x)\,dx}}.
        \end{displaymath}
    It is clear that $m^{{\ast}}\,\,{\leq}\,\,{\mu}\,\,{\leq}\,\,M^{{\ast}}$.

        If $f$ is also continuous on $[a,b]$, then the Intermediate-Value Theorem for Continuous Functions implies that ${\mu} \,=\, f(c)$ for some $c$ in $[a,b]$.

\V

        (b) This follows easily from Part~(a).

\V
\V


        \subsection{\small{{\bf Theorem} (Second Mean-Value Theorem for Riemann Integrals)}}
        \label{ThmH50.40}

\V

        Suppose that $f:[a,b] \,{\rightarrow}\, {\RR}$ is a monotonic-up function that is differentiable on $[a,b]$ and for which $f'{\in}{\cal R}_{[a,b]}$, and that $g$ is continuous on $[a,b]$.
    Then there exists $c$ in $[a,b]$ such that
        \begin{displaymath}
        \int_{a}^{b} f(x)g(x)\,dx \,=\, f(a)\int_{a}^{c} g(x)\,dx + f(b)\int_{c}^{b} g(x)\,dx
        \end{displaymath}

\V

        {\bf Proof} Define $G:[a,b] \,{\rightarrow}\, {\RR}$ by the rule $G(x) \,=\, {\displaystyle \int_{a}^{x} g(t)\,dt}$.
    Note that $G$ is differentiable because of the hypothesis that $g$ is continuous, and of course $G' \,=\, g$ is Riemann integrable on $[a,b]$.
    Then Theorem~\Ref{ThmH50.20}, `Integration-by-Parts', can be apllied to the functions $f$ and $G$ to yield
        \begin{displaymath}
        \int_{a}^{b} f(x)g(x)\,dx \,=\, \int_{a}^{b} f(x)G'(x)\,dx \,=\, 
        \left(f(b)G(b) - f(a)G(a)\right) - \int_{a}^{b} G(x)f'(x)\,dx \h ({\ast})
        \end{displaymath}
    Now apply Part~(a) of the preceding theorem, with $G$ and $f'$ here playing the roles of $f$ and $g$, respectively, in that earlier theorem.
    Then, since $G$ is certainly continuous on $[a,b]$, one gets
        \begin{displaymath}
        \int_{a}^{b} G(x)f'(x)\,dx \,=\, G(c)\int_{a}^{b} f'(x)\,dx \,=\, G(c)(f(b)-f(a)) \,=\, \left(\int_{a}^{c} g(x)\,dx\right)(f(b)-f(a))
        \end{displaymath}
    for some $c$ in $[a,b]$. Combine this last result with Equation~$({\ast})$ to get
        \begin{displaymath}
        \int_{a}^{b} f(x)g(x)\,dx\,=\, 
        \left(f(b)G(b) - f(a)G(a)\right) - G(c)(f(b)-f(a)) \h ({\ast}{\ast})
        \end{displaymath}
    By the definition of $G$ one has $G(a) \,=\, 0$, $G(b) \,=\, {\displaystyle \int_{a}^{b} g(x)\,dx}$, and ${\displaystyle G(c) \,=\, \int_{a}^{c} g(x)\,dx}$.
    After doing the obvious simplification, including noting that
        \begin{displaymath}
         f(b)G(b) - f(b)G(b) \,=\, f(b)\left(\int_{a}^{b} g(x)\,dx - \int_{a}^{c} g(x)\,dx\right) \,=\, f(b)\int_{c}^{b} g(x)\,dx,
        \end{displaymath}
    the desired result follows.

\V
\V

        \subsection{\small{{\bf Theorem} (Change-of-Variables Theorem for Riemann Integrals)}}
        \label{ThmH50.50}

\V

        Suppose that $f:[a,b] \,{\rightarrow}\, {\RR}$ is continuous on an interval $[a,b]$,
    and that $g:[c,d] \,{\rightarrow}\, {\RR}$ has continuous first derivative on an interval $[c,d]$.
    Suppose further that $a\,\,{\leq}\,\,g(t)\,\,{\leq}\,\,b$ for all $t$ in $[c,d]$.
    Then
        \begin{equation}
        \label{EqnH.50}
        \int_{g(c)}^{g(d)} f(x)\,dx \,=\, \int_{c}^{d} f((g(t)))g'(t)\,dt
        \end{equation}

\V

        {\bf Proof} Let $F$ be an antiderivative of $f$ on $[a,b]$; such $F$ exists because $f$ is continuous.
    Likewise, let $H \,=\, F{\circ}g:[c,d] \,{\rightarrow}\, {\RR}$; the composition makes sense because $g$ maps $[c,d]$ to a subset of $[a,b]$.
    Then $H$ is the composition of differentiable functions, so the Chain Rule can be used to say that $H$ is differentiable on $[c,d]$, and that $H'(t) \,=\, F'(g(t)){\cdot}g'(t) \,=\, f(g(t)){\cdot}g'(t)$ for all $t$ in $[c,d]$.
    Since $f$, $g$ and $g'$ are all continuous, the function $H' \,=\, (f{\circ}g){\cdot}g'$ is certainly continuous on $[c,d]$.
    It now follows from Theorem~\Ref{ThmH40.40} -- applied to $(f{\circ}g){\cdot}g'$ that
        \begin{displaymath}
        \int_{c}^{d} (f{\circ}g){\cdot}g' \,=\, H(d) - H(c) \,=\, F(g(d)) - F(g(c)).
        \end{displaymath}
    However, since $f$ is continuous on $[a,b]$, one can use Theorem~\Ref{ThmH40.40} again, but this time on $f$, to say that
        \begin{displaymath}
        \int_{u}^{v} f(x)\,dx \,=\, F(u) - F(v) \mbox{ for all $u$, $v$ in $[a,b]$}.
        \end{displaymath}
    In particular, one has $F(g(d)) - F(g(c)) \,=\, {\displaystyle \int_{g(c)}^{g(d)} f(x)\,dx}$.
    The desired result now follows.

\V

        {\bf Remarks}

\V

        (1) Some analysis texts impose the requirement that the function $g$ be strictly increasing on $[a,d]$, so that $g$ is a bijection of $[c,d]$ onto $[a,b]$.
    Clearly this restriction is not needed; however, it does make the name `change of variables' seem more appropriate.

\V

        (2) Note that there is no restriction that $g(c)\,<\,g(d)$. Indeed, we even allow the possibility that $g(c) \,=\, g(d)$.
    If this occurs, Equation~\Ref{EqnH.50} takes the particularly simple form
        \begin{displaymath}
        \int_{g(c)}^{g(c)} f(x)\,dx \,=\, \int_{c}^{d} f((g(t)))g'(t)\,dt; 
        \end{displaymath}
    that is,
        \begin{displaymath}
        \int_{c}^{d} f((g(t)))g'(t)\,dt \,=\, 0.
        \end{displaymath}
    Note that this holds independently of the choice of function~$f$.
    For instance, suppose that $f(x) \,=\, e^{-x^{2}}$ and $g(t) \,=\, {\sin}\,t$ for $0\,\,{\leq}\,\,t\,\,{\leq}\,\,2{\pi}$.
    Then, when read backwards, Equation~\Ref{EqnH.50} takes the form
        \begin{displaymath}
        \int_{0}^{2{\pi}} e^{-{\sin}^{2}\,t}\,{\cos}\,t\,dt \,=\, \int_{0}^{0} e^{-x^{2}}\,dx.
        \end{displaymath}
    One cannot write down an antiderivative for the integrand $f(x) \,=\, e^{-x^{2}}$ which appears on the right side of this last equation,
    but one can still see that the value of the integral on the right -- and thus the value of the integral on the left -- must equal~$0$.

\V
\V

%}%\EndSkip
%----------------------B

\VV


        \section{{\bf Existence Theorems for the Riemann Integral}}
        \label{SectH27}\IndB{ZZ Sections}{\Ref{SectH27} Existence Theorems for the Riemann Integral}

\V


        In his original treatment of the integral, Riemann carries out a more detailed analysis of the quantity ${\Delta}(f;{\cal P})$ in Statement~(iii); see [RIEMANN~$1854]$.
    Several decades later Henri Lebesgue greatly improved this analysis; see [LEBESGUE~$1901$]. The approach followed here,
    however, focuses on a somewhat simpler approach arising from the work of Camille Jordan; see [JORDAN~????]. The next definition clarifies the ideas involved.

\V

        \subsection{\small{{\bf Definition}}}
        \label{DefH20.45A}

\V

        Let $f:X \,{\rightarrow}\, {\RR}$ be a function such that $f$ bounded on the (nonempty) set~$X$.
    Then the {\bf oscillation of $\Bfm{f}$ over the set $\Bfm{X}$}\IndB{functions}{oscillation over a set} is the number ${\Omega}(f;X)$ given by the formula
        \begin{displaymath}
        {\Omega}\,(f;X) \,=\, {\sup}\,\{|f(x_{2})-f(x_{1})|: x_{1},x_{2}{\in}X\}.
        \end{displaymath}
    As usual, the `function diagram' notation $f:X \,{\rightarrow}\, {\RR}$ allows the possibility that the full domain of $f$ includes points outside the set~$X$.

\V

        \subsection{\small{{\bf Remarks}}}
        \label{RemrkH20.50}

\V

        (1) Some authors define ${\Omega}(f;X)$ by the formula
        \begin{displaymath}
        {\Omega}\,(f;X) \,=\, {\sup}\,\{f(x_{2})-f(x_{1}): x_{1},x_{2}{\in}X\};
        \end{displaymath}
    that is, they omit absolute-value signs found in the definition above. Likewise, some authors define this expression by
        \begin{displaymath}
        {\Omega}\,(f;X) \,=\, {\sup}\,\{f(x): x{\in}X\} - {\inf}\,\{f(x): x{\in}X\}.
        \end{displaymath}
    It is a simple exercise to show that these definitions all yield the same value for ${\Omega}(f;X)$. In {\ThisText} we freely use all these variations.
    It is also easy to show that if $Y$ is a nonempty subset of $X$, then $0\,\,{\leq}\,\,{\Omega}(f;Y)\,\,{\leq}\,\,{\Omega}(f;X)$. %% EXERCISES

\V

        (2) It is clear that the quantities $m_{j}$ and $M_{j}$ that appear in Definition~\Ref{DefH20.20} are related by the rule
        \begin{displaymath}
        M_{j} - m_{j} \,=\, {\Omega}\,(f;[x_{j-1},x_{j}]).
        \end{displaymath} 

\V

        (3) The symbol ${\Omega}$ is the upper-case version of the Greek letter `omega'; the lower-case version of this letter is~${\omega}$.
    This Greek letter corresponds closely to the English letter `O'; in the present context it is used to remind one of the first letter of the word `oscillation'.

\VV

        There is a close relationship between the concept of `oscillation over a set' and `continuity'. For simplicity, we consider here only sets which are closed intervals.

        Thus, suppose, as usual, that $f:[a,b] \,{\rightarrow}\, {\RR}$ is bounded on~$[a,b]$, and let $c$ be a point of~$[a,b]$.
    It follows easily from Theorem~\Ref{ThmC30.20} that a necessary and sufficient condition for $f$ to \underline{not} be continuous at $c$ is this:

\VA

        \h `There exists a sequence $(x_{1},x_{2},\,{\ldots}\,x_{n},\,{\ldots}\,)$ in $[a,b]$, converging to~$c$,
    such that the corresponding sequence $(f(x_{1}),f(x_{2}),\,{\ldots}\,f(x_{n}),\,{\ldots}\,)$ of values converges to some number $L \,\,{\neq}\,\, f(c)$.'
    (The possibility $L \,=\,  \,{\pm}\, {\infty}$ is excluded by the boundedness hypothesis on~$f$.)

\VA

    Of course if $c$ equals one of the endpoints $a$ or $b$, then `continuity' in the present context means the appropriate one-sided continuity.

\V

        It follows that a reasonable indicator of the degree to which $f$ fails to be continuous at $c$ is the width of the set $S_{(f;c)}$
    consisting of all numbers $L$ which can be expressed as such a limit. Note that this set is bounded, because the function $f$~is, and it is nonempty, because it contains the number~$f(c)$.
    A commonly used measure of the size of $S_{(f;c)}$ is the number ${\omega}_{f}(c) \,=\, {\sup}\,S_{(f;c)} - {\inf}\,S_{(f;c)}$, called the {\bf oscillation of $\Bfm{f}$ at~$c$};\IndB{continuity}{oscillation at a point}
    this is often abbreviated to ${\omega}(c)$ if the function $f:[a,b] \,{\rightarrow}\, {\RR}$ is understood from the context.

        It is clear that for each $c$ in $[a,b]$ one has ${\omega}(c)\,\,{\geq}\,\,0$, and that ${\omega}(c) \,=\, 0$ if, and only if, $f$ is continuous at~$c$.
    Otherwise stated, the set $D_{(f;[a,b])}$ of all the discontinuities of $f$ in $[a,b]$ is precisely the set of $c$ such that ${\omega}(c)\,>\,0$.
    For each $k$ in ${\NN}$ let $D_{k}$ denote the set of points $c$ in $[a,b]$ such that ${\omega}(c)\,\,{\geq}\,\,1/k$.

\V

        \subsection{\small{{\bf Definition}}}
        \label{DefH20.55}

\V

        Let $S$ be a subset of a closed interval $[a,b]$ in ${\RR}$. One says that $S$ is a {\bf null set in the sense of Jordan},
    or, briefly, the set $S$ is {\bf Jordan null}, provided that the following condition holds:\IndB{null sets}{Jordan-null sets}


        \h For every ${\varepsilon}\,>\,0$ there exists a partition ${\cal P} \,=\, \{a \,=\, x_{0}\,<\,x_{1}\,<\,\,{\ldots}\,\,<\,x_{n-1}\,<\,x_{n} \,=\, b\}$
    of $[a,b]$ such that the sum ${\Sigma}(S;{\cal P})\,<\,{\varepsilon}$, where ${\Sigma}(S;{\cal P})$ denotes the sum of the lengths of those subintervals
    $[x_{j-1},x_{j}]$ of ${\cal P}$ having nonempty intersection with $S$. If $S \,=\, {\emptyset}$, this sum is defined to equal~$0$.

\VA

        \subsection{\small{{\bf Remark}}}
        \label{RemrkH20.55A}

\V

\hspace*{\parindent} The specific choice of interval $[a,b]$ containing a nonempty set $S$ as a subset does not affect whether $S$ is Jordan~null.
    In particular, there is no loss in generality by assuming that $a\,<\,{\inf}\,S$ and $b\,>\,{\sup}\,S$. In what follows it is often convenient to make that assumption.
        
\V

        \subsection{\small{{\bf Lemma}}}
        \label{LemmaH20.55B}

\V

        In this Lemma $[a,b]$ is an interval in~${\RR}$.

\V

        (a) Suppose that $S$ is a subset of $[a,b]$.

\VA

      \h  (i)\,\, If ${\cal P}$ and ${\cal Q}$ are partitions of $[a,b]$ such that 
${\cal Q}$ is a refinement of~${\cal P}$, then ${\Sigma}(S;{\cal Q})\,\,{\leq}\,\,{\Sigma}(S;{\cal P})$.

\VA

      \h  (ii)\, If $T$ is also a subset of $[a,b]$, then ${\Sigma}(S\,{\cup}\,T;{\cal R})\,\,{\leq}\,\,{\Sigma}(S;{\cal R}) + {\Sigma}(T;{\cal R})$ for every partition ${\cal R}$ of~$[a,b]$.

\VA

       \h (iii) If $T$ is a subset of $[a,b]$ such that $S \,{\subseteq}\, T$, then ${\Sigma}(S;{\cal R})\,\,{\leq}\,\,{\Sigma}(T;{\cal R})$ for every partition ${\cal R}$ of $[a,b]$.

\V


        (b) Assume now that $S$ is a nonempty subset of $[a,b]$ such that $a\,<\,{\inf}\,S$ and $b\,>\,{\sup}\,S$, as in Remark~\Ref{RemrkH20.55A} above.

\VA


    \h (i)\, Let ${\cal Q}$ be any partition of~$[a,b]$. Then there exists a partition~${\cal P} \,=\, \{a \,=\, x_{0}\,<\,\,{\ldots}\,\,<\,x_{n} \,=\, b\}$,
    satisfying ${\Sigma}(S;{\cal P}) \,=\, {\Sigma}(S;{\cal Q})$, such that if a subinterval $[x_{k-1},x_{k}]$ of ${\cal P}$ has nonempty intersection with~$S$,
    then each subinterval of ${\cal P}$ adjacent to $[x_{k-1},x_{k}]$ is disjoint from~$S$. In particular,
    for such ${\cal P}$ none of the partition points $x_{j}$, $j \,=\, 0,1,\,{\ldots}\,n$ is in~$S$.

\VA

        \h (ii) Let $U \,=\, (J_{1}, J_{2},\,{\ldots}\,J_{N})$ be any finite ordered list of open subintervals of $[a,b]$ such that $S \,{\subseteq}\, {\bigcup}_{i=1}^{N} J_{i}$;
    that is, the intervals in the list $U$ form an open cover of~$S$. Write $J_{i} \,=\, (c_{i},d_{i})$ for each $i \,=\, 1,2,\,{\ldots}\,N$.
    Let ${\cal P} \,=\, \{a, c_{1}, d_{1},\,{\ldots}\,c_{N}, d_{N}, b\}$ be the partition of $[a,b]$ formed from the endpoints of these open intervals, together with the numbers $a$ and~$b$.
    (There is no assumption that these endpoints are in any particular order or even that they are distinct; recall that a partition of ${\cal P}$ is simply a finite subset of ${\cal P}$ containing both $a$ and~$b$.)
    Then
        \begin{displaymath}
        {\Sigma}(S;{\cal P})\,\,{\leq}\,\,\sum_{i=1}^{N} (d_{i}-c_{i}).
        \end{displaymath}

\V

        {\bf Remark}\, The conclusions are obviously true if $S \,=\, {\emptyset}$.

\V

        {\bf Proof}\, (a) The simple proof is left as an exercise. %EXERCISE

\V

        (b) (i)\,Among all the partitions ${\cal R}$ for which ${\Sigma}(S;{\cal R}) \,=\, {\Sigma}(S;{\cal Q})$,
    there is at least one for which the number of subintervals of $[a,b]$ determined by ${\cal R}$ is a minimum.
    Choose ${\cal P}$ to be such a partition. Suppose that there exists an index $k$ such that the adjacent subintervals $[x_{k-1},x_{k}]$ and $[x_{k},x_{k+1}]$ both contain points of~$S$;
    note that this implies that $1\,\,{\leq}\,\,k\,\,{\leq}\,\,n-1$. Then the sum forming the quantity ${\Sigma}(S;{\cal P})$ includes the summands $(x_{k}-x_{k-1})$ and $(x_{k+1}-x_{k})$,
    which together add up to $x_{k+1}-x_{k-1}$. Now let ${\cal T}$ be the partition of $[a,b]$ obtained by removing the number $x_{k}$ from the set~${\cal P}$.
    Then ${\Sigma}(S;{\cal T}) \,=\, {\Sigma}(S;{\cal P}) \,=\, {\Sigma}(S;{\cal Q})$, but the partition ${\cal T}$
    has one fewer subintervals than the partition~${\cal P}$, contradicting the definition of~${\cal P}$.
    That is, ${\cal P}$ has the desired `non-adjacency' property. In particular, if $1\,\,{\leq}\,\,k\,\,{\leq}\,\,n-1$, then $x_{k}$, which belongs to adjacent subintervals,
    cannot be an element of~$S$. The fact that $a \,=\, x_{0}$ and $b \,=\, x_{n}$ also cannot be in $S$ follows from the hypothesis that $a\,<\,{\inf}\,S$ and $b\,>\,{\sup}\,S$.

\VA

        (ii) The proof here is by mathematical induction on the number $N$.

        \underline{Initial Step}\,Suppose that $N \,=\, 1$, so that $S \,{\subseteq}\, (c_{1},d_{1})$. Let ${\cal P} \,=\,\{a\,\,{\leq}\,\,c_{1}\,<\,d_{1}\,\,{\leq}\,\,b\}$ be the corresponding partition of $[a,b]$.
    Then $[c_{1},d_{1}]$ is a subinterval of the partition ${\cal P}$ such that $S \,{\subseteq}\, (c_{1},d_{1}) \,{\subseteq}\, [c_{1},d_{1}]$,
    so that clearly ${\Sigma}(S;{\cal P}) \,=\, (d_{1}-c_{1})$, and thus ${\Sigma}(S;{\cal P})\,\,{\leq}\,\,(d_{1}-c_{1})$, as required.

        \underline{Induction Step}\, Suppose that the claim is true for $N \,=\, k$. To see that it remains true when $N \,=\, k+1$, let $S$ be a subset of $[a,b]$,
    let $U \,=\, (J_{1}, J_{2},\,{\ldots}\,J_{k+1})$ be an ordered list of intervals $J_{i} \,=\, [c_{i},d_{i}]$ in $[a,b]$ which form an open cover of~$S$.
    Let $V \,=\, (J_{1},\,{\ldots}\,J_{k})$ be the sublist of $U$ formed from the first $k$ terms in~$U$, and let $T_{1} \,=\, {\bigcup}_{i=1}^{k} (S\,{\cap}\,J_{i})$;
    clearly the intervals in the list $V$ form an open cover of~$T_{1}$. Let ${\cal Q}$ be the partition of $[a,b]$ formed, as above, from the endpoints of the intervals $J_{i}$, $1\,\,{\leq}\,\,i\,\,{\leq}\,\,k$.
    It follows from the induction hypothesis that
        \begin{displaymath}
        {\Sigma}(T_{1};{\cal Q})\,\,{\leq}\,\,\sum_{i=1}^{k} (d_{i}-c_{i}).
        \end{displaymath}
    Similarly, let $T_{2} \,=\, S\,{\setminus}\,T_{1}$. It is clear that for each $i \,=\, 1,2,\,{\ldots}\,k$ one has
    $T_{2}\,{\cap}\,J_{i} \,=\, {\emptyset}$, and thus $T_{2} \,{\subseteq}\, (c_{k},d_{k})$. Let ${\cal R} \,=\, \{a,c_{k+1}, d_{k+1},b\}$.
    It then follows from the truth of the given result in the case $N \,=\, 1$ that
        \begin{displaymath}
        {\Sigma}(T_{2},{\cal R})\,\,{\leq}\,\,(d_{k+1},c_{k+1}).
        \end{displaymath}
    Finally, let ${\cal P} \,=\, {\cal Q}\,{\cup}\,{\cal R}$. Then it follows from Part~(a) of Lemma~\Ref{LemmaH20.55B}, together with the equation $S \,=\, T_{1}\,{\cup}\,T_{2}$, that
        \begin{displaymath}
        {\Sigma}(S;{\cal P})\,\,{\leq}\,\,{\Sigma}(T_{1};{\cal P}) + {\Sigma}(T_{2};{\cal P})
    \,\,{\leq}\,\,
        {\Sigma}(T_{1}; {\cal Q}) + {\Sigma}(T_{2};{\cal R})
    \,\,{\leq}\,\,
        \left(\sum_{i=1}^{k} (d_{i}-c_{i})\right) + (d_{k+1}-c_{k+1}).
        \end{displaymath}
    The desired result now follows. \Q

\V

        \subsection{\small{{\bf Corollary}}}
        \label{CorH20.55BC}

\V

\hspace*{\parindent}(a)   Every subset of a Jordan-null set is also Jordan null.

\V

        (b) A necessary and sufficient condition for a bounded subset $S$ of ${\RR}$ to be Jordan null is that its closure $\overline{S}$ in ${\RR}$ be Jordan null.

\V

        (c) Let $S$ be a nonempty subset of an interval $[a,b]$. The condition for $S$ to be Jordan null can be weakened slightly to the following:
    For every ${\varepsilon}\,>\,0$ there exists a partition ${\cal P}$ of $[a,b]$ and a Jordan-null subset $T$ of $S$, which may depend on~${\cal P}$,
    such that ${\Sigma}(S\,{\setminus}\,T;{\cal P})\,<\,{\varepsilon}$.

\V

        (d) Let $S$ be a nonempty subset of an interval $[a,b]$. Then a necessary and sufficient condition for $S$ to be Jordan null is that
    for every ${\varepsilon}\,>\,0$ there exists a finite open cover $U \,=\, \{J_{1}, J_{2},\,{\ldots}\,J_{N}\}$,
    formed from open intervals $J_{i} \,=\, (c_{i},d_{i})$, $i \,=\, 1,2,\,{\ldots}\,N$, such that $\sum_{i=1}^{N} (d_{i}-c_{i})\,<\,{\varepsilon}$.


\V
        {\bf Proof}\,(a) The simple proof is left as an exercise. %EXERCISE

\V

         (b) It is clear from Part (a) that if the closure $\overline{S}$ is a Jordan null set, then so is~$S$ itself.
        Now suppose, conversely, that $S$ is Jordan null. Without loss of generality assume that $S$ is a subset of an interval $[a.b]$ such that $a\,<\, {\inf}\,S$ and $b\,>\,{\sup}\,S$. Let ${\cal P} \,=\, \{a \,=\, x_{0}\,<\,x_{1}\,<\,\,{\ldots}\,\,<\,x_{n-1}\,<\,x_{n} \,=\, b\}$ be a partition of $[a,b]$,
    and let $K$ be the set of indices $k$ for which $S\,{\cap}\,[x_{k-1}, x_{k}]  \,\,{\neq}\,\,  {\emptyset}$.
    Assume that ${\cal P}$ is chosen, as in Part~(b) of the preceding lemma, so that if $k{\in}K$, then each subinterval of ${\cal P}$ adjacent to the subinterval $[x_{k-1},x_{k}]$ is disjoint from~$S$.
    One certainly has
        \begin{displaymath}
        S \,=\, \,{\bigcup}_{k{\in}K} \left(S\,{\cap}\,[x_{k-1},x_{k}]\right),
        \end{displaymath}
    which, because each interval $[x_{k-1},x_{k}]$ is closed in~${\RR}$, implies that
        \begin{displaymath}
        \overline{S} \,=\, \,{\bigcup}_{k{\in}K} \left(\overline{S}\,{\cap}\,[x_{k-1},x_{k}]\right).
        \end{displaymath}
    Unfortunately, the latter equation does {\em not} imply that $K$ is also the set of all indices $k$ such that $\overline{S}\,{\cap}\,[x_{k-1},x_{k}] \,\,{\neq}\,\, {\emptyset}$.
    Indeed, it is possible that $\overline{S}\,{\cap}\,[x_{k-1},x_{k}]$ includes one or both of the endpoints $x_{k-1}$ or $x_{k}$,
    and thus that an adjacent interval {\em does} include points of~$\overline{S}$. The simple solution is to modify the partition ${\cal P}$ by widening, slightly,
    the intervals of the form $[x_{k-1},x_{k}]$ with $k{\in}K$, thus simultaneously narrowing the other subintervals.
    For example, let ${\delta}\,>\,0$ be small enough that if $k{\in}K$ and $1\,<\,k\,<\,n$, then
        \begin{displaymath}
        \frac{x_{k-2}+x_{k-1}}{2}\,<\,x_{k-1} - {\delta}
        \mbox{ and }
        x_{k} + {\delta}\,<\,\frac{x_{k+1}+x_{k}}{2}.
        \end{displaymath}
    Similarly, let ${\delta}$ also be small enough so that if $k \,=\, 1$ is in $K$, then $x_{1} + {\delta}\,<\,(x_{1}+x_{2})/2$,
    while if $k \,=\, n$ is in~$K$, then $x_{n-1}-{\delta}\,>\,(x_{n-2} + x_{n-1})/2$. For each $k{\in}K$ replace $x_{k}$ by $x_{k}+{\delta}$ and $x_{k-1}$ by $x_{k-1}-{\delta}$,
    as appropriate, in ${\cal P}$, and leave the other partition points of ${\cal P}$ unchanged, to obtain a new partition
    ${\cal Q} \,=\, \{a \,=\, y_{0}\,<\,y_{1}\,<\,\,{\ldots}\,y_{n-1}\,<\,y_{n} \,=\, b\}$ of $[a,b]$.
    With this partition, the original set $K$ is now the set of indices for which $\overline{S}\,{\cap}\,[y_{k-1},y_{k}] \,\,{\neq}\,\, {\emptyset}$,
    and the adjacent subintervals for ${\cal Q}$ do not intersect~$\overline{S}$. Since $K$ has at most $n$ elements,
    and for each $k$ in $K$ there are at most two extensions of length ${\delta}$, it follows that
        \begin{displaymath}
        {\Sigma}(\overline{S};{\cal Q}) \,=\, \sum_{k{\in}K} (y_{k}-y_{k-1})
    \,\,{\leq}\,\,
        2\,n\,{\delta} + {\Sigma}(S;{\cal P}).
        \end{displaymath}
    Finally, let ${\varepsilon}\,>\,0$ be given, let ${\cal P}$ be chosen so that in addition one has ${\Sigma}(S;{\cal P})\,<\,{\varepsilon}/2$,
    and choose ${\delta}\,>\,0$ small enough as above and also small enough that ${\delta}\,<\,{\varepsilon}/(4\,n)$.
    It follows that ${\Sigma}(\overline{S};{\cal P})\,<\,{\varepsilon}$, and thus $\overline{S}$ is a Jordan null set, as claimed.

\V


        (c) Let ${\varepsilon}\,>\,0$ be given, and let ${\cal P}$ and $T$ be chosen so that ${\Sigma}(S\,{\setminus}\,T;{\cal P})\,<\,{\varepsilon}/2$ and $T$ is Jordan~null.
    Let ${\cal Q}$ be a partition of $[a,b]$ such that ${\Sigma}(T;{\cal Q})\,<\,{\varepsilon}/2$, and let ${\cal R} \,=\, {\cal P}\,{\cup}\,{\cal Q}$.
    Then, since $S \,=\, (S\,{\setminus}\,T)\,{\cup}\,T$, and ${\cal R}$ is a refinement of both ${\cal P}$ and~${\cal Q}$, it follows from Part~(a) of Lemma~\Ref{LemmaH20.55B} that
        \begin{displaymath}
        {\Sigma}(S;{\cal R})\,\,{\leq}\,\,{\Sigma}(S\,{\setminus}\,T;{\cal R}) + {\Sigma}(T;{\cal R})
    \,\,{\leq}\,\,
        {\Sigma}(S\,{\setminus}\,T;{\cal P}) + {\Sigma}(T;{\cal Q})\,<\,\frac{{\varepsilon}}{2} + \frac{{\varepsilon}}{2} \,=\, {\varepsilon}.
        \end{displaymath}

\V

        (d)\,This follows directly from Part~(b) of Lemma~\Ref{LemmaH20.55B} above.

\V

        \subsection{\small{{\bf Examples}}}
        \label{ExampH20.55C}

\V

\hspace*{\parindent}(1) It is easy to show that every finite subset of ${\RR}$ is Jordan null.

\V

        (2) Suppose that $S$ and $T$ are Jordan null sets in~${\RR}$. Then $W \,=\, SS\,{\cup}\,T$ is also a Jordan null set.
    Indeed, since (by definition) $S$ and $T$ are bounded subsets of~${\RR}$, it follows that so is $W$. Let $[a,b]$ be any interval such that $W \,{\subseteq}\, [a,b]$,
    so that $S$ and $T$ are also subsets of~$[a,b]$. (Recall that, by Remark~\Ref{RemrkH20.55A}~(2), the choice of the particular interval $[a,b]$does not matter here.)
    Let ${\varepsilon}\,>\,0$ be given, and let ${\cal P}$ and ${\cal Q}$ be partitions of $[a,b]$ such that ${\Sigma}(S;{\cal P})\,<\,{\varepsilon}/2$ and ${\Sigma}(T;{\cal Q})\,<\,{\varepsilon}/2$.
    Let ${\cal R} \,=\, {\cal P}\,{\cup}\,{\cal Q}$, so that ${\cal R}$ is a refinement of both ${\cal P}$ and~${\cal Q}$. Then, by Remark~\Ref{RemrkH20.55A}~(1), one has
        \begin{displaymath}
        {\Sigma}(W;{\cal R})\,\,{\leq}\,\,{\Sigma}(S;{\cal R}) + {\Sigma}(T;{\cal R})
    \,\,{\leq}\,\,
        {\Sigma}(S;{\cal P}) + {\Sigma}(T;{\cal Q})
    \,<\,
        \frac{{\varepsilon}}{2} + \frac{{\varepsilon}}{2} \,=\, {\varepsilon}.
        \end{displaymath}
    The desired result follows.

        By repeatedly applying this result, it is easy to see that the union of finitely many Jordan null sets is also Jordan null.

\V

        (3) Let $S$ be the set of rational numbers in the unit interval $[0,1]$. It is clear that if ${\cal P}$ is any  partition of $[0,1]$,
    then {\em every} subinterval of ${\cal P}$ has nonempty intersection with~$S$, so that ${\Sigma}(S;{\cal P}) \,=\, 1$. In particular, $S$ is {\em not} Jordan null.
    Of course, this set $S$ is countable, and thus is the union of countably many Jordan null sets, namely, singleton sets.

\V

        (4) Consider, instead, the set $S \,=\, \{1,1/2,1/3,\,{\ldots}\,1/n,\,{\ldots}\,\}$ consisting of all the reciprocals of natural numbers.
    This set is also the union of countably many Jordan-null sets, but here it is an easy exercise to show that $S$ {\em is} Jordan null. %EXERCISE Convergent sequence is J-null

\V

        (5) It is an interesting exercise to show that the Cantor Middle-Thirds set, although uncountable, {\em is} Jordan null. % EXERCISE
    
\VV

    The preceding examples illustrate the complexity of the relation between `Jordan null' and `countable unions'. The next result clarifies this relation somewhat.

\V

        \subsection{\small{{\bf Theorem}}}
        \label{ThmH20.50A}

\V

        Suppose that $S$ is a bounded subset of ${\RR}$ which can be expressed as the union of a countable family of Jordan-null sets.
    If, in addition, $S$ is a closed subset of ${\RR}$, then $S$ is also Jordan null.

\V

        {\bf Proof} Without loss of generality, assume that $S \,\,{\neq}\,\, {\emptyset}$ and that $a$ and $b$ are numbers such that $a\,<\,{\inf}\,S$ and $b\,>\,{\sup}\,S$.
    Then, by hypothesis, there is a sequence $X_{1}$, $X_{2}$,\,{\ldots}\,$X_{n}$,\,{\ldots}\, of Jordan-null subsets of $[a,b]$ such that $S \,=\, {\bigcup}_{j=1}^{{\infty}} X_{j}$,
    and it follows from Part~(b) of Lemma~\Ref{LemmaH20.55B} that for each index $j$ there exists a partition
    ${\cal P}_{j}$ of $[a,b]$ such that for which no partition point of ${\cal P}_{j}$ is in the set~$X_{j}$; denote the $k$-th subinterval of ${\cal P}_{j}$ by $[c_{jk},d_{jk}]$.
    For each index $j$ let $K_{j}$ denote the set of numbers $k$ such that the $k$-th subinterval of the partition ${\cal P}_{j}$ has nonempty intersection with~$X_{j}$,
    and let $V_{j} \,=\, {\bigcup}_{k{\in}K_{j}}\,(c_{jk},d_{jk})$ be the union of the interiors of such subintervals of~${\cal P}_{j}$.
    Then $X_{j} \,{\subseteq}\, V_{j}$, since $X_{j}$ has no partition points of ${\cal P}_{j}$, and ${\Sigma}(X_{j};{\cal P}_{j}) \,=\, \sum_{k{\in}K_{j}}\,(d_{jk}-c_{jk})\,<\,{\varepsilon}/2^{j}$.
    Since $S \,=\, {\bigcup}_{j \,=\, 1}^{{\infty}} X_{j} \,{\subseteq}\, {\bigcup}_{j \,=\, 1}^{{\infty}} V_{j}$, it follows that the open sets $V_{j}$, $j \,=\, 1,2,\,{\ldots}\,$ form an open cover of~$S$.
    Since, by hypothesis, $S$ is closed and bounded in~${\RR}$, it follows from the Heine-Borel Theorem that there is a finite subcollection $V_{j_{1}}, V_{j_{2}},\,{\ldots}\,V_{j_{m}}$,
    with $j_{1}\,<\,j_{2}\,<\,\,{\ldots}\,\,<\,j_{m}$, of these open sets which also covers~$S$. To simplify the notation, let $N \,=\, j_{m}$.
    Then the collection of sets $V_{j}, 1\,\,{\leq}\,\,j\,\,{\leq}\,\,N$, is also a finite open cover of $S$,
    as is the collection of open intervals of the form $(c_{jk},d_{jk})$, with $k{\in}K_{j}$ and $1\,\,{\leq}\,\,j\,\,{\leq}\,\,N$.
    The sum of the lengths of these open intervals satisfies the conditions
        \begin{displaymath}
        \sum_{j=1}^{N} \sum_{k{\in}K_{j}} (d_{jk}-c_{jk}) \,=\, \sum_{j=1}^{N} {\Sigma}(X_{j};{\cal P}_{j})
    \,<\,
    \sum_{j=1}^{N} \frac{{\varepsilon}}{2^{j}}\,<\,{\varepsilon}.
        \end{displaymath}
    Since ${\varepsilon}$ can be any positive number, it follows from Part~(b) of Lemma~\Ref{LemmaH20.55B} that $S$ is Jordan null, as claimed. \Q

\V
        \subsection{\small{{\bf Example}}}
        \label{ExampH20.50B}

\V

\hspace*{\parindent} The subset $S \,=\, [a,b]$ cannot be expressed as the countable union of Jordan-null subsets of~$[a,b]$. Indeed, since $S$ is a closed subset of $[a,b]$,
    if it could be so expressed, then it would itself be Jordan null, contraary to what has already been proved.


\VV

        The preceding theorem can be used to provide an important characterization of Riemann integrabilty.

\V

        \subsection{\small{{\bf Theorem}}}
        \label{ThmH20.60}

\V

        Suppose that $g:[c,d]\,{\rightarrow}\,{\RR}$ is a bounded function on the interval $[c,d]$, and let $D$ denote the set of discontinuities of $g$ on~$[c,d]$;
    at the endpoints $c$ and $d$ use the appropriate `one-sided' notion of continuity. Then a necessary and sufficient condition for $g$ to be Riemann integrable on $[c,d]$
    is that $D$ can be expressed as the union of a countable family of Jordan-null sets.

\V

        {\bf Proof}\,To simplify the discussion, let $a$ and $b$ be numbers so that $a\,<\,c$ and $b\,>\,d$, so that $a\,<\,{\inf}\,D$ and $b\,>\,{\sup}\,D$.
    Define $f:[a,b] \,{\rightarrow}\, {\RR}$ by the rule
        \begin{displaymath}
        f(x) \,=\, \left\{
        \begin{array}{cl}
        0 & \mbox{if $a\,\,{\leq}\,\,x\,<\,c$} \\
        g(x) & \mbox{if $c\,\,{\leq}\,\,x\,\,{\leq}\,\,d$} \\
        0 & \mbox{if $d\,<\,x\,\,{\leq}\,\,b$}
        \end{array}
        \right.
        \end{displaymath}
    Denote the set of discontinuities of $f$ on $[a,b]$ by $S$, again using the appropriate notion of one-sided continuity at the endpoints $a$ and~$b$.
    Note that $S$ consists of the points of $D$, possibly augmented by one or both of the endpoints $c$ and~$d$.
    It is clear from Examples~\Ref{ExampH20.55C}~(2) and~(3) that $D$ can be expressed as the union
    of a countable family of Jordan-null sets if, and only if, the same property holds for~$S$.
    Likewise, it is clear from Parts~(b) and~(d) of Theorem~\Ref{ThmH20.38} that $g$ is integrable on $[c,d]$ if, and only if, $f$ is integrable on~$[a,b]$.
    Thus, it suffices to consider the integrability of $f$ on~$[a,b]$, for which the set $S$ of discontinuities
    satisfies the simplifying condition $a\,<\,{\inf}\,S$ and $b\,>\,{\sup}\,S$.

        Suppose first that $f$ is integrable on~$[a,b]$. For each natural number $q$ let $S_{q}$ denote the set of points $x$ in $[a,b]$ such that ${\omega}_{f}(x)\,\,{\geq}\,\,1/q$.
    For such $q$ let ${\varepsilon}\,>\,0$ be given, and let ${\cal P} \,=\, \{a \,=\, x_{0}\,<\,x_{1}\,<\,\,{\ldots}\,\,<\,x_{n-1}\,<\,x_{n} \,=\, b\}$
    be any partition of $[a,b]$ such that ${\Delta}(f;{\cal P})\,<\,{\varepsilon}\,q$. Let $T \,=\, {\cal P}$.
    Clearly if $x{\in}S_{q}\,{\setminus}\,T$, then $x_{k-1}\,<\,x\,<\,x_{k}$ for exactly one index~$k$.
    It follows that for such $x$ one has $1/q\,\,{\leq}\,\,{\omega}_{f}(x)\,\,{\leq}\,\,M_{k}-m_{k}$, and thus $(x_{k}-x_{k-1})\,\,{\leq}\,\,(M_{k}-m_{k})\,(x_{k}-x_{k-1})\,q$.
    Let $K$ be the set of indices $k$ such that $(S_{q}\,{\setminus}\,T)\,{\cap}\,[x_{k-1},x_{k}] \,\,{\neq}\,\, {\emptyset}$. Then one has
        \begin{displaymath}
        {\Sigma}(S_{q}\,{\setminus}\,T;{\cal P})
     \,=\, 
        \sum_{k{\in}K} (x_{k}-x_{k-1})
    \,\,{\leq}\,\,
        \sum_{k{\in}K} (M_{k} - m_{k})\,(x_{k}-x_{k-1})\,q
    \,\,{\leq}\,\,
        \sum_{j=1}^{n} (M_{j} - m_{j})\,(x_{j}-x_{j-1})\,q
    \,<\,
        \frac{{\varepsilon}}{q}\,q \,=\, {\varepsilon}.
        \end{displaymath}
    It now follows from Part~(c) of Corollary~\Ref{CorH20.55BC} that $S_{q}$ is Jordan null. Clearly $S$ is the union of the countable family of Jordan null sets of the form $S_{q}$ with $q$ in~${\NN}$.

        Conversely, suppose that the stated condition holds, and let $Y_{1}$, $Y_{2}$,\,{\ldots}\,$Y_{n}$,\,{\ldots}\, be a sequence of Jordan-null sets whose union is~$S$.
    For each $q$ in ${\NN}$ let $S_{q}$ be as above. It is easy to see that $S_{q}$ is a closed and bounded subset of~${\RR}$. % EXERCISE
    Since $S_{q}$ is the union of the Jordan null sets of the form $X_{j} \,=\, Y_{j}\,{\cap}\,S_{q}$ for $j$ in~${\NN}$,
    it follows from Theorem~\Ref{ThmH20.50} that, for each $q$, $S_{q}$ is also Jordan null. Theorem??? (REFERENCE???)
    then implies that $f$ is integrable on~$[a,b]$, hence $g$ is integrable on~$[c,d]$, as claimed. \Q

\VV

        The preceding result illustrates the importance of a set being expressible as the countable union of Jordan-null subsets of~${\RR}$.
    This property is equivalent to a related concept due to Lebesgue.

\V

                \subsection{\small{{\bf Definition}}}
                \label{DefH20.70}
\V

        A subset $S$ of ${\RR}$ is said to be {\bf of Lebesgue measure zero}, or, more briefly, {\bf Lebesgue null},\IndB{Lebesgue}{of Lebesgue measure zero; Lebesgue null},
    provided it can be expressed as the union of countably many Jordan null subsets of~${\RR}$.

\V

        {\bf Remarks} (1) In contrast with the definition of `Jordan null', the definition `Lebesgue null' does not restrict $S$ to be a {\em bounded} subset of~${\RR}$.
    This makes sense because every subset $S$ of ${\RR}$ can be expressed as a countable union of bounded sets; for example, the sets of the form $S\,{\cap}\,[k,k+1]$ with $k$ in~${\ZZ}$.

\V

        (2) Lebesgue's original formulation of a set being of (Lebesgue) measure zero is slightly different. The equivalence of his definition and the one given here is outlined in an exercise. % EXERCISE

\VV

        With this new terminology, one can now phrase Theorem~\Ref{ThmH20.60} in the following more standard form:

\V

        \subsection{\small{{\bf Theorem}}}
        \label{ThmH20.75A}

\V

        Suppose that $g:[c,d]\,{\rightarrow}\,{\RR}$ is bounded on the interval $[c,d]$, and let $D$ denote the set of discontinuities of $g$ on~$[c,d]$;
    at the endpoints $c$ and $d$ use the appropriate `one-sided' notion of continuity. Then a necessary and sufficient condition for $g$ to be Riemann integrable on $[c,d]$
    is that $D$ be a set of Lebesgue measure~zero.

\VV

        The significance of the preceding result is that it shows that the (Riemann) integrability of a bounded function on an interval depends only on the set $D$ of discontinuitiesof that function,
    and not the detailed nature of those discontinuities at individual points of~$D$. The next result illustrates the power of this formulation.

\VV

        It is convenient for future reference to point out the following simple facts about sets which are Lebesgue~null. The simple proofs are left as exercises. %% EXERCISES

\V

        \subsection{\small{{\bf Theorem}}}
        \label{ThmH20.75AB}

\V

\hspace*{\parindent}(a) Every Jordan-null set is Lebesgue null. In particular, every finite sybset of ${\RR}$ is Lebesgue null.

\V

        (b) Every subset of a Lebesgue-null set is Lebesgue null.

\V

        (c) The union of a countable family of Lebesgue-null sets is Lebesgue null. (Recall that the analogous statement with `Lebesgue' replaced by `Jordan' is not true.)
    In particular, every countable subset of ${\RR}$ is Lebesgue null.

\V

        (d) If $S$ contains an interval as a subset, then $S$ is {\em not} Lebesgue null.

\VV

        In Examples~\Ref{ExampH20.30}, Theorem~\Ref{ThmH25.20} and Theorem~\Ref{ThmH25.30} we determine the integrability, or nonintegrability, in several particular cases.
    Each case requires a separate argument based on the specific nature of the function at hand.
 The following result shows how Theorem~\Ref{ThmH20.75A} allows a unified treatment of such cases.

\VV


        \subsection{\small{{\bf Examples}}}
        \label{ExampH20.75B}

\V%

\hspace*{\parindent}(1) If $f:[a,b] \,{\rightarrow}\, {\RR}$ is continuous on~$[a,b]$, then it is Riemann integrable on~$[a,b]$.
    Indeed, in this case the set $D$ of discontinuities of $f$ is the empty set, which is a finite set.

\V

        (2) The Dirichlet function is {\em not} Riemann integrable on any interval~$[a,b]$, since the corresponding set of discontinuities is $[a,b]$ itself, which is not Lebesgue null.

\V

        (3) If $f:[a,b] \,{\rightarrow}\, {\RR}$ is bounded on~$[a,b]$ and the set of discontinuities of $f$ in~$[a,b]$ is countable, then $f$ is Riemann integrable on~$[a,b]$.

        \underline{Special Cases}\,The set of discontinuities of the Thomae function in the interval $[0,1]$ is a subset of the rationals, hence is countable.
    Likewise, the set of discontinuities of a function $f:[a,b] \,{\rightarrow}\, {\RR}$ that is monotonic on $[a,b]$ is countable.
    Since both functions are clearly bounded, it follows that both are Riemann intergable.

\V

        (4) Suppoose that $f:[a,b] \,{\rightarrow}\, {\RR}$ is bounded on $[a,b]$ and that $c$ is a number such that $a\,<\,c\,<\,b$.
    Then $f$ is Riemann integrable on $[a,b]$ if, and only if, it is Riemann integrable on each of the subintervals $[a,c]$ and~$[c,b]$.
    Indeed, let $D$ be the  set of discontinuities of $f$ on~$[a,b]$. Likewise, let $D_{1}$ and $D_{2}$ be the corresponding sets for $[a,c]$ and $[c,b]$,
    respectively, but using the appropriate one-sided limits at~$c$. It is clear that either $D \,=\, D_{1}\,{\cup}\,D_{2}$ or $D \,=\, D_{1}\,{\cup}\,D_{2}\,{\cup}\,\{c\}$.
    In either case, it follows from Theorem~\Ref{ThmH20.75AB} that $D$ is Lebesgue null if, and only if, both of the sets $D_{1}$ and $D_{2}$ are Lebesgue null.

\V

        (5) Suppose that $f_{1}$ and $f_{2}$ are both Riemann integrable on~$[a,b]$. Let $D_{1}$ be the set of discontinuities of $f_{1}$ in~$[a,b]$, and let $D_{2}$ be the corresponding set for $f_{2}$.
    Then the set $D$ of discontinuities in $[a,b]$ of the function $f \,=\, f_{1} + f_{2}$ is a subset of the finite union $D_{1}\,{\cup}\,D_{2}$ of Lebesgue-null sets, and thus is itself Lebesgue null.
    Since $f_{1}$ and $f_{2}$ must be bounded on~$[a,b]$, so is their sum~$f$. Thus $f$ is also Riemann integrable on~$[a,b]$.

    A similar argument shows that the product $g \,=\, f_{1}{\cdot}f_{2}$ of the Riemann integrable functions $f_{1}$ and $f_{2}$ is Riemann integrable on~$[a,b]$, then $g$ is Riemann integrable on~$[a,b]$.
    Likewise, if there is a constant $c\,>\,0$ such that $|f_{2}(x)|\,\,{\geq}\,\,c$ for all $x$ in~$[a,b]$, then the quotient $h \,=\, f_{1}/f_{2}$ is bounded on $[a,b]$,
    and its set of discontinuities is a subset of the Lebesgue-null set $D_{1}\,{\cup}\,D_{2}$, so that $h$ is also Riemann integrable on~$[a,b]$.

\V

        {\bf Remark} In Part~(b) of Theorem~\Ref{ThmH25.20} we prove not only that the sum $f \,=\, f_{1} + f_{2}$ of two Riemann integrable functions is integrable,
    but also the formula ${\displaystyle \int_{a}^{b} f \,=\, \int_{a}^{b} f_{1} + \int_{a}^{b} f_{2}}$. The proof of that formula can be simplified if one already has proved,
    as above, that $f$ is integrable on~$[a,b]$. Indeed, from the integrability of $f_{1}$ and $f_{2}$ one has
        \begin{displaymath}
        L(f_{1};{\cal P})\,\,{\leq}\,\,\int_{a}^{b} f_{1}\,\,{\leq}\,\,U(f_{1};{\cal P})
    \mbox{ and }
        L(f_{2};{\cal P})\,\,{\leq}\,\,\int_{a}^{b} f_{2}\,\,{\leq}\,\,U(f_{2};{\cal P})
        \end{displaymath}
    for every partition ${\cal P}$ of $[a,b]$. It then follows from the standard properties of Darboux sums that for each such partition ${\cal P}$ one has
        \begin{displaymath}
        L(f_{1}+f_{2};{\cal P})
    \,\,{\leq}\,\,
        L(f_{1}; {\cal P}) + L(f_{2};{\cal P})
    \,\,{\leq}\,\,
        \left(\int_{a}^{b} f_{1}\right) + \left(\int_{a}^{b} f_{2}\right)
    \,\,{\leq}\,\,
        U(f_{1};{\cal P}) + U(f_{2}; {\cal P})
    \,\,{\leq}\,\,
        U(f_{1}+ f_{2};{\cal P}).
        \end{displaymath}
    In particular, one has
        \begin{displaymath}
        L(f;{\cal P})
    \,\,{\leq}\,\,
        \left(\int_{a}^{b} f_{1}\right) + \left(\int_{a}^{b} f_{2}\right)
    \,\,{\leq}\,\,
        U(f;{\cal P}).
        \end{displaymath}
    for every such partition~${\cal P}$.
   However, since $f$ is already proved to be integrable on~$[a,b]$, it follows that there is only one number which lies between $L(f;{\cal P})$ and $U(f;{\cal P})$
    for each such partition~${\cal P}$, namely ${\displaystyle \int_{a}^{b} f}$. The desired equation now follows.

\VV

        The following consequnce of Theorem~\Ref{ThmH20.75A} makes it easy to prove the Riemann integrability of a wide class of functions.

\V

        \subsection{\small{{\bf Corollary}}}
        \label{CorH20.75C}

\V
    Suppose that $f:[a,b] \,{\rightarrow}\, {\RR}$ is Riemann integrable on~$[a,b$.
    Suppose further that $H:X \,{\rightarrow}\, {\RR}$ is a function which is defined on a subset $X$ of ${\RR}$ containing all numbers of the form $h(x)$ with $x$ in~$[a,b]$.
    If $H$ is continuous and bounded on~$X$, then the composition $g \,=\, H{\circ}f:[a,b] \,{\rightarrow}\, {\RR}$ is also Riemann integrable on~$[a,b]$.

\V

        {\bf Proof}\, The hypotheses certainly imply that the expression $(F{\circ}g)(x)$ is defined for every $x$ in~$[a,b]$.
    In addition, Theorem~\Ref{ThmD20.70} implies that $g$ is continuous at each $x$ in $[a,b]$ at which $f$ is continous,
    hence the set $S$ of discontinuities of $g$ in $[a,b]$ is a subset of the set $D$ of discontinuities of $f$ in~$[a,b]$.
    It follows easily that $S$ is a set of Lebesgue measure zero, and thus $h$ is Riemann integrable on~$[c,d]$, as claimed.

\VV

        {\bf Example} It was proved in Theorem~\Ref{ThmH25.30} that if $f:[a,b] \,{\rightarrow}\, {\RR}$ is Riemann integrable on~$[a,b]$,
    then so is the function~$|f|$. This fact also follows from the preceding corollary by choosing $H$  in that result to be the absolute-value function.


\VV

%----------------------C
\StartSkip{

        So far all the integrals we have studied have been defined on closed {\em bounded} intervals in ${\RR}$;
    likewise, all the functions have been {\em bounded} functions.
    The next definition allows one to extend the concept of `definite integral' to integrals of unbounded functions and to integrals over unbounded intervals.

\V

        \subsection{\small{{\bf Definition} (Improper Riemann Integrals)}}
        \label{DefH50.60}

\V

       (a) Let $b$ be either a real number or the symbol $+{\infty}$, and let $f:[c,b) \,{\rightarrow}\, {\RR}$ be a function on the half-open interval $[c,b)$, where $c$ is a real number such that $c\,<\,b$.
    Assume that $f{\in}{\cal R}_{[c,d]}$ for every real number $d$ such that $c\,<\,d\,<\,b$.

        Suppose that the limit ${\displaystyle \lim_{x{\nearrow}b} \int_{c}^{b} f}$ exists and equals~$L$;
    the quantity $L$ can be either a (finite) number or one of the symbols $+{\infty}$ or $-{\infty}$.
    Then one says that the {\bf improper (Riemann) integral ${\displaystyle \int_{c}^{b} f(x)\,dx}$ exists and equals $L$}.
    If $L$ is finite, one says that the integral ${\displaystyle \int_{c}^{b} f(x)\,dx}$ is convergent.
    If, instead, the limit diverges to $L$, where $L \,=\, +{\infty}$ or $L \,=\, -{\infty}$,
    then one says that the integral ${\displaystyle \int_{c}^{b} f(x)\,dx}$ {\bf diverges to $L$}.

\V

        (b) In like manner, let $a$ be either a real number or the symbol $-{\infty}$,
    and let $f:(a,c] \,{\rightarrow}\, {\RR}$ be a function on the half-open interval $(a,c]$, where $c$ is a real number such that $c\,>\,b$.
    Assume that $f{\in}{\cal R}_{[c,d]}$ for every real number $d$ such that $a\,<\,c\,<\,d$.
    Then the improper integral ${\displaystyle \int_{a}^{c} f(x)\,dx}$ is defined in much the same way as in Part~(a).

\V

        (c) Suppose that $f$ is defined on an open interval $(a,b)$, where $-{\infty}\,\,{\leq}\,\,a\,<\,b\,\,{\leq}\,\,+{\infty}$.
    Assume that $f{\in}{\cal R}_{[c,d]}$ for every closed bounded subinterval $[c,d]$ of $(a,b)$.
    Let $p$ be any point of $(a,b)$.
    If the improper integrals ${\displaystyle \int_{a}^{p} f(x)\,dx}$ and ${\displaystyle \int_{p}^{b}}$ both exist,
    and it is {\em not} the case that one of these integrals equals $+{\infty}$ while the other equals $-{\infty}$,
    then one defines the improper integral ${\displaystyle \int_{a}^{b} f(x)\,d{\alpha}}$ by the rule
        \begin{displaymath}
        \int_{a}^{b} f(x)\,dx \,=\, \int_{a}^{p} f(x)\,dx + \int_{p}^{b} f(x)\,dx.
        \end{displaymath}
    It is convenient to refer to such an integral as a {\bf doubly improper integral}.

\V

        \subsection{\small{{\bf Remarks}}}
        \label{RemrkH50.70}

\V

\hspace*{\parindent}(1) It is easy to show that in Part~(c) the choice of the point $p$ is irrelevant.

\V

        (2) If a function $f:I \,{\rightarrow}\, {\RR}$ has the property that $f{\in}{\cal R}_{[c,d]}$ for all closed bounded subintervals of an interval~$I$,
    one says that $f$ is {\bf locally Riemann integrable on $I$}.

\V

        (3) It follows from Part~(e) of Theorem~\Ref{ThmH30.20} that if $f$ is locally integrable on an interval $I$, then so is~$|f|$.
    (Of course the converse is {\em not} true.)


\V
\V

        \subsection{\small{{\bf Examples}}}
        \label{DefH50.80}

\V

\hspace*{\parindent}(1) Let $p$ be a positive number, and suppose that $f:(0,+{\infty}) \,{\rightarrow}\, {\RR}$ is given by the rule $f(x) \,=\, 1/x^{p}$.
    Then $f$ is continuous on $(0,+{\infty})$, and one has
        \begin{displaymath}
        D^{-1}_{1} f(x) \,=\, \left\{
        \begin{array}{cl}
        {\displaystyle \frac{1}{p-1} - \frac{1}{(p-1)x^{p-1}}} & \mbox{if $p\,>\,1$}   \\
                                                               & \\
            {\ln}\,x                           & \mbox{if $p \,=\, 1$} \\
                                               &                       \\
        {\displaystyle \frac{x^{1-p}-1}{1-p}}                 & \mbox{if $0\,<\,p\,<\,1$}
        \end{array}
                                        \right.
        \end{displaymath}
    It follows easily that
        \begin{displaymath}
        \lim_{x{\nearrow}+{\infty}} D^{-1}_{1} f(x) \,=\, \left\{
        \begin{array}{cl}
        {\displaystyle \frac{1}{p-1}} & \mbox{if $p\,>\,1$} \\
                                      &                     \\
            +{\infty}                 & \mbox{if $0\,<\,p\,\,{\leq}\,\,1$}
        \end{array}
                                    \right.
        \end{displaymath}
    Likewise,
        \begin{displaymath}
        \lim_{x{\searrow}0} D^{-1}_{1} f(x) \,=\, \left\{
        \begin{array}{cl}
        -{\infty} & \mbox{if $p\,\,{\leq}\,\,1$} \\
                                      &                     \\
            -\frac{1}{1-p}                 & \mbox{if $0\,<\,p\,<\,1$}
        \end{array}
                                    \right.
        \end{displaymath}
    It follows that ${\displaystyle \int_{1}^{{\infty}} \frac{dx}{x^{p}}}$ is convergent if $p\,>\,1$ and diverges to $+{\infty}$ when $0\,<\,p\,\,{\leq}\,\,1$.
    Likewise, ${\displaystyle \int_{0}^{1} \frac{dx}{x^{p}}}$ is divergent to  $+{\infty}$ when $p\,\,{\geq}\,\,1$ and convergent $0\,<\,p\,<\,1$.

        It also follows that for all $p\,>\,0$ the doubly improper integral ${\displaystyle \int_{0}^{+{\infty}} \frac{dx}{x^{p}}}$ diverges to~$+{\infty}$.

\V

        (2) It is easy to see that the improper integral ${\displaystyle \int_{0}^{{\infty}} e^{-x}\,dx}$ exists and equals~$1$.
    Indeed, one computes that if $d\,>\,0$ then ${\displaystyle \int_{0}^{d} e^{-x}\,dx \,=\, \left.-e^{-x}\right|_{0}^{d} \,=\, 1-e^{-d}}$,
    so ${\displaystyle \lim_{d{\nearrow}+{\infty}} \int_{0}^{d} e^{-x}\,dx \,=\, 1}$.

        Similarly, by using Integration-by-Parts and Mathematical Induction, it is easy to show that for each $n$ in ${\NN}$ one has ${\displaystyle \int_{0}^{{\infty}} x^{n}e^{-x}\,dx \,=\, n!}$.

\V

        (3) The improper integral ${\displaystyle \int_{0}^{{\infty}} {\cos}\,x\,dx}$ clearly does not exist, since ${\displaystyle \int_{0}^{d} {\cos}\,x}\,dx  \,=\, {\sin}\,d$, and $\lim_{d \,{\rightarrow}\, {\infty}} {\sin}\,d$ does not exist.

\V

        (4) Suppose that $f:(a,b) \,{\rightarrow}\, {\RR}$ is locally Rieman integrable on $(a,b)$.
    Assume that $f(x)\,\,{\geq}\,\,0$ for all $x$ in $(a,b)$.
    Then the improper integral ${\displaystyle \int_{a}^{b} f(x)\,dx}$ exists: its value is either a positive number or~$+{\infty}$.

\V
\V


        \subsection{\small{{\bf Theorem}} (The Comparison Test)}
        \label{ThmH50.90}

\V

        Suppose that $f$ and $g$ are locally integrable functions on an interval $(a,b)$ such that $0\,\,{\leq}\,\,f(x)\,\,{\leq}\,\,g(x)$ for all $x$ in $(a,b)$.

\V

        (a) If ${\displaystyle \int_{a}^{b} g(x)\,dx}\,<\,+{\infty}$, then one has ${\displaystyle 0\,\,{\leq}\,\,\int_{a}^{b} f(x)\,dx\,\,{\leq}\,\,\int_{a}^{b}g(x)\,<\,+{\infty}}$.

\V

        (b) If ${\displaystyle \int_{a}^{b} f(x)\,dx \,=\, +{\infty}}$, then ${\displaystyle \int_{a}^{b} g(x) \,=\, +{\infty}}$.

\V

        The simple proof is left as an exercise.

\V

        \subsection{\small{{\bf Corollary}}}
        \label{CorH50.100}

\V

        Let $a$, $b$ be such that $-{\infty}\,\,{\leq}\,\,a\,<\,b\,\,{\leq}\,\,+{\infty}$.
    If $f$ is locally Riemann integrable on $(a,b)$, and if ${\displaystyle \int_{a}^{b} |f(x)|\,dx}$ is finite,
    then the improper integral ${\displaystyle \int_{a}^{b} f(x)\,dx}$ converges.

\V

        {\bf Proof} Since $f$ and $|f|$ are locally integrable on $(a,b)$ (see Remark~\Ref{RemrkH50.70}~(3) above),
    it is clear that $|f| + f$ is locally integrable on $(a,b)$. Note that one has
        \begin{displaymath}
        0\,\,{\leq}\,\,|f(x)| + f(x)\,\,{\leq}\,\,2|f(x)| \mbox{ for all $x$ in $(a,b)$}.
        \end{displaymath}
    Thus, by the preceding theorem, one has
        \begin{displaymath}
        0\,\,{\leq}\,\,\int_{a}^{b} \left(|f(x)| + f(x)\right)\,dx\,\,{\leq}\,\,2\int_{a}^{b} |f(x)|\,dx\,<\,+{\infty}.
        \end{displaymath}
    The desired result follows by noting that $f(x) \,=\, (|f(x)| + f(x)) - |f(x)|$.

\V
\V

        \subsection{\small{{\bf Examples}}}
        \label{ExampH50.110}

\V

\hspace*{\parindent}(1) Since $e^{-x^{2}}\,\,{\leq}\,\,e^{-x}$ for all $x$ in $[0,+{\infty})$, it follows that ${\displaystyle \int_{1}^{{\infty}} e^{-x^{2}}\,dx}$ is convergent.

\V

        (2) One has ${\displaystyle \int_{{\pi}}^{{\infty}} \left|\frac{{\sin}\,x}{x}\right|} \,=\, +{\infty}$.
    Indeed, for each $k{\in}{\NN}$ if $k{\pi} + {\pi}/4\,\,{\leq}\,\,x\,\,{\leq}\,\,3{\pi}/4 + k{\pi}$, then $|{\sin}\,x|/|x|\,\,{\geq}\,\,1/((k+1){\pi}\sqrt{2})$.
 Thus
        \begin{displaymath}
        \int_{k{\pi}}^{(k+1){\pi}} \left|\frac{{\sin}\,x}{x}\right|\,dx\,\,{\geq}\,\,\frac{{\pi}}{(k+1){\pi}\sqrt{2}} \,=\, \frac{1}{(k+1)\sqrt{2}}.
        \end{displaymath}
    The claimed result follows from the fact that the harmonic series diverges to~$+{\infty}$.

\V

        (3) The improper integral ${\displaystyle \int_{{\pi}}^{{\infty}} \frac{{\sin}\,x}{x}\,dx}$ converges.
    Indeed, note that if $d\,>\,{\pi}$ then one has
        \begin{displaymath}
        \int_{{\pi}}^{d} \frac{{\sin}\,x}{x}\,dx \,=\, -\left.\frac{{\cos}\,x}{x}\right|_{{\pi}}^{d} - \int_{{\pi}}^{d} \frac{{\cos}\,x}{x^{2}}\,dx.
        \end{displaymath}
    It follows easily that the right side of this equation approaches a finite limit as $d$ approaches $+{\infty}$.
%

}% \EndSkip
%-----------------C

\newpage

\input{Exercises_M140AB_H_2017} %% NOTE: Automatically starts on a new page
%
%

